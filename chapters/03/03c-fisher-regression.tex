\index{regression}
\index{Fisher information!regression function}
We are now equipped to derive the Fisher information for our regression function.
For convenience, we restate the regression model and its assumptions.
The regression model relating response variables $y_i \in \bbR$ and the covariates $x_i \in \cX $, for $i=1,\dots,n$ is
\begin{gather}
  y_i = \alpha + f(x_i) + \epsilon_i \retag{eq:model1} \\
  (\epsilon_1, \dots, \epsilon_n)^\top \sim \N_n(0, \bPsi^{-1}) \retag{eq:model1ass}
\end{gather}
where $\alpha \in \bbR$ is an intercept and $f$ is in an RKKS $\cF$ with kernel $h:\cX \times \cX \to \bbR$.
Note that the dependence of the kernel on parameters $\eta$ is implicitly assumed.

\begin{lemma}[Fisher information for regression function]\label{thm:fisherregf}
  For the regression model \cref{eq:model1} subject to \cref{eq:model1ass} and $f \in \cF$ where $\cF$ is an RKKS with kernel $h$, the Fisher information for $f$ is given by
  \[
    \cI(f) = \sum_{i=1}^n\sum_{j=1}^n \psi_{ij} h(\cdot,x_i) \otimes h(\cdot,x_j)
  \]
  where $\psi_{ij}$ are the $(i,j)$'th entries of the precision matrix $\bPsi$ of the normally distributed model errors.
  More generally, suppose that $\cF$ has a feature space $\cV$ such that the mapping $\phi:\cX \to \cV$ is its feature map, and if $f(x)=\ip{\phi(x),v}_\cV$, then the Fisher information $I(v) \in \cV \otimes \cV$ for $v$ is
  \[
    \cI(v) = \sum_{i=1}^n\sum_{j=1}^n \psi_{ij} \phi(x_i) \otimes \phi(x_j).
  \]
\end{lemma}

\begin{proof}
	For $x \in \mathcal X$, let $k_x:\mathcal V \rightarrow \mathbb R$ be defined by $k_x(v) = \langle \phi(x), v \rangle_{\mathcal V}$. 
	Clearly, $k_x$ is linear and continuous.
%	By the reproducing property, $k_x(f) = f(x)$. 
	Hence, the Gâteaux derivative of $k_x(v)$ in the direction $u$ is
	\begin{align*}
		\partial_u k_x(v)	
		&= \lim_{t \rightarrow 0} \frac{k(v+t u) - k(v)}{t} \\
		&= \lim_{t \rightarrow 0} \frac{\langle \phi(x), v+t u \rangle_{\cV} - \langle \phi(x), v \rangle_{\cV}}{t} \\
		&= \lim_{t \rightarrow 0} \frac{\langle \phi(x), \cancel{v} + tu - \cancel{v} \rangle_{\cV}}{t} \\		
		&= \lim_{t \rightarrow 0} \frac{t\langle \phi(x), u \rangle_{\cV}}{t} \\
		&= \langle \phi(x), u \rangle_{\cV}.
	\end{align*}
    Since clearly $\partial_u k_x(v)$ is a continuous linear operator for any $u\in\cV$, it is bounded, so the Fréchet derivative exists and $\d k_x(v) = \partial k_x(v)$.
%	Thus, the gradient is $\nabla k_x(v) = \phi(x)$ by definition. 
	Let $\by = \{y_1,\dots,y_n\}$, and denote the hyperparameters of the regression model by $\theta = \{ \alpha,\bPsi,\eta \}$.
	Without loss of generality, assume $\alpha = 0$, and even if this is not so, we can always add back $\alpha$ to the $y_i$'s later.
	Regardless, both $\alpha$ and $\by$ are constant in the differential of $L(v|\by, \theta)$.
	The log-likelihood of $v$ is given by
	\begin{align*}
		L(v|\by,\theta) 
		&= \const - \frac{1}{2} \sum_{i=1}^n\sum_{j=1}^n \psi_{ij}\big(y_i - k_{x_i}(v)\big)\big(y_j  - k_{x_j}(v)\big)
	\end{align*}
	and the score by
	\begin{align*}
	  \d L(\cdot|\by,\theta)
	  &= - \frac{1}{2} \sum_{i=1}^n\sum_{j=1}^n \psi_{ij} \, \d (k_{x_i}k_{x_j} - y_jk_{x_i} - y_ik_{x_j} + y_iy_j )  \\
	  &= - \frac{1}{2} \sum_{i=1}^n\sum_{j=1}^n \psi_{ij} (k_{x_j} \d k_{x_i}  + k_{x_i}\d k_{x_j} - y_j \d k_{x_i} - y_i \d k_{x_j} ).
	\end{align*}
%	where we had made the substitution $\nabla k_x(v) = \phi(x)$. 
	Differentiating again gives
	\begin{align*}
		\d^2 L(\cdot|\by,\theta)
		&= - \frac{1}{2} \sum_{i=1}^n\sum_{j=1}^n \psi_{ij} (\d k_{x_j} \d k_{x_i}  + \d k_{x_i}\d k_{x_j}) \\
		&= - \sum_{i=1}^n\sum_{j=1}^n \psi_{ij} \, \d k_{x_i} \d k_{x_j} \\
		&= - \sum_{i=1}^n\sum_{j=1}^n \psi_{ij}  \langle \phi(x_i), \cdot \rangle_{\cV} \, \langle \phi(x_j), \cdot \rangle_{\cV} ,
	\end{align*}
	since the derivative of $\d k_{x} = \ip{\phi(x),\cdot}_\cV$ is zero (it is the derivative of a constant).
	We can then calculate the Fisher information to be
	\begin{align*}
		\cI(v) = -\E \big[ \d^2 L(v|\by,\theta) \big] 
		&=\E \left[ \sum_{i=1}^n\sum_{j=1}^n \psi_{ij}  \langle \phi(x_i), \cdot \rangle_{\cV} \, \langle \phi(x_j), \cdot \rangle_{\cV} \right] \\
		&= \sum_{i=1}^n\sum_{j=1}^n \psi_{ij} \, \langle \phi(x_i) \otimes \phi(x_j), \cdot \rangle_{\cV\otimes\cV}  \\		
		&=\sum_{i=1}^n\sum_{j=1}^n \psi_{ij} \cdot \phi(x_i) \otimes \phi(x_j).
%		&\rlap {\color{grymath}\text{by substituting $\nabla k_x(f) = h(\cdot,x)$, the expectation}} \\
%		&\rlap {\color{grymath}\text{is free of $f$}} 
	\end{align*}	 	
	Here, we had treated $\phi(x_i) \otimes \phi(x_j)$ as a bilinear operator, since $\cI(v) \in \cV \otimes \cV$ as well.
	Also, the expectation is free of the random variable under expectation (i.e. $\by$), which makes the second line possible.
	
	By taking the canonical feature $\phi(x)=h(\cdot,x)$, we have that $\phi \equiv h(\cdot,x):\cX\to\cF \equiv \cV$ and therefore for $f\in\cF$, the reproducing property gives us $f(x) = \ip{h(\cdot,x),f}_\cF$, so the formula for $\cI(f) \in \cF \otimes \cF$ follows.
\end{proof}

The above lemma gives the form of the Fisher information for $f$ in a rather abstract fashion.
Consider the following example of applying \cref{thm:fisherregf} to obtain the Fisher information for a standard linear regression model.

\begin{example}[Fisher information for linear regression]
  \index{inner product!dot product}
  \index{outer product|seealso{tensor product}}
  As before, suppose model \cref{eq:model1} subject to \cref{eq:model1ass} and $f\in\cF$, an RKHS.
  For simplicity, we assume iid errors, i.e. $\bPsi = \psi \bI_n$.
  Let $\cX = \bbR^p$, and the feature space $\cV = \bbR^p$ be equipped with the usual dot product $\ip{\cdot,\cdot}_\cV:\cV \otimes \cV \to \bbR$ defined by $v^\top v$.
  Consider also the identity feature map $\phi:\cX \to \cV$ defined by $\phi(\bx)=\bx$.
  For some $\boldsymbol\beta \in \cV$, the linear regression model is such that $f(\bx) = \bx^\top \boldsymbol\beta = \ip{\phi(\bx),\boldsymbol\beta}_\cV$.
  Therefore, according to \cref{thm:fisherregf}, the Fisher information for $\bbeta$ is
  \begin{align*}
    \cI(\boldsymbol\beta) 
    &= \sum_{i=1}^n\sum_{j=1}^n \psi \cdot \phi(\bx_i) \otimes \phi(\bx_j) \\
    &= \psi \sum_{i=1}^n\sum_{j=1}^n \bx_i \otimes \bx_j \\
    &= \psi \bX^\top \bX.
  \end{align*}
  Note that the operation `$\otimes$' on two vectors in Euclidean space is simply their outer product.
  The resulting $\bX$ is a $n \times p$ matrix containing the entries $\bx_1^\top,\dots,\bx_n^\top$ row-wise.
  This is of course recognised as the Fisher information for the regression coefficients in the standard linear regression model.
\end{example}

We can also compute the Fisher information for linear functionals of $f$, and in particular, for point evaluation functionals of $f$, thereby allowing us to compute the Fisher information at two points $f(x)$ and $f(x')$.

\begin{corollary}[Fisher information between two linear functionals of $f$]\label{thm:fisherreglinfunc}
	For our regression model as defined in \cref{eq:model1} subject to \cref{eq:model1ass} and $f$ belonging to an RKKS $\mathcal F$ with kernel $h$, the Fisher information at two points $f(x)$ and $f(x')$
%	, written $\cI\big(f(x),f(x')\big)$, is given by
	is given by
	\[
		\cI\big(f(x),f(x')\big) = \sum_{i=1}^n\sum_{j=1}^n \psi_{ij} h(x,x_i)h(x',x_j).
	\]
\end{corollary}

\begin{proof}
  In an RKKS $\mathcal F$, the reproducing property gives $f(x) = \langle f, h(\cdot, x) \rangle_{\mathcal F}$ and in particular, $\langle h(\cdot,x), h(\cdot, x') \rangle_{\mathcal F} = h(x,x')$. 
  By \cref{eq:fisher-linear-functional}, we have that
  \begin{align*}
    \cI(f)\big(h(\cdot,x),h(\cdot,x') \big)
    &= \big\langle \cI(f), h(\cdot, x) \otimes h(\cdot, x') \big\rangle_{\mathcal F \otimes \mathcal F} \\
    &= \Bigg\langle \sum_{i=1}^n\sum_{j=1}^n \psi_{ij} h(\cdot,x_i) \otimes h(\cdot,x_j) \ , \ h(\cdot, x) \otimes h(\cdot, x') \Bigg\rangle_{\mathcal F \otimes \mathcal F} \\
    &= \sum_{i=1}^n\sum_{j=1}^n \psi_{ij} \big\langle h(\cdot,x_i), h(\cdot, x) \big\rangle_{\mathcal F} \big\langle h(\cdot,x_j), h(\cdot, x') \big\rangle_{\mathcal F } \\
    %		&{\color{grymath}\text{(by using the fact that inner products are linear, and that $\forall a_1, a_2 \in \mathcal A$}} \\
    %		&{\color{grymath}\text{and $\forall b_1, b_2 \in \mathcal B$, $\langle a_1 \otimes b_1, a_2 \otimes b_2 \rangle_{\mathcal A \otimes \mathcal B} = \langle a_1, a_2 \rangle_{\mathcal A}\langle b_1, b_2 \rangle_{\mathcal B}$)}} \\
    &= \sum_{i=1}^n\sum_{j=1}^n \psi_{ij} h(x,x_i) h(x', x_j).
    %		\mycomment{(by the reproducing property)}
%    &=: \cI\big(f(x),f(x')\big).
  \end{align*}
  The second to last line follows from the definition of the usual inner product for tensor spaces, and the last line follows by the reproducing property.
\end{proof}

An inspection of the formula in \cref{thm:fisherreglinfunc} reveals the fact that the Fisher information for $f(x)$, $\cI \big(f(x),f(x) \big)$, is positive if and only if $h(x,x_i)\neq 0$ for at least one $i \in \{1,\dots,n \}$.
In practice, this condition is often satisfied for all $x$, so this result might be considered both remarkable and reassuring, because it suggests we can estimate $f$ over its entire domain, no matter how big, even though we only have a finite amount of data points.
