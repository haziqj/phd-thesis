From \cref{thm:fisherregf}, the formula for the Fisher information uses $n$ points of the observed data $x_i\in\cX$.
This seems to suggest that the Fisher information only exists for a finite subspace of the RKKS $\cF$.
Indeed, this is the case, and we will be specific about the subspace for which there is Fisher information.
Consider the following set, a similar one considered in the proof of the Moore-Aronszajn theorem \colp{\cref{thm:moorea}}, \mypageref{thm:moorea}):
\begin{align}\label{eq:subspaceFn}
\cF_n = \left\{ f:\cX \to \bbR \, \bigg| \, f(x) = \sum_{i=1}^n h(x,x_i)w_i, \ w_i \in \bbR, \ i=1,\dots,n \right\}.  
\end{align}
Since $h(\cdot,x_i) \in \cF$, any $f \in \cF_n$ is also in $\cF$ by linearity, and thus $\cF_n$ is a subset of $\cF$.
Further, $\cF_n$ is closed under addition and multiplication by a scalar, and is therefore a subspace of $\cF$.
Unlike \cref{thm:moorea}, $\cF_n$ defined here is a finite subspace of dimension $n$.

Let $\cF_n^\bot$ be the orthogonal complement of $\cF_n$ in $\cF$.
By the orthogonal decomposition theorem \colp{\cref{thm:orthdecomp}, \mypageref{thm:orthdecomp}}, any regression function $f\in\cF$ can be uniquely decomposed as $f = f_n + r$, with $f_n \in \cF_n$ and $r \in \cF_n^\bot$, where $\mathcal F = \mathcal F_n \oplus  \cF_n^\bot$.
We saw in the proof of \cref{thm:moorea} that $\cF$ is the closure of $\cF_n$, so therefore $\cF$ is dense in $\cF_n$, and hence by \cref{thm:orthdecomp2} \colp{\mypageref{thm:orthdecomp2}} we have that $\cF_n^\bot = \{0\}$.
Alternatively, we could have argued that any $r \in \cF_n^\bot$ is orthogonal to each of the $h(\cdot,x_i)\in\cF$, so by the reproducing property of $h$, $r(x_i) = \ip{r,h(\cdot,x_i)}_\cF = 0$.
This suggests the following corollary.

\begin{corollary}
  With $g \in \cF$, the Fisher information for $g$ is zero if and only if $g\in\cF_n^\bot$, i.e. if and only if $g(x_1) = \cdots = g(x_n) = 0$.
\end{corollary}

\begin{proof}  
  Let $\cI(f)$ be the Fisher information for $f$.
  The Fisher information for $\ip{f,r}_\cF$ is
  \begin{align*}
    \cI(f)(r,r) 
    &= \ip{\cI(f), r\otimes r}_{\cF\otimes\cF} \\
    &= \sum_{i=1}^n\sum_{j=1}^n \psi_{ij} \ip{h(\cdot,x_i),r}_\cF\ip{h(\cdot,x_j),r}_\cF \\
    &= \sum_{i=1}^n\sum_{j=1}^n \psi_{ij} r(x_i)r(x_j).
  \end{align*}
  So if $r\in\cF_n^\bot$, then $r(x_1) = \cdots = r(x_n) = 0$, and thus the Fisher information at $r\in\cF_n^\bot$ is zero.
  Conversely, if the Fisher information is zero, it must necessarily mean that $r(x_1) = \cdots = r(x_n) = 0$ since $\psi_{ij}>0$, and thus $r\in\cF_n^\bot$.
%  Conversely, suppose that the Fisher information is zero, and that $r \notin \cF_n^\bot$. 
%  Then, $r$ must be in $\cF_n$, i.e. $r=\sum_{i=1}^n h(\cdot,x_i) w_i$, for some $\{w_i\}$.
%  We can then argue
%  \begin{align*}
%    \ip{r,f}_\cF
%    &= \left\langle \sum_{i=1}^n h(\cdot,x_i) w_i, f \right\rangle_\cF \\
%    &= \sum_{i=1}^n w_i \left\langle h(\cdot,x_i), f \right\rangle_\cF \\
%    &= \sum_{i=1}^n w_if(x_i)
%  \end{align*}
\end{proof}

The above corollary implies that the Fisher information for our regression function $f\in\cF$ exists only on the $n$-dimensional subspace $\mathcal F_n$. 
More subtly, as there is no Fisher information for $r\in\cF_n^\bot$, $r$ cannot be estimated from the data.
Thus, in estimating $f$, we will only ever consider the finite subspace $\mathcal F_n \subset \mathcal F$ where there is information about $f$.

As it turns out, $\cF_n$ can be identified as a RKHS with reproducing kernel equal to the Fisher information for $f$.
That is, the real, symmetric, and positive-definite function $h_n$ over $\cX\times\cX$ defined by $h_n(x,x') = \cI\big(f(x),f(x')\big)$ is associated to the RKHS which is $\cF_n$, equipped with the squared norm 
$\norm{f}_{\cF_n}^2 = \sum_{i,j=1}^n w_i(\bPsi^{-1})_{ij}w_j$.
%$||f||_{\mathcal F_n}^2 = w^\top\Psi^{-1}w$.
This is stated in the next lemma.

\begin{lemma}\label{thm:subspaceFn}
  Let $\cF_n$ as in \cref{eq:subspaceFn} be equipped with the inner product
  \begin{align}\label{eq:Fninnerprod}
    \ip{f, f'}_{\cF_n} = \sum_{i=1}^n\sum_{j=1}^n w_i(\bPsi^{-1})_{ij}w_j' = \bw^\top\bPsi\bw' 
  \end{align}
  for any two $f=\sum_{i=1}^n h(\cdot,x_i)w_i$ and $f'=\sum_{j=1}^n h(\cdot,x_j)w_j'$ in $\cF_n$.
  Then, $h_n:\cX\times\cX\to\bbR$ as defined by
  \[
    h_n(x,x') = \sum_{i=1}^n\sum_{j=1}^n \psi_{ij}h(x,x_i)h(x',x_j)
  \]
  is the reproducing kernel of $\cF_n$.
\end{lemma}

\begin{proof}
%  Since $\cF_n$ is a finite subspace of $\cF$, it is complete, and thus a Hilbert space.
  What needs to be proven is the reproducing property of $h_n$ for $\cF_n$.
  First note that by defining $w_j(x) = \sum_{k=1}^n \psi_{jk} h(x,x_k)$, we see that
  \begin{align*}
    h_n(x,\cdot) 
    &= \sum_{j=1}^n\sum_{k=1}^n \psi_{jk} h(x,x_j) h(\cdot,x_k) 
    = \sum_{j=1}^n w_j(x)h(\cdot,x_j)
  \end{align*}
  Furthermore, writing $h(\cdot,x_j) = \sum_{k=1}^n \delta_{jk} h(\cdot,x_k)$, with $\delta$ being the Kronecker delta, we see that $h(\cdot,x_j)$ is also an element of $\cF_n$, and in particular,
  \[
    \big\langle h(\cdot,x_i), h(\cdot,x_k) \big\rangle_{\cF_n} 
    = \sum_{j=1}^n\sum_{l=1}^n \delta_{ij} (\bPsi^{-1})_{jl} \delta_{lk} = (\bPsi^{-1})_{ik}.
  \]
  Denote by $\psi_{ij}^{-}$ the $(i,j)$'th element of $\bPsi^{-1}$.
  A fact we will use later is $\sum_{k=1}^n \psi_{jk}\psi_{ik}^{-} = (\bPsi\bPsi^{-1})_{ji} = (\bI_n)_{ji} = \delta_{ji}$.  
  In the mean time,
  \begin{align*}
    \ip{f, h_n(x,\cdot)}_{\cF_n}
    &= \left\langle 
    \sum_{i=1}^n h(\cdot,x_i)w_i ,
    \sum_{j=1}^n\sum_{k=1}^n \psi_{jk} h(x,x_j) h(\cdot,x_k)
    \right\rangle_{\cF_n} \\
    &= \sum_{i=1}^n w_i \sum_{j=1}^n\sum_{k=1}^n \psi_{jk} h(x,x_j) 
    \big\langle h(\cdot,x_i) , h(\cdot,x_k) \big\rangle_{\cF_n} \\
    &= \sum_{i=1}^n w_i \sum_{j=1}^n\sum_{k=1}^n \psi_{jk}h(x,x_j)\psi_{ik}^{-} \\
    &= \sum_{i=1}^n w_i \sum_{j=1}^n \delta_{ji}h(x,x_j) \\
    &= \sum_{i=1}^n w_i h(x,x_i) \\
    &= f(x).
  \end{align*}
  Therefore, $h_n$ is a reproducing kernel for $\cF_n$.
  Obviously, $h_n$ is positive definite (it is a squared kernel), and hence it defines the RKHS $\cF_n$.
\end{proof}

%\hltodo{Is the Fisher information metric and semi-norm over $\cF$ useful?}

