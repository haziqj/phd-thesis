It was \citet{ra1922mathematical} who introduced the method of maximum likelihood as an objective way of conducting statistical inference.
This method of inference is distinguished from the Bayesian school of thought in that only the data may inform deductive reasoning, but not any sort of  prior probabilities.
Towards the later stages of his career\footnote{The introductory chapter of \citet{pawitan2001all} and the citations therein give a delightful  account of the evolution of the Fisherian view regarding statistical inference.}, his work reflected the view that the likelihood is to be more than simply a device to obtain parameter estimates; it is also a vessel that carries uncertainty about estimation.
In this light and in the absence of the possibility of making probabilistic statements, one should look to the likelihood in order to make rational conclusions about an inference problem.
Specifically, we may ask two things of the likelihood function: where is the maxima and what does the graph around the maxima look like?
The first of these two problems is maximum likelihood estimation, while the second concerns the Fisher information.

In simple terms, the Fisher information measures the amount of information that an observable random variable $Y$ carries about an unknown parameter $\theta$ of the statistical model that models $Y$.
To make this concrete, $Y$ has the density function $p(\cdot|\theta)$ which depends on $\theta$.
Write the log-likelihood function of $\theta$ as $L(\theta) = \log p(Y|\theta)$, and the gradient function of the log-likelihood (the \emph{score function}) with respect to $\theta$ as $S(\theta) = \partial L(\theta)/\partial\theta$.
The \emph{Fisher information} about the parameter $\theta$ is defined to be expectation of the second moment of the score function, 
%$\cI(\theta) = \E[S(\theta)^2] = \int S(\theta)^2 \, p(y|\theta) \d y$.
\[
  \cI(\theta) = \E\left[\left(\frac{\partial}{\partial\theta} \log p(Y|\theta)\right)^2\right].
\]
Here, expectation is taken with respect to the random variable $Y$ under its true distribution.
Under certain regularity conditions, it can be shown that $\E[S(\theta)] = 0$, and thus the Fisher information is in fact the variance of the score function, since $\Var[S(\theta)] = \E[S(\theta)^2] - \E^2[S(\theta)]$.
Further, if $\log p(Y|\theta)$ is twice differentiable with respect to $\theta$, then it can be shown that under certain regularity conditions,
\[
  \cI(\theta) = \E\left[-\frac{\partial ^ 2}{\partial\theta^2} \log p(Y|\theta)\right].
\]
Many textbooks provides a proof of this fact---see, for example, \citet[Section 9.7]{wasserman2013all}.

From the last equation above, we see that the Fisher information is related to the curvature or concavity of the graph of the log-likelihood function, averaged over the random variable $Y$.
The curvature, defined as the second derivative on the graph\footnote{Formally, the graph of a function $g$ is the set of all ordered pairs $(x, g(x))$.}~of a function, measures how quickly the function changes with changes in its input values.
This then gives an intuition regarding the uncertainty surrounding $\theta$ at its maximal value; high Fisher information is indicative of a sharp peak at the maxima and therefore small variance, while low Fisher information is indicative of a shallow maxima for which many $\theta$ share similar log-likelihood values.
%Fisher information may be added much in the same way as log-likelihood may be added---the \emph{total Fisher information} from $n$ independent and identically distributed random variables $Y_1,\dots,Y_n$ is simply the sum of the $n$ \emph{unit Fisher information}, i.e. $\cI_n(\theta) = n \cI(\theta)$.
%\hltodo[Check if total Fisher information is relevant.]{}
