\documentclass[a4paper,showframe,11pt,draft]{report}
\usepackage{standalone}
\standalonetrue
\ifstandalone
  \usepackage{../../haziq_thesis}
  \usepackage{../../haziq_maths}
  \usepackage{../../haziq_glossary}
  \addbibresource{../../bib/haziq.bib}
  \externaldocument{../01/.texpadtmp/introduction}
\fi

\begin{document}
\hChapterStandalone[3]{Fisher information and the I-prior}

Traditionally, Fisher information is calculated for unknown parameters $\theta$ of probability distribution from observable random variables.
In a similar light, we can treat the regression function $f$ in the model stated in \eqref{eq:model1}, subject to \eqref{eq:model1ass}, as the unknown ``parameter'' for which we would like information regarding.
In this chapter, we extend the notion of Fisher information to abstract objects in Hilbert spaces, and also to linear functionals of these objects.
This will allow us to achieve our aim of deriving the Fisher information for our regression function.

Following this, we shall discuss the notion of prior distributions for regression functions, and how one might assign a suitable prior.
In our case, we choose an objective prior following \citep{jaynes1957a,jaynes1957b}---in the absence of any prior knowledge, a prior distribution which maximises entropy should be used.
It turns out, the entropy maximising prior for $f$ is Gaussian with mean chosen a priori and covariance kernel proportional to the Fisher information.
Such a distribution on $f$ is called the I-prior distribution.

\section{The traditional Fisher information}
%\input{traditional-fisher.tex}

\section{Fisher information for Hilbert space objects}

We extend the idea beyond thinking about parameters as merely numbers in the usual sense, to abstract objects in Hilbert spaces. 
This generalisation allows us to extend the concept of Fisher information to regression functions in RKHSs later.
The score and Fisher information is derived in a familiar manner, but extra care is required when taking derivatives with respect to Hilbert space objects.  

Let $Y$ be a random variable with density in the parametric family $\{p(\cdot|\theta) \,|\, \theta \in \Theta \}$, where $\Theta$ is now assumed to be a Hilbert space with inner product $\ip{\cdot,\cdot}_\Theta$.
If $p(Y|\theta) > 0$\hltodo[Why wouldn't it be >0 ?]{}, the log-likelihood function of $\theta$ is the real-valued function $L(\cdot|Y):\Theta\to\bbR$ defined by $\theta \mapsto \log p(Y|\theta)$. 
To discuss derivatives of the log-likelihood function for $\theta\in\Theta$, we require a generalisation of the concept of differentiability from real-valued functions of a single, real variable, as is common in calculus, to functions between Banach spaces.

%\begin{definition}[Directional derivative and gradient]
%  Let ($\cH$, $\ip{\cdot,\cdot}_\cH$) be an inner product space, and consider a function $g:\mathcal H \rightarrow \mathbb R$. 
%  Denote the directional derivate of $g$ in the direction $z$ by $\nabla_z g$, that is, 
%	\[
%		\nabla_z g(x) = \lim_{\delta \rightarrow 0} \frac{g(x + \delta z) - g(x)}{\delta}.
%	\]
%	The gradient of $g$, denoted by $\nabla g$, is the unique vector field satisfying 
%	\[
%		\langle \nabla g(x), z \rangle_{\mathcal H} = \nabla_z g(x), \ \ \ \forall x,z \in \mathcal H.
%	\]
%\end{definition}

\begin{definition}[Fréchet derivative]
  Let $\cV$ and $\cW$ be two normed spaces, and $\cU \subseteq \cV$ be an open subset.
  A function $f:\cU\to\cW$ is called \emph{Fréchet differentiable} at $x \in \cU$ if there exists a bounded, linear operator $T:\cV\to\cW$ such that 
  \[
    \lim_{v\to 0} \frac{\big\Vert f(x+v) - f(x) - T v \big\Vert_{\cW}}{\norm{v}_\cV} = 0
  \]
  If this relation holds, then the operator $T$ is unique, and we write $\d f(x) := T$ and call it the \emph{Fréchet derivative} or \emph{Fréchet differential} of $f$ at $x$.
  If $f$ is differentiable at every point $\cU$, then $f$ is said to be \emph{differentiable} on $\cU$.
\end{definition}

\begin{remark}
  Since $\d f(x)$ is a bounded, linear operator, by Lemma X, it is also continuous. 
\end{remark}

\begin{remark}
  While many authors in the calculus of variations literature write the Fréchet derivative as derivative between Banach spaces, the definition also applies to Hilbert spaces.
  On the other hand, in the functional analysis literature, it is presented as derivatives in Hilbert spaces.
  A. V. Balakrishnan, Applied Functional Analysis. Springer, 1976.
  Extension of Wirtinger's Calculus to Reproducing Kernel Hilbert Spaces and the Complex Kernel LMS Pantelis Bouboulis, Sergios Theodoridis .
  For Gateaux derivative, $\cV$ need only be a vector space, while $\cW$ a topological space.
  For continuous linear functionals on $\bbR$ then this is fine.
\end{remark}


The intuition here is similar to that of regular differentiability, that the linear operator $T$ well approximates the change in $f$ at $x$ (the numerator), relative to the change in $x$ (the denominator)---the fact that the limit exists and is zero, it must mean that the numerator converges faster to zero than the denominator does.
In Landau notation, we have the familiar expression $f(x+h) = f(x) + \d f(x) h + o(h)$, that is, the tangent line to $f$ at $x$ gives the best linear approximation to $f$ near $x$.
The limit in the definition is meant in the usual sense of convergence of functions with respect to the norms of $\cV$ and $\cW$.
Of course, we may use Fréchet derivatives in Hilbert spaces too by using the inner product norm of the space.

For the avoidance of doubt, $\d f(x)$ is not a vector in $\cW$, but is an element of the set of bounded, linear operators from $\cV$ to $\cW$, denoted $\L(\cV,\cW)$.
That is, if $f:\cU\to\cW$ is a differentiable function at all points in $\cU\subseteq\cV$, then its derivative is a linear map
\begin{align*}
  \d f: \cU &\to \text{L}(\cV,\cW) \\
  x &\mapsto \d f(x).
\end{align*}
It follows that this function may also have a derivative, which by definition will be a linear map as well:
\begin{align*}
  \d^2 f: \cU &\to \text{L}\big(\cV,\text{L}(\cV,\cW)\big) \\
  x &\mapsto \d^2f(x).
\end{align*}
The space on the righthand side is identified with the Banach space $\L(\cV \times \cV, \cW)$ of all continuous bilinear maps from $\cV$ to $\cW$.
In other words, an element $\phi\in \text{L}\big(\cV,\text{L}(\cV,\cW)\big)$ is identified with $\psi \in \text{L}(\cV \times \cV, \cW)$ such that for all $x,y\in\cV$, $\phi(x)(y) = \psi(x,y)$.
Simply put, a function $\phi$ linear in $x$ with $\phi(x)$ linear in $y$ is the same as a bilinear function $\psi$ in $x$ and $y$.
The second derivative $\d^2 f(x)$ is therefore a bounded, bilinear operator from $\cV\times\cV$ to $\cW$.

Another closely related type of differentiability is the concept of \emph{Gâteaux differentials}, which is the formalism of functional derivatives in calculus of variations.
Let $\cV$, $\cW$ and $\cU$ be as before, and consider the function $f:\cU\to\cW$.

\begin{definition}[Gâteaux derivative]
  The \emph{Gâteaux differential} or the \emph{Gâteaux derivative} $\partial_v f(x)$ of $f$ at $x \in \cU$ in the direction $v\in\cV$ is defined as
  \[
    \partial_v f(x) = \lim_{t \to 0} \frac{f(x + t v) - f(x)}{t},  % = \frac{\partial}{\partial t}f(x+tv)\bigg|_{t=0}.
  \]  
  for which this limit is taken relative to the topology of $\cW$.
  The function $f$ is said to be \emph{Gâteaux differentiable} at $x\in\cU$ if $f$ has a directional derivative along all directions at $x$.
  We name the operator $\partial f(x):\cV\to\cW$ which assigns $v \mapsto \partial_v f(x) \in \cW$ the \emph{Gâteaux derivative} of $f$ at $x$, and the operator $\partial f:\cU\to(\cV,\cW) = \{A \,|\, A:\cV\to\cW \}$ which assigns $x \mapsto \partial f(x)$ simply the \emph{Gâteaux derivative} of $f$.
  
\end{definition}

\begin{remark}
  The space $(\cV,\cW)$ of operators from $\cV$ to $\cW$ is not a topological space, and there is no obvious way to define a topology on it.
  Consequently, we cannot consider the Gâteaux derivative of the Gâteaux derivative.
  Furthermore, unlike the Fréchet derivative, which is by definition a linear operator, the Gâteaux derivative may fail to satisfy the additive condition of linearity\footnote{Although, for all scalars $\lambda \in \bbR$, the Gâteaux derivative is homogenous: $\partial_{\lambda v}f(x) = \lambda \partial_v f(x)$.}.
  Finally, even if it is linear, it may fail to depend continuously on $v$ if $\cV$ and $\cW$ are infinite dimensional.
\end{remark}

Nevertheless, the reason we bring up Gâteaux differentials is that it may motivate higher-order Fréchet differentials.
First note the connection between the two, by again considering the function $f:\cU\to\cW$.

\begin{lemma}[Fréchet differentiability implies Gâteaux differentiability]
  If $f$ is Fréchet differentiable at $x\in\cU$, then $f$ is Gâteaux differentiable at that point too, and $\d f(x) = \partial f(x)$.
\end{lemma}

\begin{proof}
  Since $f$ is Fréchet differentiable at $x\in\cU$, we can write $f(x+v) \approx f(x) + \d f(x)(v)$ for some $v\in\cV$.
  Then, 
  \begin{align*}
    \left\Vert \frac{f(x + t v) - f(x)}{t} - \d f(x)(v) \right\Vert_\cW
    &= \frac{1}{t} \big\Vert f(x + t v) - f(x) - \d f(x)(tv)  \big\Vert_\cW \\
    &= \frac{\big\Vert f(x + t v) - f(x) - \d f(x)(tv) \big\Vert_\cW }{\norm{tv}_\cV}\cdot \norm{v}_\cV 
  \end{align*}
  which converges to 0 since $f$ is Fréchet differentiable at $x$, and  $t\to 0$ if and only if $\norm{tv}_\cV \to 0$.
  Thus, $f$ is Gâteaux differentiable at $x$, and the Gâteaux derivative $\partial_v f(x)$ of $f$ at $x$ in the direction $v$ coincides with the Fréchet derivatiave of $f$ at $x$ evaluated at $v$.
\end{proof}

Consider now the function $\d f(x):\cV\to\cW$ and suppose that $f$ is twice Fréchet differentiable at $x\in\cU$, i.e. $\d f(x)$ is Fréchet differentiable at $x\in\cU$ with derivative $\d^2 f(x):\cV\times\cV \to \cW$.
Then, $\d f(x)$ is also Gâteaux differentiable at the point $x$ and the two differentials coincide.
In particular, we have
\begin{align}\label{eq:frech2gat}
  \left\Vert \frac{\d f(x + t v)(v') - \d f(x)(v')}{t} - \d^2 f(x)(v,v') \right\Vert_\cW \to 0 \text{ as } t \to 0,
\end{align}
by a similar argument in the proof above.
We will use this fact when we describe the Hessian in a little while.

There is also the concept of \emph{gradients} in Hilbert space.
Recall that the Riesz representation theorem says that the mapping $A:\cV\to\cV'$ from the Hilbert space $\cV$ to its continuous dual space $\cV'$ defined by $A = \ip{\cdot,v}_\cV$ for some $v\in\cV$ is an isometric isomorphism.
Again, let $\cU \subseteq \cV$ be an open subset, and let $f:\cU\to\bbR$ be a (Fréchet) differentiable function with derivative $\d f: \cU \to \L(\cV,\bbR) \equiv \cV'$.
We define the gradient as follows.

\begin{definition}[Gradients in Hilbert space]
  The \emph{gradient} of $f$ is the operator $\nabla f: \cU \to \cV$ defined by $\nabla f = A^{-1} \circ \d f$.
  Thus, for $x \in \cU$, the gradient of $f$ at $x$, denoted $\nabla f(x)$, is the unique element of $\cV$ satisfying
  \[
    \ip{\nabla f(x), v}_\cV = \d f(x)(v)
  \]
  for any $v \in \cV$.
  Note that $\nabla f$ being a composition of two continuous functions, is itself continuous.
\end{definition}

Since the gradient of $f$ is an operator on $\cU$ to $\cV$, it may itself have a (Fréchet) derivative.
Assuming existence, i.e., $f$ is twice Fréchet differentiable at $x \in \cU$, we call this derivative the \emph{Hessian} of $f$.
From \eqref{eq:frech2gat}, it must be that
\begin{align*}
  \d^2 f(x)(v,v') &= \lim_{t\to 0} \frac{\d f(x + t v)(v') - \d f(x)(v')}{t} \\
  &= \lim_{t\to 0} \frac{\ip{\nabla f(x+tv), v'}_\cV - \ip{\nabla f(x), v'}_\cV}{t} \\
  &= \left\langle \lim_{t\to 0} \frac{\nabla f(x+tv) - \nabla f(x)}{t} , v' \right\rangle_\cV \\%\hspace{10pt} \rlap{\color{gray} \text{by linearity}} \\
  &= \left\langle \partial_v \nabla f(x) , v' \right\rangle_\cV.
\end{align*}
The second line follows from the definition of gradients, and the third line follows by linearity of inner products.
Note that since the Fréchet and Gâteaux differentials coincide, we have that $\partial_v \nabla f(x) = \d \nabla f(x) (v)$.
Letting $\cV$, $\cW$ and $\cU$ be as before, we now define the Hessian for the function $f:\cU \to \cW$.

\begin{definition}[Hessian]
  The Fréchet derivative of the gradient of $f$ is known as the \emph{Hessian} of $f$.
  Denoted $\nabla^2 f$, it is the mapping $\nabla^2 f: \cU \to \L(\cV,\cV)$ defined by $\nabla^2 f  = \d \nabla f$, and it satisfies
  \[
    \left\langle \nabla^2 f(x)(v) , v' \right\rangle_\cV = \d^2 f(x)(v,v').
  \]
  for $x\in\cU$ and $v,v'\in\cV$.
\end{definition}

\begin{remark}
  Since $\d^2 f(x)$ is a bilinear form in $\cV$, we can equivalently write
  \[
    \d^2 f(x)(v,v') = \ip{\d^2 f(x), v \otimes v'}_{\cV\otimes\cV}
  \]
  following the correspondence between bilinear forms and tensor product spaces.  
\end{remark}

We can now define the score $S$, assuming existence, as the (Fréchet) derivative of $L(\cdot|Y)$, i.e. $S:\Theta \to \L(\Theta,\bbR) \equiv \Theta'$ defined by $S = \d L(\cdot | Y)$.
The second (Fréchet) derivative of $L(\cdot|Y)$ is then $\d^2 L(\cdot|Y): \Theta \to \L(\Theta \times \Theta,\bbR)$.
The Fisher information $\cI(\theta)$ at $\theta\in\Theta$ is defined to be
\[
  \cI(\theta) = -\E[\d^2 L(\theta|Y)] \in \Theta \otimes \Theta.
\]
or \hltodo[Is this required?]{alternatively}
\begin{align*}
  \cI(\theta) 
  &= \E[\d L(\theta|Y) \otimes \d L(\theta|Y)] \\ %\in \Theta \otimes \Theta 
  &= \E[\ip{\nabla L(\theta|Y), \cdot}_{\Theta} \otimes \ip{\nabla L(\theta|Y), \cdot}_{\Theta}] \\
  &= \E\ip{\nabla L(\theta|Y) \otimes \nabla L(\theta|Y), \cdot}_{\Theta\otimes\Theta} \\
  &= \ip{\E[\nabla L(\theta|Y) \otimes \nabla L(\theta|Y)], \cdot}_{\Theta\otimes\Theta}.
\end{align*}

Since $\cI(\theta) \in \Theta \otimes \Theta$ we may view it also as a bilinear form. 
That is, for any $b,b'\in\Theta$, we have
\begin{align}\label{eq:fisher-linear-functional}
  \cI(\theta)(b,b') = \ip{\cI(\theta), b \otimes b'}_{\Theta\otimes\Theta}.
\end{align}
\hltodo[This part seems sketchy.]{We call this the Fisher information for $\theta$ evaluated at two points $b$ and $b'$ in $\Theta$}.
Setting $\theta_b = \ip{\theta,b}_\Theta$ for some $b\in\Theta$, we may view this also as the Fisher information between two continuous, linear functionals of $\theta$.
That is, $\cI(\theta)(x,\cdot)$ and $\cI(\theta)(\cdot,x')$ are both \hltodo[I think it is, because derivatives are continuous]{continuous}, linear functionals on $\Theta$, and thus belong to the continuous dual space $\Theta'$.
By the Riesz representation theorem, $\cI(\theta)(x,\cdot) = \ip{\cdot,b}$ for some $b\in\Theta$...

\subsection{Tensor product spaces}
\hltodo{Move this to Chapter 2}

\begin{definition}[Tensor products]
  Let $x_1\in\cH_1$ and $x_2\in\cH_2$ be two elements of two real Hilbert spaces.
  Then, the tensor product $x_1 \otimes x_2:\cH_1\times\cH_2\to\bbR$, is a bilinear form defined as
  \[
    (x_1\otimes x_2)(y_1,y_2) = \ip{x_1,y_1}_{\cH_1}\ip{x_2,y_2}_{\cH_2}
  \]
  for any $(y_1,y_2)\in\cH_1\times\cH_2$.
\end{definition}

\begin{definition}[Tensor product space]
  The tensor product space $\cH_1\otimes\cH_2$ is the completion of the space
  \[
    \cA = \left\{ \sum_{j=1}^J x_{1j} \otimes x_{2j} \,\Bigg\vert\, x_{1j}\in\cH_1, x_{2j}\in\cH_2, J \in \bbN \right\}
  \]
  with respect to the norm induced by the inner product
  \[
    \left\langle  \sum_{j=1}^J x_{1j} \otimes x_{2j},  \sum_{k=1}^K y_{1k} \otimes y_{2k} \right\rangle_\cA = \sum_{j=1}^J\sum_{k=1}^K \ip{x_{1j},y_{1k}}_{\cH_1} \ip{x_{2j},y_{2k}}_{\cH_2}.
  \]
\end{definition}

An operator interpretation of the tensor product.
For each pair of elements $(x_1,x_2) \in \cH_1\times\cH_2$, we define the operator $A:\cH_1\to\cH_2$ in the following way:
\begin{align*}
  A_{x_1,x_2}:\cH_1&\to\cH_2 \\
  y_1&\mapsto \ip{x_1,y_1}_{\cH_1}x_2
\end{align*}
For some $y_1\in\cH_1$ and $y_2\in\cH_2$, we have that
\begin{align*}
  \ip{A_{x_1,x_2}(y_1),y_2}_{\cH_2} 
  &= \big\langle \ip{x_1,y_1}_{\cH_1}x_2 , y_2 \big\rangle_{\cH_2} \\
  &= \ip{x_1,y_1}_{\cH_1}\ip{x_2,y_2}_{\cH_2} \\
  &= (x_1\otimes x_2)(y_1,y_2).
\end{align*}
It is seen that the tensor product $x_1\otimes x_2$ is is associated with the rank one operator $B:\cH_1'\to\cH_2$ defined by $z \mapsto z(x_1)x_2$ with $z = \ip{x_1,\cdot}_{\cH_1}$.
We write $B = x_1 \otimes x_2$.
\hltodo[From Wikipedia. But don't really get it, although it might explain the Fisher information between linear functionals.]{Therefore, this extends a linear identification between $\cH_1\otimes\cH_2$ and the space of finite-rank operators from $\cH_1'$ to $\cH_2$.}
We now have three distinct interpretations of the tensor product:
\begin{itemize}
  \item \textbf{Bilinear form} (as defined in Definition 3.5). 
  \begin{align*}
    x_1 \otimes x_2:\cH_1 \times \cH_2 &\to \bbR \\
    (y_1,y_2) &\mapsto \ip{x_1,y_1}_{\cH_1}\ip{x_2,y_2}_{\cH_2}
  \end{align*}
  for $x_1,y_1\in\cH_1$ and $x_2,y_2\in\cH_2$.
  \item \textbf{Operator}.
  \begin{align*}
    x_1 \otimes x_2:\cH_1 &\to \cH_2 \\
    y_1 &\mapsto \ip{x_1,y_1}_{\cH_1}x_2
  \end{align*}  
  \item \textbf{General form} (as an element in the tensor space).
  \[
    x_1 \otimes x_2 \in \cH_1 \otimes \cH_2.
  \]
\end{itemize}

\subsection{Random elements in a Hilbert space}

\hltodo{Move this to Chapter 2}

Let $\cH$ be a real Hilbert space. 
We can define a metric on $\cH$ using $D(x,x') = \norm{x-x'}_\cH$, where the norm on $\cH$ is the norm induced by the inner product.
A collection $\Sigma$ of subsets of $\cH$ is called a \emph{$\sigma$-algebra} if $\emptyset \in \Sigma$, $S \in \Sigma$ implies its complement $S^c \in \Sigma$, and $S_j\in\Sigma$, $j\geq 1$ implies $\bigcup_{j=1}^\infty S_j \in \Sigma$.
The smallest $\sigma$-algebra containing all open subsets of $\cH$ is called the \emph{Borel $\sigma$-algebra}, and its members the Borel sets.
Denote by $\cB(\cH)$ the Borel $\sigma$-algebra of $\cH$.
The metric space $(\cH,D)$ is called \emph{separable} if it has a countable dense subset, i.e., there are $x_1,x_2,\cdots$ in $\cH$ such that the closure $\overline{\{x,_1,x_2,\cdots\}} = \cH$.

A function $\nu:\Sigma\to[0,\infty]$ is called a \emph{measure} if it satisfies
\begin{itemize}
  \item \textbf{Non-negativity}. $\nu(S) \geq 0$ for all $S$ in $\Sigma$
  \item \textbf{Null empty set}. $\nu(\emptyset) = 0$
  \item \textbf{$\sigma$-additivity}. For all countable, mutually disjoint sets $\{S_i\}_{i=1}^\infty$,
  \[
    \nu\left(\bigcup_{i=1}^\infty S_i \right) = \sum_{i=1}^\infty \nu(S_i)
  \] 
\end{itemize}
A measure $\nu$ on $\big(\cH,\cB(\cH)\big)$ is called a \emph{Borel measure} on $\cH$.
We shall only concern ourselves with finite Borel measures. 
In addition, if $\nu(\cH) = 1$ then $\nu$ is a \emph{(Borel) probability measure} and the measure space $\big(\cH,\cB(\cH),\nu\big)$ is a \emph{(Borel) probability space}.

Let $(\Omega,\cE,\Prob)$ be a probability space.
We say that a mapping $X:\Omega\to\cH$ is a \emph{random element} in $\cH$ if $X^{-1}(B)\in\cE$ for every Borel set, i.e., $X$ is a function such that for every $B\in\cB(\cH)$, its preimage $X^{-1}(B) = \{\omega \in \Omega \,|\, X(\omega) \in B \}$ lies in $\Sigma$.
This is simply a generalised version of the definition of random variables in regular Euclidean space.
From this definition, we can also properly define random functions $f$ in some Hilbert space of functions $\cF$.
In any case, every random element $X$ induces a probability measure on $\cH$ defined by
\[
  \nu(B) = \Prob\big(X^{-1}(B)\big) = \Prob\big( \omega \in \Omega | X(\omega) \in B  \big) = \Prob(X \in B).
\]
The measure $\nu$ is called the \emph{distribution} of $X$.
The \emph{density} $\pi$ of $X$ is a measurable function with the property that
\[
  \Prob(X \in B) = \int_{X^{-1}(B)} \omega \dint\Prob(\omega) = \int_B \pi(x) \dint\nu(x).
\]

\begin{definition}[Mean vector]
  Let $\nu$ be a Borel probability measure on a real Hilbert space $\cH$.
  Supposing that a random element $X$ of $\cH$ is \emph{integrable}, that is to say
  \[
    \E \norm{X}_\cH = \int_\cH \norm{x}_\cH \dint\nu(x) < \infty,
  \]
  then the unique element $\mu\in\cH$ satisfying 
  \[
    \ip{\mu,x'} = \int_\cX \ip{x,x'}_\cX \dint\nu(x) = \E\ip{X,x'}_\cH
  \]
  for all $x' \in \cH$ is called the \emph{mean vector}. 
\end{definition}

\begin{definition}[Covariance operator]
  Let $\nu$ be a Borel probability measure on a real Hilbert space $\cH$.
  Suppose that a random element $X$ of $\cH$ is \emph{square integrable}, i.e., $\E \norm{X}_\cH^2 < \infty$, and let $\mu$ be the mean vector of $X$.
  Then the \emph{covariance operator} $C$ is defined by the mapping
  \begin{align*}
    C:\cH &\to \cH \\
    x &\mapsto \E\big[\ip{X - \mu, x}_{\cH}(X - \mu) \big].
  \end{align*}
  The covariance operator $C$ is also an element of $\cH\otimes\cH$ that satisfies
  \begin{align*}
    \ip{C,x\otimes x'}_{\cH\otimes\cH} 
    &= \int_\cH \ip{z-\mu, x}_\cH\ip{z-\mu,x'}_\cH \dint\nu(z) \\
    &= \E\big[\ip{X-\mu,x}_\cH \ip{X-\mu,x'}_\cH\big]
  \end{align*}
  for all $x,x'\in\cH$.
\end{definition}

From the definition of the covariance operator, we see that it induces a symmetric, bilinear form, which we shall denote by $\Cov:\cH\times\cH\to\bbR$, through
\begin{align*}
  \ip{Cx,x'}_\cH 
  &= \big\langle\E\big[\ip{X - \mu, x}_{\cH}(X - \mu) \big], x' \big\rangle_\cH \\
  &= \E\big[\ip{X-\mu,x}_\cH \ip{X-\mu,x'}_\cH\big] \\
  &=: \Cov(x,x').
\end{align*}

\begin{definition}[Gaussian vectors]
  A random element $X$ is called \emph{Gaussian} if $\ip{X,x}_\cH$ has a normal distribution for all fixed $x\in\cH$.
  A Gaussian vector $X$ is characterised by its mean element $\mu\in\cH$ and its covariance $C\in\cH\otimes\cH$.
\end{definition}


%The Fisher information $\cI(\theta) \in \mathcal \cH \otimes \mathcal \cH$ for $\theta \in \Theta$ is
\[
	\cI(\theta) = \E[\nabla L(\theta) \otimes \nabla L(\theta)],
\]
or equivalently,
\[
	\cI(\theta) = -\E[\nabla^2 L(\theta)],
\]
where again, stated for clarity, expectations are taken with respect to the random variable $Y$ under the true distribution $p(\cdot|\theta)$.
In the above definitions, $\nabla^2$ is the second-order gradient, and the operation $\otimes:\cH \times \cH \to \cH \otimes \cH$ is the tensor product, mapping elements from $\cH^2$ to the tensor product space $\cH \otimes \cH$.
It is a Hilbert space when equipped with the inner product
\[
  \ip{x\otimes y, x'\otimes y'}_{\cH\otimes\cH} = \ip{x, x'}_\cH\ip{y, y'}_\cH.
\]

Taking this concept further, we can also define the Fisher information for a linear functional of $\theta$, or between two linear functionals of $\theta$.
This is essence of the next lemma.

\begin{lemma}[Fisher information for linear functionals]\label{thm:fisherlinfunc}
	Following the above definitions, suppose that the Fisher information for $\theta \in \Theta$ is $\cI(\theta)$, with $\Theta$ a Hilbert space with inner product $\ip{\cdot,\cdot}_\Theta$.
	For some $b\in\Theta$, denote $\theta_b = \ip{\theta,b}_\Theta$.
	Then, the Fisher information for $\theta_b$ is given as
	\[
		\cI(\theta_b) = \ip{\cI(\theta), b \otimes b}_{\Theta \otimes \Theta},
	\]
	and, more generally, the Fisher information between $\theta_b$ and $\theta_{b'}$ is given as
	\[
		\cI(\theta_b,\theta_{b'}) = \ip{\cI(\theta), b \otimes b'}_{\Theta \otimes \Theta}
	\]
\end{lemma}

\begin{proof}
  Let $\cB$ be a set containing an orthonormal sequence of points in $\Theta$, i.e. $\cB$ is a Hilbert basis for the Hilbert space $\Theta$. 
  Then, by definition, every $\theta \in \Theta$ can be written as
  \[
    \theta = \sum_{\beta \in \cB} \ip{\theta,\beta}_\Theta \beta.
  \]
  Now, the score function with respect to the linear functional $\theta_b = \ip{\theta,b}_\Theta$ is
  \begin{align*}
    \frac{\partial}{\partial\theta_b} L(\theta) 
    &= \dots \\
    &= \nabla_b L(\theta) \\
    %\frac{\partial}{\partial\theta_b} L\big(\textstyle\sum_{\beta \in \cB} \theta_\beta \beta\big) \\
    &= \langle \nabla L(\theta), b \rangle_{\Theta}
  \end{align*}
  Differentiating again gives
  \begin{align*}
    \frac{\partial^2}{\partial\theta_b\partial\theta_{b'}} L(\theta) 
    &= \langle \nabla L^2(\theta), b \otimes b' \rangle_{\Theta \otimes \Theta}
  \end{align*}  
%  Provided $\E \norm{\nabla L^2(\theta)}_{\Theta \otimes \Theta} < \infty$, then it follows from Fubini's theorem that taking expectations and multiplying both sides by minus one gives the desired result.
  Note that by the bilinear property of tensor products,
    \begin{align*}
    -\frac{\partial^2}{\partial\theta_b\partial\theta_{b'}} L(\theta)
    &= (-1) \cdot \langle \nabla L^2(\theta), b \otimes b' \rangle_{\Theta \otimes \Theta} \\
    &= \langle -\nabla L^2(\theta), b \otimes b' \rangle_{\Theta \otimes \Theta}.
  \end{align*}  
  Provided $\E \norm{\nabla L^2(\theta)}_{\Theta \otimes \Theta} < \infty$, taking expectations of both sides gives the desired result, since $b\otimes b'$ is free of $Y$ and is therefore constant under the expectation. 
  \hltodo{Not really convinced of this proof.}
\end{proof}

alternative

\begin{align*}
  \ip{\nabla L(\theta) \otimes \nabla L(\theta), b \otimes b'}_{\Theta \otimes \Theta} 
  &= \ip{\nabla L(\theta), b }_\Theta  \ip{\nabla L(\theta), b' }_\Theta \\
  &= \frac{\partial}{\partial\theta_b} L(\theta)\frac{\partial}{\partial\theta_{b'}} L(\theta)
\end{align*}


\section{Fisher information for regression functions}
\input{fisher-regression}

\section{The induced Fisher information RKHS}
\hltodo[Rewrite]{Next, let us see for which linear functionals of $f$ there is Fisher information.}
Let 
\begin{align}
\cF_n = \left\{ f:\cX \to \bbR \, \bigg| \, f(x) = \sum_{i=1}^n h(x,x_i)w_i, \ w_i \in \bbR, \ i=1,\dots,n \right\}.  
\end{align}
Since $h(\cdot,x_i) \in \cF$, then any $f \in \cF_n$ is also in $\cF$ by linearity, and thus $\cF_n$ is a subset of $\cF$.
Further, $\cF_n$ is closed under addition and multiplication by a scalar, and is therefore a subspace of $\cF$.
Let $\cF_n^\bot$ be the orthogonal complement of $\cF_n$ in $\cF$.
Then, any $r \in \cF_n^\bot$ is orthogonal to each of the $h(\cdot,x_i)$, so by the reproducing property of $h$, $r(x_i) = \ip{r,h(\cdot,x_i)}_\cF = 0$.

\begin{corollary}
  With $g \in \cF$, the Fisher information for $g$ is zero if and only if $g\in\cF_n^\bot$, i.e. if and only if $g(x_1) = \cdots = g(x_n) = 0$.
\end{corollary}

Hence, $r$ cannot be estimated from the data and has to be estimated by a prior guess.

\hltodo{OLD, but some stuff relevant here.}
Note that any regression function $f\in \mathcal F$ can be decomposed into $f = f_n + r$, with $f \in \mathcal F_n$ and $r \in \mathcal R$ where $\mathcal F = \mathcal F_n + \mathcal R$ and $\mathcal F_n \perp \mathcal R$. Fisher information exists only on the $n$-dimensional subspace $\mathcal F_n$, while there is no information for $\mathcal R$. Thus, we will only ever consider the RKHS $\mathcal F_n \subset \mathcal F$ where there is Fisher information. Let $h$ be a real symmetric and positive definite function over $\mathcal X$ defined by $h(x,x') = I[f(x),f(x')]$. As we saw earlier, $h$ defines a RKHS, and it can be shown that the RKHS induced is in fact $\mathcal F_n$ spanned by the reproducing kernel on the dataset with the squared norm $||f||_{\mathcal F_n}^2 = w^\top\Psi^{-1}w$.

\begin{lemma}
  Let $\cF_n$ be equipped with the inner product
  \[
    \ip{f_w, f_{w'}}_{\cF_n} = \bw^\top \bPsi^{-1} \bw',
  \]
  where $\bw = (w_1,\dots,w_n)$ and $f_w(x)=\sum_{i=1}^n h(x,x_i)w_i$.
  Then, $h_n$ defined by
  \[
    h_n(x,x') = \sum_{i=1}^n\sum_{j=1}^n \psi_{ij}h(x,x_i)h(x',x_j)
  \]
  is the reproducing kernel of $\cF_n$.
\end{lemma}

\begin{proof}
  \hltodo{Prove $\cF_n$ is a Hilbert space?}
  \begin{align*}
    f_j = \sum h(\cdot,x_i)w_{ij}
  \end{align*}
  \begin{align*}
    \norm{f_j-f}_{\cF_n}^2 &= \ip{f_j-f,f_j-f} \\
    &\leq \ip{f_j,f_j} + \ip{f,f} \\
    &= w_j\Psi w_j + w\Psi w \\
    &= \Psi(w_jw_j^\top + ww^\top)
  \end{align*}
  Note that by defining $w_j(x) = \sum_{k=1}^n \psi_{jk} h(x,x_k)$, we see that
  \begin{align*}
    h_n(\cdot,x) 
    &= \sum_{j=1}^n\sum_{k=1}^n \psi_{jk} h(\cdot,x_j) h(x,x_k) \\
    &= \sum_{j=1}^n w_j(x)h(\cdot,x_j)
  \end{align*}
  is an element of $\cF_n$.
  Now, we just need to prove the reproducing property. 
  Denote by $\psi_{ij}^{-}$ the $(i,j)$th element of $\bPsi^{-1}$.
  \hltodo[How?]{Since $\ip{h(\cdot,x_i), h(\cdot,x_j)}_{\cF_n} = \psi_{ij}^{-}$}, we have
  \begin{align*}
    \ip{f_w, h_n(\cdot,x)}_{\cF_n}
    &= \left\langle 
    \sum_{i=1}^n h(\cdot,x_i)w_i ,
    \sum_{j=1}^n\sum_{k=1}^n \psi_{jk} h(\cdot,x_j) h(x,x_k)
    \right\rangle_{\cF_n} \\
    &= \sum_{i=1}^n w_i \sum_{j=1}^n\sum_{k=1}^n \psi_{jk}h(x,x_k) 
    \big\langle h(\cdot,x_i)w_i , h(\cdot,x_j) \big\rangle_{\cF_n} \\
    &= \sum_{i=1}^n w_i \sum_{j=1}^n\sum_{k=1}^n \psi_{jk}h(x,x_k)\psi_{ij}^{-} \\
    &= \sum_{i=1}^n w_i \sum_{k=1}^n \delta_{ik}h(x,x_k) \\
    &= \sum_{i=1}^n w_i h(x,x_i) \\
    &= f_w(x)
  \end{align*}
  Therefore, $h_n$ is a reproducing kernel for $\cF_n$.
\end{proof}

\hltodo{Is the Fisher information metric and semi-norm over $\cF$ useful?}



\section{The I-prior}
%Here we consider data dependent priors---seemingly data dependent (i.e. dependent on X) but the whole model is conditional on $X$ implicitly, so there is no issue.
If prior depended on $y$ then there is a problem, at least, violates Bayesian first principles (using the data twice such that a priori and a posteriori same amount of information).
Rather, more of a principled prior. One that is based on objectivity of maximum entropy---if one does not know anything, best to choose prior which maximises uncertainty.
We see that it coincides with the Fisher information induced RKHS.

Goal is always to estimate $f \in \cF$ based on finite amount of data points.
We know MLE is not so good, so want regularise by some prior.
Unfortunately, $\cF$ might be huge such that data don't provide enough information for $f$ to be estimated sufficiently well.
We ask: What is the smallest subset for which there is full information coming from the data? 
Intuitively, it must be of $n$-dimensions, the sample size of the data.
Rather separately, we found out what the Fisher information for $f$ looks like, and deduced that there is Fisher information only on an orthogonal projection of $\cF$ on to $\cF_n$.
There is this flavour of dimension reduction---no need to consider the entire space, because this is futile, but just consider functions in the smaller subspace, as this is the best we can do anyway.
Therefore, we just look in this subspace $\cF_n$ for an appropriate approximation to $f$. 
In particular, what prior should I use? On the basis of maximum entropy principle, I figure out that the form of our I-prior.
The connection of $\cF_n$ to Fisher information is this: $\cF_n$ is the subspace of $\cF$ for which Fisher information exists. Equipping this space with a particular inner product reveals that $\cF_n$ is a RKHS with reproducing kernel equal to the Fisher information for $f$.

The set $\cF$ is potentially ``too big'' for the purpose of estimating $f$, that is, for certain pairs of functions $\cF$, the data do not allow an assessment of whether one is closer to the truth than the other.
In particular, the data do not contain information to distinguish betwen any $f$ and $f'$ for which $f(x_i) = f'(x_i), i=1,\dots,n$.
  \hltodo[If data do not provide enough information, isn't the purpose of the prior to provide the missing information?]{A prior for $f$ therefore need not have support $\cF$, instead it is sufficient to consider priors with support $f_0 + \cF_n$}, where $f_0 \in \cF$ is fixed and chosen a priori as a ``best guess'' of $f$.
Since the Fisher information for $\ip{g,f}_\cF$ is non-zero for any non-zero $g \in \cF_n$, there is information to allow a comparison between any pair of functions in $f_0 + \cF_n$.

Key questions:
\begin{itemize}
  \item What does it mean to say that the measure space $(\cF,\nu)$ has a probability density function $\pi$? A probability density function $p$ on $(\cF, \nu)$ is a $\nu$-measurable function from $\cF$ to $[0, \infty)$ such that $p \d\nu$ is a probability measure on $\cF$.
  \item What does it mean for $f \in \cF$ to be Gaussian?
\end{itemize}

Let $(\Theta,D)$ be a metric space and let $\nu = \nu_D$ be a volume measure induced by $D$ (e.g. Hausdorff measure).
  Denote by $\pi$ a density of $\Theta$ relative to $\nu$, i.e. if $\theta$ is a random variable with density $\pi$, then for any measurable subset $A \subset \Theta$, $\Prob(\theta \in A) = \int_A \pi(t)\nu(\d t)$.


\begin{definition}[Entropy]
  The entropy of a distribution $\pi$ over $\cF$ relative to a measure $\nu$ is defined as
  \[
    \cE(\pi) = - \int_\cF \pi(f) \log \pi(f) \, \d\nu(f).
  \]
  This converges if $\pi \log\pi$ is Lebesgue integrable, i.e. $\pi\log\pi \in \text{L}^1(\cF,\nu)$.
\end{definition}

\begin{definition}[Functional derivative]
  Given a manifold $M$ representing continuous/smooth functions $\rho$ with certain boundary conditions, and a functional $F:M\to\bbR$, the functional derivative of $F[\rho]$ with respect to $\rho$, denoted $\partial F/\partial\rho$, is defined by
  \begin{align*}
    \int \frac{\partial F}{\partial\rho}(x)\phi(x)\d x
    &= \lim_{\epsilon\to 0} \frac{F[\rho + \epsilon\phi] - F[\rho]}{\epsilon} \\
    &= \left[ \frac{\d}{\d \epsilon} F[\rho + \epsilon\phi] \right]_{\epsilon=0},
  \end{align*}
  where $\phi$ is an arbitrary function.
  The function $\partial F/\partial\rho$ as the gradient of $F$ at the point $\rho$, and
  \[
    \partial F(\rho,\phi) = \int \frac{\partial F}{\partial\rho}(x)\phi(x) \d x
  \]
  as the directional derivative at point $\rho$ in the direction of $\phi$.
  Analogous to vector calculus, the inner product with the gradient gives the directional derivative.
\end{definition}

\begin{example}[Functional derivative of entropy]
  Let $X$ be a discrete random variable with probability mass function $p(x) \geq 0$, for $\forall x \in \Omega$, a finite set.
  The entropy is a functional of $p$, namely
  \[
    \cE[p] = - \sum_{x\in\Omega} p(x)\log p(x).
  \]
  Equivalently, using the counting measure $\nu$ on $\Omega$, we can write
  \[
    \cE[p] = -\int_\Omega p(x) \log p(x) \d\nu(x).
  \]
  \begin{align*}
    \int_\Omega \frac{\partial\cE}{\partial p}(x)\phi(x) 
    &= \left[ \frac{\d}{\d \epsilon} \cE[p +  \epsilon\phi] \right]_{\epsilon=0} \\
    &= \left[ -\frac{\d}{\d \epsilon} 
    \big( p(x) + \epsilon\phi(x) \big) 
    \log \big(p(x) + \epsilon\phi(x) \big) 
    \right]_{\epsilon=0} \\
    &= -\int_\Omega \left( 
    \frac{p(x)\phi(x)}{p(x)+\epsilon\phi(x)}
    + \frac{\epsilon\phi(x)}{p(x) + \epsilon\phi(x)}
    + \phi(x)\log\big( p(x) + \epsilon\phi(x) \big)
    \right) \d x \\
    &= -\int_\Omega \left( 1 + \log p(x) \right) \phi(x) \d x.
  \end{align*}
  Thus, $(\partial\cE/\partial p)(x) = -1 -\log p(x)$.
\end{example}

\begin{lemma}[Maximum entropy distribution]\label{thm:maxentr}
  Let $(\cX,d)$ be a metric space and let $\nu=\nu_d$ be a volume measure induced by $d$.
  Let $p$ be a probability density function on $(\cX,d)$.
  The entropy maximising density, which satisfies
  \[
    \argmax_{p} \cE(p) = - \int_\cX p(x) \log p(x) \, \d\nu(x),
  \]
  subject to the constraints
  \begin{align*}
    \begin{gathered}
      \E\big[d(x,x_0)^2\big] = \int_\cX d(x,x_0)^2 p(x) \d\nu(x) = \const, \hspace{1cm} 
      \int_\cX p(x) \d\nu(x) = 1, \\
      \text{and} \hspace{0.5cm} p(x) \geq 0,
    \end{gathered}
  \end{align*}
  is the density given by
  \[
    \tilde p(x) \propto \exp \left(-\half d(x,x_0)^2 \right),
  \]
  for some $x_0\in\cX$.
  If $(\cX,d)$ is a Euclidean space and $\nu$ a flat (Lebesgue) measure then $\tilde p$ represent a (multivariate) normal density.
\end{lemma}

\begin{proof}
  This follows from standard calculus of variations.
  We provide a sketch proof here.
  Set up the Langrangian
  \begin{align*}
      \cL(p,\gamma_1,\gamma_2) &= 
      - \int_\cX p(x) \log p(x) \, \d\nu(x) +
      \gamma_1 \left(\int_\cX d(x,x_0)^2 p(x) \d\nu(x) - \const \right) \\
      &\phantom{==} + \gamma_2 \left( \int_\cX p(x) \d\nu(x) - 1 \right).
  \end{align*}
  From the above lemma and example, taking derivatives with respect to $p$ yields
  \begin{align*}
    \frac{\partial}{\partial p} \cL(p,\gamma_1,\gamma_2)(x)
    = - 1 - \log p(x) + \gamma_1 d(x,x_0)^2 + \gamma_2.
  \end{align*}
  Set this to zero, and solve for $p$:
  \begin{align*}
    p(x) &= \exp \left( \gamma_1 d(x,x_0)^2 + \gamma_2 - 1 \right) \\
    &\propto \exp \left( \gamma_1 d(x,x_0)^2 \right)
  \end{align*}
  which is positive for any values of $\gamma_1$ (and $\gamma_2$).
  This density normalises to one if $\gamma_1 < 0$, so we choose $\gamma_1=-1/2$.
  If $\cX = \bbR^n$ and that $\nu$ is the Lebesgue measure then $d(x,x_0) = \norm{x-x_0}_{\bbR^n}$, so $\tilde p$ is recognised as a multivariate normal density centred at $x_0$ with identity covariance matrix.
\end{proof}

\begin{theorem}[The I-prior]
  Let $\cF$ be an RKHS with kernel $h$, and consider the finite dimensional affine subspace $\cF_n$ of $\cF$ equipped with an inner product as in Lemma 2.5.
%  Suppose $\Theta$ is a finite dimensional affine subspace of a Hilbert space with norm $\norm{\cdot}_\Theta$. 
%  We have a metric space $(\cF_n,d)$, where $d(f,f')^2 = \ip{f-f',f-f'}$.
  Let $\nu$ be a volume measure induced by the norm $\norm{\cdot}_{\cF_n} = \sqrt{\ip{\cdot,\cdot}_{\cF_n}}$.
  With $f_0 \in \cF$, let $\Pi_0$ be the class of distributions $p$ such that 
  \[
    \E[\norm{f-f_0}^2_{\cF_n}] = \int_{\cF_n} \norm{f-f_0}^2_{\cF_n} \ p(f) \d\nu(f) = \const
  \]
  Denote by $\tilde p$ the density of the entropy maximising distribution among the class of distributions within $\Pi_0$.
  Then, $\tilde p$ is Gaussian over $\cF$ with mean $f_0$ and covariance kernel equal to the reproducing kernel of $\cF_n$, i.e.
  \[
    \Cov\big(f(x),f(x')\big) = h_n(x,x').
  \]
  We call $\tilde p$ the I-prior for $f$.
\end{theorem}

\begin{proof}
  Recall the fact that any $f \in \cF$ can be decomposed into $f = f_n + r_n$, with $f_n \in \cF_n$ and $r_n \in \cR_n$, the orthogonal complement of $\cF_n$.
  Also recall that there is no Fisher information about any $r \in \cR_n$, and therefore it is not possible to estimate $r_n$ from the data.
  Therefore, $p(r_n) = 0$, and one needs only consider distributions over $\cF_n$ when building distributions over $\cF$.
  
  The norm on $\cF_n$ induces the metric $d(f,f') = \norm{f - f'}_{\cF_n}$.
  Thus, for $f \in \cF$ of the form $f = \sum_{i=1}^n h(\cdot,x_i)w_i$ (i.e., $f \in \cF_n$) and provided $f_0 \in \cF_n \subset \cF$,
  \begin{align*}
    d(f,f_0)^2 
    &= \norm{f - f_0}_{\cF_n}^2 \\
    &= \left\Vert \sum_{i=1}^n h(\cdot,x_i)w_i - \sum_{i=1}^n h(\cdot,x_i)w_{i0} \right\Vert_{\cF_n}^2 \\
    &= \left\Vert \sum_{i=1}^n h(\cdot,x_i)(w_i - w_{i0}) \right\Vert_{\cF_n}^2 \\
    &= (\bw - \bw_0)^\top\bPsi^{-1} (\bw - \bw_0) \\
%    &= \bw^\top\bPsi^{-1}\bw - 2\bw^\top\bPsi^{-1}\bw' + \bw'^\top\bPsi^{-1}\bw'j
  \end{align*}
  Thus, by Lemma \ref{thm:maxentr}, the maximum entropy distribution for $f = \sum_{i=1}^n h(\cdot,x_i)w_i$ is
  \[
    (w_1,\dots,w_n)^\top \sim \N_n(\bw_0,\bPsi).
  \]
  This implies that $f$ is Gaussian, since
  \begin{align*}
    \langle f,f' \rangle_{\cF}
    = \left\langle \sum_{i=1}^n h(\cdot,x_i)w_i, f' \right\rangle_{\cF} 
    = \sum_{i=1}^n w_i \left\langle  h(\cdot,x_i), f' \right\rangle_{\cF}  
  \end{align*}
  is a sum of normal random variables, and therefore $\langle f,f' \rangle_{\cF}$ is normally distributed for any $f' \in \cF$.
  The mean $\mu\in\cF$ of this random vector $f$ satisfies $\E\ip{f,f'}_{\cF}  = \ip{\mu,f'}_{\cF}$ for all $f'\in\cF_n$, but
  \begin{align*}
    \E\ip{f,f'}_{\cF}  
    &= \E \left\langle \sum_{i=1}^n h(\cdot,x_i)w_i, f' \right\rangle_{\cF} \\
    &= \E \left[ \sum_{i=1}^n w_i \left\langle  h(\cdot,x_i), f' \right\rangle_{\cF} \right] \\
    &= \sum_{i=1}^n w_{i0} \left\langle  h(\cdot,x_i), f' \right\rangle_{\cF} \\
    &= \left\langle \sum_{i=1}^n h(\cdot,x_i)w_{i0}, f' \right\rangle_{\cF} \\
    &= \langle f_0,f' \rangle_{\cF},
  \end{align*}
  so $\mu \equiv f_0 = \sum_{i=1}^n h(\cdot,x_i)w_{i0}$. 
  The covariance kernel $\Sigma$ is the bilinear form satisfying
  \begin{align*}
    \Cov\big(f(x),f(x')\big) 
    &= \Cov\big(\ip{f,h(\cdot,x)}_{\cF}, \ip{f,h(\cdot,x')}_{\cF} \big) \\
    &= \left\langle \Sigma, h(\cdot,x) \otimes h(\cdot,x') \right\rangle_{\cF \otimes \cF}.
  \end{align*}
  Write $h_{x} := \langle h(\cdot,x),f \rangle_{\cF}$. 
  Then, by the usual definition of covariances, we have that 
  \begin{align*}
    \Cov(h_x,h_{x'}) = \E[h_xh_{x'}] - \E[h_x]\E[h_{x'}],
  \end{align*}
  where, making use of the reproducing property, the first term on the left hand side is
  \begin{align*}
    \E[h_xh_{x'}] 
    &= \E \left[ 
    \left\langle h(\cdot,x), \sum_{i=1}^n h(\cdot,x_i)w_i \right\rangle_{\cF} 
    \left\langle h(\cdot,x'), \sum_{j=1}^n h(\cdot,x_j)w_j \right\rangle_{\cF} 
    \right] \\
    &= \E \left[ 
    \sum_{i=1}^n\sum_{j=1}^n w_iw_j \left\langle  h(\cdot,x), h(\cdot,x_i) \right\rangle_{\cF} 
     \left\langle h(\cdot,x'), h(\cdot,x_j)\right\rangle_{\cF} 
    \right] \\
    &= \sum_{i=1}^n\sum_{j=1}^n (\psi_{ij} + w_{i0}w_{j0}) h(x,x_i) h(x',x_j),
  \end{align*}
  while the second term on the left hand side is
  \begin{align*}
    \E[h_x]\E[h_{x'}]
    &= \left( \sum_{i=1}^n w_{i0} \left\langle  h(\cdot,x), h(\cdot,x_i)  \right\rangle_{\cF} \right)
    \left( \sum_{j=1}^n w_{j0} \left\langle  h(\cdot,x'), h(\cdot,x_j)  \right\rangle_{\cF} \right) \\
    &= \sum_{i=1}^n \sum_{j=1}^n w_{i0}w_{j0} h(x,x_i)h(x',x_j).
  \end{align*}  
  Thus,
  \[
    \Cov\big( f(x),f(x') \big) = \sum_{i=1}^n\sum_{j=1}^n \psi_{ij} h(x,x_i) h(x',x_j),
  \]
  the reproducing kernel for $\cF_n$.
\end{proof}




\section{Rate of convergence}

\section{Conclusion}
We used the true Fisher information. \citet{efron1978assessing} say favour the observed information instead. Does this change if we use MLE $\hat f$ instead? Probably not... we don't use MLE anyway!

https://stats.stackexchange.com/questions/179130/gaussian-process-proofs-and-results

https://stats.stackexchange.com/questions/268429/do-gaussian-process-regression-have-the-universal-approximation-property

\hClosingStuffStandalone
\end{document}