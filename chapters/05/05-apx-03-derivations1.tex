\section{Derivation of the CAVI algorithm}

Let $\cZ = \{\by^*,\bw,\balpha,\eta,\bPsi \}$.
Approximate the posterior for $\cZ$ by a mean-field variational distribution
\begin{align*}
  p(\by^*, \bw, \alpha, \eta, \bPsi | \by) 
  &\approx q(\by^*)q(\bw)q(\balpha)q(\eta)q(\bPsi) \\
  &= \prod_{i=1}^n q(\by_{i}^*)q(\bw)q(\balpha)q(\eta)q(\bPsi).
\end{align*}
The first line is by assumption, while the second line follows from an induced factorisation on the latent propensities, as we will see later. 
If needed, we also assume that $q(\eta)$ factorises into its constituents components.
Recall that, for each $\xi\in\cZ$, the optimal mean-field variational density $\tilde q$ for $\xi$ satisfies
\[
  \log \tilde q(\xi) = \E_{-\xi} [\log p(\by, \cZ)] + \const \tag{\ref{eq:qtilde}}
\]
Write $\bff = \bH_\eta \bw \in  \bbR^{n\times m}$.
The joint likelihood $p(\by, \cZ)$ is given by
\begin{align*}
  p(\by, \cZ) 
  &= p(\by|\cZ)p(\cZ) \\
  &= p(\by|\by^*) p(\by^* | \balpha,\bw,\eta,\bPsi) p(\bw|\bPsi) p(\eta) p(\bPsi)p(\balpha).
\end{align*}
For reference, the relevant distributions are listed below.

\begin{itemize}
  \item {\boldmath$p(\by|\by^*)$}. For each observation $i\in\{1,\dots,n\}$, given the corresponding latent propensities $\by^*_i = (y_{i1}^*,\dots,y_{im}^*)$, the distribution for $y_i$ is a degenerate distribution which depends on the $j$'th component of $\by^*_i$ being largest, where the value observed for $y_i$ was $j$. Since each of the $y_i$'s are independent, everything is multiplicative.
  \begin{align*}
    p(\by|\by^*) 
    &= \prod_{i=1}^n \prod_{j=1}^m p_{ij}^{[y_i = j]} 
    = \prod_{i=1}^n \prod_{j=1}^m \ind[y_{ij}^* 
    = \max_k y_{ik}^*]^{\ind[y_i = j]}.
  \end{align*}
  
  \item {\boldmath$p(\by^*|\balpha,\bw,\eta,\bPsi)$}. Given values for the parameters and I-prior random effects, the distribution of the latent propensities is matrix normal
  \[
    \by^*|\balpha,\bw,\eta,\bPsi \sim \MN_{n,m}(\bone_n\balpha^\top + \bH_\eta\bw, \bI_n, \bPsi^{-1}).
  \]
%  Equivalently, 
%  \[
%    \vecc \by^* | \balpha,\bw,\eta,\bPsi \sim \N_{nm}\big(\vecc(\bone_n\balpha^\top + \bH_\eta\bw), \bPsi^{-1} \otimes \bI_n \big).
%  \]
%  The $n$ rows of $\by^* = (\by_1^{*\top},\dots,\by_n^{*\top})^\top$ are independent of each other, with each row following a $m$-variate normal distribution $\by_i^* \sim \N_{m}(\balpha + \bff(x_i), \bPsi^{-1})$.
  Write $\bmu = \bone_n\balpha^\top + \bH_\eta\bw$.
  Its pdf is
  \begin{align*}
    p(\by^*|\balpha,\bw,\eta,\bPsi)
%    &= \prod_{i=1}^n \phi(\by_i^*|\balpha + \bff(x_i), \bPsi^{-1}) \\
    &= \exp \left[-\half[nm]\log 2\pi + \half[n]\log\abs{\bPsi} - \half\tr \big((\by^* - \bmu) \bPsi (\by^* - \bmu)^\top  \big)  \right] \\
    &= \exp \left[-\half[nm]\log 2\pi + \half[n]\log\abs{\bPsi} - \half\sum_{i=1}^n (\by^*_{i \bigcdot} - \bmu_{i \bigcdot})^\top \bPsi (\by^*_{i \bigcdot} - \bmu_{i \bigcdot})   \right],
  \end{align*}
  where $\by^*_i \in\bbR^m$ and $\bmu_i \in\bbR^m$ are the rows of $\by^*$ and $\bmu$ respectively.
  The second line follows directly from the definition of the trace, but  also emanates from the fact that $\by_i^*$ are independent multivariate normal with mean $\bmu_i$ and variance $\bPsi^{-1}$.
  
  \item {\boldmath$p(\bw|\bPsi)$}. The $\bw$'s are normal random matrices $\bw \sim \MN_{n,m}(\bzero, \bI_n,\bPsi)$ with pdf
  \begin{align*}
    p(\bw|\bPsi) 
    &= \exp \left[-\half[nm]\log 2\pi - \half[n]\log\abs{\bPsi} - \half\tr \big( \bw \bPsi^{-1} \bw^\top \big)  \right] \\
    &= \exp \left[-\half[nm]\log 2\pi - \half[n]\log\abs{\bPsi} - \half\sum_{i=1}^n  \bw_{i \bigcdot}^\top \bPsi^{-1} \bw_{i \bigcdot}   \right].
  \end{align*}
  
  \item {\boldmath$p(\eta)$}. The most common scenario would be $\eta = \{\lambda_1,\dots,\lambda_p\}$ only. In this case, choose independent normal priors for each $\lambda_k \sim \N(m_k,v_k)$, $k=1,\dots,p$, whose pdf is
  \begin{align*}
    p(\eta) = \prod_{k=1}^p \exp\left[-\half\log 2\pi - \half\log v_k - \frac{1}{2v_k} (\lambda_k - m_k)^2 \right].
  \end{align*} 
  An improper prior $p(\eta) \propto \const$ can be used as well, and this is the same as letting $m_k \to 0$ and $v_k\to 0$.
  The resulting posterior will be proper.
  If $\eta$ contains other parameters as well, such as the Hurst coefficient $\gamma \in (0,1)$, SE lengthscale $l >0$ or polynomial offset $c>0$, then appropriate priors should be used to match the support of the parameter.
  Choices include $p(\gamma) = \ind\big(\gamma \in (0,1)\big)$ and $l,c \sim \Gamma(a,b)$.
  
  \item {\boldmath$p(\bPsi)$}. Our analysis shows that regardless of prior choice of $\bPsi$, be it in the full or independent I-probit model, the posterior for $\bPsi$ will not be of a recognisable form. Without giving too much thought, assume an improper prior on $\bPsi$, i.e. $p(\bPsi) \propto \const$
%  For the precision matrix, a Wishart prior with scale matrix $\bG^{-1}$ and $g$ degrees of freedom, denoted $\bPsi \sim \Wis_m(\bG^{-1},g)$, is convenient. It has pdf
%  \[
%    p(\bPsi) = \exp\left[\const + \half[g-m-1]\log \abs{\bPsi} - \half\tr(\bG\bPsi)  \right].
%  \]
%  For the independent I-probit model, $\bPsi = \diag(\psi_1,\dots,\psi_m)$, and we choose independent Gamma distributions for each precision $\sigma_j^{-2} \sim \Gamma(s_j,r_j)$, where $s_j$ and $r_j$ are the shape and rate parameters.
%  Then,
%  \[
%    p(\bPsi) = \prod_{j=1}^m \exp\left[\const + (s_j - 1)\log \psi_j - r_j\psi_j \right].
%  \] 
  
  \item {\boldmath$p(\balpha)$}. Choose independent normal priors for the intercept, $\alpha_j \sim \N(a_j,A_j)$ for $j=1,\dots,m$. The pdf is
  \[
    p(\balpha) = \prod_{j=1}^m \exp\left[
    -\half\log 2\pi - \half\log A_j - \frac{1}{2A_j} (\alpha_j - a_j)^2  
    \right].
  \]
\end{itemize}

\begin{remark}
  The priors on the parameters $\{ \balpha,\eta \}$ can be set to very vague or even improper priors, and the resulting posterior will still yield a proper distribution.
  Using improper priors eases the algebra slightly.
  For the precision matrix $\bPsi$, it is best to stick with the Wishart prior to avoid positive-definite issues, unless the independent I-probit model is used, in which case Jeffreys' prior for the precisions $p(\sigma_j^{-2})\propto \sigma_j^2$ is a convenient choice.
\end{remark}

\subsection{Derivation of \texorpdfstring{$\tilde q(\by^*)$}{$\tilde q(y^*)$}}
% [Derivation of q ystar]

The rows of $\by^*$ are independent, and thus we can consider the variational density for each $\by_i^*$ separately.
Consider the case where $y_i$ takes one particular value $j \in \{1,\dots,m\}$. The mean-field density $q(\by_{i}^*)$ for each $i=1,\dots,n$ is found to be
\begin{align*}
  \log \tilde q(\by_{i}^*) 
  &=  \ind[y_{ij}^* = \max_k y_{ik}^*] \, \E_{\cZ\backslash\{\by^*\}\sim q} \left[ - \half (\by^*_i - \bmu_i)^\top \bPsi (\by^*_i - \bmu_i)  \right] + \const \\
  &= \ind[y_{ij}^* = \max_k y_{ik}^*] \, \left[ - \half (\by^*_i - \tilde\bmu_i)^\top \tilde\bPsi (\by^*_i - \tilde\bmu_i)  \right] + \const \tag{$\star$} \\
%  &\equiv
%  \begin{cases}
%    \prod_{k=1}^m \N(\tilde f_{ik}, 1) & \text{ if } y_{ij}^* > y_{ik}^*, \forall k \neq j \\
%    0 & \text{ otherwise} \\
%  \end{cases}
  &\equiv
  \begin{cases}
    \phi(\by_i^*|\tilde\bmu_i,\tilde\bPsi) & \text{ if } y_{ij}^* > y_{ik}^*, \forall k \neq j \\
    0 & \text{ otherwise} \\
  \end{cases}
\end{align*}
where $\tilde\bmu_i = \E\balpha + (\E\bH_\eta \E\bw)_i$, and expectations are taken under the optimal mean-field distribution $\tilde q$. 
The distribution $q(\by_i^*)$ is a truncated $m$-variate normal distribution such that the $j$'th component is always largest. 
Unfortunately, the expectation of this distribution cannot be found in closed-form, and must be approximated by techniques such as Monte Carlo integration.
If, however, the independent I-probit model is used and $\tilde\bPsi$ is diagonal, then \cref{thm:contruncn} provides a simplification.

\begin{remark}
  In ($\star$)  above,  we needn't consider the second order terms in the expectations because they do not involve $\by^*$ and can be absorbed into the constant.
  To see this,
  \begin{align*}
    \E[(\by^*_i - \bmu_i)^\top\bPsi(\by^*_i - \bmu_i)]
    &= \E[\by^{*\top}_i\bPsi \by^*_i + \bmu_i^\top\bPsi\bmu_i - 2\bmu_i^\top\bPsi\by^*_i] \\
    &= \by^{*\top}_i\bPsi \by^*_i - 2 \E[\bmu_i^\top] \E[\bPsi]\by^*_i + \const \\
    &= \by^{*\top}_i\bPsi \by^*_i - 2 \tilde\bmu_i^\top\tilde\bPsi\by^*_i + \const \\
    &= (\by^*_i - \tilde\bmu_i)^\top \tilde\bPsi (\by^*_i - \tilde\bmu_i) + \const
  \end{align*}
  We will see this occurring a lot later on and we shall take note of this fact.
\end{remark}


\subsection{Derivation of \texorpdfstring{$\tilde q(\bw)$}{$\tilde q(w)$}}

The terms involving $\bw$ in \cref{eq:qtilde} are the $p(\by^*|\balpha,\bw,\eta,\bPsi)$ and $p(\bw|\bPsi)$ terms, and the rest are absorbed into the constant.
The easiest way to derive $\tilde q(\bw)$ is to vectorise $\by^*$ and $\bw$.
We know that
\begin{gather*}
  \vecc \by^* |  \balpha,\bw,\eta,\bPsi \sim \N_{nm}\big(\vecc (\bone_n\balpha^\top + \bH_\eta\bw), \bPsi^{-1}\otimes\bI_n\big) \\
  \text{and}\\
  \vecc \bw | \bPsi \sim \N_{nm} (\bzero, \bPsi\otimes\bI_n)
\end{gather*}
using properties of matrix normal distributions.
We also use the fact that $\vecc (\bH_\eta\bw) = (\bI_m \otimes \bH_\eta)\vecc\bw$.  %\diag(\bH_\eta,\dots,\bH_\eta)\vecc\bw = 
For simplicity, write $\bar\by^* = \vecc(\by^* - \bone_n\balpha^\top)$, and $\bM = (\bI_m \otimes \bH_\eta)$.
Thus,
\begin{align*}
  \log \tilde q(\bw) 
  &= \E_{\cZ\backslash\{\bw \}\sim q} \left[ 
  -\half (\bar\by^* - \bM\vecc\bw )^\top(\bPsi^{-1} \otimes \bI_n)^{-1} (\bar\by^* - \bM\vecc\bw )
  \right] \\
  &\phantom{==} + \E_{\cZ\backslash\{\bw \}\sim q} \left[ 
  -\half (\vecc \bw )^\top(\bPsi \otimes \bI_n)^{-1} \vecc (\bw ) \right] + \const \\
  &= -\half\E_{\cZ\backslash\{\bw \}\sim q} \left[ 
  (\vecc\bw)^\top \Big( \,
  \greyoverbrace{\bM^\top(\bPsi \otimes \bI_n)\bM + (\bPsi^{-1} \otimes \bI_n)}{\bA} 
  \, \Big) \vecc (\bw )
  \right] \\
  &\phantom{==} + \E_{\cZ\backslash\{\bw \}\sim q} \Big[ 
  \greyoverbrace{\bar\by^{*\top} (\bPsi \otimes \bI_n) \bM}{\ba^\top} \vecc (\bw )
  \Big] + \const \\
  &= -\half\E_{\cZ\backslash\{\bw \}\sim q} \left[
  (\vecc\bw - \bA^{-1}\ba)^\top \bA (\vecc\bw - \bA^{-1}\ba)
  \right] + \const
\end{align*}
This is recognised as a multivariate normal of dimension $nm$ with mean and precision given by $\vecc \tilde\bw = \E[\bA^{-1}\ba]$ and $\tilde\bV_w ^{-1}= \E[\bA]$ respectively.
With a little algebra, we find that
\begin{align*}
  \bV_w^{-1} 
  &= \E_{\cZ\backslash\{\bw \}\sim q} [\bA] \\
  &= \E_{\cZ\backslash\{\bw \}\sim q} \left[(\bI_m \otimes \bH_\eta)^\top(\bPsi \otimes \bI_n)(\bI_m \otimes \bH_\eta) + (\bPsi^{-1} \otimes \bI_n) \right] \\
  &= \E_{\cZ\backslash\{\bw \}\sim q} \left[(\bPsi \otimes \bH_\eta^2) + (\bPsi^{-1} \otimes \bI_n) \right] \\
  &= (\tilde\bPsi \otimes \tilde\bH_\eta^2) + (\tilde\bPsi^{-1} \otimes \bI_n) 
\end{align*}
and making a first-order approximation $(\E \bA )^{-1} \approx \E [\bA^{-1}]$\footnotemark,
\begin{align*}
  \vecc \wtilde 
  &= \E_{\cZ\backslash\{\bw \}\sim q} [\bA^{-1}\ba] \\
  &= \tilde\bV_w \E_{\cZ\backslash\{\bw \}\sim q} \big[(\bI_m \otimes \bH_\eta) (\bPsi \otimes \bI_n) \vecc(\by^* - \bone_n\balpha^\top)  \big] \\
  &= \tilde\bV_w \E_{\cZ\backslash\{\bw \}\sim q} \big[(\bPsi \otimes \bH_\eta) \vecc(\by^* - \bone_n\balpha^\top)  \big] \\
  &= \tilde\bV_w (\tilde\bPsi \otimes \tilde\bH_\eta) \vecc(\tilde\by^* - \bone_n\tilde\balpha^\top).
\end{align*}
Ideally, we do not want to work with the $nm \times nm$ matrix $\bV_w$, since its inverse is expensive to compute.
Refer to \cref{sec:complxiprobit} for details.

\footnotetext{
  \citet{groves1969note} show that $\E [\bA^{-1}] = (\E \bA )^{-1} + \bB$, where $\bB$ is a positive-definite matrix. This approximation has been used also by \citet{girolami2006variational} in their work.
}

In the case of the I-probit model, where $\bPsi = \diag(\psi_1,\dots,\psi_m)$, then the covariance matrix takes a simpler form.
Specifically, it has the block diagonal structure:
\begin{align*}
  \tilde\bV_w
  &= \E \big[\diag(\psi_1,\dots,\psi_m) \otimes \bH_\eta^2 + \diag(\psi_1,\dots,\psi_m) \otimes \bI_n \big]^{-1} \\
  &= \diag\Big(
  \E\big( \psi_1\bH_\eta^2 + \psi_1^{-1}\bI_n\big)^{-1},
  \cdots,
  \E\big(\psi_m\bH_\eta^2 + \psi_m^{-1}\bI_n\big)^{-1}
  \Big) \\
  &\approx \diag\Big(
  \big( \tilde\psi_1\tilde\bH_\eta^2 + \tilde\psi_1^{-1}\bI_n\big)^{-1},
  \cdots,
  \big(\tilde\psi_m\tilde\bH_\eta^2 + \tilde\psi_m^{-1}\bI_n\big)^{-1}
  \Big) \\
  &=: \diag(\tilde\bV_{w_1},\dots,\tilde\bV_{w_m}).
  \end{align*}
The mean $\vecc \tilde\bw$ is
\begin{align*}
  \vecc \tilde\bw 
  &= \tilde\bV_w (\diag(\tilde\psi_1,\dots,\tilde\psi_m) \otimes \tilde\bH_\eta) \vecc(\tilde\by^* - \bone_n\tilde\balpha^\top) \\
  &= \diag(\tilde\bV_{w_1},\dots,\tilde\bV_{w_m})
  \diag(\tilde\psi_1\tilde\bH_\eta,\dots,\tilde\psi_m\tilde\bH_\eta)  
  \vecc(\tilde\by^* - \bone_n\tilde\balpha^\top) \\
  &= \diag(\tilde\psi_1\tilde\bV_{w_1}\tilde\bH_\eta,\dots,\tilde\psi_m\tilde\bV_{w_m}\tilde\bH_\eta)  
  (\tilde\by^* - \bone_n\tilde\balpha^\top) \\
  &= 
  \bordermatrix{
  &\color{gray}\tilde\bw_{\bigcdot 1} 
  &\color{gray}\cdots 
  &\color{gray}\tilde\bw_{\bigcdot m} \cr
  &\tilde\psi_1\tilde\bV_{w_{1}}\tilde\bH_\eta(\tilde\by^*_{\bigcdot 1} - \tilde\alpha_1\bone_n)      
  &\cdots 
  &\tilde\psi_m\tilde\bV_{w_{m}}\tilde\bH_\eta(\tilde\by^*_{\bigcdot m} - \tilde\alpha_m\bone_n) 
  }{}^\top.
\end{align*}
Therefore, we can consider the distribution of $\bw = (\bw_{\bigcdot 1},\dots,\bw_{\bigcdot m})$ columnwise, and each are normally distributed with mean and variance
\[
  \tilde\bw_{\bigcdot j} = \tilde\sigma_j^{-2}\tilde\bV_{w_{j}}\tilde\bH_\eta(\tilde\by^*_{\bigcdot j} - \tilde\alpha_j\bone_n) 
  \hspace{0.5cm}\text{and}\hspace{0.5cm}
  \tilde\bV_{w_{j}} = \big(\tilde\sigma_j^{-2}\tilde\bH_\eta^2 + \tilde\sigma_j^2\bI_n\big)^{-1}.
\]

A quantity that we will be requiring time and again will be $\tr(\bC\E[\bw^\top\bD\bw ])$, where $\bC \in \bbR^{m \times m}$ and $\bD \in \bbR^{n \times n}$ are both square and symmetric matrices.
Using the definition of the trace directly, we get
\begin{align}
  \tr(\bC\E[\bw^\top\bD\bw ])
  &= \sum_{i,j=1}^m \bC_{ij} \E[\bw^\top\bD\bw ]_{ij} \nonumber \\
  &= \sum_{i,j=1}^m \bC_{ij} \E[\bw_{\bigcdot i}^\top\bD\bw_{\bigcdot j} ]. \label{eq:trCEwDw}
\end{align}
The expectation of the univariate quantity $\bw_{\bigcdot i}^\top\bD\bw_{\bigcdot j}$ is inspected below:
\begin{align*}
  \E[\bw_{\bigcdot i}^\top\bD\bw_{\bigcdot j}]
  &= \tr(\bD \E[\bw_{\bigcdot j}\bw_{\bigcdot i}^\top]) \\
  &= \tr\big(\bD (\Cov(\bw_{\bigcdot j},\bw_{\bigcdot i}) + \E[\bw_{\bigcdot j}]\E[\bw_{\bigcdot i}]^\top) \big) \\
  &= \tr\big(\bD (\bV_w[i,j]  + \tilde\bw_{\bigcdot j}\tilde\bw_{\bigcdot i}^\top)\big).
\end{align*}
where $\bV_w[i,j] \in \bbR^{n\times n}$ refers to the $(i,j)$'th submatrix block of $\bV_w$.
Of course, in the independent the I-probit model, this is equal to 
\[
  \bV_w[i,j] = \delta_{ij}(\psi_j\bH_\eta^2 + \psi_j^{-1}\bI_n)^{-1}
\]
where $\delta$ is the Kronecker delta. 
Continuing on \cref{eq:trCEwDw} leads us to
\begin{align*}
  \tr(\bC\E[\bw^\top\bD\bw ])
  &= \sum_{i,j=1}^m \bC_{ij} \left( 
  \tr\big(\bD (\delta_{ij}\bV_{w_j}  + \tilde\bw_{\bigcdot j}\tilde\bw_{\bigcdot i}^\top)\big).
  \right).
\end{align*}
If $\bC = \diag(c_1,\dots,c_m)$, then
\begin{align*}
  \tr(\bC\E[\bw^\top\bD\bw ]) 
  &= \sum_{j=1}^m c_j\left( 
  \tr \big( \bD \tilde\bV_{w_j} \big)  +
  \tilde\bw_{\bigcdot j}^\top \bD \tilde\bw_{\bigcdot j}
  \right) \\
  &= \sum_{j=1}^m c_j
  \tr \big( \bD (\tilde\bV_{w_j} + \tilde\bw_{\bigcdot j} \tilde\bw_{\bigcdot j}^\top) \big)
\end{align*}

\subsection{Derivation of $\tilde q(\eta)$}


By looking at only the terms involving $\eta$ in \cref{eq:qtilde}, we deduce that $\tilde q$ for $\eta$ satisfies
\begin{align*}
  \log\tilde q(\eta) 
  &=  -\half\tr\E_{\cZ\backslash\{\eta\}\sim q} \Big[ 
  (\by^* - \bone_n\balpha^\top - \bH_\eta\bw) \bPsi (\by^* - \bone_n\balpha^\top - \bH_\eta\bw)^\top \Big] + \log p(\eta) \\
  &\phantom{==} + \const \\
  &=  -\half\tr \E_{\cZ\backslash\{\eta\}\sim q} \Big(
  \bPsi\bw^\top\bH_\eta^2\bw - 2\bPsi\bw^\top\bH_\eta(\by^*-\balpha)
  \Big) + \log p(\eta)  + \const \\
  &= -\half \tr \Big( 
   \tilde\bPsi \E[\bw^\top\bH_\eta^2\bw] - 2\tilde\bPsi\tilde\bw^\top\bH_\eta(\tilde\by^* - \tilde\balpha)
   \Big) + \log p(\eta) + \const
\end{align*}
with some appropriate prior $p(\eta)$.
In general, this does not have a recognisable form in $\eta$, especially when it is not linearly dependent on the kernel matrix.
This happens when considering parameters other than the scales of the RKHSs.
Our interest would be to obtain $\tilde\bH_\eta := \E_{\eta\sim q}\bH_\eta$ and $\tilde\bH_\eta^2 := \E_{\eta\sim q}\bH_\eta^2$.
We use a Metropolis random-walk algorithm to obtain these quantities, as detailed in the algorithm below.

\begin{algorithm}[hbt]
\caption{Metropolis random-walk to sample $\eta$}
\begin{algorithmic}[1]
  \State \textbf{inputs} $\tilde\balpha$, $\tilde\bw$, $\tilde\bPsi$, and  $s$ Metropolis sampling s.d.
  \State \textbf{initialise} $\eta^{(0)} \in \bbR^q$ and $t \gets 0$
  \For{$t=1,\dots,T$}
    \State Draw $\eta^* \sim \N_q(\eta^{(t)}, s^2)$
    \State Accept/reject proposal state, i.e.
    \[
      \eta^{(t+1)} \gets 
      \begin{cases}
        \eta^* &\text{if } u\sim\Unif(0,1) < \pi_{\text{acc}} \\
        \eta^{(t)} &\text{otherwise}       
      \end{cases}
    \]
    \hspace{1.8em}where
    \[
      \pi_{\text{acc}} = \min\left(1, 
%      \frac{\tilde q(\eta^*)}{\tilde q(\eta^{(t)})}
      \exp\big(\log \tilde q(\eta^*) - \log \tilde q(\eta^{(t)})\big)
      \right).
    \]
  \EndFor
  \State $\tilde\bH_\eta \gets \frac{1}{T}\sum_{i=1}^T \bH_{\eta^{(t)}}$ and $\tilde\bH_\eta^2 \gets \frac{1}{T}\sum_{i=1}^T \bH_{\eta^{(t)}}^2$
\end{algorithmic}
\end{algorithm}

%In calculating the acceptance probabilities, we need the matrix $\E[\bw^\top\bH_\eta^2\bw]$.
%This is a bit awkward to compute---it requires sampling $\vecc \bw^{(t)}$ from the distribution $\tilde q(\vecc \bw)$, an $nm$-variate normal distribution, and the sample mean of $\bw^{(t)\top}\bH_\eta^2\bw^{(t)}$ calculated.
%If the independent I-probit model is considered, then 
%\[
%  \tr(\tilde\bPsi \E[\bw^\top\bH_\eta^2\bw])
%  = \sum_{i,j=1}^m \tilde\bPsi_{ij}\E[\bw_{\bigcdot i}^\top\bH_\eta^2\bw_{\bigcdot j}]
%\]
%where each $\bw_{\bigcdot j} \sim \N_n(\tilde\bw_{\bigcdot j},\bV_{w_j})$ and are independent of each other, so
%$\E[\bw_{\bigcdot i}^\top\bH_\eta^2\bw_{\bigcdot j}] = \tilde\bw_{\bigcdot i} \bH_\eta^2 \tilde\bw_{\bigcdot j}$.


%
%\begin{align*}
%  \vecc \bH_\eta\bw \sim \N_{nm}(\vecc \bH_\eta\wtilde, (\bI_m\otimes\bH_\eta)\bV_w(\bI_m\otimes\bH_\eta))
%\end{align*}
%
%\begin{align*}
%  \E[(\vecc \bH_\eta\bw)(\vecc \bH_\eta\bw)^\top] &=
%  (\bI_m\otimes\bH_\eta)\bV_w(\bI_m\otimes\bH_\eta) + \vecc \bH_\eta\wtilde (\vecc \bH_\eta\wtilde)^\top \\
%  \E[\bw^\top\bH_\eta^2\bw] &=
%  (\bI_m\otimes\bH_\eta)\bV_w(\bI_m\otimes\bH_\eta) + \vecc \bH_\eta\wtilde (\vecc \bH_\eta\wtilde)^\top  
%\end{align*}

Now consider the case where $\eta = \{\lambda_1,\dots,\lambda_p \}$ (RKHS scale parameters only), and the scenario described in the exponential family EM algorithm of \cref{sec:expfamEM} applies.
In particular, for $k=1,\dots,p$, we can decompose the kernel matrix as $\bH_\eta = \lambda_k \bR_k + \bS_k$ and its square as $\bH_\eta^2 = \lambda_k^2 \bR_k^2 + \lambda_k \bU_k + \bS_k^2$.
Then, for $j = 1,\dots,m$, assuming each of the $q(\lambda_k)$ densities are independent of each other, we find that
\begin{align*}
  \log \tilde q(\lambda_k) 
  &= \E_{\cZ\backslash\{\eta\}\sim q} \left[ 
  - \half\tr \big((\by^* - \bmu) \bPsi (\by^* - \bmu)^\top  \big)
  \right] 
  - \frac{1}{2v_k^2} (\lambda_k - m_k)^2 + \const \\
  &= - \half\tr \E_{\cZ\backslash\{\eta\}\sim q} \left[ \bPsi\bw^\top \bH_\eta^2 \bw - 2\bPsi(\by^* - \bone\balpha^\top)^\top\bH_\eta \bw \right] \\
  &\phantom{==} - \frac{1}{2v_k^2} (\lambda_k - m_k)^2 + \const \\
  &= - \half\tr \E_{\cZ\backslash\{\eta\}\sim q} \left[ \bPsi\bw^\top (\lambda_k^2\bR_k^2 + \lambda_k\bU_k) \bw - 2\bPsi(\by^* - \bone\balpha^\top)^\top(\lambda_k\bR_k) \bw \right] \\
  &\phantom{==} - \frac{1}{2v_k^2} (\lambda_k^2 - 2m_k\lambda_k) + \const \\
  &= - \half\tr \E_{\cZ\backslash\{\eta\}\sim q} \left[ 
  \lambda_k^2\bPsi\bw^\top \bR_k^2 \bw 
  -2 \lambda_k \Big(
  \bPsi(\by^* - \bone\balpha^\top)^\top\bR_k \bw 
  -\half\bPsi\bw^\top \bU_k \bw
  \Big) \right] \\
  &\phantom{==} - \half\left(\frac{1}{v_k^2}\lambda_k^2  - 2 \frac{m_k}{v_k^2} \lambda_k \right) + \const \\
  &= -\half \Big[ 
 \lambda_k^2 \big( 
 \greyoverbrace{\tr(\tilde\bPsi\E[\bw^\top\bR_k^2\bw]) + v_k^{-2}}{c_k}
 \big) \\ 
  &\hspace{2cm} - 2\lambda_k \Big( 
  \greyoverbrace{
  \tr\Big( \tilde\bPsi(\tilde\by^* - \bone_n\tilde\balpha^\top)^\top\bR_k \tilde\bw 
  - \half\tilde\bPsi\E[\bw^\top \bU_k \bw] \Big) 
  + m_k v_k^{-2}}{d_k}
  \Big) \Big]
\end{align*}
By completing the squares, we recognise this is as the kernel of a univariate normal density. 
Specifically, $\lambda_k \sim \N(d_k/c_k,1/c_k)$.
The quantity $\tilde\bH_\eta$ can be obtained by substituting $\lambda_k \mapsto \E_{\lambda_k\sim q}[\lambda_k]$ in the expression $\bH_\eta = \lambda_k \bR_k + \bS_k$.
However, in the calculation of $\tilde\bH_\eta^2$ using $\lambda_k^2 \bR_k^2 + \lambda_k \bU_k + \bS_k^2$, we must replace occurrences of  $\lambda_k^2$ with $\E_{\lambda_k\sim q}[\lambda_k]^2 +  \Var_{\lambda_k\sim q}[\lambda_k]$.
This can be cumbersome, so if felt necessary, use the approximation $\lambda_k^2 \mapsto \E_{\lambda_k\sim q}[\lambda_k]^2$ instead.

\begin{example}
  Suppose $k=1$, and we only have $\lambda$ to estimate.
  Then, $\bH_\eta = \lambda \bH$, $\bR_k = \bH$, $\bR_k^2 = \bH^2$, and $\bU_k = \bzero$.
  Suppose also we use an improper prior $\lambda_k \propto \const$, which is the same as having $v_k^2 \to 0$ and $m_k v_k^{-2} \to 0$.
  The mean field distribution for $\lambda$ is then
  \[
    \lambda \sim \N \left( 
    \frac{\tr\big(\tilde\bPsi(\tilde\by^* - \bone\tilde\balpha^\top)^\top \bH \tilde\bw \big)}{\tr(\tilde\bPsi\E[\bw^\top\bH^2\bw])},
    \frac{1}{\tr(\tilde\bPsi\E[\bw^\top\bH^2\bw])}
    \right)
  \]
  Further, if $\tilde\bPsi = \tilde\psi\bI_m$, then
  \[
    \lambda \sim \N \left( 
    \frac{\sum_{j=1}^m (\tilde\by^*_{\bigcdot j} - \tilde\alpha_j\bone)^\top \bH \tilde\bw_{\bigcdot j} }{\sum_{j=1}^m
  \tr \big( \bH^2 \E [\bw_{\bigcdot j} \bw_{\bigcdot j}^\top] \big)},
    \frac{1}{\sum_{j=1}^m
  \tr \big( \bH^2 \E [\bw_{\bigcdot j} \bw_{\bigcdot j}^\top] \big)}
    \right)
  \]
  which bears a resemblance to the exponential family EM algorithm solutions described in Chapter 4.
  Now, $\tilde \bH_\eta = \E[\lambda \bH] = \tilde\lambda \bH$, and $\tilde \bH_\eta^2 = \E[\lambda^2 \bH^2] = (\Var\lambda + \tilde\lambda^2) \bH^2$.
\end{example}

\subsection{Derivation of \texorpdfstring{$\tilde q(\bPsi)$}{$\tilde q(\Psi)$}}

%Introduce the transformed random matrix $\bu = \bw\bPsi^{-1} \in \bbR^{n\times m}$.
%Since we have that  $\vecc \bu = (\vecc \bw)^\top (\bPsi^{-1} \otimes \bI_n)$, the optimal mean-field distribution for $\bu$ is normal with mean $\vecc \tilde\bu = \vecc (\tilde\bw\tilde\bPsi^{-1} )$ and variance
%\begin{align*}
%  \tilde \bV_u
%  &= (\tilde\bPsi^{-1} \otimes \bI_n) \tilde \bV_w (\tilde\bPsi^{-1} \otimes \bI_n). 
%\end{align*}
%In the case of the independent model, its mean is $\tilde\bu_{\bigcdot j} = \tilde\psi_j^{-1} \tilde\bu_{\bigcdot j}$ for $j=1,\dots,m$ and its variance is
%\begin{align*}
%  \tilde \bV_u 
%  &= \diag(\tilde\psi_1^{-2}\tilde\bV_{w_1},\dots,\tilde\psi_m^{-2}\tilde\bV_{w_m}).
%\end{align*}

We find that $q(\bPsi)$ satisfies
\begin{align*}
  \log q(\bPsi)
  &= \E_{\cZ\backslash\{\bPsi\}\sim q} \Big[ 
%  \half[n]\log\abs{\bPsi} 
  - \half\tr \big((\by^* - \bmu)^\top (\by^* - \bmu)\bPsi  \big)
%  - \half[n]\log\abs{\bPsi} 
  - \half\tr \big( \bw^\top\bw\bPsi^{-1} \big)
  \Big] \\
  &\phantom{==} 
  + \log p(\bPsi)
%  + \half[g-m-1]\log \abs{\bPsi} - \half\tr(\bG\bPsi) 
  + \const \\
  &= -\half \tr \Big(
  \big( %\bG + 
  \greyoverbrace{\E[(\by^* - \bmu)^\top (\by^* - \bmu)]}{\bG_1}
  \big)\bPsi +
  \greyoverbrace{\E[\bw^\top\bw]}{\bG_2} \bPsi^{-1}
  \Big) \\
  &\phantom{==} 
  + \log p(\bPsi)
%  + \half[g-m-1]\log \abs{\bPsi} - \half\tr(\bG\bPsi) 
  + \const 
\end{align*}
This seems to be the pdf of $\Wis(\bG+\bG_1,g)$ plus the pdf of a distribution which almost resembles an inverse Wishart pdf.
Unfortunately, the properties such as its moments and entropy are unknown.
%However, we can compute the posterior mode:
%\begin{align*}
%  \frac{\partial}{\partial\bPsi} \log q(\bPsi)
%  &= -\half \tr \left(\frac{\partial}{\partial\bPsi} (\bG_1\bPsi) + \frac{\partial}{\partial\bPsi}(\bG_2\bPsi^{-1}) \right) \\
%  &= -\half \tr \left( \bG_1 - \bG_2\bPsi^{-2} \right)
%\end{align*}
%equated to zero means solving $m$ quadratic equations 

The matrix $\bG_1$ is 
\begin{align*}
  \bG_1 
  &= \E[(\by^* - \bmu)^\top (\by^* - \bmu)] \\
  &= \E \big[\by^{*\top}\by^* + \balpha\bone_n^\top \bone_n\balpha^\top + \bw^\top\bH_\eta^2\bw -2\by^{*\top}\bone_n\balpha^\top -2\by^{*\top}\bH_\eta\bw -2 \balpha\bone_n^\top\bH_\eta\bw \big] \\
  &= \E \big[\by^{*\top}\by^*] + n\E[\balpha\balpha^\top] + \E[\bw^\top\bH_\eta\bw] -2(\tilde\by^{*\top}\bone_n\tilde\balpha^\top + \tilde\by^{*\top}\tilde\bH_\eta\tilde\bw + \tilde\balpha\bone_n^\top\tilde\bH_\eta\tilde\bw),
\end{align*}
and this involves second order moments of a conically truncated multivariate normal distribution, which needs to be obtained via simulation.
Meanwhile,
\begin{align*}
  \bG_{2,ij}
  &= \E[\bw^\top\bw]_{ij} \\
  &= \E[\bw_{\bigcdot i}^\top \bw_{\bigcdot j}] \\
  &= \tilde\bV_w[i,j] + \tilde\bw_{\bigcdot i}^\top  \tilde\bw_{\bigcdot j}.
\end{align*}

In the case of the independent I-probit model, we use a gamma prior on each of the precisions in the diagonal entries of $\bPsi=\diag(\psi_1,\dots,\psi_m)$.
Then, the variational density for each $\psi_j$ is found to be
\begin{align*}
  \log  q(\psi_j)
  &= \E_{\cZ\backslash\{\bPsi\}\sim q} \Big[ 
  \half[n]\log (\psi_1\cdots\psi_m) - \half\sum_{j=1}^m \sum_{i=1}^n \psi_j(\by^*_{ij} - \bmu_{ij})^2 
  \Big] \\
  &\phantom{==} +\E_{\cZ\backslash\{\bPsi\}\sim q} \Big[ 
  -\half[n]\log (\psi_1\cdots\psi_m) - \half\sum_{j=1}^m \sum_{i=1}^n \psi_j^{-1} \bw_{ij}^2 
  \Big] \\
  &\phantom{==} + \sum_{j=1}^m \big( (s_j - 1)\log \psi_j - r_j\psi_j \big) + \const \\
  &= (s_j - 1)\log \psi_j 
  - \psi_j \left( \half \E\norm{\by_{\bigcdot j}^* -  \bmu_{\bigcdot j}}^2 +  r_j \right) \\
  &\phantom{==}
  - \psi_j^{-1} \left( \half\E\norm{\bw_{\bigcdot j}}^2  \right) + \const
\end{align*}
which again, is a pdf of an unknown distribution.
However, its posterior mode can be computed.
Write $a = -\left( \half \E\norm{\by_{\bigcdot j}^* -  \bmu_{\bigcdot j}}^2 + r_j \right)$,
$b = s_j - 1$, and $c=\left( \half\E\norm{\bw_{\bigcdot j}}^2  \right)$.
Then,
\begin{align*}
  \frac{\partial}{\partial\psi_j} \log q(\psi_j)
  = \frac{\partial}{\partial\psi_j} \left( a\psi_j + b\log \psi_j - c\psi_j^{-1} \right) 
  = a +b\psi_j^{-1} + c\psi_j^{-2} 
\end{align*}
equated to zero means solving a quadratic equation in $\psi_j$.
Suppose that $p(\psi_j) \propto \const$, then $s_j=1$ and $r_j = 0$ so $\tilde\psi_j$ can be solved directly to be
\[
  \hat\psi_j = \sqrt{\frac{ \E\norm{\by_{\bigcdot j}^* -  \bmu_{\bigcdot j}}^2 }{\E\norm{\bw_{\bigcdot j}}^2}}.
\]
%The mean is given by $\tilde\psi_j = (s_j + n)(\half \E\norm{\by_{\bigcdot j}^* - \bmu_{\bigcdot j}}^2 + \half \E\norm{\bu_{\bigcdot j}}^2  + r_j)^{-1}$.
If the posterior mean is close to its mode, then $\hat\psi_j$ is a good approximation for $\tilde\psi_j$.

To calculate $\E\norm{y_{\bigcdot j}^* -  \bmu_{\bigcdot j}}^2 = \E \sum_{i=1}^n (\by_{ij}^* - \mu_{ij})^2$, one first needs $\E (y_{ij}^* - \alpha_j - \bw_{\bigcdot j}^\top\bh_\eta(x_i))^2$.
This, in itself, presents a challenge to compute analytically, because it requires, among other things, the second moments $\E y_{ij}^{*2}$ and $\E [\bw_{\bigcdot j}^\top\bh_\eta(x_i)\bh_\eta(x_i)^\top \bw_{\bigcdot j}]$.
Although not entirely accurate, it is simpler to use the approximation
\[
  \E\norm{\by_{\bigcdot j}^* -  \bmu_{\bigcdot j}}^2 \approx 
  \norm{\tilde \by_{\bigcdot j}^* -  \tilde \bmu_{\bigcdot j}}^2.
\]
% from the distribution of $\by_{i \bigcdot} \sim \tN(\bmu_{i \cdot}, \tilde\bPsi^{-1}, \cC_{y_i})$.
%Because of the independence structure, these can be computed componentwise.
(see note \labelcref{note2} on page \pageref{note2}).
Also, we have $\bw_{\bigcdot j} \sim \N_n(\tilde\bw_{\bigcdot j}, \tilde\bV_{w_j})$, and so $  \E\norm{\bw_{\bigcdot j}}^2 = \tr(\tilde\bV_{w_j} + \tilde\bw_{\bigcdot j}\tilde\bw_{\bigcdot j}^\top)$.

\subsection{Derivation of \texorpdfstring{$\tilde q(\balpha)$}{$\tilde q(\alpha)$}}

Let $\bA = \diag(A_1,\dots,A_m)$ and $\ba = (a_1,\dots,a_m)^\top$.
The terms involving $\alpha_j$ in \cref{eq:qtilde} are
\begin{align*}
  \log q(\balpha)
  &= \E_{\cZ\backslash\{\balpha\}\sim q} \left[ 
  -\half\sum_{i=1}^n 
  \big(\by^*_{i \bigcdot} - \balpha - \bw^\top\bh_\eta(x_i)\big)^\top 
  \bPsi \big(\by^*_{i \bigcdot} - \balpha - \bw^\top\bh_\eta(x_i)\big)
  \right] \\
  &\phantom{==}
  - \half (\balpha - \ba)^\top \bA^{-1} (\balpha - \ba) + \const \\
  &= -\half \Bigg[ \balpha^\top(
  \greyoverbrace{n\bPsi + \bA^{-1}}{\tilde \bA}
  )\balpha -2\Bigg( \, 
   \greyoverbrace{\sum_{i=1}^n \bPsi\big( \tilde\by^*_{i \bigcdot} - \tilde\bw^\top\tilde\bh_\eta(x_i) \big) + \bA^{-1} \ba}{\tilde \ba} 
  \, \Bigg)^\top \balpha \Bigg]
\end{align*}
which implies a normal mean-field distribution for $\balpha$ whose mean and variance are $\tilde\balpha = \tilde \bA^{-1}\tilde \ba$ and $\tilde \bA^{-1}$ respectively.
If $\bPsi$ is diagonal, the components of $\balpha$ would be independent.

As a remark, due to identifiability, only $m-1$ of these intercept are estimable.
We can either put a constraint that one of the intercepts is fixed at zero, or the sum of the intercepts equals zero.
The latter constraint is implemented in this thesis, and this is realised by estimating all the intercepts and then centring them.

