\subsection{A brief introduction to variational inference}
\label{sec:varintro}

Consider a statistical model for which we have observations $\by := \{y_1,\dots,y_n\}$, but also some latent variables $\bz := \{z_1,\dots,z_n\}$.
Typically, in such models, there is a want to to evaluate the integral 
\begin{equation}\label{eq:varint}
  \cI = \int p(\by|\bz) p(\bz) \dint \bz.
\end{equation}
Models that include latent variables are plenty, for example: Gaussian mixture models, latent class analysis, factor models, random coefficient models, and so on.
Marginalising out the latent variables in \cref{eq:varint} is usually a precursor to obtaining a log-likelihood function to be maximised, in a frequentist setting.
In Bayesian analysis, the $\bz$'s are parameters which are treated as random, and the integral corresponds to the marginal density for $\by$, on which the posterior depends.

In many instances, for one reason or another, evaluation of $\cI$ is difficult, in which case inference is halted unless a way of overcoming the intractable integral \cref{eq:varint} is found.
Here, we discus \emph{variational inference} (VI), a fully Bayesian treatment of the statistical model with a deterministic algorithm, i.e. does not involve sampling from posteriors.
The crux of variational inference is this: find a suitably close distribution function $q(z)$ that approximates the true posterior $p(\bz|\by)$, where closeness here is defined in the Kullback-Leibler divergence sense,
\[
  \KL(q\Vert p) = \int \log \frac{q(\bz)}{p(\bz|\by)} q(\bz) \dint \bz.
\]
Posterior inference is then conducted using $q(\bz)$ in lieu of $p(\bz|\by)$.
Advantages of this method are that 1) it is fast to implement computationally (compared to MCMC); 2) convergence is assessed simply by monitoring a single convergence criterion; and 3) it works well in practice, as attested to by the many studies implementing VI.

Briefly, we present the motivation behind variational inference and the minimisation of the KL divergence.
Denote by $q(\cdot)$ some density function of $\bz$.
One may show that log marginal density (the log of the intractable integral \cref{eq:varint}) holds the following bound:
\begin{align}
  \log p(y) &= \log p(\by,\bz) - \log p(\bz|\by) \mycomment{\footnotesize (Bayes' theorem)} \nonumber \\
  &= \int \left\{ \log \frac{p(\by,\bz)}{q(\bz)} - \log \frac{p(\bz|\by)}{q(\bz)} \right\} q(\bz) \dint z \mycomment{\footnotesize (expectations both sides)} \hspace{0.7cm} \nonumber \\    
  &=  \cL(q) +  \KL(q \Vert p) \nonumber \\
  &\geq \cL(q) \label{eq:varbound}
\end{align}
since the KL divergence is a non-negative quantity.
The functional $\cL(q)$ given by 
\begin{align}
  \cL(q) 
  &= \int \log \frac{p(\by,\bz)}{q(\bz)} q(\bz) \dint \bz \nonumber \\
%  &= \E_{\bz\sim q}[\log p(\by,\bz) - \log q(\bz)] \nonumber \\
  &= \E_{\bz\sim q}\log p(\by,\bz) + H(q), \label{eq:elbo1}
\end{align}
where $H$ is the entropy functional, is known as the \emph{evidence lower bound} (ELBO).
Evidently, the closer $q$ is to the true $p$, the better, and this is achieved by maximising $\cL$, or equivalently, minimising the KL divergence from $p$ to $q$.
Note that the bound \cref{eq:varbound} achieves equality if and only if $q(\bz) \equiv p(\bz|\by)$, but of course the true form of the posterior is unknown to us---see \cref{sec:varEM} for a discussion.
Maximising $\cL(q)$ or minimising $\KL(q\Vert p)$ with respect to the density $q$ is a problem of calculus of variations, which incidentally, is where variational inference takes its name.
The astute reader will realise that $\KL(q||p)$ is impossible to compute, since one does not know the true distribution $p(\bz|\by)$. Efforts are concentrated on maximising the ELBO instead.

Maximising $\cL$ over all possible density functions $q$ is not possible without considering certain constraints.
Two such constraints are described. 
The first, is to make a distributional assumption regarding $q$, for which it is parameterised by $\nu$.
For instance, we might choose the closest normal distribution to the posterior $p(\bz|\by)$ in terms of KL divergence.
In this case, the task is to find optimal mean and variance parameters of a normal distribution.

\begin{figure}[htb]
  \centering
  \begin{tikzpicture}
    \draw[ultra thick] (0,0) ellipse (4cm and 2.2cm);
    \node at (-2.8,0.7) {$q(\bz;\nu)$};
    \draw[thick,colred] (-0.8,-0.5) to [curve through={(-2,0) .. (-2,-0.8) .. (1,-0.8) .. (-1,1.5) .. (0,1.5) ..(0.3,0.1) .. (0.5,1.5) .. (2,0) .. (2.1,-0.1) .. (2.2,1.1) .. (3,0.9)}] (3.317,1.225);  % curve using hobby tikz
    \draw[dashed, very thick,black!50] (3.317,1.225) -- (4.05,2.1);
    \fill (-0.8,-0.5) circle (2.2pt) node[below] {$\nu^{\text{init}}$};
    \fill (3.317,1.225) circle (2.2pt) node[left,yshift=-1] {$\nu^*$};
    \fill (4.05,2.1) circle (2.2pt) node[right,yshift=1] {$p(\bz|\by)$};
    \node[gray] at (4.5,1.4) {$\KL(q\Vert p)$};
  \end{tikzpicture}
  \vspace{-1em}
  \caption[Schematic view of variational inference. The aim is to find the closest distribution $q$ (parameterised by a variational parameter $\nu$) to $p$ in terms of KL divergence within the set of variational distributions, represented by the ellipse.]{Schematic view of variational inference\footnotemark. The aim is to find the closest distribution $q$ (parameterised by a variational parameter $\nu$) to $p$ in terms of KL divergence within the set of variational distributions, represented by the ellipse.}
\end{figure}
\footnotetext{Reproduced from the talk by David Blei entitled `Variational Inference: Foundations and Innovations', 2017. URL: \url{https://simons.berkeley.edu/talks/david-blei-2017-5-1}.}

The second type of constraint, and the one considered in this thesis, is simply an assumption that the approximate posterior $q$ factorises into $M$ disjoint factors.
Partition $\bz$ into $M$ disjoint groups $\bz = (z_{[1]},\dots,z_{[M]})$.
Note that each factor $z_{[k]}$ may be multidimensional.
Then, the structure
\[
  q(\bz) = \prod_{k=1}^M q_k(z_{[k]})
\]
for $q$ is considered.
This factorised form of variational inference is known in the statistical physics literature as the \emph{mean-field theory} \citep{itzykson1991statistical}.

Let us denote the distributions which minimise the Kullback-Leibler divergence (maximise the variational lower bound) by the use of tildes.
By appealing to \citet[equation 10.9, p. 466]{bishop2006pattern}, we find that for each $z_{[k]}$, $k=1,\dots,M$, $\tilde q_k$ satisfies
\begin{align}\label{eq:qtilde}
  \log \tilde q_k(z_{[k]}) = \E_{-k} \log p(\by,\bz) + \const
\end{align}
where expectation of the joint log density of $\by$ and $\bz$ is taken with respect to all of the unknowns $\bz$, except the one currently in consideration $z_{[k]}$, under their respective $\tilde q_k$ densities. 
%Estimates of the latent variables and parameters are then obtained by taking the mean of their respective approximate posterior distribution.

In practice, rather than an explicit calculation of the normalising constant, one simply needs to inspect \cref{eq:qtilde} to recognise it as a known log-density function, which is the case when exponential family distributions are considered.
That is, suppose that each complete conditional $p(z_{[k]}|\bz_{-k}, \by)$, where $\bz_{-k} = \{z_{[i]} | i \neq k\}$, follows an exponential family distribution
\[
  p(z_{[k]}| \bz_{-k}, \by ) 
  = B(z_{[k]})\exp \big(\ip{\zeta_k(\bz_{-k}, \by) , z_{[k]}} - A(\zeta_k) \big).
\]
Then, from \cref{eq:qtilde},
\begin{align*}
  \tilde q(z_{[k]})
  &\propto \exp\big(\E_{-k}\log p(z_{[k]}| \bz_{-k}, \by) \big) \\
  &= \exp \Big(\log B(z_{[k]}) + \E \ip{\zeta_k(\bz_{-k}, \by) , z_{[k]}} - \E [ A(\zeta_k) ] \Big) \\
  &\propto B(z_{[k]})\exp \E\ip{\zeta_\xi(\bz_{-k}, \by) , z_{[k]}}
\end{align*}
is also in the same exponential family.
In situations where there is no closed form expression for $\tilde q$, then one resorts to sampling methods such as a Metropolis random walk to estimate quantities of interest.
This stochastic step within a deterministic algorithm has been explored before in the context of a Monte Carlo EM algorithm---see \citet[§4, pp. 537--538]{meng1997algorithm} and references therein.

One notices that the optimal mean-field variational densities for each component are coupled with one another, in the sense that the distribution $\tilde q_k$ depends on the moments of the rest of the components $\bz_{-k}$.
For very simple problems, an exact solution for each $\tilde q_k$ can be found, but usually, the way around this is to employ an iterative procedure.
The \emph{coordinate ascent mean-field variational inference} (CAVI) algorithm cycles through each of the distributions in turn, updating them in sequence starting with arbitrary distributions as initial values.

\begin{algorithm}[H]
\caption{The CAVI algorithm}\label{alg:cavi}
  \begin{algorithmic}[1]
    \State \textbf{initialise} Variational factors $q_k(z_{[k]})$
    \While{ELBO $\cL(q)$ not converged}
      \For{$k = 1,\dots,M$}
        \State $\tilde q_k(z_{[k]}) \gets \const \times \exp\E_{-k}\log p(\by,\bz)$ \Comment{from \eqref{eq:qtilde}}
      \EndFor
      \State $\cL(q) \gets \E_{\bz \sim \prod_k\tilde q_k}\log p(\by, \bz) + \sum_{k=1}^m H\big[ q_k(z_{[k]})\big]$ \Comment{Update ELBO}
    \EndWhile
    \State \textbf{return} $\tilde q(\bz) = \prod_{k=1}^M \tilde q_j(z_{[k]})$ 
  \end{algorithmic}
\end{algorithm}

Each iteration of the CAVI brings about an improvement in the ELBO (hence the name coordinate ascent).
The algorithm terminates when there is no more significant improvement in the ELBO, indicating a convergence of the CAVI.
\citet{blei2017variational} notes that the ELBO is typically a non-convex function, in which case convergence may be to (one of possibly many) local optima.
A simple solution would be to restart the CAVI at multiple initial values, and the solution giving the highest ELBO is the distribution that is closest to the true posterior.

\subsection{Variational methods and the EM algorithm}
\label{sec:varEM}
% https://tex.stackexchange.com/questions/124269/energy-level-diagrams-with-tex
\tikzset{
    level/.style = {
        ultra thick,
    },
    connect/.style = {
        dashed,
        red
    },
    notice/.style = {
        draw,
        rectangle callout,
        callout relative pointer={#1}
    },
    label/.style = {
        text width=2cm
    }
}

\newcommand{\energyem}[1][1.5]{
\begin{tikzpicture}
  % Draw all levels
%  \draw[draw=none] (0, 8/#1) -- node[above] {EM Algorithm} (5, 8/#1);  % new top
  \draw[level,colred] (0, 6.5/#1) -- node[above] {} node[below] {} (5, 6.5/#1);  % top
  \draw[level,colblu] (0, 3.25/#1) -- node[above] {} node[below] {} (2.5, 3.25/#1);  % mid
  \draw[level] (0,0) -- node[above] {} (5, 0);  % bottom
  
  % Draw arrows
  \draw[<->,very thick] (0.75, 3.25/#1) -- node[left] {$\KL(q\Vert p)$} (0.75, 6.5/#1);
  \draw[<->,very thick] (1.75,0) -- node[left] {$\cL_\theta(q)$} (1.75,3.25/#1);
  \draw[<->,very thick] (3.5,0) -- node[right] {$\log p(\by|\theta)$} (3.5,6.5/#1);
\end{tikzpicture}
}

\newcommand{\energyemEstep}[1][2.15]{
\begin{tikzpicture}
  % Draw all levels
  \draw[draw=none] (0,8/#1) -- node[above] {EM (E-step)} (5,8/#1);  % new top
  \draw[level,colred] (0,6.5/#1) -- node[above] {} node[below] {} (5,6.5/#1);  % top
  \draw[level,colblu,dashed] (0,3.25/#1) -- node[above] {} node[below] {} (2.5,3.25/#1);  % mid
  \draw[level] (0,0) -- node[above] {} (5,0);  % bottom
  
  % Draw arrows
  \draw[<->,very thick,opacity=0] (0.75,3.25/#1) -- node[left] {\footnotesize $\KL(q\Vert p)$} (0.75,6.5/#1);
  \draw[<->,very thick] (3.5,0) -- node[right] {\footnotesize $\log p(\by|\theta^{\text{old}})$} (3.5,6.5/#1);
  
  % E-step
  \draw[level,colblu] (0,6.5/#1) -- node[above] {\color{black!99}\footnotesize $ \KL(\tilde q\Vert p) = 0$} (2.5,6.5/#1);  % top      
  \draw[->,very thick,colblu] (2,3.25/#1) -- node[left] {} (2,6.5/#1);  
  \draw[<->,very thick] (1.25,0) -- node[left,yshift=-33/#1] {\footnotesize $\cL_{\theta^{\text{old}}}(\tilde q)$} (1.25,6.5/#1);      
  
  \draw[white] (0,0) -- (0,-0.5);
\end{tikzpicture}
}

\newcommand{\energyemMstepfade}[1][2.15]{
\begin{tikzpicture}[opacity=0.25]
  % Draw all levels
  \draw[level,colred] (0,8/#1) -- node[above] {\color{black!99} EM (M-step)} (5,8/#1);  % new top
  \draw[level,colblu,dashed] (0,6.5/#1) -- node[above] {} node[below] {} (2.5,6.5/#1);  % mid
  \draw[level,colred,dashed] (2.5,6.5/#1) -- node[above] {} node[below] {} (5,6.5/#1);  % mid
  \draw[level] (0,0) -- node[above] {} (5,0);  % bottom
  \draw[level,colblu] (0,7.1/#1) -- node[above] {} node[below] {} (2.5,7.1/#1);  % mid
  
  % Draw arrows
  \draw[<->,very thick] (0.75,7.1/#1) -- node[left] {\footnotesize $\KL(\tilde q\Vert p)$} (0.75,8/#1);
  \draw[<->,very thick] (1.25,0) -- node[left] {\footnotesize $\cL_{\theta^*}(\tilde q)$} (1.25,7.1/#1);         
  \draw[<->,very thick] (3.5,0) -- node[right] {\footnotesize $\log p(\by|\theta^*)$} (3.5,8/#1);
  \draw[->,very thick,colblu] (2,6.5/#1) -- node[left] {} (2,7.1/#1);  
  \draw[->,very thick,colred] (4.5,6.5/#1) -- node[left] {} (4.5,8/#1);  
  
  \draw[white] (0,0) -- (0,-0.5);
\end{tikzpicture}
}


\newcommand{\energyemMstep}[1][2.15]{
\begin{tikzpicture}
  % Shaded region
  \fill[colred!10] (0,6.5/#1) rectangle (5,8/#1);
  
  % Draw all levels
  \draw[level,colred] (0,8/#1) -- node[above] {\color{black!99} EM (M-step)} (5,8/#1);  % new top
  \draw[level,colblu,dashed] (0,6.5/#1) -- node[above] {} node[below] {} (2.5,6.5/#1);  % mid
  \draw[level,colred,dashed] (2.5,6.5/#1) -- node[above] {} node[below] {} (5,6.5/#1);  % mid
  \draw[level] (0,0) -- node[above] {} (5,0);  % bottom
  \draw[level,colblu] (0,7.1/#1) -- node[above] {} node[below] {} (2.5,7.1/#1);  % mid
  
  % Draw arrows
  \draw[<->,very thick] (0.75,7.1/#1) -- node[left] {\footnotesize $\KL(\tilde q\Vert p)$} (0.75,8/#1);
  \draw[<->,very thick] (1.25,0) -- node[left] {\footnotesize $\cL_{\theta^*}(\tilde q)$} (1.25,7.1/#1);         
  \draw[<->,very thick] (3.5,0) -- node[right] {\footnotesize $\log p(\by|\theta^*)$} (3.5,8/#1);
  \draw[->,very thick,colblu] (2,6.5/#1) -- node[left] {} (2,7.1/#1);  
  \draw[->,very thick,colred] (4.5,6.5/#1) -- node[left] {} (4.5,8/#1);   

  \draw[white] (0,0) -- (0,-0.5);
\end{tikzpicture}
}

\newcommand{\energyvbEstep}[1][2.15]{
\begin{tikzpicture}
  % Draw all levels
  \draw[draw=none] (0,8/#1) -- node[above] {VB (E-step)} (5,8/#1);  % new top
  \draw[level,colred] (0,6.5/#1) -- node[above] {} node[below] {} (5,6.5/#1);  % top
  \draw[level,colblu,dashed] (0,3.25/#1) -- node[above] {} node[below] {} (2.5,3.25/#1);  % mid
  \draw[level] (0,0) -- node[above] {} (5,0);  % bottom
  
  % Draw arrows
  \draw[<->,very thick] (0.75,5.25/#1) -- node[left] {\footnotesize $\KL(\tilde q \Vert p)$} (0.75,6.5/#1);
  \draw[<->,very thick] (3.5,0) -- node[right] {\footnotesize $\log p(\by|\theta^{\text{old}})$} (3.5,6.5/#1);
  
  % E-step
  \draw[level,colblu] (0,5.25/#1) -- node[above] {} (2.5,5.25/#1);  % new KL level     
  \draw[->,very thick,colblu] (2,3.25/#1) -- node[left] {} (2,5.25/#1);  % increase in L  
  \draw[<->,very thick] (1.25,0) -- node[left,yshift=-33/#1] {\footnotesize $\cL_{\theta^{\text{old}}}(\tilde q)$} (1.25,5.25/#1);    

  \draw[white] (0,0) -- (0,-0.5);
\end{tikzpicture}
}

\newcommand{\energyvbMstepa}[1][2.15]{
\begin{tikzpicture}
  % Shaded region
  \fill[colred!10] (0,6.5/#1) rectangle (5,8/#1);  
  
  % Draw all levels
  \draw[level,colred] (0,8/#1) -- node[above] {\color{black!99} VB (M-step)---scenario A} (5,8/#1);  % new top
  \draw[level,colred,dashed] (2.5,6.5/#1) -- node[above] {} node[below] {} (5,6.5/#1);  % top
  \draw[level,colblu] (0,6.5/#1) -- node[above] {} node[below] {} (2.5,6.5/#1);  % top
  \draw[level] (0,0) -- node[above] {} (5,0);  % bottom
  \draw[level,colblu,dashed] (0,5.25/#1) -- node[above] {} (2.5,5.25/#1);  % new KL level     

  % Draw arrows
  \draw[<->,very thick] (0.75,6.5/#1) -- node[left] {\footnotesize $\KL(\tilde q\Vert p)$} (0.75,8/#1);   
  \draw[<->,very thick] (1.25,0) -- node[left] {\footnotesize $\cL_{\theta^*}(\tilde q)$} (1.25,6.5/#1);                  
  \draw[<->,very thick] (3.5,0) -- node[right] {\footnotesize $\log p(\by|\theta^*)$} (3.5,8/#1);
  \draw[->,very thick,colblu] (2,5.25/#1) -- node[left] {} (2,6.5/#1);        
  \draw[->,very thick,colred] (4.5,6.5/#1) -- node[left] {} (4.5,8/#1);  

  \draw[white] (0,0) -- (0,-0.5);
\end{tikzpicture}
}

\newcommand{\energyvbMstepb}[1][2.15]{
\begin{tikzpicture}
  % Shaded region
  \fill[colred!10] (0,6.5/#1) rectangle (5,7.25/#1);  
  
  % Draw all levels
  \draw[draw=none] (0,8/#1) -- node[above] {\color{black} VB (M-step)---scenario B} (5,8/#1);  % new top
  \draw[level,colred] (0,7.25/#1) -- node[above] {} (5,7.25/#1);  % new top
  \draw[level,colred,dashed] (0,6.5/#1) -- node[above] {} node[below] {} (5,6.5/#1);  % top
  \draw[level,colblu] (0,6/#1) -- node[above] {} node[below] {} (2.5,6/#1);  % top
  \draw[level] (0,0) -- node[above] {} (5,0);  % bottom
  \draw[level,colblu,dashed] (0,5.25/#1) -- node[above] {} (2.5,5.25/#1);  % new KL level     

  % Draw arrows
  \draw[<->,very thick] (0.75,6/#1) -- node[left,yshift=6.5/#1] {\footnotesize $\KL(\tilde q\Vert p)$} (0.75,7.25/#1);   
  \draw[<->,very thick] (1.25,0) -- node[left] {\footnotesize $\cL_{\theta^*}(\tilde q)$} (1.25,6/#1);                  
  \draw[<->,very thick] (3.5,0) -- node[right] {\footnotesize $\log p(\by|\theta^*)$} (3.5,7.25/#1);
  \draw[->,very thick,colblu] (2,5.25/#1) -- node[left] {} (2,6/#1);        
  \draw[->,very thick,colred] (4.5,6.5/#1) -- node[left] {} (4.5,7.25/#1); 
  
  \draw[white] (0,0) -- (0,-0.5);
\end{tikzpicture}
}

\newcommand{\energyvbMstepc}[1][2.15]{
\begin{tikzpicture}
  % Draw all levels
  \draw[draw=none] (0,8/#1) -- node[above] {\color{black} VB (M-step)---scenario C} (5,8/#1);  % new top
  \draw[level,colred] (0,6.5/#1) -- node[above] {} node[below] {} (5,6.5/#1);  % top
  \draw[level,colblu] (0,5.8/#1) -- node[above] {} node[below] {} (2.5,5.8/#1);  % top
  \draw[level] (0,0) -- node[above] {} (5,0);  % bottom
  \draw[level,colblu,dashed] (0,5.25/#1) -- node[above] {} (2.5,5.25/#1);  % new KL level     

  % Draw arrows
  \draw[<->,very thick] (0.75,5.8/#1) -- node[left] {\footnotesize $\KL(\tilde q\Vert p)$} (0.75,6.5/#1);   
  \draw[<->,very thick] (1.25,0) -- node[left] {\footnotesize $\cL(\tilde q;\theta^*)$} (1.25,5.8/#1);                  
  \draw[<->,very thick] (3.5,0) -- node[right] {\footnotesize $\log p(y;\theta^*)$} (3.5,6.5/#1);
  \draw[->,very thick,colblu] (2,5.25/#1) -- node[left] {} (2,5.8/#1);    
  
  \draw[white] (0,0) -- (0,-0.5);  
\end{tikzpicture}
}

Consider again the latent variable setup described in \cref{sec:varintro}, but suppose the goal now is to maximise the (marginal) log-likelihood of the parameters $\theta$ of the model.
We will see how the EM algorithm relates to minimising the KL divergence between a density $q(\bz)$ and the posterior of $\bz$, and connect this idea to variational methods.

\begin{figure}[t]
  \centering
  \energyem 
  \caption[Illustration of the decomposition of the log likelihood.]{Illustration\footnotemark~of the decomposition of the log-likelihood into $\cL_\theta(q)$ and $\KL[q(\bz) \Vert p(\bz|\by)]$. The quantity $\cL_\theta(q)$ is a lower bound for the log-likelihood.}
  \label{fig:loglikdecomp}
\end{figure}
\footnotetext{Reproduced from \citet[Figure 9.11]{bishop2006pattern}.}

As we did in deriving \cref{eq:varbound}, we decompose the marginal log-likelihood as
\begin{align*}
  \log p(y|\theta) 
  &= \E \left[\log \frac{p(\by,\bz|\theta)}{q(\bz)} \right] - \E \left[\log \frac{p(\bz|\by,\theta)}{q(\bz)} \right]  = \cL(q) + \KL(q\Vert p).
%  &= \log p(\by,\bz|\theta) - \log p(\bz|\by) \mycomment{\footnotesize (Bayes' theorem)}  \\
%  &= \int \left\{ \log \frac{p(\by,\bz)}{q(\bz)} - \log \frac{p(\bz|\by)}{q(\bz)} \right\} q(\bz) \dint z \mycomment{\footnotesize (expectations both sides)} \hspace{0.7cm}  \\
%  &\geq \cL(q)
\end{align*}
This decomposition is shown in \cref{fig:loglikdecomp}.
We realise that the KL divergence non-negative, and is zero exactly when $q(\bz) \equiv p(\bz|\by,\theta)$.
Substituting this into the above equation yields the relationship
\begin{align*}
  \log p(y|\theta) 
  &= \E \left[\log \frac{p(\by,\bz|\theta)}{p(\bz|\by,\theta)} \right] 
  - \cancel{\E \left[\log \frac{p(\bz|\by,\theta)}{p(\bz|\by,\theta)} \right]}  \\
  &= \E \log p(\by,\bz|\theta) - \E p(\bz|\by,\theta).
\end{align*}
By taking expectations under the posterior distribution with known parameter values $\theta^{(t)}$, the term on the left becomes the $Q$ function of the E-step
\[
  Q(\theta) = Q(\theta|\theta^{(t)}) = \E_\bz\left[ \log p(\by,\bz|\theta) \,\big|\, \by,\theta^{(t)} \right],
\]
while the term on the left is an entropy term.
Thus, minimising the KL divergence corresponds to the E-step in the EM algorithm.
As a side fact, for any $\theta$, we find that
\begin{align*}
  \log p(\by|\theta) - \log p(\by|\theta^{(t)}) 
  &= Q(\theta|\theta^{(t)}) - Q(\theta^{(t)}|\theta^{(t)}) + \Delta \, \text{entropy} \\
  &\geq Q(\theta|\theta^{(t)}) - Q(\theta^{(t)}|\theta^{(t)}).
\end{align*}
because entropy differences are positive by Gibbs' inequality.
We see that maximising $Q$ with respect to $\theta$ (the M-step) brings about an improvement to the log-likelihood value.
To summarise, the EM algorithm is seen as
\begin{itemize}
  \item \textbf{E-step}. Maximise $\cL_\theta\big[q(\bz)\big]$ with respect to $q$, keeping $\theta$ fixed. This is equivalent to minimising $\KL(q\Vert p)$.
  \item \textbf{M-step}. Maximise $\cL\big[q(\bz|\theta)\big]$ with respect to $\theta$, keeping $q$ fixed. 
\end{itemize}

When the true posterior distribution $p(\bz|\by)$ is not tractable, then the E-step becomes intractable as well.
By constraining the maximisation in the E-step to consider $q$ belonging to a family of tractable densities, the E-step yields a variational approximation $\tilde q$ to the true posterior.
In \cref{sec:varintro}, we saw that constraining $q$ to be of a factorised form, then $\tilde q$ is a mean-field density.
This form of the EM is known as \emph{variational Bayes EM algorithm} (VB-EM) \citep{beal2003}.

In variational inference, a fully Bayesian treatment of the parameters is considered, with the aim of obtaining approximation to their posterior distributions.
In VB-EM, the variational approximation is only performed on the latent, or `missing' variables, to use the EM nomenclature.
After a variational E-step, the M-step proceeds as usual, and as such, all of the material relating to the EM in the previous chapter is applicable.
The VB-EM can also be seen as obtaining (approximate) maximum a posteriori estimates with diffuse priors on the parameters.

\begin{figure}[p]
  \centering
  \energyemEstep \hspace{0.5cm}
  \energyvbEstep
  \energyemMstep \hspace{0.5cm}
  \energyvbMstepa
  \energyemMstepfade \hspace{0.5cm}
  \energyvbMstepb
  \energyemMstepfade \hspace{0.5cm}
  \energyvbMstepc
  \vspace{-1em} 
  \caption{Illustration of EM vs VB-EM. Whereas the EM guarantees an increase in log-likelihood value (red shaded region), the VB-EM does not.}
\end{figure}



\subsection{The EM algorithm for I-probit models is intractable---variational Bayes EM?}

% IT TURNS OUT THE EM IS DIFFICULT! BECAUSE THERE IS NO INHERENT ASSUMPTION OF INDEPENDENCE BETWEEN YSTAR AND W, SO THE DISTRIBUTIONS ARE DIFFICULT TO ASCERTAIN. 
% IN THE VARIATIONAL ALGORITHM, THIS IS ASSUMED IN A MEAN-FIELD APPROXIMATION

Consider employing an EM algorithm, similar to the one seen in the previous chapter, to estimate I-probit models.
This time, treat both the latent propensities $\by^*$ and the I-prior random effects $\bw$ as `missing', so the complete data is $\{\by,\by^*,\bw\}$.
Now, due to the independence of the observations $i=1,\dots,n$, the complete data log-likelihood is
\begin{align*}
  \log p&(\by,\by^*,\bw) \\
  &= \sum_{i=1}^n \Big\{ 
  \log p(y_i|\by^*_{i \bigcdot}) + \log p(\by^*_{i \bigcdot}|\bw_{i \bigcdot}) + \log p(\bw_{i \bigcdot}) 
  \Big\} \\
  &= - \half \sum_{i=1}^n \ind[y_{ij}^* = \max_k y_{ik}^*] \Bigg[
%   \cancel{\half\log\abs{\bPsi}} 
    (\by^*_{i \bigcdot} - \balpha - \bw_{i \bigcdot}^\top\bh_\eta(x_i))^\top \bPsi (\by^*_{i \bigcdot} - \balpha - \bw_{i \bigcdot}^\top\bh_\eta(x_i)) \\
  &\hspace{9.5cm} + \bw_{i \bigcdot}^\top \bPsi^{-1} \bw_{i \bigcdot} \Bigg] 
%  \cancel{- \half\log\abs{\bPsi}} 
   + \const
\end{align*}
which looks like the complete data log-likelihood seen previously in \cref{eq:QfnEstep}, except that here, together with the $\bw_{i \bigcdot}$'s, the $\by^*_{i \bigcdot}$'s are never observed.

For the E-step, it is of interest to determine the posterior density $p(\by^*,\bw|\by) = p(\by^*|\bw,\by)p(\bw|\by)$, which apparently is hard to obtain.
We can go as far as determining that the full conditional of the latent propensities is multivariate subject to a conical truncation $\cC_j = \{y_{ij}^* > y_{ik}^* \,|\, \forall k \neq j \}$, i.e. $\by^*_{i \bigcdot}|\bw_{i \bigcdot},\{y_i=j\} \iid \tN_m(\balpha + \bw_{i \bigcdot}^\top\bh_\eta(x_i), \bPsi^{-1},\cC_j)$, for each $i=1,\dots,n$, and that $\vecc \bw|\by^*\sim \N(\tilde\bw,\tilde\bV_w)$ is found to be similar to the distribution in \cref{eq:varipostw}.
%To be specific, $\vecc \bw | \by^* \sim \N_{nm}(\vecc \tilde\bw, \tilde \bV_w)$, where
%\begin{align*}
%  \vecc \tilde\bw = \tilde\bV_w (\bPsi \otimes \bH_\eta) \vecc(\by^* - \bone_n\balpha^\top)
%  \hspace{0.5cm}\text{and}\hspace{0.5cm}
%  \tilde \bV_w^{-1} = (\bPsi \otimes \bH_\eta^2) + (\bPsi^{-1} \otimes \bI_n).
%\end{align*}
To obtain the first and second posterior moments for the I-prior random effects, we can use the law of total expectations:
\begin{gather*}
  \E[\vecc \bw|\by]
  = \E_{\by^*} \big[ \E[\vecc \bw | \by^*] \big| \by \big] =: \hat\bw \\
  \text{and}\\
  \E[\vecc \bw (\vecc \bw)^\top|\by]
  = \E_{\by^*} \big[ \E[\vecc \bw (\vecc \bw)^\top| \by^*] \big| \by \big] =: \hat\bW ,
\end{gather*}
but this requires $p(\by^*|\by)$ which does not come by easily.
A similar problem has been faced by \citet{chan1997maximum}, who analysed binary linear probit models with random effects.
The authors ultimately resort to Monte Carlo sampling within an EM framework to overcome the difficult distributions of interest.

Suppose that, instead of the true posterior distribution $p(\by^*,\bw|\by)$ being used, a mean-field variational approximation $q(\by^*,\bw) = q(\by^*)q(\bw)$ is used instead.
As we know from \hltodo{Section X}, $q(\by^*)$ is a truncated multivariate normal distribution, and $q(\bw)$ is multivariate normal, whose means and second moments can be computed with some  effort.
Let $\bar\by^* = \by^* - \bone_n\balpha^\top$.
The (approximate) E-step then entails computing
\begin{align*}
  Q(\theta) 
  &= \E_{\by^*,\bw\sim q}  \log p(\by,\by^*,\bw|\theta) \\
  &= \const -\half \tr\E_{\by^*,\bw\sim q} \left[ 
  \bPsi(\bar\by^{*\top}\bar\by^* + \bw^\top\bH_\eta^2\bw - 2\bar\by^*\bPsi\bw^\top\bH_\eta )
  + \bPsi^{-1} \bw^\top\bw 
  \right].
\end{align*}
In the M-step, this is maximised with respect to $\theta$.
This is the VB-EM algorithm described in \cref{sec:varEM}.
As per the discussion in \hltodo{Section X}, this alleviates the problem of non-conjugacy of the complete conditional for $\bPsi$.
One downside to VB-EM is that it is not entirely certain how one could obtain standard errors for the parameters, other than by bootstrapping, which for the I-probit model, is likely to be computationally intensive.
