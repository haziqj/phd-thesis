Post-estimation procedures such as obtaining predictions for a new data point, the credibility interval for such predictions, and model comparison, are of interest.
These are performed in an empirical Bayes manner using the variational posterior density of the regression function obtained from the output of the variational EM algorithm.

We first describe prediction of a new data point $x_{\text{new}}$.
Step one is to determine the distribution of the posterior regression functions in each class, $\bff(x_{\text{new}}) = \bw^\top\bh_\eta(x_{\text{new}})$, where $\bh_\eta(x_{\text{new}})$ is the vector of length $n$ containing entries $h_\eta(x_i,x_\new)$, given values for the parameters $\theta$ of the I-probit model.
To this end, we use the ELBO estimates for $\theta$, i.e. $\hat\theta = \argmax_\theta \cL_q(\theta)$, as obtained from the variational EM algorithm.
As we know, the variational distribution of $\vecc\bw$ is normally distributed with mean and variance according to \cref{eq:varipostw}.
By writing $\vecc \tilde\bw = (\tilde \bw_{\bigcdot 1}, \dots, \tilde \bw_{\bigcdot m})^\top$ to separate out the I-prior random effects per class, we have that $\bw_{\bigcdot j}|\hat\theta \sim \N_n(\tilde \bw_{\bigcdot j}, \tilde\bV_w[j,j])$, and $\Cov[\bw_{\bigcdot j},\bw_{\bigcdot k}] = \tilde\bV_w[j,k]$, where the `$[\cdot,\cdot]$' indexes the $n\times n$ sub-block of the block matrix structured matrix $\bV_w$.
Thus, for each class $j=1,\dots,m$ and any $x \in \cX$,
\[
  f_j(x)|\by,\hat\theta \sim \N\big(\,
  \bh_{\hat\eta}(x)^\top \tilde\bw_{\bigcdot j} \, , \,
  \bh_{\hat\eta}(x)^\top\tilde\bV_w[j,j]\bh_{\hat\eta}(x)
  \big),
\]
and the covariance between the regression functions in two different classes is
\[
  \Cov\big[f_j(x),f_k(x)|\by,\hat\theta \big] = 
  \bh_{\hat\eta}(x)^\top \tilde\bV_w[j,k] \tilde\bh_{\hat\eta}(x).
\]
Then, in step two, using the results obtained in the previous chapter in \cref{sec:ipriorpostest} \colp{\mypageref{sec:ipriorpostest}}, we have that the latent propensities $y_{\new,j}^*$ for each class are normally distributed with mean, variance, and covariances
\begin{alignat*}{3}
  \E[y_{\new,j}^*|\by,\hat\theta] 
  &= \hat\alpha_j + \E \big[ f_j(x_\new) |\by,\hat\theta \big] 
  &&=: \hat\mu_j(x_\text{new}) \\
  \Var[y_{\new,j}^*|\by,\hat\theta] 
  &= \Var\big[f(x_\new)|\by,\hat\theta \big] + \bPsi^{-1}_{jj}
  &&=: \hat\sigma_j^2(x_\text{new}) \\
  \Cov[y_{\new,j}^*,y_{\new,k}^*|\by,\hat\theta]
  &= \Cov\big[f_j(x),f_k(x)|\by,\hat\theta \big] + \bPsi^{-1}_{jk}
  &&=: \hat\sigma_{jk}(x_\text{new}).
\end{alignat*}

From here, step three would be to extract class information of data point $x_\text{new}$, which are contained in the normal distribution $\N_{m}\big(\hat\bmu_\new, \hat\bV_\new \big)$, where
\begin{equation*}
  \hat\bmu_\new = \big(\mu_1(x_\text{new}),\dots, \mu_m(x_\text{new}) \big)^\top 
  \hspace{0.5cm}\text{and}\hspace{0.5cm}
  \hat\bV_{\new,jk} = 
  \begin{cases}
    \hat\sigma_j^2(x_\text{new}) & \text{if } j = k \\
    \hat\sigma_{jk}(x_\text{new}) & \text{if } j \neq k. \\
  \end{cases}
\end{equation*}
The predicted class is inferred from the latent variables using
\[
  \hat y_{\text{new}} = \argmax_k \hat\mu_k(x_\new), 
\]
while the probabilities for each class are obtained by way of integration of a multivariate normal density, as per \cref{eq:pij}:
\begin{align}
  \hat p_{\text{new},j} 
  &=  \idotsint\displaylimits_{\{y^*_j > y^*_k | \forall k \neq j \}} \phi(y_{1}^*, \dots, y_{m}^*|\hat\bmu_\new, \hat\bV_{\new}) \dint y_{1}^* \cdots \dint y_{m}^*.
\end{align}
For the independent I-probit model, class probabilities are obtained in a more compact manner via
\[
  \hat p_{\text{new},j} 
  = \E_Z \Bigg[ \mathop{\prod_{k=1}^m}_{k\neq j} 
  \Phi \left(\frac{\hat\sigma_j(x_\text{new})}{\hat\sigma_k(x_\text{new})} Z + \frac{\hat\mu_j(x_\new) - \hat\mu_k(x_\new)}{\hat\sigma_k^2(x_\text{new})} \right) \Bigg],
\]
as per \cref{eq:pij2}, since the $m$ components of $\bff(x_\new)$, and hence the $\by^*_{\new,j}$'s, are independent of each other ($\bPsi$ and $\hat\bV_\new$ are diagonal).
Prediction of a single new data point takes $O(n^2m)$ time, because there are essentially $m$ I-prior posterior regression functions, and each take $O(n^2)$ to evaluate.
This is assuming negligible time to compute the class probabilities.

We are able to take advantage of the Bayesian machinery to obtain credibility intervals for probability estimates or any transformation of these probabilities (e.g. log odds or odds ratios).
The procedure is as follows.
First, obtain samples $\bw^{(1)},\dots,\bw^{(T)}$ by drawing from its variational posterior distribution $\vecc\bw^{(t)}|\hat\theta \sim \N_{nm}(\vecc \tilde\bw,\bV_w)$.
Then, obtain samples of class probabilities $\{p_{xj}^{(1)},\dots, p_{xj}^{(T)} \}_{j=1}^m$, for a given data point $x\in\cX$ by evaluating
\[
  p_{xj}^{(t)} = \idotsint\displaylimits_{\{y^*_j > y^*_k | \forall k \neq j \}} \phi\big(y_{1}^*, \dots, y_{m}^*|\hat\bmu^{(t)}(x), \hat\bV(x)\big) \dint y_{1}^* \cdots \dint y_{m}^*,
\]
where $\hat\bmu^{(t)}(x) = \hat\balpha + \bw^{(t)\top}\bh_{\hat\eta}(x)$, and $\hat\bV(x)_{jk}$ equals $\hat\sigma^2_j(x)$ if $j = k$, and $\hat\sigma_{jk}(x)$ otherwise.
To obtain a statistic of interest, say, a 95\% credibility interval of a function $r(p_{xj})$ of the probabilities, simply take the empirical lower 2.5th and upper 97.5th percentile of the transformed sample $\big\{ r(p_{xj}^{(1)}),\dots, r(p_{xj}^{(T)}) \big\}$.
%In this manner, all aspects of uncertainty, from the parameters to the latent variables of the generative model, are accounted for.

\begin{remark}
  Unfortunately, with the variational EM algorithm, standard errors for the parameters $\theta$ are not so easy to obtain.  
  We could not ascertain as to the availability of an unbiased estimate of the asymptotic covariance matrix for $\theta$ under a variational framework.
  One strategy for obtaining standard errors is bootstrap \citep{chen2017use}:
  \begin{enumerate}
    \item Obtain $\hat\theta = \argmax_\theta \cL_q(\theta)$ using $\cS = \{(y_1,x_1),\dots, (y_n,x_n)\}$.
    \item For $t=1,\dots,T$, do
    \begin{enumerate}
      \item Obtain $\cS^{(t)} = \{ (y_1^{(t)},x_1^{(t)}),\dots, (y_n^{(t)},x_n^{(t)})\}$ by sampling $n$ points with replacement from $\cS$.
      \item Compute $\hat\theta^{(t)} = \argmax_\theta \cL_q(\theta)$ using the data $\cS^{(t)}$.
    \end{enumerate}
    \item For the $l$-th component of $\theta$, compute its variance estimator using
    \[
      \widehat\Var(\hat\theta_l) = \frac{1}{T} \sum_{t=1}^T (\hat\theta_l^{(t)} - \bar\theta_l)^2
      \hspace{0.5cm}\text{where}\hspace{0.5cm}
      \bar\theta_l = \frac{1}{T} \sum_{t=1}^T \hat\theta_l^{(t)}.
    \]
%    where
%    \[
%      \bar\theta_l = \frac{1}{T} \sum_{t=1}^T \hat\theta_l^{(t)}.
%    \]
  \end{enumerate}
  The obvious downside to this bootstrap scheme is computational time.
\end{remark}

Finally, a discussion on model comparison, which, in the variational inference literature, is achieved by comparing ELBO values of competing models \citep{beal2003}.
The rationale is that the ELBO serves as a conservative estimate for the log marginal likelihood, which would allow model selection via (empirical) Bayes factors.
This stems from the fact that 
\begin{align*}
  \log p(\by|\theta) 
  &= \cL_q(\theta) + \KL(q \Vert p ) 
  > \cL_q(\theta),
\end{align*}
since the Kullback-Leibler divergence from the true posterior density $p(\by^*,\bw|\by)$ to the variational density $q(\by^*,\bw)$ is strictly positive (it is zero if and only if the two densities are equivalent), and is minimised under a variational inference scheme.
\citet{kass1995bayes} suggest \cref{tab:BF} as a way of interpreting  observed Bayes factor values $\BF(M_1,M_0)$ for comparing model $M_1$ against model $M_0$, where $\BF(M_1,M_0)$ is approximated by
\[
  \BF(M_1,M_0) \approx \frac{\cL_q(\theta|M_1)}{\cL_q(\theta|M_0)},
\]
and $\cL_q(\theta|M_k)$, $k=0,1$, is the ELBO for model $M_k$.
It should be noted that while this works in practice, there is no theoretical basis for model comparison using the ELBO \citep{blei2017variational}.

\vspace{0.3em}
\begin{table}[htbp]\label{tab:BF}
  \centering
  \caption[Guidelines for interpreting Bayes factors]{Guidelines for interpreting Bayes factors \citep{kass1995bayes}.}
  \label{tab:bf}
  \begin{tabular}{lll}
    \toprule
    $2\log \BF(M_1,M_0)$ &$\BF(M_1,M_0)$ & Evidence against $M_0$ \\
    \midrule
    0--2  &1--3    &Not worth more than a bare mention \\ 
    2--6  &3--20   &Positive \\ 
    6--10 &20--150 &Strong \\ 
    >10   &>150    &Very strong \\ 
  \end{tabular}
\end{table}

\begin{remark}
  In the previous chapter on normal I-prior models, the I-prior could be integrated out of the model completely, resulting in a normal  log-likelihood for the parameters.
  Model comparison can be validly done using likelihood ratio tests and asymptotic chi-square distributions.
  Here however, we only have a lower bound to the log-likelihood, and most likely the asymptotic results of likelihood ratio tests do not hold.
  Then, the concept of approximate (empirical) Bayes factors seem most intuitive, even if not rooted in theory.
\end{remark}
\vspace{-1em}
