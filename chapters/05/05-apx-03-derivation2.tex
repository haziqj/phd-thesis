\section{Deriving the ELBO expression}

The evidence lower bound (ELBO) expression involves the following calculation:
\begin{align*}
  \cL &= \idotsint q(\by^*,\bw,\theta) 
  \log \frac{p(\by,\by^*,\bw,\theta)}{q(\by^*,\bw,\theta)}
  \dint\by^* \dint\bw \dint\theta \\
  &= \E \log \greyoverbrace{p(\by,\by^*,\bw,\theta)}{\text{joint likelihood}}
  +
  (\greyoverbrace{-\E \log q(\by^*,\bw,\theta)}{\text{entropy}}) \\
  &= \E\bigg[
  \cancel{\sum_{i=1}^n \sum_{j=1}^m \log  p(y_{i}|y_{ij}^*)} + 
  \sum_{i=1}^n \log  p(\by_{i \bigcdot}^*|\balpha,\bw,\bPsi,\eta) +
  \log p(\bw|\bPsi) +
  \log p(\bPsi) \\ 
  &\hspace{2cm} +
  \log p(\eta) +
  \log p(\balpha)
  \bigg] \\
  &\phantom{==} 
  + \sum_{i=1}^n H\big[q(\by^*_{i \bigcdot}) \big]
  + H\big[q(\bw) \big]
  + H\big[q(\bPsi) \big]
  + H\big[q(\eta) \big]
  + H\big[q(\balpha) \big].
\end{align*}
As we saw earlier, the distribution of $q(\bPsi)$ is not of recognisable form. 
This makes computation of $\E \log \abs\bPsi$, $\E \log p(\bPsi)$, and $H\big[q(\bPsi)\big]$, which are required in the expression of the ELBO, problematic.
For simplicity, we present the ELBO calculations for when $\bPsi$ is treated to be fixed.

\begin{remark}
  As discussed, given the latent propensities $\by^*$, the pdf of $\by$ is degenerate and hence can be disregarded.  
\end{remark}

\begin{remark}
  When using improper priors for the hyperparameters, i.e. $p(\eta,\balpha) \propto \const$, then these terms can be disregarded.  
\end{remark}

\subsection{Terms involving distributions of \texorpdfstring{$\by^*$}{$y^*$}}

\begin{align*}
  \sum_{i=1}^n  \bigg( &
  \E  \log p(\by_{i \bigcdot}^*|\balpha,\bw,\bPsi,\eta) 
  + H \big[q(\by^*_{i \bigcdot}) \big] 
  \bigg) \\
  &=  -\half[nm]\log 2\pi + \half[n] \log \abs{\bPsi} - \half \E \sum_{i=1}^n (\by^*_{i \bigcdot} - \bmu_{i \bigcdot})^\top \bPsi (\by^*_{i \bigcdot} - \bmu_{i \bigcdot}) \\
  &\phantom{==} +\half[nm]\log 2\pi - \half[n] \log \abs{\tilde\bPsi} + \half \E \sum_{i=1}^n (\by^*_{i \bigcdot} - \tilde\bmu_{i \bigcdot})^\top \bPsi (\by^*_{i \bigcdot} - \tilde\bmu_{i \bigcdot}) + \log C_i  \\
   &=\const + \sum_{i=1}^n \log C_i 
\end{align*}
where $C_i$ is the normalising constant for the distribution of multivariate truncated normal $\by_{i \bigcdot}$.

Notes:
\begin{enumerate}
  \item $p(\by_{i \bigcdot}^*)$ is the pdf of $\N(\bmu_{i \bigcdot}, \bPsi^{-1})$, and $q(\by_{i \bigcdot}^*)$ is the pdf of $\tN(\tilde\bmu_{i \bigcdot}, \bPsi^{-1}, \cC_{y_i})$, where $\bmu_{i \bigcdot} = \balpha + \bw^\top\bh_\eta(x_i) \in \bbR^m$.
%  \item For $\bPsi\sim \Wis(\cdot,\cdot)$ with mean $\tilde\bPsi$, $\E \log \abs{\bPsi} = \log \tilde\bPsi + \const$ \citep[§10.2]{bishop2006pattern}.
  \item It is simpler to use the approximation
  \begin{align}\label{eq:elboyaprx}
    \E (\by^*_{i \bigcdot} - \bmu_{i \bigcdot})^\top \bPsi (\by^*_{i \bigcdot} - \bmu_{i \bigcdot})
    \approx \E (\by^*_{i \bigcdot} - \tilde\bmu_{i \bigcdot})^\top \bPsi (\by^*_{i \bigcdot} - \tilde\bmu_{i \bigcdot}).   
  \end{align}
  rather than work out the actual quantity, which is
  \begin{align}\label{eq:elboyact}
    \E (\by^*_{i \bigcdot} - \bmu_{i \bigcdot})^\top \bPsi (\by^*_{i \bigcdot} - \bmu_{i \bigcdot})
    &= \E (\by^*_{i \bigcdot} - \tilde\bmu_{i \bigcdot})^\top \bPsi (\by^*_{i \bigcdot} - \tilde\bmu_{i \bigcdot}) + \tr(\bPsi\Var \bmu_{i \bigcdot})
  \end{align}
  where $\Var \bmu_{i \bigcdot} = \Var \balpha + \Var \bw^\top\bh_\eta(x_i)$, obtained by taking expectations with respect to everything except $\by^*_{i \bigcdot}$.
  The first term is a diagonal matrix of the posterior variances of the intercepts.
  The second term is where things get complicated.
  Let $\bOmega_i = \Var \bw^\top\bh_\eta(x_i)$. 
  Then $\bOmega_{i,kj} \approx \Cov(\bw^\top_{\bigcdot k}\bh_\eta(x_i), \bw^\top_{\bigcdot j}\bh_\eta(x_i)) = \bh_\eta(x_i)^\top \tilde\bV_w[k,j]\bh_\eta(x_i)$. 
  So
  \[
    \tr(\bPsi \bOmega_i) \approx \sum_{k,j=1}^m \bPsi_{kj} \bh_\eta(x_i)^\top \tilde\bV_w[k,j]\bh_\eta(x_i)
  \]
  However, we know that $\Var XY = \E X^2Y^2 - (\E XY)^2 = \Var X \Var Y + \Var X (\E Y)^2 + \Var Y (\E X)^2$, so there is actually some covariance terms which need to be considered, and these are not so easily computed.
  In practice, we find that using \cref{eq:elboyaprx} gives satisfactory results as far as determining convergence for the variational algorithm goes. 
\end{enumerate}

\subsection{Terms involving distributions of $\bw$}

\begin{align*}
  \E \log p(\bw|\bPsi) + H \big[q(\bw) \big] 
  &= \cancel{-\half[nm]\log 2\pi} - \half[n]\log\abs{\bPsi} - \half\E\tr \big( \bw \bPsi^{-1} \bw^\top \big) \\
  &\phantom{==} + \half[nm](1 + \cancel{\log 2\pi}) + \half\log\abs{\tilde\bV_w} \\
  &= \const 
  - \half \sum_{j=1}^m \tr \big(\bPsi^{-1} (\tilde\bV_w[j,j]  + \tilde\bw_{\bigcdot j}\tilde\bw_{\bigcdot j}^\top)\big)
\end{align*}

Notes:
\begin{enumerate}
  \item $p(\bw)$ is the pdf of $\MN(\bzero,\bI_n,\bPsi)$, and $q(\bw)$ is the pdf of $\N(\vecc \tilde\bw,\tilde\bV_w)$.
%  \item We used the first order approximation $\E \bPsi^{-1} \approx (\E \bPsi)^{-1} = \tilde\bPsi^{-1}$.
  \item $\tilde\bV_w[j,j]$ are the $n \times n$ sub matrices along the diagonal of $\tilde\bV_w$.
\end{enumerate}

\subsection{Terms involving distributions of $\eta$}

If no closed-form expression for $q(\eta)$ is found, then the expression $\E [\log p(\eta) - q(\eta)]$ must be obtained by sampling methods.
Otherwise, consider the case where $\eta =\{\lambda_1,\dots,\lambda_p\}$.
Then, the contribution to the ELBO is
\begin{align*}
  \E \log p(\lambda_1 & ,\dots,\lambda_p) + H\big[q(\lambda_1,\dots,\lambda_p) \big] \\
  &=  -\half[p]\log 2\pi - \half\log v_1\cdots v_k - \half \sum_{k=1}^p  \frac{\E(\lambda_k - m_k)^2}{v_k} \\
  &\phantom{==} + \half[p](1 + \log 2\pi) + \half\log \tilde v_1\cdots \tilde v_p \\
  &= \const + \half \sum_{k=1}^p \log \tilde v_k - \half \sum_{k=1}^p  \frac{\tilde v_k + \tilde\lambda_k^2 - 2\tilde\lambda_k m_k}{v_k}
\end{align*}

Notes:
\begin{enumerate}
  \item The priors on the $\lambda_k$'s are $\N(m_k,v_k)$, and  $q(\lambda_k)$ is the density of $\N(\tilde\lambda_k, v_{\lambda_k})$.
  \item When using improper priors $\lambda_k \propto \const$, then we need only consider the middle term involving the sums of $\log \tilde v_{\lambda_k}$.
\end{enumerate}

%\subsection{Terms involving distributions of \texorpdfstring{$\bPsi$}{$\Psi$}}
%
%The terms involving $\bPsi$ are $\E \log p(\bPsi) + H\big[q(\bPsi)\big]$.
%In the case of the full I-probit model, this becomes
%\begin{align*}
%  \const & + \half[g-m-1]\E \log \abs{\bPsi} - \half \E \tr(\bG\bPsi) 
%  + \log B - \half[\tilde g-m-1]\E \log \abs{\bPsi} + \half[\tilde g m] \\
%  &= \const  + \log B + \half[\tilde g m] + \half[g-\tilde g] \log \abs{\tilde\bPsi} - \half \tr(\bG\tilde\bPsi)
%%  + \frac{m+1}{2}\log\abs{\bG + \bG_1 + \bG_2} + \half m(m+1)\log 2 + \log \Gamma_p \left(\half[s] \right) - \half[s-m-1] \psi_m \left(\half[s] \right) + \half[sm]
%\end{align*}
%where $B = \half[\tilde g] \log \abs{\tilde G} + \half[\tilde g m]\log 2 + \log \Gamma_m(\tilde g / 2)$.
%In the case of the independent I-probit model, we have
%\begin{align*}
%  \const & + \sum_{j=1}^m \Big\{ (s_j - 1)\E \log \psi_j -  r_j \E\psi_j \Big\}
%  - \sum_{j=1}^m \log \tilde r_j \\
%  &= \const + \sum_{j=1}^m \Big\{ (s_j - 1)\log \tilde \psi_j   -  r_j \tilde\psi_j \Big\}
%  - \sum_{j=1}^m \log \tilde r_j 
%\end{align*}
%
%Notes:
%\begin{enumerate}
%  \item The priors on the $\bPsi$ is $\Wis(\bG,g)$, or if $\bPsi = \diag(\psi_1,\dots,\psi_m)$, then each $\psi_j\sim \Gamma(s_j,r_j)$. $q(\bPsi)$ is the density of $\Wis(\tilde\bG,\tilde g)$ or in the case of the independent model, each $q(\psi_j)$ is the density of $\Gamma(s_j,\tilde r_j)$.
%  \item Use the first order Taylor expansion about $\E\psi_j$ to approximate $\E \log \psi_j \approx \log \E \psi_j = \log \tilde \psi_j$, as per \citet{teh2007collapsed}.
%\end{enumerate}

\subsection{Terms involving distribution of \texorpdfstring{$\balpha$}{$\alpha$}}

For the intercepts, consider only
\begin{align*}
  \E \log p(\balpha) + H \big[q(\balpha) \big] 
  &=  \const  - \half \E \sum_{j=1}^m \frac{(\alpha_j - a_j)^2}{A_j}    + \half \log \tilde v_{\alpha_1}\cdots\tilde v_{\alpha_m} \\ 
  &= \const + \half \sum_{j=1}^m \log \tilde v_{\alpha_j} - \half \sum_{j=1}^m \frac{v_{\alpha_j} + \tilde\alpha_j^2  -2a_j \tilde\alpha_j}{A_j} 
\end{align*}

Notes:
\begin{enumerate}
  \item $p(\balpha)$ is $\prod_{j=1}^m \phi(\alpha_j|a_j,A_j)$, and $q(\balpha)$ $\prod_{j=1}^m \phi(\alpha_j|\tilde \alpha_j,\tilde v_{\alpha_j})$.
\end{enumerate}

\subsection{ELBO summarised}

In the example section of Chapter 5, we considered  only 1) the independent I-probit model; 2) fixed $\bSigma = \bI_m$; 3) only RKHS scale parameters to estimate; and 4) and improper priors on the hyperparameters.
In such situations, the ELBO expression is simply
\begin{align*}
  \cL 
  &= \const + \sum_{i=1}^n  \log C_i 
  - \half \sum_{j=1}^m \tr \big(\tilde\bV_{w_j}  + \tilde\bw_{\bigcdot j}\tilde\bw_{\bigcdot j}^\top \big) + \half \sum_{k=1}^p \log \tilde v_k.
\end{align*}

As a final remark, often times the ELBO is treated as a proxy for the (penalised) marginal likelihood of the model, in which case it must be noted that the ELBO as we had derived is correct up to a constant.
We find that keeping track of the constants is slightly tedious, and hence decided not to do so.
When comparing ELBOs of two or more models, the comparison is still valid as only differences between the ELBOs matter, in which case the constants would cancel out.
