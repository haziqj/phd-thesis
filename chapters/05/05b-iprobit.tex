\index{I-prior!categorical|see{I-probit}}
\index{I-probit}
We derive the I-probit model through a latent variable motivation.
It is convenient, as we did in \cref{sec:naiveclass} \colp{\mypageref{sec:naiveclass}}, to again think of the responses $y_i \in \{1,\dots,m\} $ as comprising of a binary vector $\by_{i \bigcdot} = (y_{i1},\dots,y_{im})^\top$, with a single `1' at the position corresponding to the value that $y_i$ takes. 
That is,
\[
  y_{ij} =
  \begin{cases}
    1 &\text{ if } y_i = j \\
    0 &\text{ if } y_i \neq j.
  \end{cases}
\]
With $y_i\iid\Cat(p_{i1},\dots,p_{im})$ for $i=1,\dots,n$, each $y_{ij}$ is distributed as Bernoulli with probability $p_{ij}$, $j=1,\dots,m$ according to the above formulation. 
Now, assume that, for each $y_{i1}, \dots, y_{im}$, there exists corresponding \emph{continuous, underlying, latent variables} $y_{i1}^*, \dots, y_{im}^*$ such that\index{latent variable}\index{latent propensity}
\begingroup
\setlength{\abovedisplayskip}{7pt}
\setlength{\belowdisplayskip}{12pt}
\begin{align}\label{eq:latentmodel}
  y_i =
  \begin{cases}
    1 &\text{ if } y_{i1}^* \geq y_{i2}^*, y_{i3}^*, \dots, y_{im}^* \\
    2 &\text{ if } y_{i2}^* \geq y_{i1}^*, y_{i3}^*, \dots, y_{im}^* \\
    \,\vdots \\
    m &\text{ if } y_{im}^* \geq y_{i2}^*, y_{i3}^*, \dots, y_{i\,m-1}^*. \\
  \end{cases}  
\end{align}
\endgroup
In other words, 
%$y_{ij} = [y_{ij}^* = \max_k y_{ik}^*]$.
$y_{ij} = \argmax_{k=1}^m y_{ik}^*$.
Such a formulation is common in economic choice models, and is rationalised by a utility-maximisation argument: an agent faced with a choice from a set of alternatives will choose the one which benefits them most.
In this sense, the $y_{ij}^*$'s represent individual $i$'s \emph{latent propensities} for choosing alternative $j$.

Instead of modelling the observed $y_{ij}$'s directly, we model instead, for observation $i=1,\dots,n$, the $m$ latent variables corresponding to each class or response category $j=1,\dots,m$ according to the regression problem
\begin{equation}\label{eq:multinomial-latent}
  \begin{gathered}
    y_{ij}^* = \alpha + \alpha_j + f_j(x_i) + \epsilon_{ij} \\
    (\epsilon_{i1}, \dots, \epsilon_{im})^\top  \iid \N_m(\bzero, \bPsi^{-1}), 
  \end{gathered}
\end{equation}
with $\alpha$ being the grand intercept, $\alpha_j$ group or class intercepts, and $f_j:\cX \to \bbR$ a regression function belonging to some RKKS $\cF$ of functions over the covariate set $\cX$ with reproducing kernel $h_\eta$.
We can see some semblance of this model with the one in \cref{eq:naiveclassiprior}, and ultimately the aim is to assign I-priors to the regression function of these latent variables, which we shall describe shortly.
For now, write $\bmu(x_i) \in \bbR^m$ whose $j$'th component is $\alpha + \alpha_j + f_j(x_i)$, and realise that each $\by_{i \bigcdot}^* = (y_{i1}^*, \dots, y_{im}^*)^\top$ has the distribution $\N_m(\bmu(x_i), \bPsi^{-1})$, conditional on the data $x_i$,  the intercepts $\alpha,\alpha_1,\dots,\alpha_m$, the evaluations of the functions at $x_i$ for each class $f_1(x_i), \dots, f_m(x_i)$, and the error covariance matrix $\bPsi^{-1}$.

\newcommand{\intset}{\{y_{ij}^* > y_{ik}^* \,|\, \forall k \neq j\}}
The probability $p_{ij}$ of observation $i$ belonging to class $j$ (or responding into category $j$) is then calculated as 
\begin{align}
  p_{ij} 
  &= \Prob(y_i = j) \nonumber \\
  &= \Prob\big(\intset\big) \nonumber \\
  &= \idotsint\displaylimits_{\intset} \phi(y_{i1}^*, \dots, y_{im}^*|\bmu(x_i), \bPsi^{-1}) \dint y_{i1}^* \cdots \dint y_{im}^*,\label{eq:pij}
\end{align}
where $\phi(\cdot|\bmu,\bSigma)$ is the density of the multivariate normal with mean $\bmu$ and variance $\bSigma$.
This is the probability that the normal random variable $\by_{i \bigcdot}^*$ belongs to the set $\cC_j := \intset$, which are cones in $\bbR^m$.
Since the union of these cones is the entire $m$-dimensional space of reals, the probabilities add up to one and hence they represent a proper probability mass function (pmf) for the classes.
For reference, we define our \emph{probit link function} $  g_j^{-1}(\cdot|\bPsi): \bbR^{m} \to [0,1]$ by the mapping\index{probit}
\begin{equation}\label{eq:probitlink}
  \bmu(x_i) \mapsto \int_{\cC_j} \phi(\by^*|\bmu(x_i),\bPsi^{-1}) \dint \by^*.
\end{equation}
While this does not have a closed-form expression and highlights one of the difficulties of working with probit models, the integral is by no means impossible to compute---see \cref{sec:mnint} for a note regarding this matter.

Now, we'll see how to specify an I-prior on the regression problem \cref{eq:multinomial-latent}.
In the naïve I-prior classification model \colp{\cref{sec:naiveclass}, \mypageref{sec:naiveclass}}, we wrote $f(x_i,j) = \alpha_j + f_j(x_i)$, and called for $f$ to belong to an ANOVA RKKS with kernel defined in \cref{eq:anovaclass}.
Instead of doing the same, we take a different approach.
Treat the $\alpha_j$'s in \cref{eq:multinomial-latent} as intercept parameters to estimate with the additional requirement that $\sum_{j=1}^m \alpha_j = 0$.
Further, let $\cF$ be a (centred) RKHS/RKKS of functions over $\cX$ with reproducing kernel $h_\eta$.
Now, consider putting an I-prior on the regression functions $f_j \in \cF$, $j=1\dots,m$, defined by
\[
  f_j(x_i) = f_0(x_i,j) + \sum_{k=1}^n h_\eta(x_i,x_k)w_{ik}
\]
with $\bw_{i \bigcdot} := (w_{i1},\dots,w_{im})^\top \iid \N(0,\bPsi)$.
This is similar to the naïve I-prior specification \cref{eq:naiveclassiprior}, except that the intercept  have been treated as parameters rather than accounting for them using an RKHS of functions (Pearson RKHS or identity kernel RKHS).
Importantly, the overall regression relationship still satisfies the ANOVA functional decomposition, because the $\alpha_j$'s sum to zero.
We find that this approach, rather than the I-prior specification described in the naïve classification, bodes well down the line computationally.

We call the multinomial probit regression model of \cref{eq:latentmodel} subject to \cref{eq:multinomial-latent} and I-priors on $f_j \in \cF$, the \emph{I-probit model}.
For completeness, this is stated again: for $i=1,\dots,n$, $y_i = \argmax_{k=1}^m y_{ik}^* \in \{1,\dots,m\}$, where, for $j=1,\dots,m$,
\begingroup
\setlength{\abovedisplayskip}{5pt}
\setlength{\belowdisplayskip}{5.5pt}
\begin{align}\label{eq:iprobitmod}
  \begin{gathered}
    y_{ij}^* = \alpha + \alpha_j + 
    \myoverbrace{f_0(x_i, j) + \sum_{k=1}^n h_\eta(x_i,x_k)w_{ik}}{f_j(x_i)}
    + \epsilon_{ij} \\
    \bepsilon_{i \bigcdot} := (\epsilon_{i1}, \dots, \epsilon_{im})^\top  \iid \N_m(\bzero, \bPsi^{-1}) \\
    \bw_{i \bigcdot} := (w_{i1},\dots,w_{im})^\top \iid \N_m(\bzero,\bPsi).
  \end{gathered}
\end{align}
\endgroup
The parameters of the I-probit model are denoted by $\theta = \{\alpha_1,\dots,\alpha_m,\eta,\bPsi \}$.
To establish notation, let 
\begin{itemize}
  \item $\bepsilon \in \bbR^{n \times m}$ denote the matrix containing $(i,j)$ entries $\epsilon_{ij}$, whose rows are $\bepsilon_{i \bigcdot}$, columns are $\bepsilon_{\bigcdot j}$, and is distributed $\bepsilon \sim \MN_{n,m}(\bzero, \bI_n,\bPsi^{-1})$;
  \item $\bw \in \bbR^{n \times m}$ denote the matrix containing $(i,j)$ entries $w_{ij}$, whose rows are $\bw_{i \bigcdot}$, columns are $\bw_{\bigcdot j}$, and is distributed $\bw \sim \MN_{n,m}(\bzero, \bI_n,\bPsi)$;
  \item $\bff,\bff_0 \in \bbR^{n \times m}$ denote the matrices containing $(i,j)$ entries $f_j(x_i)$ and $f_0(x_i,j)$ respectively, so that $\bff = \bff_0 + \bH_\eta\bw \sim \MN_{n,m}(\bone_n\bff_0^\top, \bH_\eta^2, \bPsi)$;
  \item $\balpha = (\alpha + \alpha_1,\dots,\alpha + \alpha_m)^\top\in\bbR^{m}$ be the vector of intercepts;
  \item $\bmu = \bone_n\balpha^\top + \bff$, whose $(i,j)$ entries are $\mu_{j}(x_i) = \alpha + \alpha_j + f_j(x_i)$; and
  \item $\by^* \in \bbR^{n \times m}$ denote the matrix containing $(i,j)$ entries $y_{ij}^*$, that is, $\by^* = \bmu + \bepsilon$, so $\by^*|\bw \sim \MN_{n,m}(\bmu = \bone_n\balpha^\top + \bH_\eta\bw, \bI_n, \bPsi^{-1})$ and $\vecc \by^* \sim \N_{nm}\big(\vecc (\bone_n\balpha^\top), \bPsi \otimes \bH_\eta^2 + \bPsi^{-1} \otimes \bI_n\big)$---note that the marginal distribution of $\by^*$ cannot be expressed as a matrix normal, except when $\bPsi=\bI_m$.
\end{itemize}\index{matrix normal distribution}
In the above, we have made use of matrix normal distributions, denoted by $\MN(\cdot,\cdot)$.
The definition and properties of matrix normal distributions can be found in \colp{\cref{apx:matrixnormal}, \mypageref{apx:matrixnormal}}.

Before proceeding with estimating the I-probit model \cref{eq:iprobitmod}, we lay out several standing assumptions:
\begin{enumerate}[label=A\arabic*,ref=A\arabic*]
  \setcounter{enumi}{3}
  \item \textbf{Centred responses}. Set $\alpha = 0$. \label{ass:A4}  
  \item \textbf{Zero prior mean}. Assume a zero prior mean $f_0(x) = 0$ for all $x\in\cX$. \label{ass:A5} 
  \item \textbf{Fixed error precision}. Assume $\bPsi$ is fixed. \label{ass:A6} 
\end{enumerate}
Assumption \ref{ass:A4} is a requirement for identifiability, while \ref{ass:A5} is motivated by a similar argument to assumption \ref{ass:A2} in the normal I-prior model.
While estimation of $\bPsi$ would add flexibility to the model, several computational issues were not able to be resolved within the time limitations of completing this project (see \cref{sec:difficultPsi}).
