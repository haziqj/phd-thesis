It is convenient, as we did in the previous subsection, to again think of the responses $y_i \in \{1,\dots,m\} = \cM$ as comprising of a binary vector $(y_{i1},\dots,y_{im})$, with a single `1' at the position corresponding to the value that $y_i$ takes. 
%That is, $y_i = (y_{i1}, \dots, y_{im})$ with
%\[
%  y_{ik} =
%  \begin{cases}
%    1 &\text{ if } y_i = k \\
%    0 &\text{ if } y_i \neq k.
%  \end{cases}
%\]
In this formulation, each $y_{ij}$ is distributed as Bernoulli with probability $p_{ij}$. Now, assume that, for each $y_i = (y_{i1}, \dots, y_{im})$, there exists corresponding \emph{continuous, underlying, latent variables} $y_{i1}^*, \dots, y_{im}^*$ such that
\begin{align}\label{eq:latentmodel}
  y_i =
  \begin{cases}
    1 &\text{ if } y_{i1}^* \geq y_{i2}^*, y_{i3}^*, \dots, y_{im}^* \\
    2 &\text{ if } y_{i2}^* \geq y_{i1}^*, y_{i3}^*, \dots, y_{im}^* \\
    \,\vdots \\
    m &\text{ if } y_{im}^* \geq y_{i2}^*, y_{i3}^*, \dots, y_{i\,m-1}^*. \\
  \end{cases}  
\end{align}
In other words, 
%$y_{ij} = [y_{ij}^* = \max_k y_{ik}^*]$.
$y_{ij} = \argmax_{k=1}^m y_{ik}^*$.
Such a formulation is common in economic choice models, and is rationalised by a utility-maximisation argument: an agent faced with a choice from a set of alternatives will choose the one which benefits them most.
In this sense, the $y_{ij}^*$'s represent individual $i$'s \emph{latent propensities} for choosing alternative $j$.

Instead of modelling the observed $y_{ij}$'s directly, we model instead the $n$ latent variables in each class $j=1,\dots,m$ according to the regression problem
\begin{equation}\label{eq:multinomial-latent}
  \begin{gathered}
    y_{ij}^* = \alpha + \alpha_j + f_j(x_i) + \epsilon_{ij} \\
    (\epsilon_{i1}, \dots, \epsilon_{im})^\top  \iid \N_m(\bzero, \bPsi^{-1}). 
  \end{gathered}
\end{equation}
%with $\alpha_j$ being an intercept, and $f_j:\cX \to \bbR$ a regression function belonging to some RKHS/RKKS of functions $\cF$.
%having the reproducing kernel $h_{\eta_j}: \cX \times \cX \to \bbR$. 
We can see some semblance of this model with the one in \cref{eq:naiveclassiprior}, and ultimately the aim is to assign I-priors to the regression function of these latent variables, and we will describe this shortly.
For now, write $\bmu(x_i) \in \bbR^m$ whose $j$'th component is $\alpha + \alpha_j + f_j(x_i)$, and realise that each $(y_{i1}^*, \dots, y_{im}^*)^\top$ has the distribution $\N_m(\bmu(x_i), \bPsi^{-1})$, conditional on the data $x_i$,  the intercepts $\alpha,\alpha_1,\dots,\alpha_m$, the evaluations of the functions at $x_i$ for each class $f_1(x_i), \dots, f_m(x_i)$, and the error covariance matrix $\bPsi^{-1}$.

\newcommand{\intset}{\{y_{ij}^* > y_{ik}^* \,|\, \forall k \neq j\}}
The probability of belonging to class $j$ for observation $i$, i.e. $p_{ij}$ is calculated as
\begin{align}
  p_{ij} 
  &= \Prob(y_i = j) \nonumber \\
  &= \Prob\big(\intset\big) \nonumber \\
  &= \idotsint\displaylimits_{\intset} \phi(y_{i1}^*, \dots, y_{im}^*|\bmu(x_i), \bPsi^{-1}) \dint y_{i1}^* \cdots \dint y_{im}^*,\label{eq:pij}
%  =: g_{j}^{-1} ( \bmu(x_i) | \bPsi ),
\end{align}
where $\phi(\cdot|\mu,\Sigma)$ is the density of the multivariate normal with mean $\mu$ and variance $\Sigma$.
This is the probability that the normal random variable $\by_i^*$ belongs to the set $\intset$, which are cones in $\bbR^m$.
Since the union of these cones is the entire $m$-dimensional space of reals, the probabilities add up to one and hence they represent a proper probability mass function of the classes.
%Upon knowing all values for $\alpha_j$, $f_j(x_i)$, and $\bSigma$, one is able to calculate $p_{ij}$ through the relationship \eqref{eq:pij}, which we denote as $g^{-1}$.
While this does not have a closed-form expression and highlights one of the difficulties of working with probit models, the integral is by no means impossible to compute---see \cref{misc:mnint} for a note regarding this matter.


Now we'll see how to specify an I-prior on the regression problem \cref{eq:multinomial-latent}.
In the naïve I-prior model, we wrote $f(x_i,j) = \alpha_j + f_j(x_i)$, and specified for $f$ to belong to an ANOVA RKKS with kernel defined in \cref{eq:anovaclass}.
Instead of doing the same, we take a different approach.
Treat the $\alpha_j$'s in \cref{eq:multinomial-latent} as intercept parameters to estimate with the additional requirement that $\sum_{j=1}^m \alpha_j = 0$.
Further, let $\cF$ be a (centred) RKHS/RKKS of functions over $\cX$ with reproducing kernel $h_\eta$.
Now, consider putting an I-prior on the regression functions $f_j \in \cF$, $j=1\dots,m$, defined by
\[
  f_j(x_i) = f_0(x_i) + \sum_{k=1}^n h_\eta(x_i,x_k)w_{ik}
\]
with $\bw_i := (w_{i1},\dots,w_{im})^\top \iid \N(0,\bPsi)$.
This is similar to the naïve I-prior specification \cref{eq:naiveclassiprior}, except that the intercept  have been treated as parameters rather than accounting for them using an RKHS of constant functions.
In particular, the overall regression relationship still satisfies the ANOVA functional decomposition.
We find that this method bodes well down the line computationally.

We call the multinomial probit regression model of \cref{eq:latentmodel} subject to \cref{eq:multinomial-latent} and I-priors on $f_j \in \cF$, the \emph{I-probit model}.
For completeness, this is stated again: for $i=1,\dots,n$, $y_i = \argmax_{k=1}^m y_{ik}^* \in \{1,\dots,m\}$, where, for $j=1,\dots,m$,
\begin{align}\label{eq:iprobitmod}
  \begin{gathered}
    y_{ij}^* = \alpha + \alpha_j + 
    \greyoverbrace{f_0(x_i) + \sum_{k=1}^n h_\eta(x_i,x_k)w_{ik}}{f_j(x_i)}
    + \epsilon_{ij} \\
    (\epsilon_{i1}, \dots, \epsilon_{im})^\top  \iid \N_m(\bzero, \bPsi^{-1}) \\
    (w_{i1},\dots,w_{im})^\top \iid \N_m(\bzero,\bPsi).
  \end{gathered}
\end{align}
The parameters of the I-probit model are denoted by $\theta = \{\alpha_1,\dots,\alpha_m,\eta,\bPsi \}$.
To establish notation, let 
\begin{itemize}
  \item $\bepsilon \in \bbR^{n \times m}$ denote the matrix containing $(i,j)$ entries $\epsilon_{ij}$, whose rows are $\bepsilon_{i \bigcdot}$ and columns are $\bepsilon_{\bigcdot j}$. Its distribution is $\bepsilon \sim \MN_{n,m}(\bzero, \bI_n,\bPsi^{-1})$;
  \item $\bw \in \bbR^{n \times m}$ denote the matrix containing $(i,j)$ entries $w_{ij}$, whose rows are $\bw_{i \bigcdot}$ and columns are $\bw_{\bigcdot j}$. Its distribution is $\bw \sim \MN_{n,m}(\bzero, \bI_n,\bPsi)$;
  \item $\bff \in \bbR^{n \times m}$ denote the matrix containing $(i,j)$ entries $f_j(x_i)$, and $\bff_0$ a vector equal to $\big(f_0(x_1),\dots,f_0(x_n) \big)^\top$.  We then have $\bff = \bone_n\bff_0^\top + \bH_\eta\bw \sim \MN_{n,m}(\bone_n\bff_0^\top, \bH_\eta^2, \bPsi)$;
  \item $\balpha = (\alpha + \alpha_1,\dots,\alpha + \alpha_m)^\top\in\bbR^{m}$ be the vector of intercepts;
  \item $\bmu = \bone_n\balpha^\top + \bff$, whose $(i,j)$ entries are $\mu_{j}(x_i) = \alpha + \alpha_j + f_j(x_i)$; and
  \item $\by^* \in \bbR^{n \times m}$ denote the matrix containing $(i,j)$ entries $y_{ij}^*$. That is, $\by^* = \bmu + \bepsilon$, so $\by^*|\bw \sim \MN_{n,m}(\bmu = \bone_n\balpha^\top + \bH_\eta\bw, \bI_n, \bPsi^{-1})$ and $\vecc \by^* \sim \N_{nm}\big(\vecc (\bone_n\balpha^\top), (\bPsi \otimes \bH_\eta^2) + (\bPsi^{-1} \otimes \bI_n)\big)$. The marginal distribution of $\by^*$ cannot be written as a matrix normal, except when $\bPsi=\bI_m$. 
\end{itemize}

Before proceeding with estimating the I-probit model \cref{eq:iprobitmod}, we lay out several standing assumptions:
\begin{enumerate}[label=A\arabic*,ref=A\arabic*]
  \setcounter{enumi}{3}
  \item \textbf{Centred responses}. Set $\alpha = 0$. \label{ass:A4}  
  \item \textbf{Zero prior mean}. Assume a zero prior mean $f_0(x) = 0$ for all $x\in\cX$. \label{ass:A5} 
  \item \textbf{Fixed error precision}. Assume $\bPsi$ is fixed. \label{ass:A6} 
\end{enumerate}
Assumption \ref{ass:A4} is a requirement for identifiability.
Assumption \ref{ass:A5} is motivated by a similar argument to assumption \ref{ass:A2} in the normal I-prior model.
As for assumption \ref{ass:A6}, we do not consider estimation of the error precision in this thesis mainly due to time limitations. 
More on this in \hltodo{Section X}.

