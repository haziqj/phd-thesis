It is convenient, as we did in the previous subsection, to again think of the responses $y_i \in \{1,\dots,m\} = \cM$ as comprising of a binary vector $(y_{i1},\dots,y_{im})$, with a single `1' at the position corresponding to the value that $y_i$ takes. 
%That is, $y_i = (y_{i1}, \dots, y_{im})$ with
%\[
%  y_{ik} =
%  \begin{cases}
%    1 &\text{ if } y_i = k \\
%    0 &\text{ if } y_i \neq k.
%  \end{cases}
%\]
In this formulation, each $y_{ij}$ is distributed as Bernoulli with probability $p_{ij}$. Now, assume that, for each $y_i = (y_{i1}, \dots, y_{im})$, there exists corresponding \emph{continuous, underlying, latent variables} $y_{i1}^*, \dots, y_{im}^*$ such that
\begin{align}\label{eq:latentmodel}
  y_i =
  \begin{cases}
    1 &\text{ if } y_{i1}^* \geq y_{i2}^*, y_{i3}^*, \dots, y_{im}^* \\
    2 &\text{ if } y_{i2}^* \geq y_{i1}^*, y_{i3}^*, \dots, y_{im}^* \\
    \,\vdots \\
    m &\text{ if } y_{im}^* \geq y_{i2}^*, y_{i3}^*, \dots, y_{i\,m-1}^*. \\
  \end{cases}  
\end{align}
In other words, 
%$y_{ij} = [y_{ij}^* = \max_k y_{ik}^*]$.
$y_{ij} = \argmax_{k=1}^m y_{ik}^*$.
Such a formulation is common in economic choice models, and is rationalised by a utility-maximisation argument: an agent faced with a choice from a set of alternatives will choose the one which benefits them most.

Instead of modelling the observed $y_{ij}$'s directly, we model instead the $n$ latent variables in each class $j=1,\dots,m$ according to the regression problem
\begin{equation}\label{eq:multinomial-latent}
  \begin{gathered}
    y_{ij}^* = \alpha_j + f_j(x_i) + \epsilon_{ij} \\
    \bepsilon_{i} = (\epsilon_{i1}, \dots, \epsilon_{im})^\top  \iid \N_m(\bzero, \bPsi^{-1}). 
  \end{gathered}
\end{equation}
%with $\alpha_j$ being an intercept, and $f_j:\cX \to \bbR$ a regression function belonging to some RKHS/RKKS of functions $\cF$.
%having the reproducing kernel $h_{\eta_j}: \cX \times \cX \to \bbR$. 
We can see some semblance of this model with the one in \cref{eq:naiveclassiprior}, and ultimately the aim is to assign I-priors to the regression function of these latent variables, and we will describe this shortly.
For now, realise that each $\by_i^* := (y_{i1}^*, \dots, y_{im}^*)^\top$ has the distribution $\N_m(\balpha + \bff(x_i), \bPsi^{-1})$, conditional on the data $x_i$,  the intercepts $\balpha = (\alpha_1,\dots,\alpha_m)^\top$, the evaluations of the functions at $x_i$ for each class $\bff(x_i) = \big(f_1(x_i), \dots, f_m(x_i)\big)^\top$, and the error covariance matrix $\bPsi^{-1}$.

\newcommand{\intset}{\{y_{ij}^* > y_{ik}^* \,|\, \forall k \neq j\}}
The probability of belonging to class $j$ for observation $i$, i.e. $p_{ij}$, is calculated as
\begin{align}
  p_{ij} 
  &= \Prob(y_i = j) \nonumber \\
  &= \Prob\big(\intset\big) \nonumber \\
  &= \int\displaylimits_{\intset} \phi(\by_i^*|\balpha + \bff(x_i), \bPsi^{-1}) \dint\by^* \label{eq:pij},
  %&=: g_{\bSigma}^{-1} \big( \alpha_j + f_j(x_i) \big)_{j=1}^m \nonumber
\end{align}
where $\phi(\cdot|\mu,\Sigma)$ is the density of the multivariate normal with mean $\mu$ and variance $\Sigma$.
This is the probability that the normal random variable $\by_i^*$ belongs to the set $\intset$, which are cones in $\bbR^m$.
Since the union of these cones is the entire $m$-dimensional space of reals, the probabilities add up to one and hence they represent a proper probability mass function of the classes.
%Upon knowing all values for $\alpha_j$, $f_j(x_i)$, and $\bSigma$, one is able to calculate $p_{ij}$ through the relationship \eqref{eq:pij}, which we denote as $g^{-1}$.
While this does not have a closed-form expression and highlights one of the difficulties of working with probit models, the integral is by no means impossible to compute---see \cref{misc:mnint} for a note regarding this matter.

Note that the dimension of the integral \eqref{eq:pij} is $m-1$, since the $j$'th coordinates is fixed relative to the others.
Alternatively, we could have specified the model in terms of \emph{relative differences} of the latent variables.
Choosing the first category as the reference category, define new random variables $z_{ij} = y_{ij}^* - y_{i1}^*$, for $j = 2,\dots,m-1$. 
The model \cref{eq:latentmodel} is equivalently represented by
\begin{equation}
  y_i = 
  \begin{cases}
    1 & \text{if } \max (z_{i2},\dots,z_{im}) < 0 \\
    j & \text{if } \max (z_{i2},\dots,z_{im}) = z_{ij} \geq 0.
  \end{cases}
\end{equation} 
Write $\bz_i = (z_{i2},\dots,z_{im})^\top \in \bbR^{m-1}$.
Then $\bz_i = \bQ\by^*_i$, where $\bQ \in \bbR^{(m-1)\times m}$  is the $(m-1)$ identity matrix pre-augmented with a column vector of minus ones.
We have that $\bz_i \iid \N_{m-1}\big(\bQ(\balpha + \bff(x_i)), \bQ\bPsi^{-1}\bQ^\top)\big)$.
Thus, the class probabilities for $j=2,\dots,m$ are
\begin{align}
  p_{ij} = 
  \int\displaylimits_{\{z_{ik} < 0 \,|\, \forall k \neq j \}} \ind(z_{ij} \geq 0) \,\phi(\bz_i) \dint\bz_i \label{eq:pij2},
\end{align}
with $\phi(\bz_i)$ representing the $(m-1)$-variate normal density for $\bz_i$.
The class probability $p_{i1}$ is simply
\[
  p_{i1} = \int\displaylimits_{\{ z_{ik} <0 \}}  \phi(\bz_i) \dint\bz_i = 1 - \sum_{k\neq 1} p_{ik}.
\]
From this representation of the model, with $m=2$ (binary outcomes) we see that
\[
  p_{i1} = 
%  \int \ind(z_{i2} < 0) \phi(z_{i2}) \dint z_{i2} = 
  \Phi \left( \frac{z_{i2} - \mu}{\sigma} \right)
  \hspace{0.5cm}\text{and}\hspace{0.5cm}
  p_{i2} = 
%  \int \ind(z_{i2} \geq 0) \phi(z_{i2}) \dint z_{i2} = 
  1 - \Phi \left( \frac{z_{i2} - \mu}{\sigma} \right),
\]
where $\Phi(\cdot)$ is the CDF of the standard normal univariate distribution, and $\mu$ and $\sigma$ are the mean and standard deviation of the random variable $z_{i2}$.

%\begin{figure}[t]
%  \centering
%  \begin{tikzpicture}[scale=1.1, transform shape]
%    \tikzstyle{main}=[circle, minimum size=10mm, thick, draw=black!80, node distance=16mm]
%    \tikzstyle{connect}=[-latex, thick]
%    \tikzstyle{box}=[rectangle, draw=black!100]
%      \node[main, draw=black!0] (blank) [xshift=-0.55cm] {};  % pushes image to right slightly
%      \node[main, fill=black!10] (H) [] {$x_i$};
%      \node[main] (Sigma) [below=of H, yshift=-1.2cm, xshift=0.6cm] {$\bSigma$};
%      \node[main, double, double distance=0.6mm] (f) [right=of H, xshift=0.5cm] {$f_{ij}$};
%      \node[main, double, double distance=0.6mm] (ystar) [right=of f, xshift=0cm] {$y_{ij}^*$};
%      \node[main, double, double distance=0.6mm] (pij) [right=of ystar, xshift=0cm] {$p_{ij}$};
%      \node[main] (lambda) [above=of f, xshift=0cm, yshift=-0.3cm] {$\eta_j$};        
%      \node[main] (alpha) [above=of ystar, xshift=0cm, yshift=-0.3cm] {$\alpha_j$};  
%      \node[main, fill = black!10] (y) [right=of pij, xshift=0.2cm] {$y_{i}$};
%      \node[main] (w) [below=of f, yshift=0.3cm] {$w_{ij}$};  
%      \path (alpha) edge [connect] (ystar)
%            (lambda) edge [connect] (f)
%            (H) edge [connect] node [above] {$h \ \ $} (f)
%    		(f) edge [connect] (ystar)
%    		(ystar) edge [connect] node [above] {$g^{-1}$}  (pij)
%            (pij) edge [connect] (y)
%            (Sigma) edge [connect] (w)
%    		(w) edge [connect] (f);
%      \node[rectangle, draw=black!100, fit={($(H.north west) + (0.2,0cm)$) ($(y.north east) + (-0.2,0.4cm)$) (w)}] {}; 
%      \node[rectangle, fit= (w) (y), label=below right:{$i=1,\dots,n$}, xshift=0.95cm, yshift=0.5cm] {};  % the label
%      \node[rectangle, draw=black!100, fit={(lambda) ($(pij.north east) + (0.5cm,0.7cm)$) ($(w.south west) + (-0.5,-0.7cm)$)}] {}; 
%      \node[rectangle, fit={(f) ($(ystar.north east) + (0.5cm,0.7cm)$) ($(w.south west) + (-0.5,-0.7cm)$)}, label=below right:{$j=1,\dots,m$}, xshift=0.63cm, yshift=0.37cm] {}; 
%    \end{tikzpicture}
%    \caption{A DAG of the probit I-prior  model. Observed nodes are shaded, while double-lined nodes represented known or calculable quantities. There are at most $m-1$ sets of intercept ($\alpha_j$) and RKHS parameters ($\eta_j$)  to estimate due to identifiability. Depending on the specification of $\bSigma$, this may need to be estimated too.}
%\end{figure}

Now we'll see how to specify an I-prior on the regression problem \cref{eq:multinomial-latent}.
In the naïve I-prior model, we wrote $f(x_i,j) = \alpha_j + f_j(x_i)$, and specified for $f$ to belong to an ANOVA RKKS with kernel defined in \cref{eq:anovaclass}.
Instead of doing the same, we take a different approach.
Treat the $\alpha_j$'s in \cref{eq:multinomial-latent} as intercept parameters to estimate with the additional requirement that $\sum_{j=1}^m \alpha_j = 0$.
Further, let $\cF$ be a (centred) RKHS/RKKS of functions over $\cX$ with reproducing kernel $h_\eta$.
Now, consider putting an I-prior on the regression functions $f_j \in \cF$, $j=1\dots,m$, defined by
\[
  f_j(x_i) = \sum_{k=1}^n h_\eta(x_i,x_k)w_{ik}
\]
with $\bw_i := (w_{i1},\dots,w_{im})^\top \iid \N(0,\bPsi)$.
This is similar to the naïve I-prior specification \cref{eq:naiveclassiprior}, except that the intercept  have been treated as parameters rather than accounting for them using an RKHS of constant functions.
In particular, the overall regression relationship still satisfies the ANOVA functional decomposition.
We find that this method bodes well down the line computationally.

We call the multinomial probit regression model of \cref{eq:latentmodel} subject to \cref{eq:multinomial-latent} and I-priors on $f_j \in \cF$, the \emph{I-probit model}.
For completeness, this is stated again: for $i=1,\dots,n$, $y_i = \argmax_{k=1}^m y_{ik}^* \in \{1,\dots,m\}$, where, for $j=1,\dots,m$,
\begin{align}
  \begin{gathered}
    y_{ij}^* = \alpha_j + 
    \greyoverbrace{\sum_{k=1}^n h_\eta(x_i,x_k)w_{ik}}{f_j(x_i)}
    + \epsilon_{ij} \\
    \bepsilon_{i} = (\epsilon_{i1}, \dots, \epsilon_{im})^\top  \iid \N_m(\bzero, \bPsi^{-1}) \\
    \bw_i := (w_{i1},\dots,w_{im})^\top \iid \N(0,\bPsi).
  \end{gathered}
\end{align}
The parameters of the I-probit model are denoted by $\theta = \{\alpha_1,\dots,\alpha_m,\eta,\bPsi \}$.
Let $\by^* \in \bbR^{n \times m}$ denote the matrix containing $(i,j)$ entries $y_{ij}^*$.
Using the results in Chapter 4, the marginal distribution of the latent variables is
\[
  \vecc \by^* \sim \N_{nm}\big(\balpha, (\bPsi \otimes \bH_\eta^2) + (\bPsi^{-1} \otimes \bI_n)\big).
\]

\subsection{IIA}

In decision theory, the independence axiom states that an agent's choice between a set of alternatives should not be affected by the introduction or elimination of a (new) choice option.
The probit model is suitable for modelling multinomial data where the independence axiom, which is also known as the \emph{independence of irrelevant alternatives} (IIA) assumption, is not desired. 
Such cases arise frequently in economics and social science, and the famous Red-Bus-Blue-Bus example is often used to illustrate IIA.
Suppose commuters face the decision between taking cars and red busses. 
The addition of blue busses to commuters' choice should in theory be more likely chosen by those who prefer taking the bus over cars.
That is, assuming commuters are indifferent about the colour of the bus, commuters who are predisposed to taking the red bus would see the blue bus as an identical alternative.
 Yet, if IIA is imposed, then the three choices are distinct, and the fact that red and blue busses are substitutable is ignored.

In the I-probit model, the choice dependency is controlled by the error precision matrix $\bPsi$.
Specifically, the off-diagonal elements $\bPsi_{jk}$ capture the correlation between choices $j$ and $k$.
Allowing all $m(m+1)/2$ covariance elements of $\bPsi$ leads to the \emph{full I-probit model}, and would not automatically assume an IIA position.

%Although crucial in choice models, it is not so much necessary for classification tasks when the alternatives under consideration are distinctly different.
%In such cases, one may choose to abandon the IIA

While it is an advantage to be able to model the correlations across choices (unlike in logistic models), it would be a major simplification algorithmically to consider all covariances in $\bPsi$ to be zero.
This would trigger the IIA assumption in the I-probit model.
There are applications where the IIA assumption would not adversely affect the analysis, such as when all the choices are mutually exclusive and exhaustive.
In these situations, it would be beneficial to reduce the I-probit model to a simpler version by assuming $\bPsi = \diag(\psi_1,\dots,\psi_m)$.

The independence assumption causes the distribution of the latent variables to be $y_{ij}^* \sim \N(\alpha_j + f_j(x_i), \sigma_j^2)$ for $j=1,\dots,m$.
As a continuation of line \eqref{eq:pij}, we can show the class probability $p_{ij}$ to be
\begin{align*}
  p_{ij} 
  &= \idotsint\displaylimits_{\{y_{ij}^* > y_{ik}^* | \forall k \neq j\}} 
  \prod_{k=1}^m \Big\{ p(y_{ik}^*|\alpha_j + f_k(x_i), \sigma_j^2) \d y_k^* \Big\} \\
  &= \int \mathop{\prod_{k=1}^m}_{k\neq j} 
  \Phi \left( \frac{y_{ij}^* - \alpha_k - f_{ik}}{\sigma_k} \right) \cdot
  \frac{1}{\sigma_j} \phi \left( \frac{y_{ij}^* - \alpha_j - f_{ij}}{\sigma_j} \right)  \d y_{ij}^* \\
  &= \E_Z \Bigg[ \mathop{\prod_{k=1}^m}_{k\neq j} 
  \Phi \left(\frac{\sigma_j}{\sigma_k} Z + \frac{\alpha_j + f_{ij} - \alpha_k - f_{ik}}{\sigma_k} \right) \Bigg] \nonumber
\end{align*}
where $Z\sim\N(0,1)$, and $\phi(\cdot)$ and $\Phi(\cdot)$ are its PDF and CDF respectively.
%In the binary case where $m=2$, we set $\sigma_1^2 = 2$ and fix $f_{i2} = 0$, and we get that 
%\[
%  p_{i1} = 1 - \Phi(\alpha_1 + f_{i1}) \ \text{ and } \ p_{i2} =  \Phi(\alpha_1 + f_{i1}),
%\]
%which clearly shows the probit relationship between the class probabilities and the latent regression.
The proof of this fact is included in the Appendix.
With the exception of the binary case, these probabilities still do not have a closed-form expression (per se) and numerical methods are required to calculate them.
In this simplified version of the I-probit model, the integral is unidimensional and involves the Gaussian PDF, and this can be efficiently obtained using quadrature methods.


%\begin{table}[]
%\centering
%\caption{My caption}
%\label{my-label}
%\begin{tabular}{@{}lp{5cm}p{5cm}@{}}
%\toprule
%                    & Naïve model                                                                              & I-probit model                                            \\ \midrule
%Grand intercept     & Fixed at $\alpha = 1/m$, equivalently $y_{ij}\mapsto y_{ij} - 1/m$ and set $\alpha = 0$. & $\alpha=0$ (otherwise, model is not location identified). \\
%Intercepts          & $\alpha_j \in \cF_\cM$, a centred RKHS with kernel $a$                                                  & $\alpha_j\in\bbR$, $\sum_{j=1}^m \alpha_j = 0$            \\
%Interaction effects & $f_j \in \cF_\cM \otimes \cF_\cX$, tensor product interaction space with kernel $ah_\eta$                     & $f_j\in\cF$ with kernel $h_\eta$                                                \\ \bottomrule
%\end{tabular}
%\end{table}




%A rearrangement of the error terms is beneficial for the process of deriving the required I-priors.
%Recall that since each $\bepsilon_i \iid \N_m(\bzero,\bSigma)$ then concatenating all of the error vectors result in $\bepsilon = (\bepsilon_1^\top,\dots,\bepsilon_n^\top)^\top$, and this has distribution $\N_{nm}(\bzero, \bOmega)$, where $\bOmega = \diag(\bSigma,\dots,\bSigma)$.
%By rearranging the entries of $\bepsilon$ according to class (instead of by observation) and defining $\bepsilon_j'=(\epsilon_{1j},\dots,\epsilon_{nj})^\top$ for $j=1,\dots,m$, we observe that
%\[
%  \bepsilon' = 
%  \begin{pmatrix}
%    \bepsilon_1' \\
%    \vdots \\
%    \bepsilon_m'
%  \end{pmatrix}
%  \sim \N_{nm}(\bzero,\bOmega')
%\]
%where $\bOmega'$ contains the appropriately rearranged entries of $\bOmega$. 
%For illustration, an example of this rearrangement is shown in Figure \ref{fig:covrearrange} for $m = 2$ and $n=5$.
%The subtle difference is that whereas $\bOmega$ is a block diagonal matrix, $\bOmega'$ is comprised of $m^2$ equally sized $n \times n$ partitions of diagonal matrices.
%However, the number of unique elements in $\bOmega$ and $\bOmega'$ is the same as in the $m \times m$ matrix $\bSigma$, i.e. $m(m+1)/2$.
%We can also express these matrices in terms of Kronecker products $\bOmega = \bI_n \otimes \bSigma$ and $\bOmega = \bSigma \otimes \bI_n$.
%
%Denote $f_{ij} = f_j(x_i)$ as the evaluation of the function $f_j(\cdot)$ at $x_i$, and also $\bff_j = (f_{1j}, \dots, f_{nj})^\top$ as the vector containing all $n$ evaluations pertaining to the $j$th class.
%Concatenate all of the $\bff_j$'s into one long vector of length $nm$: $\bff = (\bff_1^\top, \dots, \bff_m^\top)^\top$.
%The I-prior is $\bff \sim \N_{nm}(\bzero,\bV_f)$, where $\bzero$ is an vector of length $nm$ containing zeroes and the covariance matrix $\bV_f$ is has the block matrix structure
%\begin{align}\label{eq:vf}
%  \bV_f = \begin{pmatrix}
%    \bV_{11} & \bV_{12} & \dots  & \bV_{1m} \\
%    \bV_{21} & \bV_{22} & \dots  & \bV_{2m} \\
%    \vdots   & \vdots   & \ddots & \vdots \\
%    \bV_{m1} & \bV_{m2} & \dots  & \bV_{mm} \\
%  \end{pmatrix},
%\end{align}
%and each block entry above is given by
%\begin{align}\label{eq:fisherinformation}
%  \bV_{jk}(r,s) = \cI\big( f_j(x_r), f_k(x_s) \big) = \bPsi_{jk} \sum_{r'=1}^n \sum_{s'=1}^n  h_{\eta_j}(x_r, x_{r'}) h_{\eta_k}(x_{s'}, x_s),
%\end{align}
%where $\bPsi = \bSigma^{-1}$. 
%Equation \eqref{eq:fisherinformation} gives the Fisher information for the regression functions $f_j$ and $f_k$, $j,k\in\{1,\dots,m\}$, evaluated at two points $x_r$ and $x_s$, $r,s\in\{1,\dots,n\}$---see \cite{bergsma2017} for details.
%
%One may also write $\bff = \bG\bw$ where $\bw \sim \N_{nm}(\bzero, \bOmega'^{-1})$, $\bG = \diag(\bH_{\eta_1},\dots,\bH_{\eta_m})$, and $\bH_{\eta_j}$ are the $n\times n$ kernel matrices with $(r,s)$ entries $h_{\eta_j}(x_r,x_s)$.
%Realise that $\bOmega^{-1}$ is obtainable by repeating $\bSigma^{-1}$ diagonally, and that $\bOmega'^{-1}$ is then obtained using a similar rearrangement of its rows and columns to that of $\bOmega$.
%Substituting equation \eqref{eq:fisherinformation} into the model \eqref{eq:multinomial-latent}, we get that
%\begin{align}
%  \begin{gathered}\label{eq:ipriorw}
%    y_{ij}^* = \alpha_j + \sum_{r=1}^n h_{\eta_j}(x_i,x_r)w_{rj} + \epsilon_{ij} \\
%    (w_{i1},\dots,w_{im}) \iid \N_m(\bzero, \bSigma^{-1}) \\
%    (\epsilon_{i1},\dots,\epsilon_{im}) \iid \N_m(\bzero, \bSigma)
%  \end{gathered}
%\end{align}
%This formulation will be useful in the variational algorithm later on.
%



