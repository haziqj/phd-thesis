\documentclass[a4paper,showframe,11pt]{report}
\usepackage{standalone}
\standalonetrue
\ifstandalone
  \usepackage{../../haziq_thesis}  
  \usepackage{../../haziq_maths}
  \usepackage{../../haziq_glossary}
  \usepackage{../../knitr}
  \addbibresource{../../bib/haziq.bib}
  \externaldocument{../01/.texpadtmp/chapter1}
  \externaldocument{../02/.texpadtmp/chapter2}
  \externaldocument{../03/.texpadtmp/chapter3}
  \externaldocument{../04/.texpadtmp/chapter4}
\fi

\begin{document}
\hChapterStandalone[5]{I-priors for categorical responses}

In a regression setting, consider polytomous response variables $y_1,\dots,y_n$, where each $y_i$ takes on exactly one of the values $\{1,\dots,m\}$ from a set of $m$ possible choices.
Modelling categorical response variables is of profound interest in statistics, econometrics and machine learning, with applications aplenty. 
In the social sciences, categorical variables often arise from survey responses, and one may be interested in studying correlations between  explanatory variables and the categorical response of interest.
Economists are often interested in discrete choice models to explain and predict choices between several alternatives, such as consumer choice of goods or modes of transport.
In this age of big data, machine learning algorithms are used for classification of observations based on what is usually a large set of variables or features.

As an extension to the I-prior methodology, we propose a flexible modelling framework suitable for regression of categorical response variables.

In the spirit of generalised linear models \citep{mccullagh1989}, we relate the class probabilities of the observations to the I-prior regression model via a link function.
Perhaps though, it is more intuitive to view it as machine learners do: Since the regression function is ranged on the entire real line, it is necessary to ``squash'' it through some sigmoid function to conform it to the interval $[0,1]$ suitable for probability measures.
As in GLMs, the $y_i$'s are assumed to follow an\hltodo[Exponential family for $y$ not really necessary, it just follows nicely from the latent variable motivation.]{exponential family distribution}, and in this case, the categorical distribution.
We denote this by
\[
  y_i \sim \Cat(p_{i1},\dots,p_{im}),
\]
with the class probabilities satisfying $p_{ij} \geq 0, \forall j=1,\dots,m$ and $\sum_{j=1}^m p_{ij} = 1$. 
The probability mass function (PMF) of $y_i$ is given by
\begin{align}\label{eq:catdist}
  p(y_i) = p_{i1}^{[y_i = 1]} \cdots p_{im}^{[y_i = m]}
\end{align}
where the notation $[\cdot]$ refers to the Iverson bracket\footnote{$[A]$ returns 1 if the proposition $A$ is true, and 0 otherwise. The Iverson bracket is a generalisation of the Kronecker delta.}. 
The dependence of the class probabilities on the covariates is specified through the relationship
\[
  g(p_{ij}) = \big(\alpha_j + f_j(x_i)\big)_{j=1}^m
\]
where $g:[0,1]\to\bbR^m$ is some specified link function.
As we will see later, the normality assumption of the errors naturally implies a \emph{probit} link function, i.e., $g$ is the inverse cumulative distribution function (CDF) of a standard normal distribution (or more precisely, a function that \emph{involves} the standard normal CDF).
Normality is also a required assumption for I-priors to be specified on the regression functions.
We call this method of probit regression using I-priors the \emph{I-probit} regression model.

Note that the probabilities are modelled per class $j\in\{1,\dots,m\}$ by individual regression curves $f_j$, and in the most general setting, $m$ sets of intercepts $\alpha_j$ and kernel hyperparameters $\eta_j$ must be estimated.
The dependence of these $m$ curves are specified through covariances $\sigma_{jk} := \Cov[\epsilon_{ij}, \epsilon_{ik}]$, 
%\[
%  \Corr(\epsilon_{ij},\epsilon_{ik}) = \frac{\sigma_{jk}}{\sigma_j\sigma_k},
%\]
%where $\sigma_{jk} = \Cov[\epsilon_{ij}, \epsilon_{ik}]$, 
for each $j,k\in\{1,\dots,m\}$ and $j\neq k$.
While it may be of interest to estimate these covariances, this paper considers cases where the regression functions are class independent, i.e. $\sigma_{jk} = 0,\forall j \neq k$.
This violates the independence of irrelevant alternatives (IIA) assumption (see Section \ref{sec:iia} for details) crucial in choice models, but not so much necessary for classification when the alternatives are distinctively different.



The many advantages of the I-prior methodology of \cite{jamil2017} transfer over quite well to the I-probit model for classification and inference.
In particular, by choosing appropriate RKHSs for the regression functions, we are able to fit a multitude of binary and multinomial models, including multilevel or random-effects models, linear and non-linear classification models, and even spatio-temporal models.
Examples of these models applied to real-world data is shown in Section \ref{sec:examples}.
Working in a Bayesian setting together with variational inference allows us to estimate the model much faster than traditional MCMC sampling methods, yet provides us with the conveniences that come with posterior estimates.
For example, inferences around log-odds is usually cumbersome for probit models, but a credibility interval can easily be obtained by resampling methods from the relevant posteriors, which are normally exponential family distributions in the I-probit model.


\newpage
\section{A naïve model}
%We describe a naïve classification model using I-priors.
Here, the responses are categorical $y_i \in \{ 1,\dots,m \} =: \cM$, and additionally, write $\by_{i \bigcdot} = (y_{i1},\dots,y_{im})^\top$ where the class responses $y_{ij}$ equal one if individual $i$'s response category is $y_i = j$, and zero otherwise.
In other words, there is exactly a single `1' at the $j$'th position in the vector $\by_{i \bigcdot}$, and zeroes everywhere else.
For $j=1,\dots,m$, we model 
\begin{equation}\label{eq:naiveclassmod}
  \begin{gathered}
    y_{ij} = \alpha + 
    \myoverbrace{\alpha_j + f_j(x_i)}{f(x_i,j)}
    + \epsilon_{ij}  \\
    (\epsilon_{i1},\dots,\epsilon_{im})^\top \iid \N_m(\bzero,\bPsi^{-1}).
  \end{gathered}
\end{equation}
The idea here being that we attempt to model the class responses $y_{ij}$ using class-specific regression functions $f_j$, and the class responses are assumed to be independent among individuals, but may or may not be correlated among classes for each individual.
The class correlations are manifest themselves in the variance of the errors $\bPsi^{-1}$, which is an $m\times m$ matrix.

Denote the regression function $f$ in \cref{eq:naiveclassmod} on the set $\cX\times\cM$ as $f(x_i,j) = \alpha_j + f_j(x_i)$.
This regression function corresponds to an ANOVA decomposition of the spaces $\cF_\cM$ and $\cF_\cX$ of functions over $\cM$ and $\cX$ respectively. 
That is, $\cF = \cF_\cM \oplus (\cF_\cM \otimes \cF_\cX)$ is a decomposition into the main effects of `class', and an interaction effect of the covariates for each class.
Let $\cF_\cM$ and $\cF_\cX$ be RKHSs respectively with kernels $a:\cM\times\cM\to\bbR$ and $b_\eta:\cX\times\cX\to\bbR$.
Then, the ANOVA RKKS $\cF$ possesses the reproducing kernel $h_\eta:(\cX \times \cM)^2 \to \bbR$ as defined by
\begin{align}\label{eq:anovaclass}
  h_\eta\big( (x,j), (x',j') \big) = a(j,j') + a(j,j')b_\eta(x,x').  
\end{align}
The kernel $b_\eta$ may be any of the kernels described in this thesis, ranging from the linear kernel, to the fBm kernel, or even an ANOVA kernel.
Choices for $a:\cM \times \cM \to \bbR$ include 
\begin{enumerate}
  \item \textbf{The Pearson kernel} (as defined in \cref{def:pearson}, \mypageref{def:pearson}). With $J\sim\Prob$, a probability measure over $\cM$,
  \[
    a(j,j') = \frac{\delta_{jj'}}{\Prob(J=j)} - 1.
  \]
  \item \textbf{The centred identity kernel}. With $\delta$ denoting the Kronecker delta function,
  \[
    a(j,j') = \delta_{jj'} - 1 / m.
  \]
\end{enumerate}
The purpose of either of these kernels is to contribute to the class intercepts $\alpha_j$, and to associate a regression function in each class.
The only difference between the two is the inverse probability weighting per class that is applied in the Pearson kernel, but not in the identity kernel.

With $f \in \cF$ (the RKKS with kernel $h_\eta$), it is straightforward to assign an I-prior on $f$. 
It is in fact
\begin{align}\label{eq:naiveclassiprior}
  \begin{gathered}
    f(x_i,j) = \sum_{j'=1}^m\sum_{i'=1}^n a(j,j')\big(1 + b_\eta(x_i,x_{i'})\big) w_{i'j'} \\
    (w_{i'1},\dots,w_{i'm})^\top \iid \N_m(\bzero,\bPsi)
  \end{gathered}
\end{align}
assuming a zero prior mean $f_0(x,j) = 0$.
The model then classifies the $i$'th data point to class $j$ if $\hat y_{ij} = \max(\hat y_{i1},\dots,\hat y_{im})$, where $\hat y_{ik} = \hat\alpha + \hat f(x_i,k)$, the prediction for the $k$'th component of $y_i$.

There are several drawbacks to using the model described above.
Unlike in the case of continuous response variables, the normal I-prior model is highly inappropriate for categorical responses.
For one, it violates the normality and homoscedasticity assumptions of the errors.
For another, predicted values may be out of the range $[0,m]$ and thus poorly calibrated.
Furthermore, it would be more suitable if the class probabilities---the probability of an observation belonging to a particular class---were also part of the model.
In \cref{chapter5}, we propose an improvement to this naïve I-prior classification model by considering a probit-like transformation of the regression functions.

%\begin{proof}
%  \[
%    f(x_i,j) = \sum_{j'=1}^m\sum_{i'=1}^n a(j,j')\big(1 + b_\eta(x_i,x_{i'})\big) w_{i'j'}
%  \]
%  
%  \[
%    \alpha_j = \sum_{j'=1}^m\sum_{i'=1}^n a(j,j') w_{i'j'}
%  \]
%  
%  \begin{align*}
%    \sum_{j=1}^m \alpha_j
%    &= \sum_{j=1}^m \sum_{j'=1}^m\sum_{i'=1}^n a(j,j') w_{i'j'} \\
%    &= \sum_{j=1}^m \sum_{j'=1}^m\sum_{i'=1}^n \delta_{jj'} w_{i'j'} \\
%    &= \sum_{j=1}^m\sum_{i'=1}^n  w_{i'j}
%  \end{align*}
%  
%  \begin{align*}
%    \sum_{j=1}^m f_j(x_i)
%    &= \sum_{j=1}^m \sum_{j'=1}^m\sum_{i'=1}^n a(j,j')b_\eta(x_i,x_{i'}) w_{i'j'} \\
%    &= \sum_{j=1}^m \sum_{j'=1}^m\sum_{i'=1}^n \delta_{jj'} b_\eta(x_i,x_{i'}) w_{i'j'} \\
%    &= \sum_{j=1}^m\sum_{i'=1}^n b_\eta(x_i,x_{i'}) w_{i'j}
%  \end{align*}
%  
%  \begin{align*}
%    \sum_{i=1}^n f_j(x_i)
%    &= \sum_{i=1}^n \sum_{j'=1}^m\sum_{i'=1}^n a(j,j')b_\eta(x_i,x_{i'}) w_{i'j'} \\
%    &= \sum_{i=1}^n \sum_{j'=1}^m\sum_{i'=1}^n \delta_{jj'} b_\eta(x_i,x_{i'}) w_{i'j'} 
%  \end{align*}
%\end{proof}

%Rearrange the $n$ observations per class.
%Let $\bff_j = \big(f_j(x_1),\dots,f_j(x_n)\big)^\top \in \bbR^n$.
%We can write the I-prior as $\bff_j = \bA_{jj} \cdot \bH\bw_j$
%Therefore, $\bff_j \sim \N_n(\bzero, \bPsi_{jj}\bA_{jj} \bH^2)$, and
%\begin{align*}
%  \Cov(\bff_j,\bff_k) &= \Cov(\bA_{jj} \cdot \bH\bw_j, \bA_{kk} \cdot \bH\bw_k) \\
%  &= \bA_{jj}\bA_{kk} \cdot \bH \Cov(\bw_j,\bw_k) \bH \\
%  &= \bA_{jj}\bA_{kk}\bPsi_{jk} \bH^2.
%\end{align*}


\section{A latent variable motivation: the I-probit model}
%It is convenient, as we did in the previous subsection, to again think of the responses $y_i \in \{1,\dots,m\} = \cM$ as comprising of a binary vector $(y_{i1},\dots,y_{im})$, with a single `1' at the position corresponding to the value that $y_i$ takes. 
%That is, $y_i = (y_{i1}, \dots, y_{im})$ with
%\[
%  y_{ik} =
%  \begin{cases}
%    1 &\text{ if } y_i = k \\
%    0 &\text{ if } y_i \neq k.
%  \end{cases}
%\]
In this formulation, each $y_{ij}$ is distributed as Bernoulli with probability $p_{ij}$. Now, assume that, for each $y_i = (y_{i1}, \dots, y_{im})$, there exists corresponding \emph{continuous, underlying, latent variables} $y_{i1}^*, \dots, y_{im}^*$ such that
\begin{align}\label{eq:latentmodel}
  y_i =
  \begin{cases}
    1 &\text{ if } y_{i1}^* \geq y_{i2}^*, y_{i3}^*, \dots, y_{im}^* \\
    2 &\text{ if } y_{i2}^* \geq y_{i1}^*, y_{i3}^*, \dots, y_{im}^* \\
    \,\vdots \\
    m &\text{ if } y_{im}^* \geq y_{i2}^*, y_{i3}^*, \dots, y_{i\,m-1}^*. \\
  \end{cases}  
\end{align}
In other words, 
%$y_{ij} = [y_{ij}^* = \max_k y_{ik}^*]$.
$y_{ij} = \argmax_{k=1}^m y_{ik}^*$.
Such a formulation is common in economic choice models, and is rationalised by a utility-maximisation argument: an agent faced with a choice from a set of alternatives will choose the one which benefits them most.

Instead of modelling the observed $y_{ij}$'s directly, we model instead the $n$ latent variables in each class $j=1,\dots,m$ according to the regression problem
\begin{equation}\label{eq:multinomial-latent}
  \begin{gathered}
    y_{ij}^* = \alpha_j + f_j(x_i) + \epsilon_{ij} \\
    \bepsilon_{i} = (\epsilon_{i1}, \dots, \epsilon_{im})^\top  \iid \N_m(\bzero, \bPsi^{-1}). 
  \end{gathered}
\end{equation}
%with $\alpha_j$ being an intercept, and $f_j:\cX \to \bbR$ a regression function belonging to some RKHS/RKKS of functions $\cF$.
%having the reproducing kernel $h_{\eta_j}: \cX \times \cX \to \bbR$. 
We can see some semblance of this model with the one in \cref{eq:naiveclassiprior}, and ultimately the aim is to assign I-priors to the regression function of these latent variables, and we will describe this shortly.
For now, realise that each $\by_i^* := (y_{i1}^*, \dots, y_{im}^*)^\top$ has the distribution $\N_m(\balpha + \bff(x_i), \bPsi^{-1})$, conditional on the data $x_i$,  the intercepts $\balpha = (\alpha_1,\dots,\alpha_m)^\top$, the evaluations of the functions at $x_i$ for each class $\bff(x_i) = \big(f_1(x_i), \dots, f_m(x_i)\big)^\top$, and the error covariance matrix $\bPsi^{-1}$.

\newcommand{\intset}{\{y_{ij}^* > y_{ik}^* \,|\, \forall k \neq j\}}
The probability of belonging to class $j$ for observation $i$, i.e. $p_{ij}$, is calculated as
\begin{align}
  p_{ij} 
  &= \Prob(y_i = j) \nonumber \\
  &= \Prob\big(\intset\big) \nonumber \\
  &= \int\displaylimits_{\intset} \phi(\by_i^*|\balpha + \bff(x_i), \bPsi^{-1}) \dint\by^* \label{eq:pij},
  %&=: g_{\bSigma}^{-1} \big( \alpha_j + f_j(x_i) \big)_{j=1}^m \nonumber
\end{align}
where $\phi(\cdot|\mu,\Sigma)$ is the density of the multivariate normal with mean $\mu$ and variance $\Sigma$.
This is the probability that the normal random variable $\by_i^*$ belongs to the set $\intset$, which are cones in $\bbR^m$.
Since the union of these cones is the entire $m$-dimensional space of reals, the probabilities add up to one and hence they represent a proper probability mass function of the classes.
%Upon knowing all values for $\alpha_j$, $f_j(x_i)$, and $\bSigma$, one is able to calculate $p_{ij}$ through the relationship \eqref{eq:pij}, which we denote as $g^{-1}$.
While this does not have a closed-form expression and highlights one of the difficulties of working with probit models, the integral is by no means impossible to compute---see \cref{misc:mnint} for a note regarding this matter.

Note that the dimension of the integral \eqref{eq:pij} is $m-1$, since the $j$'th coordinates is fixed relative to the others.
Alternatively, we could have specified the model in terms of \emph{relative differences} of the latent variables.
Choosing the first category as the reference category, define new random variables $z_{ij} = y_{ij}^* - y_{i1}^*$, for $j = 2,\dots,m-1$. 
The model \cref{eq:latentmodel} is equivalently represented by
\begin{equation}
  y_i = 
  \begin{cases}
    1 & \text{if } \max (z_{i2},\dots,z_{im}) < 0 \\
    j & \text{if } \max (z_{i2},\dots,z_{im}) = z_{ij} \geq 0.
  \end{cases}
\end{equation} 
Write $\bz_i = (z_{i2},\dots,z_{im})^\top \in \bbR^{m-1}$.
Then $\bz_i = \bQ\by^*_i$, where $\bQ \in \bbR^{(m-1)\times m}$  is the $(m-1)$ identity matrix pre-augmented with a column vector of minus ones.
We have that $\bz_i \iid \N_{m-1}\big(\bQ(\balpha + \bff(x_i)), \bQ\bPsi^{-1}\bQ^\top)\big)$.
Thus, the class probabilities for $j=2,\dots,m$ are
\begin{align}
  p_{ij} = 
  \int\displaylimits_{\{z_{ik} < 0 \,|\, \forall k \neq j \}} \ind(z_{ij} \geq 0) \,\phi(\bz_i) \dint\bz_i \label{eq:pij2},
\end{align}
with $\phi(\bz_i)$ representing the $(m-1)$-variate normal density for $\bz_i$.
The class probability $p_{i1}$ is simply
\[
  p_{i1} = \int\displaylimits_{\{ z_{ik} <0 \}}  \phi(\bz_i) \dint\bz_i = 1 - \sum_{k\neq 1} p_{ik}.
\]
From this representation of the model, with $m=2$ (binary outcomes) we see that
\[
  p_{i1} = 
%  \int \ind(z_{i2} < 0) \phi(z_{i2}) \dint z_{i2} = 
  \Phi \left( \frac{z_{i2} - \mu}{\sigma} \right)
  \hspace{0.5cm}\text{and}\hspace{0.5cm}
  p_{i2} = 
%  \int \ind(z_{i2} \geq 0) \phi(z_{i2}) \dint z_{i2} = 
  1 - \Phi \left( \frac{z_{i2} - \mu}{\sigma} \right),
\]
where $\Phi(\cdot)$ is the CDF of the standard normal univariate distribution, and $\mu$ and $\sigma$ are the mean and standard deviation of the random variable $z_{i2}$.

%\begin{figure}[t]
%  \centering
%  \begin{tikzpicture}[scale=1.1, transform shape]
%    \tikzstyle{main}=[circle, minimum size=10mm, thick, draw=black!80, node distance=16mm]
%    \tikzstyle{connect}=[-latex, thick]
%    \tikzstyle{box}=[rectangle, draw=black!100]
%      \node[main, draw=black!0] (blank) [xshift=-0.55cm] {};  % pushes image to right slightly
%      \node[main, fill=black!10] (H) [] {$x_i$};
%      \node[main] (Sigma) [below=of H, yshift=-1.2cm, xshift=0.6cm] {$\bSigma$};
%      \node[main, double, double distance=0.6mm] (f) [right=of H, xshift=0.5cm] {$f_{ij}$};
%      \node[main, double, double distance=0.6mm] (ystar) [right=of f, xshift=0cm] {$y_{ij}^*$};
%      \node[main, double, double distance=0.6mm] (pij) [right=of ystar, xshift=0cm] {$p_{ij}$};
%      \node[main] (lambda) [above=of f, xshift=0cm, yshift=-0.3cm] {$\eta_j$};        
%      \node[main] (alpha) [above=of ystar, xshift=0cm, yshift=-0.3cm] {$\alpha_j$};  
%      \node[main, fill = black!10] (y) [right=of pij, xshift=0.2cm] {$y_{i}$};
%      \node[main] (w) [below=of f, yshift=0.3cm] {$w_{ij}$};  
%      \path (alpha) edge [connect] (ystar)
%            (lambda) edge [connect] (f)
%            (H) edge [connect] node [above] {$h \ \ $} (f)
%    		(f) edge [connect] (ystar)
%    		(ystar) edge [connect] node [above] {$g^{-1}$}  (pij)
%            (pij) edge [connect] (y)
%            (Sigma) edge [connect] (w)
%    		(w) edge [connect] (f);
%      \node[rectangle, draw=black!100, fit={($(H.north west) + (0.2,0cm)$) ($(y.north east) + (-0.2,0.4cm)$) (w)}] {}; 
%      \node[rectangle, fit= (w) (y), label=below right:{$i=1,\dots,n$}, xshift=0.95cm, yshift=0.5cm] {};  % the label
%      \node[rectangle, draw=black!100, fit={(lambda) ($(pij.north east) + (0.5cm,0.7cm)$) ($(w.south west) + (-0.5,-0.7cm)$)}] {}; 
%      \node[rectangle, fit={(f) ($(ystar.north east) + (0.5cm,0.7cm)$) ($(w.south west) + (-0.5,-0.7cm)$)}, label=below right:{$j=1,\dots,m$}, xshift=0.63cm, yshift=0.37cm] {}; 
%    \end{tikzpicture}
%    \caption{A DAG of the probit I-prior  model. Observed nodes are shaded, while double-lined nodes represented known or calculable quantities. There are at most $m-1$ sets of intercept ($\alpha_j$) and RKHS parameters ($\eta_j$)  to estimate due to identifiability. Depending on the specification of $\bSigma$, this may need to be estimated too.}
%\end{figure}

Now we'll see how to specify an I-prior on the regression problem \cref{eq:multinomial-latent}.
In the naïve I-prior model, we wrote $f(x_i,j) = \alpha_j + f_j(x_i)$, and specified for $f$ to belong to an ANOVA RKKS with kernel defined in \cref{eq:anovaclass}.
Instead of doing the same, we take a different approach.
Treat the $\alpha_j$'s in \cref{eq:multinomial-latent} as intercept parameters to estimate with the additional requirement that $\sum_{j=1}^m \alpha_j = 0$.
Further, let $\cF$ be a (centred) RKHS/RKKS of functions over $\cX$ with reproducing kernel $h_\eta$.
Now, consider putting an I-prior on the regression functions $f_j \in \cF$, $j=1\dots,m$, defined by
\[
  f_j(x_i) = \sum_{k=1}^n h_\eta(x_i,x_k)w_{ik}
\]
with $\bw_i := (w_{i1},\dots,w_{im})^\top \iid \N(0,\bPsi)$.
This is similar to the naïve I-prior specification \cref{eq:naiveclassiprior}, except that the intercept  have been treated as parameters rather than accounting for them using an RKHS of constant functions.
In particular, the overall regression relationship still satisfies the ANOVA functional decomposition.
We find that this method bodes well down the line computationally.

We call the multinomial probit regression model of \cref{eq:latentmodel} subject to \cref{eq:multinomial-latent} and I-priors on $f_j \in \cF$, the \emph{I-probit model}.
For completeness, this is stated again: for $i=1,\dots,n$, $y_i = \argmax_{k=1}^m y_{ik}^* \in \{1,\dots,m\}$, where, for $j=1,\dots,m$,
\begin{align}
  \begin{gathered}
    y_{ij}^* = \alpha_j + 
    \greyoverbrace{\sum_{k=1}^n h_\eta(x_i,x_k)w_{ik}}{f_j(x_i)}
    + \epsilon_{ij} \\
    \bepsilon_{i} = (\epsilon_{i1}, \dots, \epsilon_{im})^\top  \iid \N_m(\bzero, \bPsi^{-1}) \\
    \bw_i := (w_{i1},\dots,w_{im})^\top \iid \N(0,\bPsi).
  \end{gathered}
\end{align}
The parameters of the I-probit model are denoted by $\theta = \{\alpha_1,\dots,\alpha_m,\eta,\bPsi \}$.
Let $\by^* \in \bbR^{n \times m}$ denote the matrix containing $(i,j)$ entries $y_{ij}^*$.
Using the results in Chapter 4, the marginal distribution of the latent variables is
\[
  \vecc \by^* \sim \N_{nm}\big(\balpha, (\bPsi \otimes \bH_\eta^2) + (\bPsi^{-1} \otimes \bI_n)\big).
\]

\subsection{IIA}

In decision theory, the independence axiom states that an agent's choice between a set of alternatives should not be affected by the introduction or elimination of a (new) choice option.
The probit model is suitable for modelling multinomial data where the independence axiom, which is also known as the \emph{independence of irrelevant alternatives} (IIA) assumption, is not desired. 
Such cases arise frequently in economics and social science, and the famous Red-Bus-Blue-Bus example is often used to illustrate IIA.
Suppose commuters face the decision between taking cars and red busses. 
The addition of blue busses to commuters' choice should in theory be more likely chosen by those who prefer taking the bus over cars.
That is, assuming commuters are indifferent about the colour of the bus, commuters who are predisposed to taking the red bus would see the blue bus as an identical alternative.
 Yet, if IIA is imposed, then the three choices are distinct, and the fact that red and blue busses are substitutable is ignored.

In the I-probit model, the choice dependency is controlled by the error precision matrix $\bPsi$.
Specifically, the off-diagonal elements $\bPsi_{jk}$ capture the correlation between choices $j$ and $k$.
Allowing all $m(m+1)/2$ covariance elements of $\bPsi$ leads to the \emph{full I-probit model}, and would not assume an IIA position.

%Although crucial in choice models, it is not so much necessary for classification tasks when the alternatives under consideration are distinctly different.
%In such cases, one may choose to abandon the IIA

While it is an advantage to be able to model the correlations across choices (unlike in logistic models), it would be a major simplification algorithmically to consider all covariances in $\bPsi$ to be zero.
This would trigger the IIA assumption in the I-probit model.
There are applications where the IIA assumption would not adversely affect the analysis, such as when all the choices are mutually exclusive and exhaustive.
In these situations, it would be beneficial to reduce the I-probit model to a simpler version by assuming $\bPsi = \diag(\psi_1,\dots,\psi_m)$.

The independence assumption causes the distribution of the latent variables to be $y_{ij}^* \sim \N(\alpha_j + f_j(x_i), \sigma_j^2)$ for $j=1,\dots,m$.
As a continuation of line \eqref{eq:pij}, we can show the class probability $p_{ij}$ to be
\begin{align*}
  p_{ij} 
  &= \idotsint\displaylimits_{\{y_{ij}^* > y_{ik}^* | \forall k \neq j\}} 
  \prod_{k=1}^m \Big\{ p(y_{ik}^*|\alpha_j + f_k(x_i), \sigma_j^2) \d y_k^* \Big\} \\
  &= \int \mathop{\prod_{k=1}^m}_{k\neq j} 
  \Phi \left( \frac{y_{ij}^* - \alpha_k - f_{ik}}{\sigma_k} \right) \cdot
  \frac{1}{\sigma_j} \phi \left( \frac{y_{ij}^* - \alpha_j - f_{ij}}{\sigma_j} \right)  \d y_{ij}^* \\
  &= \E_Z \Bigg[ \mathop{\prod_{k=1}^m}_{k\neq j} 
  \Phi \left(\frac{\sigma_j}{\sigma_k} Z + \frac{\alpha_j + f_{ij} - \alpha_k - f_{ik}}{\sigma_k} \right) \Bigg] \nonumber
\end{align*}
where $Z\sim\N(0,1)$, and $\phi(\cdot)$ and $\Phi(\cdot)$ are its PDF and CDF respectively.
%In the binary case where $m=2$, we set $\sigma_1^2 = 2$ and fix $f_{i2} = 0$, and we get that 
%\[
%  p_{i1} = 1 - \Phi(\alpha_1 + f_{i1}) \ \text{ and } \ p_{i2} =  \Phi(\alpha_1 + f_{i1}),
%\]
%which clearly shows the probit relationship between the class probabilities and the latent regression.
The proof of this fact is included in the Appendix.
With the exception of the binary case, these probabilities still do not have a closed-form expression (per se) and numerical methods are required to calculate them.
In this simplified version of the I-probit model, the integral is unidimensional and involves the Gaussian PDF, and this can be efficiently obtained using quadrature methods.


%\begin{table}[]
%\centering
%\caption{My caption}
%\label{my-label}
%\begin{tabular}{@{}lp{5cm}p{5cm}@{}}
%\toprule
%                    & Naïve model                                                                              & I-probit model                                            \\ \midrule
%Grand intercept     & Fixed at $\alpha = 1/m$, equivalently $y_{ij}\mapsto y_{ij} - 1/m$ and set $\alpha = 0$. & $\alpha=0$ (otherwise, model is not location identified). \\
%Intercepts          & $\alpha_j \in \cF_\cM$, a centred RKHS with kernel $a$                                                  & $\alpha_j\in\bbR$, $\sum_{j=1}^m \alpha_j = 0$            \\
%Interaction effects & $f_j \in \cF_\cM \otimes \cF_\cX$, tensor product interaction space with kernel $ah_\eta$                     & $f_j\in\cF$ with kernel $h_\eta$                                                \\ \bottomrule
%\end{tabular}
%\end{table}




%A rearrangement of the error terms is beneficial for the process of deriving the required I-priors.
%Recall that since each $\bepsilon_i \iid \N_m(\bzero,\bSigma)$ then concatenating all of the error vectors result in $\bepsilon = (\bepsilon_1^\top,\dots,\bepsilon_n^\top)^\top$, and this has distribution $\N_{nm}(\bzero, \bOmega)$, where $\bOmega = \diag(\bSigma,\dots,\bSigma)$.
%By rearranging the entries of $\bepsilon$ according to class (instead of by observation) and defining $\bepsilon_j'=(\epsilon_{1j},\dots,\epsilon_{nj})^\top$ for $j=1,\dots,m$, we observe that
%\[
%  \bepsilon' = 
%  \begin{pmatrix}
%    \bepsilon_1' \\
%    \vdots \\
%    \bepsilon_m'
%  \end{pmatrix}
%  \sim \N_{nm}(\bzero,\bOmega')
%\]
%where $\bOmega'$ contains the appropriately rearranged entries of $\bOmega$. 
%For illustration, an example of this rearrangement is shown in Figure \ref{fig:covrearrange} for $m = 2$ and $n=5$.
%The subtle difference is that whereas $\bOmega$ is a block diagonal matrix, $\bOmega'$ is comprised of $m^2$ equally sized $n \times n$ partitions of diagonal matrices.
%However, the number of unique elements in $\bOmega$ and $\bOmega'$ is the same as in the $m \times m$ matrix $\bSigma$, i.e. $m(m+1)/2$.
%We can also express these matrices in terms of Kronecker products $\bOmega = \bI_n \otimes \bSigma$ and $\bOmega = \bSigma \otimes \bI_n$.
%
%Denote $f_{ij} = f_j(x_i)$ as the evaluation of the function $f_j(\cdot)$ at $x_i$, and also $\bff_j = (f_{1j}, \dots, f_{nj})^\top$ as the vector containing all $n$ evaluations pertaining to the $j$th class.
%Concatenate all of the $\bff_j$'s into one long vector of length $nm$: $\bff = (\bff_1^\top, \dots, \bff_m^\top)^\top$.
%The I-prior is $\bff \sim \N_{nm}(\bzero,\bV_f)$, where $\bzero$ is an vector of length $nm$ containing zeroes and the covariance matrix $\bV_f$ is has the block matrix structure
%\begin{align}\label{eq:vf}
%  \bV_f = \begin{pmatrix}
%    \bV_{11} & \bV_{12} & \dots  & \bV_{1m} \\
%    \bV_{21} & \bV_{22} & \dots  & \bV_{2m} \\
%    \vdots   & \vdots   & \ddots & \vdots \\
%    \bV_{m1} & \bV_{m2} & \dots  & \bV_{mm} \\
%  \end{pmatrix},
%\end{align}
%and each block entry above is given by
%\begin{align}\label{eq:fisherinformation}
%  \bV_{jk}(r,s) = \cI\big( f_j(x_r), f_k(x_s) \big) = \bPsi_{jk} \sum_{r'=1}^n \sum_{s'=1}^n  h_{\eta_j}(x_r, x_{r'}) h_{\eta_k}(x_{s'}, x_s),
%\end{align}
%where $\bPsi = \bSigma^{-1}$. 
%Equation \eqref{eq:fisherinformation} gives the Fisher information for the regression functions $f_j$ and $f_k$, $j,k\in\{1,\dots,m\}$, evaluated at two points $x_r$ and $x_s$, $r,s\in\{1,\dots,n\}$---see \cite{bergsma2017} for details.
%
%One may also write $\bff = \bG\bw$ where $\bw \sim \N_{nm}(\bzero, \bOmega'^{-1})$, $\bG = \diag(\bH_{\eta_1},\dots,\bH_{\eta_m})$, and $\bH_{\eta_j}$ are the $n\times n$ kernel matrices with $(r,s)$ entries $h_{\eta_j}(x_r,x_s)$.
%Realise that $\bOmega^{-1}$ is obtainable by repeating $\bSigma^{-1}$ diagonally, and that $\bOmega'^{-1}$ is then obtained using a similar rearrangement of its rows and columns to that of $\bOmega$.
%Substituting equation \eqref{eq:fisherinformation} into the model \eqref{eq:multinomial-latent}, we get that
%\begin{align}
%  \begin{gathered}\label{eq:ipriorw}
%    y_{ij}^* = \alpha_j + \sum_{r=1}^n h_{\eta_j}(x_i,x_r)w_{rj} + \epsilon_{ij} \\
%    (w_{i1},\dots,w_{im}) \iid \N_m(\bzero, \bSigma^{-1}) \\
%    (\epsilon_{i1},\dots,\epsilon_{im}) \iid \N_m(\bzero, \bSigma)
%  \end{gathered}
%\end{align}
%This formulation will be useful in the variational algorithm later on.
%





\section{Identifiability and IIA}\label{sec:iia}
%The parameters in a linear multinomial probit model is well known to be unidentified \citep{Keane1992,train2009discrete}, and the reason for this is two-fold.
Firstly, an addition of a constant to the latent variables $y_{ij}^*$'s in \cref{eq:latentmodel} will not change which latent variable is maximal, and therefore leaves the model unchanged.
Secondly, all latent variables can be scaled by some positive constant without changing which latent variable is largest.
Therefore, a \emph{linear parameterisation} for the multinomial probit model is not identified as there can be more than one set of parameters for which the class probabilities are the same.
To fix this issue, constraints are imposed on location and scale of the latent variables.

However, for the I-probit model, this is not the case, because the model is not related to the parameters $\theta = \{\alpha_1,\dots,\alpha_m,\eta, \bPsi \}$ linearly.
One cannot simply add to or multiply $\theta$ by a constant and expect the model to be left unchanged.
Thus, the I-probit model is identified in the parameter set $\theta$ without having to impose any restrictions, particularly on the precision matrix $\bPsi$ (if this is to be estimated).

To see how the I-probit model is location identified, suppose assumptions \ref{ass:A4} and \ref{ass:A5} hold, and consider a constant $a$ added to the latent propensities.
This would then imply the relationship
\[
  a + y_{ij}^* = 
  \greyoverbrace{a + \alpha_j}{\alpha_j^*}  + f_j(x_i) + \epsilon_{ij},
\]
which is similar to adding the constant $a$ to all of the intercept parameters $\alpha_j$---denote these new intercepts by $\alpha_j^*$.
As a requirement of the functional ANOVA decomposition, the $\alpha_j^*$'s need to sum to zero, but we already have that $\sum_{j=1}^m \alpha_j=0$, so it must be that $a =0$.
This also highlights the reason behind assumption \ref{ass:A4} and \ref{ass:A5} for fixing the grand intercept $\alpha$ to zero.

As for identification in scale, consider multiplying the latent variables by $c>0$.
Denote by $\bV_y^*(\omega) \in \bbR^{nm \times nm}$ the marginal covariance matrix of the latent propensities, which depends on the scale parameters $\omega = \{\eta, \bPsi\}$.
The scaled latent variables $\{c^{1/2}y^*_{ij} \,|\, \forall i,j = 1,\dots \}$, which collectively has (marginal) variance and covariances given by the matrix $c \bV_y^*(\omega)$, is expected to have been generated from the model with parameters $c\omega$.
However, we have that 
\begin{align*}
  c\bV_y^*(\omega)
  &= c (\bPsi \otimes \bH_\eta^2) + c (\bPsi^{-1} \otimes \bI_n) \\
  &= (c \bPsi \otimes \bH_\eta^2) + (c \bPsi^{-1} \otimes \bI_n) \\
  &\neq \bV_y^*(c\omega).
\end{align*}

Now, we turn to a discussion of the role of $\bPsi$ in the model.
In decision theory, the independence axiom states that an agent's choice between a set of alternatives should not be affected by the introduction or elimination of a choice option.
The probit model is suitable for modelling multinomial data where the independence axiom, which is also known as the \emph{independence of irrelevant alternatives} (IIA) assumption, is not desired. 
Such cases arise frequently in economics and social science, and the famous Red-Bus-Blue-Bus example is often used to illustrate IIA:
suppose commuters face the decision between taking cars and red busses. 
The addition of blue busses to commuters' choices should, in theory, be more likely chosen by those who prefer taking the bus over cars.
That is, assuming commuters are indifferent about the colour of the bus, commuters who are predisposed to taking the red bus would see the blue bus as an identical alternative.
 Yet, if IIA is imposed, then the three choices are distinct, and the fact that red and blue busses are substitutable is ignored.

To put it simply, the model is IIA if choice probabilities depend only on the choice in consideration, and not on any other alternatives.
In the I-probit model, or rather, in probit models in general, choice dependency is controlled by the error precision matrix $\bPsi$.
Specifically, the off-diagonal elements $\bPsi_{jk}$ capture the correlation between alternatives $j$ and $k$.
Allowing all $m(m+1)/2$ covariance elements of $\bPsi$ to be non-zero leads to the \emph{full I-probit model}, and would not assume an IIA position.

\begin{figure}[hbt]
\centering\hspace{-13pt}
\begin{blockmatrixtabular}
\valignbox{
\begin{blockmatrixtabular}
&
\mblockmatrix{0.55in}{0in}{\footnotesize $j=1$}&
\mblockmatrix{0.55in}{0in}{\footnotesize $j=2$}&
\mblockmatrix{0.55in}{0in}{$\cdots$}&
\mblockmatrix{0.55in}{0in}{\footnotesize $j=m$}& \\
\mblockmatrix{0in}{0.55in}{\footnotesize $j=1$}&
\fblockmatrix[colblu!39]{0.55in}{0.55in}{\footnotesize $\bV[1,1]$}& 
\fblockmatrix[colblu!22]{0.55in}{0.55in}{\footnotesize $\bV[1,2]$}&
\fblockmatrix[colblu!24]{0.55in}{0.55in}{\footnotesize $\cdots$}& 
\fblockmatrix[colblu!46]{0.55in}{0.55in}{\footnotesize $\bV[1,m]$}\\
\mblockmatrix{0in}{0.55in}{\footnotesize $j=2$}&
\fblockmatrix[colblu!22]{0.55in}{0.55in}{\footnotesize $\bV[2,1]$}& 
\fblockmatrix[colblu!20]{0.55in}{0.55in}{\footnotesize $\bV[2,2]$}&
\fblockmatrix[colblu!42]{0.55in}{0.55in}{\footnotesize $\cdots$}& 
\fblockmatrix[colblu!40]{0.55in}{0.55in}{\footnotesize $\bV[2,m]$}\\
\mblockmatrix{0in}{0.55in}{\hspace{10pt}$\vdots$}&
\fblockmatrix[colblu!24]{0.55in}{0.55in}{\footnotesize $\vdots$}& 
\fblockmatrix[colblu!42]{0.55in}{0.55in}{\footnotesize $\vdots$}&
\fblockmatrix[colblu!33]{0.55in}{0.55in}{\footnotesize $\ddots$}& 
\fblockmatrix[colblu!30]{0.55in}{0.55in}{\footnotesize $\vdots$}\\
\mblockmatrix{0in}{0.55in}{\footnotesize $j=m$}&
\fblockmatrix[colblu!46]{0.55in}{0.55in}{\footnotesize $\bV[m,1]$}& 
\fblockmatrix[colblu!40]{0.55in}{0.55in}{\footnotesize $\bV[m,2]$}&
\fblockmatrix[colblu!30]{0.55in}{0.55in}{\footnotesize $\cdots$}& 
\fblockmatrix[colblu!20]{0.55in}{0.55in}{\footnotesize $\bV[m,m]$}\\
\end{blockmatrixtabular}
}&
\valignbox{\mblockmatrix{0.31in}{2.8in}{}}&
\valignbox{
\begin{blockmatrixtabular}
%&
\mblockmatrix{0.55in}{0in}{\footnotesize $j=1$}&
\mblockmatrix{0.55in}{0in}{\footnotesize $j=2$}&
\mblockmatrix{0.55in}{0in}{$\cdots$}&
\mblockmatrix{0.55in}{0in}{\footnotesize $j=m$}& \\
%\mblockmatrix{0in}{0.55in}{\footnotesize $j=1$}&
\fblockmatrix[colblu!39]{0.55in}{0.55in}{\footnotesize $\bV[1,1]$}& 
\fblockmatrix[none]{0.55in}{0.55in}{}&
\fblockmatrix[none]{0.55in}{0.55in}{}& 
\fblockmatrix[none]{0.55in}{0.55in}{}\\
%\mblockmatrix{0in}{0.55in}{\footnotesize $j=2$}&
\fblockmatrix[none]{0.55in}{0.55in}{}& 
\fblockmatrix[colblu!20]{0.55in}{0.55in}{\footnotesize $\bV[2,2]$}&
\fblockmatrix[none]{0.55in}{0.55in}{}& 
\fblockmatrix[none]{0.55in}{0.55in}{}\\
%\mblockmatrix{0in}{0.55in}{\hspace{10pt}$\vdots$}&
\fblockmatrix[none]{0.55in}{0.55in}{}& 
\fblockmatrix[none]{0.55in}{0.55in}{}&
\fblockmatrix[colblu!33]{0.55in}{0.55in}{\footnotesize $\ddots$}& 
\fblockmatrix[none]{0.55in}{0.55in}{}\\
%\mblockmatrix{0in}{0.55in}{\footnotesize $j=m$}&
\fblockmatrix[none]{0.55in}{0.55in}{}& 
\fblockmatrix[none]{0.55in}{0.55in}{}&
\fblockmatrix[none]{0.55in}{0.55in}{}& 
\fblockmatrix[colblu!20]{0.55in}{0.55in}{\footnotesize $\bV[m,m]$}\\
\end{blockmatrixtabular}
}&
\end{blockmatrixtabular}\\ 
\caption[Illustration of the covariance structure of the full I-probit model and the independent I-probit model.]{Illustration of the covariance structure of the full I-probit model (left) and the independent I-probit model (right). The full model has  $m^2$ blocks of $n \times n$ symmetric matrices, and the blocks themselves are arranged symmetrically about the diagonal. The independent model, on the other hand, has a block diagonal structure, and its sparsity induces simpler computational methods for estimation.}
\label{fig:iprobcovstr}
\end{figure}

%most economics articles prefer to estimate scaled probit models. in fact, it is an advantage of it! but do we care about the scale? maybe care more about IIA, which can't do without scales i suppose.

While it is an advantage to be able to model the correlations across choices (unlike in logistic models), there are applications where the IIA assumption would not adversely affect the analysis, such as classification tasks.
Some analyses might also be indifferent as to whether or not choice dependency exists.
In these situations, it would be beneficial, algorithmically speaking, to reduce the I-probit model to a simpler version by assuming $\bPsi = \diag(\psi_1,\dots,\psi_m)$, which would trigger the IIA assumption in the I-probit model.
We refer to this model as the \emph{independent I-probit model}.

The independence assumption causes the distribution of the latent variables to be $y_{ij}^* \sim \N(\mu_k(x_i), \sigma_j^2)$ for $j=1,\dots,m$, where $\sigma_j^2 = \psi_j^{-1}$.
As a continuation of line \cref{eq:pij}, we can show the class probability $p_{ij}$ to be
\begin{align}
  p_{ij} 
  &= \idotsint\displaylimits_{\{y_{ij}^* > y_{ik}^* | \forall k \neq j\}} 
  \prod_{k=1}^m \Big\{ \phi(y_{ik}^*|\mu_k(x_i), \sigma_k^2) \dint y_{ik}^* \Big\} \nonumber \\
  &= \int \mathop{\prod_{k=1}^m}_{k\neq j} 
  \Phi \left( \frac{y_{ij}^* - \mu_k(x_i)}{\sigma_k} \right) \cdot
   \phi(y_{ij}^*|\mu_j(x_i), \sigma_j^2)  \dint y_{ij}^* \nonumber \\
  &= \E_Z \Bigg[ \mathop{\prod_{k=1}^m}_{k\neq j} 
  \Phi \left(\frac{\sigma_j}{\sigma_k} Z + \frac{\mu_j(x_i) - \mu_k(x_i)}{\sigma_k} \right) \Bigg] \label{eq:pij2}
\end{align}
where $Z\sim\N(0,1)$, $\Phi(\cdot)$ its cdf, and $\phi(\cdot|\mu,\sigma^2)$ is the pdf of $X\sim\N(\mu,\sigma^2)$.
The equation \cref{eq:pij} is thus simplified to a unidimensional integral involving the Gaussian pdf and cdf, which can be computed fairly efficiently using quadrature methods.
The probit link function is evidently seen in the above equation.
%Moreover, in the binary case where $m=2$ and fixed error precision $\psi_1 = \psi_2 = 1/2$, we get
%\[
%  p_{i1} = 1 - \Phi\big(\mu_1(x_i) - \mu_2(x_i)\big) \ \text{ and } \ p_{i2} =  \Phi\big(\mu_1(x_i) - \mu_2(x_i)\big),
%\]
%which clearly shows the probit relationship between the class probabilities and the latent regression.
%The proof of this fact is included in the Appendix.
%With the exception of the binary case, these probabilities still do not have a closed-form expression (per se) and numerical methods are required to calculate them.


\section{Estimation}
%%The challenge of estimation is then to first overcome this intractability by means of a suitable approximation of the integral.
%Several methods may be employed, namely quadrature methods, Laplace approximation, and Markov chain Monte Carlo (MCMC) methods, but these all fail in the face of high dimensionality when the sample size $n$ is large.

As with the normal I-prior model, an estimate of the posterior regression function with optimised hyperparameters is sought.
The log likelihood function $L(\cdot)$ for $\theta$ using all $n$ observations $\{(y_1,x_1),\dots,(y_n,x_n)\}$ is obtained by integrating out the I-prior from the categorical likelihood, as follows:
\begin{align}
  L(\theta) 
  &= \log \int p(\by | \bw, \theta) \, p(\bw|\theta) \dint \bw \nonumber \\
  &= \log \int \prod_{i=1}^n \prod_{j=1}^m \Big( g^{-1}\big(\alpha_k + 
  \greyoverbrace{f_k(x_i)}{\hidewidth\sum_{i'=1}^n h_\eta(x_i,x_{i'})w_{i'}\hidewidth}
  \,\big)_{k=1}^m \Big)^{[y_i=j]} \cdot \phi(\bw|\bzero,  \bPsi \otimes \bI_n ) \dint \bw \label{eq:intractablelikelihood}
\end{align}
where we have denoted the probit relationship from \eqref{eq:pij} using the function $g^{-1}:\bbR^m \to [0,1]$.
Unlike in the continuous response models, the integral does not present itself in closed form due to the conditional categorical PMF of the $y_i$'s, which they themselves involve integrals of normal densities.
Furthermore, the posterior distribution of the regression function, which requires the density of $\bw|\by$, depends on the marginalisation provided by \cref{eq:intractablelikelihood}.
The challenge of estimation is then to first overcome this intractability by means of a suitable approximation of the integral.
We present three possible avenues to achieve this aim, namely the Laplace approximation, Markov chain Monte Carlo (MCMC) methods, and  variational Bayes.

%Methods of approximating the integral in \eqref{eq:intractablelikelihood} such as quadrature methods, Laplace approximation and MCMC tend to fail or are unsatisfactorily slow to accomplish.
%The main reason for this is the dimension of this integral, which is $nm$, and in particular, for large sample sizes and/or number of classes, is unfeasible for such methods.

\subsection{Laplace approximation}

One is interested in the posterior density $p(\bw|\by) \propto p(\by|\bw)p(\bw) =: e^{Q(\bw)}$, with normalising constant equal to the marginal density of $\by$, $p(\by) = \int e^{Q(\bw)} \dint \bw$, and we have established that the calculation of this marginal density is intractable.
Laplace's method \citep[§4.1.1, pp. 777--778]{kass1995bayes} entails expanding a Taylor series for $Q$ about its posterior mode $\hat\bw = \argmax_\bw p(\by|\bw)p(\bw)$, and this gives the relationship
\begin{align*}
  Q(\bw) 
  &= Q(\hat\bw) + 
  \cancelto{0}{(\bw - \hat\bw)^\top \nabla Q(\hat\bw)} 
  - \half (\bw - \hat\bw)^\top \bOmega (\bw - \hat\bw) + \cdots \\
  &\approx Q(\hat\bw) + 
  - \half (\bw - \hat\bw)^\top \bOmega (\bw - \hat\bw)
\end{align*}
because, assuming that $Q$ has a unique maxima, $\nabla Q$ evaluated at its mode is zero.
This is recognised as the logarithm of an unnormalised Gaussian density, implying $\bw|\by \sim \N_n(\hat\bw,\bOmega^{-1})$.
Here, $\bOmega = -\nabla^2 Q(\bw)|_{\bw=\hat\bw}$ is the negative Hessian of $Q$ evaluated at the posterior mode.

The marginal distribution is then approximated by
\begin{align*}
  p(\by) 
  &\approx \int \exp
  \greyoverbrace{Q(\bw)}{\hidewidth Q(\hat\bw) - \half (\bw - \hat\bw)^\top \bOmega (\bw - \hat\bw)\hidewidth}
   \dint \bw \\
  &= (2\pi)^{n/2} \abs{\bOmega}^{-1/2} e^{Q(\hat\bw)} 
  \int (2\pi)^{-n/2} \abs{\bOmega}^{1/2} \exp \left(- \half (\bw - \hat\bw)^\top \bOmega (\bw - \hat\bw) \right) \dint\bw \\
  &= (2\pi)^{n/2} \abs{\bOmega}^{-1/2} p(\by|\hat\bw)p(\hat\bw).
%  &= \cancelto{1}{\int \phi(\bw|\hat\bw,\bOmega^{-1})}
\end{align*} 
The log marginal density of course depends on the parameters $\theta$, which becomes the objective function to maximise in a likelihood maximising approach.
Note that, should a fully Bayesian approach be undertaken, i.e. priors prescribed on the model parameters using $\theta \sim p(\theta)$, then this approach is viewed as a maximum a posteriori approach.

In fact, under an EM algorithm approach, using the approximate posterior density which is normally distributed is simply using the posterior mode in lieu of the actual posterior means.
\hltodo[Expand on this further.]{}

In any case, each evaluation of the objective function $L(\theta) = \log p(\by|\btheta)$ involves finding the posterior modes $\hat\bw$.
This is a slow and difficult undertaking, especially for large sample sizes $n$, because the dimension of this integral is exactly the sample size.
Furthermore, standard errors for the parameters are cumbersome to calculate as well.
Lastly, as a comment, Laplace's method only approximates the true marginal likelihood well if the true function is small far away from the mode.

% IT TURNS OUT THE EM IS DIFFICULT! BECAUSE THERE IS NO INHERENT ASSUMPTION OF INDEPENDENCE BETWEEN YSTAR AND W, SO THE DISTRIBUTIONS ARE DIFFICULT TO ASCERTAIN. 
% IN THE VARIATIONAL ALGORITHM, THIS IS ASSUMED IN A MEAN-FIELD APPROXIMATION
%\subsection{Expectation-maximisation algorithm}
%
%An EM algorithm similar to the one seen in the previous Chapter can be employed, with a slight modification.
%This time, treat both the latent propensities $\by^*$ and the I-prior random effects $\bw$ as `missing', so the complete data is $\{\by,\by^*,\bw\}$.
%Now, due to the independence of the observations $i=1,\dots,n$, the complete data log-likelihood is
%\begin{align*}
%  \log p&(\by,\by^*,\bw) \\
%  &= \sum_{i=1}^n \Big\{ 
%  \log p(y_i|\by^*_{i \bigcdot}) + \log p(\by^*_{i \bigcdot}|\bw_{i \bigcdot}) + \log p(\bw_{i \bigcdot}) 
%  \Big\} \\
%  &= - \half \sum_{i=1}^n \ind[y_{ij}^* = \max_k y_{ik}^*] \Bigg[
%%   \cancel{\half\log\abs{\bPsi}} 
%    (\by^*_{i \bigcdot} - \balpha - \bw_{i \bigcdot}^\top\bH_\eta)^\top \bPsi (\by^*_{i \bigcdot} - \balpha - \bw_{i \bigcdot}^\top\bH_\eta)
%   +  \bw_{i \bigcdot}^\top \bPsi^{-1} \bw_{i \bigcdot} \Bigg] \\
%  &\phantom{==} 
%%  \cancel{- \half\log\abs{\bPsi}} 
%   + \const
%\end{align*}
%which looks like the complete data log-likelihood seen previously in \cref{eq:QfnEstep}, except that here, together with the $\bw_{i \bigcdot}$'s, the $\by^*_{i \bigcdot}$'s are never observed.
%
%For the E-step, it is of interest to determine the posterior density $p(\by^*,\bw|\by) = p(\by^*|\bw,\by)p(\bw|\by) = p(\bw|\by^*,\by)p(\by^*|\by)$.
%For the latent propensities, the conditional posterior mean of a normally distributed subject to a conical truncation $\cC_j = \{y_{ij}* > y_{ik}^* \,|\, \forall k \neq j \}$, i.e. $\by^*_{i \bigcdot}|\bw_{i \bigcdot},\{y_i=j\} \iid \tN_m(\balpha + \hat\bw_{i \bigcdot}^\top\bH_\eta, \bPsi^{-1},\cC_j)$, for each $i=1,\dots,n$, and $\hat\bw$ is the posterior mean of $\bw$.
%The distribution for $\bw|\by^*$ is found to be similar to the posterior distribution of $\bw$ in the normal case as in \cref{eq:posteriorw}, except with $\by$ replaced with $\by^*$.
%To be specific, $\vecc \bw | \by^* \sim \N_{nm}(\vecc \tilde\bw, \tilde \bV_w)$, where
%\begin{align*}
%  \vecc \tilde\bw = \tilde\bV_w (\bPsi \otimes \bH_\eta) \vecc(\by^* - \bone_n\balpha^\top)
%  \hspace{0.5cm}\text{and}\hspace{0.5cm}
%  \tilde \bV_w^{-1} = (\bPsi \otimes \bH_\eta^2) + (\bPsi^{-1} \otimes \bI_n).
%\end{align*}
%To obtain the first and second posterior moments for the I-prior random effects, use the law of total expectations:
%\begin{gather*}
%  \E[\vecc \bw|\by]
%  = \E \big[ \E[\vecc \bw | \by^*] \big| \by \big] =: \hat\bw \\
%  \text{and}\\
%  \E[\vecc \bw (\vecc \bw)^\top|\by]
%  = \E \big[ \E[\vecc \bw (\vecc \bw)^\top| \by^*] \big| \by \big] =: \hat\bW .
%\end{gather*}
%The $Q$ function, whose argument is $\theta = \{\balpha,\eta,\bPsi\}$, is then 
%\begin{align*}
%  Q(\theta) 
%  &= \E_{\by^*,\bw}\big[ \log p&(\by,\by^*,\bw) | \by,\theta^{(t)}\big] \\
%  &= \const - \half 
%\end{align*}
%

\subsection{Markov chain Monte Carlo methods}

\citet{albert1993bayesian} showed that the latent variable approach to probit models can be analysed using exact Bayesian methods, due to the underlying normality structure.
Paired with corresponding conjugate prior choices, sampling from the posterior is very simple using a Gibbs sampling approach.
On the other hand, this data augmentation scheme enlarges the variable space to $n+q$ dimensions, where $q$ is the number of parameters to estimate, which is inefficient and computationally challenging especially when $n$ is large.
It is no longer possible to marginalise the normal latent variables from the model, as this is intractable, just as we discussed previously.

Hamiltonian Monte Carlo is another possibility, since it does not require conjugacy.
For binary models, this is a feasible approach because the class probabilities are a function of the normal CDF, which means that it is doable in off-the-shelf software such as \proglang{Stan}.
Things get out of hand with multinomial responses, because the intractability of computing class probabilities is not addressed.

In summary, the computational challenge here stems from two sources: 1) integrating out the random effects $\bw$; and 2) evaluating class probabilities.
Point 1) is addressed using a Gibbs sampling data augmentation scheme (latent variable approach), but this is not feasible with large $n$.
Point 2) remains regardless whether Gibbs sampling or HMC is used.



\subsection{Variational inference}

We turn to variational inference as a method of estimation. Variational methods are widely discussed in the machine learning literature, and there have been efforts to popularise it in statistics \citep{blei2017variational}.

By factorising appropriately, we can obtain approximated posteriors for the regression function and the parameters of the I-prior model.
The algorithm itself typically condenses to that of a simple, sequential updating scheme, akin to the expectation-maximisation (EM) algorithm for exponential families we saw in Chapter 4, which is very fast to implement compared to the other methods described in the previous subsections.
A full derivation of the variational algorithm used by us will be described in \cref{sec:iprobitvar}.
%\citep{mclachlan2007algorithm}

\subsection{Comparison of estimation methods}

\hltodo{Compare: Laplace, variational and HMC.}

The three estimation methods described aim to overcome the intractable integral by means of either a deterministic approximation (Laplace and variational inference) or a stochastic approximation (MCMC).
In the Laplace and variational method, the posterior distribution of $\bw$ ends up being approximated by a Gaussian distribution, although the mean and variance is different in each method.
In essence, once $\bw|\by$ is approximately normal, then estimation of the parameters $\theta$ using a direct optimisation approach or an EM-type approach is straightforward.
On the other hand, MCMC approximates the density $p(\bw|\by)$ using samples generated via Gibbs sampling or HMC, and these samples would asymptotically be representative of draws from the true posterior.

Consider the data set...
Plot the data. Explain priors for HMC and variational. Compare.





\section{A variational algorithm}\label{sec:iprobitvar}
We present a variational inference algorithm to estimate the parameters of interest and values for the latent variables.
Begin by assuming some prior distribution on the parameters $p(\theta) = p(\balpha)p(\eta)p(\bSigma)$. 
Although one may devote more attention to the prior specification of these parameters, for our purposes it suffices that they are independent component-wise, and the PDFs belong to the exponential family of distributions with known hyperparameters.
The exponential family requirement greatly eases the complexity of deriving the variational algorithm later on\footnote{
Of interest, one may even opt to assign improper priors on $\theta$ and the algorithm would still work.
This is akin to obtaining empirical Bayes estimate of the $\theta$ if seen from an EM algorithm standpoint.
}.

Recall the fact that $\bff = \bG\bw$ from \eqref{eq:ipriorw}, where $\bw \sim \N_{nm}(\bzero, \bSigma^{-1} \otimes \bI_n)$.
The required posterior distribution is then $
  p(\by^*,\bw,\theta|\by) \propto p(\by|\by^*)p(\by^*|\bw,\theta)p(\bw|\theta)p(\theta)
$.
This is approximated by a mean-field distribution of the form $q(\by^*,\bw,\theta) \equiv q(\by^*)q(\bw)q(\theta)$, and also $q(\theta) = q(\balpha)q(\eta)q(\bSigma)$.
Denote by $\tilde q$ the distributions which minimise the Kullbeck-Leibler divergence (maximise the variational lower bound).
By appealing to \cite[equation 10.9, p.466]{bishop2006pattern}, we find that for each $\xi \in \{ \by^*,\bw,\theta \}$, $\tilde q$ satisfies
\begin{align}\label{eq:qtilde}
  \log \tilde q(\xi) = \E_{-\xi} [\log p(\by, \by^*, \bw, \theta)] + \const
\end{align}
where expectation of the log joint density of $(\by, \by^*, \bw, \theta)$ is taken with respect to all of the parameters except the one currently in consideration.
Estimates of the parameters are then obtained by taking the mean of this approximate posterior distribution.
In practice, rather than an explicit calculation of the normalising constant, one simply needs to inspect the equation \eqref{eq:qtilde} to recognise it as a known log-density function, which is the case when exponential family distributions are considered.
In other situations, it is possibly to perform some form of sampling method (such as a Metropolis random walk) to obtain quantities of interest, for example $\E[\xi]$.

\begin{figure}[t]
  \centering
  \begin{tikzpicture}[scale=1.1, transform shape]
    \tikzstyle{main}=[circle, minimum size=10mm, thick, draw=black!80, node distance=16mm]
    \tikzstyle{connect}=[-latex, thick]
    \tikzstyle{box}=[rectangle, draw=black!100]
      \node[main, draw=black!0] (blank) [xshift=-0.55cm] {};  % pushes image to right slightly
      \node[main, fill=black!10] (H) [] {$x_i$};
      \node[main] (Sigma) [below=of H, yshift=-1.2cm, xshift=0.6cm] {$\bSigma$};
      \node[main, double, double distance=0.6mm] (f) [right=of H, xshift=0.5cm] {$f_{ij}$};
      \node[main, double, double distance=0.6mm] (ystar) [right=of f, xshift=0cm] {$y_{ij}^*$};
      \node[main, double, double distance=0.6mm] (pij) [right=of ystar, xshift=0cm] {$p_{ij}$};
      \node[main] (lambda) [above=of f, xshift=0cm, yshift=-0.3cm] {$\eta_j$};        
      \node[main] (alpha) [above=of ystar, xshift=0cm, yshift=-0.3cm] {$\alpha_j$};  
      \node[main, fill = black!10] (y) [right=of pij, xshift=0.2cm] {$y_{i}$};
      \node[main] (w) [below=of f, yshift=0.3cm] {$w_{ij}$};  
      \path (alpha) edge [connect] (ystar)
            (lambda) edge [connect] (f)
            (H) edge [connect] node [above] {$h \ \ $} (f)
    		(f) edge [connect] (ystar)
    		(ystar) edge [connect] node [above] {$g^{-1}$}  (pij)
            (pij) edge [connect] (y)
            (Sigma) edge [connect] (w)
    		(w) edge [connect] (f);
      \node[rectangle, draw=black!100, fit={($(H.north west) + (0.2,0cm)$) ($(y.north east) + (-0.2,0.4cm)$) (w)}] {}; 
      \node[rectangle, fit= (w) (y), label=below right:{$i=1,\dots,n$}, xshift=0.95cm, yshift=0.5cm] {};  % the label
      \node[rectangle, draw=black!100, fit={(lambda) ($(pij.north east) + (0.5cm,0.7cm)$) ($(w.south west) + (-0.5,-0.7cm)$)}] {}; 
      \node[rectangle, fit={(f) ($(ystar.north east) + (0.5cm,0.7cm)$) ($(w.south west) + (-0.5,-0.7cm)$)}, label=below right:{$j=1,\dots,m$}, xshift=0.63cm, yshift=0.37cm] {}; 
    \end{tikzpicture}
    \caption{A DAG of the probit I-prior  model. Observed nodes are shaded, while double-lined nodes represented known or calculable quantities. There are at most $m-1$ sets of intercept ($\alpha_j$) and RKHS parameters ($\eta_j$)  to estimate due to identifiability. Depending on the specification of $\bSigma$, this may need to be estimated too.}
\end{figure}

We now present the $\tilde q$ distributions, which we call the posteriors, instead of the mean-field variational densities, when it is unambiguous to do so.
A note on notation: We will typically refer to posterior means of the parameters $\by^*$, $\bw$, $\theta$ and so on by use of a tilde.
For instance, we write $\tilde\bw$ to mean $\E_{\tilde q}[\bw]$, the expected value of $\bw$ under the pdf $\tilde q(\bw)$.

Write $\by_i^* = (y_{i1}^*,\dots,y_{im}^*)^\top$.
Due to the independence of each $\by_{i}^*|\theta,x_i \sim \N_m(\balpha + \bff(x_i),\bSigma)$, we have an induced factorisation of the posterior $q(\by^*) = \prod_{i=1}^n q(\by_i^*)$.
Each $q(\by_i^*)$ follows a \emph{conically truncated multivariate normal disribution}, i.e., for $i=1,\dots,n$, $\by_i^*$ is distributed according to
\begin{align}\label{eq:ystardist}
  \by_i^* \iid
  \begin{cases}
    \N_m(\tilde \alpha + \tilde \bff(x_i), \tilde\bSigma) & \text{ if } y_{ij}^* > y_{ik}^*, \forall k \neq j \\
    0 & \text{ otherwise}. \\
  \end{cases}
\end{align}
The posterior mean for the latent variables, $\tilde\by_i^* = (\tilde y_{i1}^*,\dots,\tilde y_{im}^*)$ depends on the value observed for $y_i \in \{1,\dots,m\}$. 
The expected value $\tilde\by_i^*$ for this truncated multinormal variable is tricky to compute.
One strategy might be Monte Carlo integration---using samples from $\N_m(\tilde \alpha + \tilde \bff(x_i), \tilde\bSigma)$, zero out those that do not satisfy the condition $y_{ij}^* > y_{ik}^*, \forall k \neq j$, then take the sample average.
If the independent I-probit is considered, the expected value can be considered component-wise, where each component of this expectation is given by
\begin{align}\label{eq:ystarupdate}
  \tilde y_{ik}^* =
  \begin{cases}
    \tilde\alpha_k + \tilde f_{ik} - \sigma_k C_i^{-1} \displaystyle{  \int \phi_{ik}(z) \prod_{l \neq k,j} \Phi_{il}(z) \phi(z) \d z }
    &\text{ if } k \neq y_i \\[1.5em]
    \tilde\alpha_{y_i} + \tilde f_{iy_i} - \sigma_{y_i} \sum_{k \neq y_i} \big(\tilde y_{ik}^* - \tilde f_{ik} \big) 
    &\text{ if } k = y_i \\
  \end{cases}
\end{align}
with 
\begin{align*}
  \phi_{ik}(Z) &= \phi \left(\frac{\sigma_{y_i}}{\sigma_k} Z + \frac{\tilde\alpha_{y_i} + \tilde f_{iy_i} - \tilde\alpha_k - \tilde f_{ik}}{\sigma_k} \right) \\
  \Phi_{ik}(Z) &= \Phi \left(\frac{\sigma_{y_i}}{\sigma_k} Z + \frac{\tilde\alpha_{y_i} + \tilde f_{iy_i} - \tilde\alpha_k - \tilde f_{ik}}{\sigma_k} \right) \\
  C_i &= \int \prod_{l \neq j} \Phi_{il}(z) \phi(z) \d z
\end{align*}
and $Z \sim \N(0,1)$ with PDF and CDF $\phi(\cdot)$ and $\Phi(\cdot)$ respectively. 
The integrals that appear above are functions of a unidimensional Gaussian pdf, and these can be computed rather efficiently using quadrature methods.

This time, collect all $n$ latent observations for each class and write ${\by_j^*}' = (y_{1j}^*,\dots,y_{nj}^*)^\top$ and $\bw_j = (w_{1j},\dots,w_{nj})$, $j=1,\dots,m$.
Denote also $\by^*$ and $\bw$ as the vectors concatenating all $j$ vector components which results in a vector of length $nm$ for each.
By doing so, we have that $\bff = \bG\bw$ from \eqref{eq:ipriorw}, where $\bw \sim \N_{nm}(\bzero, \bSigma^{-1} \otimes \bI_n)$ and thus $\by^* \sim \N_{nm}(\balpha + \bG\bw, \bSigma \otimes \bI_n)$.
With these two Gaussian densities, we find that the posterior for $\bw$ is also Gaussian with $\tilde q(\bw) \equiv \N_{nm}(\tilde \bw, \tilde\bV_w)$, where
\[
  \tilde \bw = \tilde \bV_w\tilde\bG (\tilde\bSigma^{-1} \otimes \bI_n) (\tilde\by^* - \tilde\balpha) \ \text{ and } \ \tilde \bV_w = \big(\tilde\bG(\tilde\bSigma^{-1} \otimes \bI_n)\tilde\bG + (\tilde\bSigma \otimes \bI_n) \big)^{-1}
\]
This multivariate normal of dimension $nm$ could present a challenge to work with, in particular the matrix inverse process required in calculating the posterior covariance matrix $\bV_w$.
Note that $\bG(\bSigma^{-1} \otimes \bI_n)\bG$ is in fact the covariance matrix for the I-prior, and has the structure of $\bV_f$ in equation \eqref{eq:vf}.
If the independent I-probit model is assumed, then the posterior covariance matrix $\bV_w$ has a simpler structure as the covariance terms disappear, which implies that the components $\bw_j$ would be independently distributed. 
This results in $m$ of these $n$-variate Gaussian distributions with mean and covariance matrix given by
\[
  \tilde \bw_j = \sigma_j^{-2}\tilde \bV_{w_j}\tilde\bH_{\eta_j} (\tilde\by^*_j - \tilde\alpha_j\bone_n) \ \text{ and } \ \tilde \bV_w = \big(\sigma_j^{-2}\tilde\bH_{\eta_j}^2 + \sigma_j^{2}\bI_n \big)^{-1}.
\]
Each of these covariance matrices require $O(n^3)$ to compute and there are $m$ of them, so in total $O(mn^3)$ computational time is required.
This is much less time than the $O(m^3n^3)$ required than the full I-probit model.\hltodo[I think this can be improved by exploiting matrix normal distributions and Kronecker products, but don't know how yet.]{}

%This time, collect all values of $w_{ij}$ into a matrix $\bw$ of dimensions $n \times m$.
%The posterior for $\bw$ is said to follow a \emph{matrix normal} distribution with mean $\tilde \bw$ and scale matrices $\bS_1$ and $\bS_2$, with dimensions $n \times n$ and $m \times m$ respectively.
%This is written as $\bw \sim \MN_{n,m}(\tilde \bw, \bS_1, \bS_2)$.
%The values of this posterior distribution are found to be
%\begin{align*}
%  \begin{gathered}
%    \tilde \bw = \bS_2^{-1} 
%    \tilde\bSigma^{-1} (\tilde\by^* - \tilde\balpha) \bH_{\eta_j}
%%    \begin{pmatrix}
%%      \bH_{\eta_1} \cdots \bH_{\eta_m}
%%    \end{pmatrix} 
%     \\
%    \bS_1 = \bI_n \\
%    \bS_2 =
%  \end{gathered}
%\end{align*}
%where $\bG = 
%\begin{pmatrix}
%  \bH_{\eta_1} \cdots \bH_{\eta_m}
%\end{pmatrix}_{n \times nm} 
%\times 
%\begin{pmatrix}
%  \bI_n \cdots \bI_n
%\end{pmatrix}_{nm \times n}^\top = \sum_{k=1}^m \bH_{\eta_k}$ is a $n \times n$ matrix.

The posterior density $\tilde q$ involving the RKHS parameters is of the form
\[
  \log\tilde q(\eta) =  -\half\E_{-\eta} \Big[ (\by^* - \balpha - \bG_\eta\bw)^\top (\bSigma^{-1} \otimes \bI_n) (\by^* - \balpha - \bG_\eta\bw) \Big] + \log p(\eta) + \const,
\]
where $p(\eta)$ is a prior distribution for $\eta$.
The RKHS parameters are contained in the kernel matrices within the $\bG_\eta$ matrix, and the subscript $\eta$ emphasises this fact.
The relevance of exponential family distributions for the priors are seen here---if the prior is normal, for example, then the PDF $\tilde q(\eta)$ is also normal.
Alternatively, samples $\eta^{(1)},\dots,\eta^{(T)}$ from $\tilde q(\eta)$ may be obtained using a Metropolis algorithm, and quantities such as $\tilde\bH_{\eta_j} = \E_q[\bH_{\eta_j}]$ may be approximated using $\frac{1}{T}\sum_{t=1}^T \tilde\bH_{\eta_j^{(t)}}$.
We note also that the RKHS parameters may be considered by class if the independent I-probit model is considered.

Moving on to the estimation of the covariance matrix $\bSigma$.
For this part, we revert back to considering the IID observations $\by_i^* = (y_{i1}^*,\dots,y_{im}^*)^\top \sim \N_m(\balpha - \bff(x_i), \bSigma)$. 
Create new random variables $\bu_1,\dots,\bu_n$ defined by $\bu_i = \bSigma\bP\bw$, where $\bw^\top = (\bw_1^\top,\dots,\bw_m^\top)$ is as before.
Since the vector $\bw$ is sorted according to class instead of observations, an appropriate permutation matrix $\bP$ is required to match this with the $\bu_i$'s.
Specifically, the permutation matrix is such that if $\bu \sim \N_{nm}(\bzero, \bSigma^{-1} \otimes \bI_n)$, then $\bP\bu \sim \N_{nm}(\bzero, \bI_n \otimes\bSigma^{-1})$.
Therefore, $\bu_i \iid \N_m(\bzero, \bSigma)$, and the posterior mean and variance for $\bu_i$ may be given as $\tilde \bu_i = \tilde\bSigma\bP_i\tilde\bw$ and $\tilde\bV_{u_i} = \tilde\bSigma\bP_i\tilde\bV_w\bP_i^\top\tilde\bSigma \vphantom{\iid}$.
This step is required so that the resulting posterior for $\bSigma$ is conjugate with an inverse Wishart prior on $\bSigma \sim \invWis(\bA, a)$.
We obtain an inverse Wishart distribution for $\bSigma$, i.e. $\tilde q(\bSigma) \equiv \invWis(\bA_1 + \bA_2 + \bA, 2nm + a)$, where
\begin{align*}
  \begin{gathered}
  \bA_1 = \E_{\by^*,\bw,\balpha} \left[ \sum_{i=1}^n (\by^*_i - \balpha - \bff(x_i))(\tilde\by^*_i - \balpha - \bff(x_i))^\top \right]   \\
  \bA_2 = \sum_{i=1}^n \left[\bu_i\bu_i^\top + \tilde\bV_{u_i} \right].
  \end{gathered}
\end{align*}
The challenge here is that it involves the second posterior moment of the conically truncated multivariate normal distribution for $\by^*$, which may be obtained by sampling or numerical integration as described earlier.

As a remark on identifiability, we might require that one of the components of $\bSigma$ be fixed, e.g. $\bSigma_{11} = 1$.
\cite{mcculloch2000bayesian} gives a Gibbs sampling algorithm which we find useful for our variational algorithm as well.
Partition $\bSigma$ as follows:
\[
  \bSigma = \begin{pmatrix}
    1       &\bzeta^\top \\
    \bzeta  &\bZ + \bzeta\bzeta^\top
  \end{pmatrix}.
\]
We can choose the priors $\bzeta \sim \N_{m-1}(\bzero,\bB)$ and $\bZ \sim \invWis(\bA,a)$, independent of each other with appropriately chosen hyperparameters.
Define $\tilde\bepsilon_i = \by^*_i - \tilde\balpha - \tilde\bff(x_i) = (\tilde\epsilon_{i1},\dots,\tilde\epsilon_{im})^\top$, and let $\upsilon_i = \surd 2 \tilde\epsilon_{i1}$ and $\bnu_i = \surd 2(\tilde\epsilon_{i2},\dots,\tilde\epsilon_{im})^\top$ such that $\surd 2\tilde\epsilon_i^\top = (\upsilon_i, \bnu_i^\top)^\top$.
The posterior distribution $\tilde q$ for $\bZ$ is inverse Wishart with scale equal to $\bA_3 + \bA$, where\hltodo[These equations can be simplified further.]{}
\begin{align*}
  \begin{gathered}
    \bA_3 = \E_{\bnu,\upsilon_i,\zeta} \left[ \sum_{i=1}^n (\bnu_i - \upsilon_i \bzeta)(\bnu_i - \upsilon_i \bzeta)^\top \right]   \\
    \end{gathered}
\end{align*}
and $2n(m-1)+a$ degrees of freedom.
The posterior distribution $\tilde q$ for $\bzeta$ is normal with mean and variance $\tilde\bzeta = \frac{1}{n}\sum_{i=1}^n \tilde\upsilon_i \tilde\bV_\zeta \tilde\bZ\tilde\bnu_i$ and $\tilde\bV_\zeta = (\tilde\bupsilon\tilde\bupsilon^\top\tilde\bZ^{-1} +\bB)^{-1}$.

If the independent I-probit model is considered, then $\bSigma = \diag(\sigma_1^2,\dots,\sigma_m^2)$, class independence holds so we can use independent inverse gamma distributions as a prior for $\bSigma$, i.e. $ p(\bSigma) = \prod_{j=1}^m  p(\sigma_j^2)$, where each $p(\sigma_j) \equiv \Gamma^{-1}(r,s)$.
The posterior for $\bSigma$ will also be of a similar factorised form , namely $\tilde q(\bSigma) = \prod_{j=1}^m \tilde q(\sigma_j^2)$, where $\tilde q(\sigma_j^2)$ is the PDF of an inverse gamma distribution with shape and scale parameters $\tilde r = 2n+r-1$ and $\tilde s = \half\Vert \tilde\by^*_j - \tilde\alpha_j - \tilde\bff_j \Vert^2 + \half \Vert \tilde\bu_j \Vert^2 + s$ respectively.

Finally, the posterior distribution for the intercepts follow a normal distribution should the prior specified on the intercepts also be a normal distribution, e.g. $\balpha \sim \N_m(\bzero,\bA)$.
The posterior mean and variance for the intercepts are given by
\[
  \tilde\balpha = \tilde\bV_\alpha \tilde\bSigma^{-1}\big(\tilde\by_i^* - \tilde\bff(x_i)\big) \ \text{ and } \ \tilde\bV_\alpha = \big(n\tilde\bSigma^{-1} + \bA^{-1}\big)^{-1}.
\]

Note that the evaluation of each of the component of the posterior depends on some of the components itself, and so this circular dependence is dealt with by using some arbitrary starting values and after which an iterative updating scheme of the components ensues.
The updating scheme is performed until a maximum number of iterations is reached, or ideally until some of convergence criterion is met.
In variational inference, the \emph{variational lower bound} is typically used to asses convergence.
The lower bound is given by
\begin{align*}
  \cL 
  &= \int q(\by^*,\bw,\theta) \log \left[ \frac{p(\by,\by^*,\bw,\theta)}{q(\by^*,\bw,\theta)} \right] \d\by^* \d\bw \d\theta \\
  &= \E[\log p(\by,\by^*,\bw,\theta)] - \E[\log q(\by^*,\bw,\theta)].
\end{align*}
These are calculable once the posterior distributions $\tilde q$ are known---the first term is the expectation of the logarithm of the joint density, whereas the second term factorises into the entropy of its individual components.
Similar to the EM algorithm, this quantity is\hltodo[Proof?]{expected to increase with every iteration.}


The following pseudocode summarises the variational algorithm for I-probit models.

\algrenewcommand{\algorithmiccomment}[1]{{\color{gray}\hskip2em$\triangleright$ #1}}
\begin{algorithm}[H]
\caption{VB-EM algorithm for the probit I-prior model}\label{alg:VBEM}
\begin{algorithmic}[1]
\Procedure{Initialise}{}
  \State $\bSigma^{(0)} \gets \bI_m$
  \For{$j=1,\dots,m$}
    \State Randomise $\alpha_j^{(0)}$, $\eta_j^{(0)}$, $\bw_j^{(0)}$
%    \State $\bw_j^{(0)} \gets \bzero_{n}$ %\Comment{or draw $w_i^{(0)} \ \sim \N(0,1)$ for $i=1,\dots,n$.}
    \State Calculate $\bH_{\eta_j}$ as per kernels chosen
  \EndFor
%  \State $\bW^{(t)} \gets \big(\bw_1^{(t+1)} \cdots  \bw_m^{(t+1)} \big)$
  \State $\bG^{(t)} \gets \diag(\bH_{\eta_1}, \dots, \bH_{\eta_m})$
\EndProcedure
\Statex
%\Procedure{Update for $\bff$ } {time $t$}
%  \For{$j=1,\dots,m$}
%    \State $\bff_j^{(t+1)} \gets \alpha_j^{(t)}\bone_{n} + \bH_{\eta_j}\bw_j^{(t)}$
%  \EndFor
%  \State $\bF^{(t+1)} \gets \big(\bff_1^{(t+1)} \cdots  \bff_m^{(t+1)} \big)$
%\EndProcedure
%\Statex
\Procedure{Update for $\by^*$ }{time $t$}
  \For{$i=1,\dots,n$}
    \State $j \gets y_i$
    \If{Independent I-probit model}
      \State $(y_{i1}^{*(t+1)},\dots,y_{im}^{*(t+1)}) \gets \E[\by_i^*]$ as per \eqref{eq:ystarupdate}
    \Else
      \State Sample from truncated normal as per \eqref{eq:ystardist}
      \State $(y_{i1}^{*(t+1)},\dots,y_{im}^{*(t+1)}) \gets$ sample mean
    \EndIf  
  \EndFor
\EndProcedure	
\Statex
\Procedure{Update for $\bw$ }{time $t$}
  \State $\bV_w^{(t+1)} \gets \big(\bG(\bSigma^{-1} \otimes \bI_n)\bG + (\bSigma \otimes \bI_n) \big)^{-1}$ 
  \Comment{Simpler if independent I-probit}
  \State $\bw^{(t+1)} \gets \bV_w\bG^{(t+1)} (\bSigma^{-1} \otimes \bI_n) (\by^* - \balpha)$ 
\EndProcedure	
\Statex
\Procedure{Update for $\eta$ }{time $t$}
  \State Metropolis sampling from density\vspace{-0.5em} \Comment{Simpler calculations if only RKHS scales}
  \[
    \tilde q(\eta) \propto \exp\left[ (\by^{*(t)} - \balpha^{(t)} - \bG^{(t)}\bw^{(t)})^\top (\bSigma^{-1} \otimes \bI_n) (\by^{*(t)} - \balpha^{(t)} - \bG^{(t)}\bw^{(t)}) \right] \vspace{-0.7em}
  \]
  \State $\eta^{(t+1)} \text{ and } \bG^{(t+1)} \gets$ sample mean
\EndProcedure	
\algstore{VBEMbreak1}	
\end{algorithmic}
\end{algorithm}


\begin{algorithm}[H]
\begin{algorithmic}[1]
\algrestore{VBEMbreak1}
\Procedure{Update for $\bSigma$ }{time $t$}
  \State $\bu^{(t)} \gets \bSigma^{(t)}\bP\bw$ and $\bV_u^{(t)} \gets \bSigma^{(t)}\bP\bV_w^{(t)}\bP^\top\bSigma^{(t)}$
  \State $\bA_1 \gets \sum_{i=1}^n  \E_{\by^*} \left[ (\by^*_i - \balpha^{(t)} - \bff_i^{(t)})(\by^*_i - \balpha^{(t)} - \bff_i^{(t)})^\top \right]$  
  \State $\bA_2 \gets \sum_{i=1}^n \left(\bu_i\bu_i^\top + \bV_{u_i} \right)$
  \State $\bSigma^{(t+1)} \gets (\bA_1 + \bA_2)/(2n)$ \Comment{Simpler if independent I-probit}
\EndProcedure	
\Statex
\Procedure{Update for $\balpha$ }{time $t$}
  \State $\balpha^{(t+1)} \gets \sum_{i=1}^n\big(\by_i^* - \bff^{(t)}(x_i)\big)/n $
\EndProcedure	
\Procedure{The VB-EM algorithm}{}
  \State $t \gets 0$ and initialise $\cL^{(0)}$
  \While{$\cL^{(t+1)} - \cL^{(t)} > \delta$ \textbf{or} $t < t_{max}$}{}
    \State \textbf{call} \Call{Update for $\by^*$}{}
    \State \textbf{call} \Call{Update for $\bw$}{}
    \State \textbf{call} \Call{Update for $\eta$}{}
    \State \textbf{call} \Call{Update for $\bSigma$}{}
    \State \textbf{call} \Call{Update for $\balpha$}{}
    \State \textbf{call} Calculate variational lower bound $\cL^{(t+1)}$
    \State $t \gets t + 1$
  \EndWhile
\EndProcedure
\end{algorithmic}
\end{algorithm}

\section{Post-estimation}
%For a new data point $x_{\text{new}}$, we calculate the predicted latent values $\tilde f_{\text{new}} = (\tilde f_{\text{new},1}, \dots, \tilde f_{\text{new},m})$ for each of the classes, using the variational estimates of the posterior means for the unknown quantities (denoted with tildes), as follows:
\[
  \tilde f_{\text{new},j} = \sum_{k=1}^n \tilde h_{\eta_j}(x_{\text{new}},x_k) \tilde w_{kj}, \hspace{0.5cm} j = 1,\dots,m.
\]
This is in fact the posterior mean for the regression functions evaluated at the new point $x_{\text{new}}$, which stems from a posterior normal distribution.
Denote $\bg(x_{\text{new}}) = \diag\big(\bh_{\eta_j}^\top(x_{\text{new}})\big)_{j=1}^m$, an $m \times nm$ matrix containing the kernel entries relating to this new data point.
From Gaussian process regression theory, we know that the prediction at $x_{\text{new}}$ for the latent variable $\by^*(x_{\text{new}})$ has $m$ components equal to $\tilde y_{\text{new},j}^* = \tilde\alpha_j + \tilde f_{\text{new},j}$ for $j=1,\dots,m$, and variance 
\[
  \bV_y(x_{\text{new}}) = \bg^\top(x_{\text{new}}) \tilde\bV_w \bg(x_{\text{new}}) + \tilde\bSigma.
\]
Thus, information relating to the class and its corresponding probabilities are contained within the normal distribution $\N_{m}\big(\by^*(x_{\text{new}}), \bV_y(x_{\text{new}}) \big)$, subject to the truncation $\cA_j := \{ y_{\text{new},j}^* > y_{\text{new},k}^* | \forall k \neq j\}$ if the new observation belongs to class $j \in \{1,\dots, m\}$.
The predicted class is inferred from the latent variables via
\[
  \hat y_{\text{new}} = \argmax_j \tilde y_{\text{new},j}^*, 
\]
while the probabilities for each class are once again obtained using the integral stated in \eqref{eq:pij}, stated here for convenience:
\begin{align*}
  \tilde p_{\text{new},j} 
  &= \int_{\cA_j} \N(\by^*(x_{\text{new}}), \bV_y(x_{\text{new}})) \d\by^*_{\text{new}}.
\end{align*}

For the independent I-probit model, each component of the latent variables $\by^*(x_{\text{new}})$ are calculated in a similar manner, with the difference being that the components would be independent of each other.
This is expressed in the form for the predictive covariance matrix $\bV_y(x_{\text{new}}) = \diag \big( v_1^2(x_{\text{new}}),\dots, v_m^2(x_{\text{new}})\big)$, where each variance component is given by
\[
  v_j^2(x_{\text{new}}) = \bh_{\eta_j}^\top(x_{\text{new}}) \tilde\bV_w \bh_{\eta_j}(x_{\text{new}}) + \tilde\sigma_{j}^2.
\]
Class prediction is the same as before, but class probabilities are obtained in a more compact manner via
\[
  \tilde p_{\text{new},j} 
  = \E_Z \Bigg[ \mathop{\prod_{k=1}^m}_{k\neq j} 
  \Phi \left(\frac{v_j}{v_k} Z + \frac{\tilde y_{\text{new},j}^* - \tilde y_{\text{new},k}^*}{v_k} \right) \Bigg],
\]
where $Z\sim\N(0,1)$ and $\Phi(\cdot)$ its CDF.

Working in a Bayesian framework means that we are able to perform inferences on any quantity of interest through posterior sampling.
In the I-probit model, we use the approximate mean-field densities in lieu of the true posterior densities, and these are, for the most part, easy to sample from.
Take for example the class probabilities. 
We can obtain posterior samples for the $p_{ij}$'s by firstly sampling from the underlying normal latent variables, and then passing it through the probit link function.
Taking the empirical lower 2.5th and upper 97.5th percentile of this sample would give the upper and lower values for a 95\% credibility interval.

\section{Examples}

\section{Discussion}

I-prior extended to non-normal data. 
Naive works good, but can be better.
Simply transform the normal model through a squashing function.
All the nice things about I-prior can be applied here too.
Probit model variety of binary and multinomial regression models.

Laplace slow, unreliable modes. MCMC also slow. Variational has similarity to EM, but advantageous: easier to calculate posterior s.d., ability to do inference on transformed parameters.

As with the normal model, storage and time requirements slow.
again, look to machine learning.
improvements in variational algorithm.

Extend to include class-specific covariates.

improvement in calculating the normal integral?
Need to see timing where takes longest

In terms of similarity to other works, the generalised additive models (GAMs) of \cite{hastie1986} comes close.
The setup of GAMs is near identical to the I-probit model, although estimation is done differently. 
GAMs do not assume smooth functions from any RKHS, but instead estimates the $f$'s using a local scoring method or a local likelihood method.
Kernel methods for classification are extremely popular in computer science and machine learning; examples include support vector machines \citep{scholkopf2002learning} and Gaussian process classification \citep{rasmussen2006gaussian}, with the latter being more closely related to the I-probit method.
I-priors differ from Gaussian process priors in the specification of the covariance kernel.
Gaussian process classification typically uses the logistic link function (or squashing function, to use machine learning nomenclature), and estimation is done most commonly using the Laplace approximation, but other methods such as expectation propagation \citep{minka2001expectation} and MCMC \citep{neal1999} have been explored as well.
Variational inference for Gaussian process probit models have been studied by \cite{girolami2006variational}, with their work providing a close reference to the variational algorithm employed by us.


\section{Miscellanea}
\subsection{A brief introduction to variational inference}

Suppose that, in a fully Bayesian setting, we append the unknown model parameters to the latent variables to form $\bz = \{\by^*, \bw, \theta\}$.
The crux of variational inference is this: find a suitably close distribution function $q(\bz)$ that approximates the true posterior $p(\bz|\by)$, where closeness here is defined in the Kullback-Leibler divergence sense,
\[
  \KL(q\Vert p) = \int \log \frac{q(\bz)}{p(\bz|\by)} q(\bz) \dint \bz.
\]
One may then show that log marginal density (the log of the intractable integral) holds the following bound:
    \begin{align}
      \log p(\by) &= \log p(\by,\bz) - \log p(\bz|\by) \nonumber \\
      &= \int \left\{ \log \frac{p(\by,\bz)}{q(\bz)} - \log \frac{p(\bz|\by)}{q(\bz)} \right\} q(\bz) \dint \bz \nonumber \\    
      &=  \cL(q) +  \KL(q \Vert p) \nonumber \\
      &\geq \cL(q) \label{eq:varbound}
    \end{align}
since the KL divergence is a non-negative quantity.
The functional $\cL(q)$ given by 
\begin{align}
  \cL(q) 
  &= \int \log \frac{p(\by,\bz)}{q(\bz)} q(\bz) \dint \bz \nonumber \\
%  &= \E_{\bz\sim q}[\log p(\by,\bz) - \log q(\bz)] \nonumber \\
  &= \E_{\bz\sim q}[\log p(\by,\bz)] + H(q), \label{eq:elbo1}
\end{align}
where $H$ is the entropy functional, is known as the \emph{evidence lower bound} (ELBO), which serves as the proxy objective function in the likelihood maximisation problem.
Evidently, the closer $q$ is to the true $p$, the better, and this is achieved by maximising $\cL$, or equivalently, minimising the KL divergence\footnote{The astute reader will realise that $\KL(q||p)$ is impossible to compute, since one does not know the true distribution $p(\bz|\by)$. Efforts are concentrated on maximising the ELBO instead.} from $p$ to $q$.
Note that the bound \cref{eq:varbound} achieves equality if and only if $q \equiv p$, but of course the true form of the posterior is unknown to us.
Maximising $\cL(q)$ or minimising $\KL(q\Vert p)$ with respect to the density $q$ is a problem of calculus of variations, which incidentally, is where variational inference takes its name.

Maximising $\cL$ over all possible density functions $q$ is not possible without considering certain constraints.
Two such constraints are described. 
The first, is to make a distributional assumption regarding $q$, for which it is parameterised by $\nu$.
For instance, we might choose the closest normal distribution to the posterior $p(\bz|\by)$ in terms of KL divergence.
In this case, the task is to find optimal mean and variance parameters of a normal distribution.

\begin{figure}[htb]
  \centering
  \begin{tikzpicture}
    \fill (4.05,2.1) circle (0pt) node[right,yshift=1] {$p(\bz|\by)$};
    \draw[ultra thick] (0,0) ellipse (4cm and 2.2cm);
    \node at (-2.8,0.7) {$q(\bz;\nu)$};
    \draw[thick,colred] (-0.8,-0.5) to [curve through={(-2,0) .. (-2,-0.8) .. (1,-0.8) .. (-1,1.5) .. (0,1.5) ..(0.3,0.1) .. (0.5,1.5) .. (2,0) .. (2.1,-0.1) .. (2.2,1.1) .. (3,0.9)}] (3.317,1.225);  % curve using hobby tikz
    \draw[dashed, very thick,black!50] (3.317,1.225) -- (4.05,2.1);
    \fill (-0.8,-0.5) circle (2.2pt) node[below] {$\nu^{\text{init}}$};
    \fill (3.317,1.225) circle (2.2pt) node[left,yshift=-1] {$\nu^*$};
    \fill (4.05,2.1) circle (2.2pt) node[right,yshift=1] {$p(\bz|\by)$};
    \node[gray] at (4.5,1.4) {$\KL(q\Vert p)$};
  \end{tikzpicture}
  \caption{Schematic view of variational inference. The aim is to find the closest distribution $q$ (parameterised by a variational parameter $\nu$) to $p$ in terms of KL divergence within the set of variational distributions, represented by the ellipse.}
\end{figure}

The second type of constraint, and the one considered in this thesis, is simply an assumption that the approximate posterior $q$ factorises into $M$ disjoint factors.
Supposing that the elements of $\bz$ may indeed be partitioned into $M$ disjoint groups $\bz = (z^{(1)},\dots,z^{(M)})$, then the structure
\[
  q(\bz) = \prod_{k=1}^M q_k(z^{(k)})
\]
for $q$ is considered.
This factorised form of variational inference is known in the statistical physics literature as the \emph{mean-field theory} \citep{itzykson1991statistical}.

Denote by $\tilde q$ the distributions which minimise the Kullbeck-Leibler divergence (maximise the variational lower bound).
By appealing to \citet[equation 10.9, p. 466]{bishop2006pattern}, we find that for each $\xi \in \{ \by^*,\bw,\theta \} =: \cZ$, $\tilde q$ satisfies
\begin{align}\label{eq:qtilde}
  \log \tilde q(\xi) = \E_{-\xi} [\log p(\by, \by^*, \bw, \theta)] + \const
\end{align}
where expectation of the log joint density of $(\by, \by^*, \bw, \theta)$ is taken with respect to all of the unknowns $\cZ$ except the one currently in consideration, under their respective $q$ densities. 
Estimates of the latent variables and parameters are then obtained by taking the mean of their respective approximate posterior distribution.

In practice, rather than an explicit calculation of the normalising constant, one simply needs to inspect \cref{eq:qtilde} to recognise it as a known log-density function, which is the case when exponential family distributions are considered.
That is, suppose that each complete conditional $p(\xi|\cZ_{-\xi}, \by)$ follows an exponential family distribution,
\[
  p(\xi|\cZ_{-\xi}, \by) = B(\xi)\exp \big(\ip{\zeta_\xi(\cZ_{-\xi}, \by) , \xi} - A(\zeta_\xi) \big).
\]
Then, from \cref{eq:qtilde},
\begin{align*}
  \tilde q(\xi)
  &\propto \exp\big(\E_{-\xi}[\log p(\xi|\cZ_{-\xi}, \by)] \big) \\
  &= \exp \Big(\log B(\xi) + \E \ip{\zeta_\xi(\cZ_{-\xi}, \by) , \xi} - \E [ A(\zeta_\xi) ] \Big) \\
  &\propto B(\xi)\exp \E\ip{\zeta_\xi(\cZ_{-\xi}, \by) , \xi}
\end{align*}
is also in the same exponential family.
In situations where there is no closed form expression for $\tilde q$, then one resorts to sampling methods such as a Metropolis random walk to estimate quantities of interest.
This stochastic step within a deterministic algorithm has been explored before in the context of a Monte Carlo EM algorithm---see \citet[§4, pp. 537--538]{meng1997algorithm} and references therein.

\subsection{Variational methods and the EM algorithm}

\hltodo{Goal: maximise log-likelihood wrt parameters theta.}

\subsection{The EM algorithm is intractable---variational Bayes EM?}

% IT TURNS OUT THE EM IS DIFFICULT! BECAUSE THERE IS NO INHERENT ASSUMPTION OF INDEPENDENCE BETWEEN YSTAR AND W, SO THE DISTRIBUTIONS ARE DIFFICULT TO ASCERTAIN. 
% IN THE VARIATIONAL ALGORITHM, THIS IS ASSUMED IN A MEAN-FIELD APPROXIMATION

Consider employing an EM algorithm, similar to the one seen in the previous chapter, to estimate I-probit models.
This time, treat both the latent propensities $\by^*$ and the I-prior random effects $\bw$ as `missing', so the complete data is $\{\by,\by^*,\bw\}$.
Now, due to the independence of the observations $i=1,\dots,n$, the complete data log-likelihood is
\begin{align*}
  \log p&(\by,\by^*,\bw) \\
  &= \sum_{i=1}^n \Big\{ 
  \log p(y_i|\by^*_{i \bigcdot}) + \log p(\by^*_{i \bigcdot}|\bw_{i \bigcdot}) + \log p(\bw_{i \bigcdot}) 
  \Big\} \\
  &= - \half \sum_{i=1}^n \ind[y_{ij}^* = \max_k y_{ik}^*] \Bigg[
%   \cancel{\half\log\abs{\bPsi}} 
    (\by^*_{i \bigcdot} - \balpha - \bw_{i \bigcdot}^\top\bh_\eta(x_i))^\top \bPsi (\by^*_{i \bigcdot} - \balpha - \bw_{i \bigcdot}^\top\bh_\eta(x_i)) \\
  &\hspace{9.5cm} + \bw_{i \bigcdot}^\top \bPsi^{-1} \bw_{i \bigcdot} \Bigg] 
%  \cancel{- \half\log\abs{\bPsi}} 
   + \const
\end{align*}
which looks like the complete data log-likelihood seen previously in \cref{eq:QfnEstep}, except that here, together with the $\bw_{i \bigcdot}$'s, the $\by^*_{i \bigcdot}$'s are never observed.

For the E-step, it is of interest to determine the posterior density $p(\by^*,\bw|\by) = p(\by^*|\bw,\by)p(\bw|\by)$, which apparently is hard to obtain.
We can go as far as determining that the full conditional of the latent propensities is multivariate subject to a conical truncation $\cC_j = \{y_{ij}^* > y_{ik}^* \,|\, \forall k \neq j \}$, i.e. $\by^*_{i \bigcdot}|\bw_{i \bigcdot},\{y_i=j\} \iid \tN_m(\balpha + \bw_{i \bigcdot}^\top\bh_\eta(x_i), \bPsi^{-1},\cC_j)$, for each $i=1,\dots,n$, and that $\vecc \bw|\by^*\sim \N(\tilde\bw,\tilde\bV_w)$ is found to be similar to the distribution in \cref{eq:varipostw}.
%To be specific, $\vecc \bw | \by^* \sim \N_{nm}(\vecc \tilde\bw, \tilde \bV_w)$, where
%\begin{align*}
%  \vecc \tilde\bw = \tilde\bV_w (\bPsi \otimes \bH_\eta) \vecc(\by^* - \bone_n\balpha^\top)
%  \hspace{0.5cm}\text{and}\hspace{0.5cm}
%  \tilde \bV_w^{-1} = (\bPsi \otimes \bH_\eta^2) + (\bPsi^{-1} \otimes \bI_n).
%\end{align*}
To obtain the first and second posterior moments for the I-prior random effects, we can use the law of total expectations:
\begin{gather*}
  \E[\vecc \bw|\by]
  = \E_{\by^*} \big[ \E[\vecc \bw | \by^*] \big| \by \big] =: \hat\bw \\
  \text{and}\\
  \E[\vecc \bw (\vecc \bw)^\top|\by]
  = \E_{\by^*} \big[ \E[\vecc \bw (\vecc \bw)^\top| \by^*] \big| \by \big] =: \hat\bW ,
\end{gather*}
but this requires $p(\by^*|\by)$ which does not come by easily.
A similar problem has been faced by \citet{chan1997maximum}, who analysed binary linear probit models with random effects.
The authors ultimately resort to Monte Carlo sampling within an EM framework to overcome the difficult distributions of interest.

Suppose that, instead of the true posterior distribution $p(\by^*,\bw|\by)$ being used, a mean-field variational approximation $q(\by^*,\bw) = q(\by^*)q(\bw)$ is used instead.
As we know from \hltodo{Section X}, $q(\by^*)$ is a truncated multivariate normal distribution, and $q(\bw)$ is multivariate normal, whose means and second moments can be computed with some  effort.
Let $\bar\by^* = \by^* - \bone_n\balpha^\top$.
The (approximate) E-step then entails computing
\begin{align*}
  Q(\theta) 
  &= \E_{\by^*,\bw\sim q}  \log p(\by,\by^*,\bw|\theta) \\
  &= \const -\half \tr\E_{\by^*,\bw\sim q} \left[ 
  \bPsi(\bar\by^{*\top}\bar\by^* + \bw^\top\bH_\eta^2\bw - 2\bar\by^*\bPsi\bw^\top\bH_\eta )
  + \bPsi^{-1} \bw^\top\bw 
  \right].
\end{align*}
In the M-step, this is maximised with respect to $\theta$.
An iterative procedure is a natural port of call to address the coupling in the posterior variational densities and also parameter dependencies.
This leads to the so-called \emph{variational Bayes EM algorithm} (VB-EM) \citep{beal2003}.

In variational inference, a fully Bayesian treatment of the parameters is considered, with the aim of obtaining approximation to their posterior distributions.
In VB-EM, the variational approximation is only performed on the latent, or `missing' variables, to use the EM nomenclature.
After a variational E-step, the M-step proceeds as usual, and as such, all of the material relating to the EM in the previous chapter is applicable.
The VB-EM can also be seen as obtaining (approximate) maximum a posteriori estimates with diffuse priors on the parameters.
As per the discussion in \hltodo{Section X}, this alleviates the problem of non-conjugacy of the complete conditional for $\bPsi$.

One downside to VB-EM is that it is not entirely certain how one could obtain standard errors for the parameters, other than by bootstrapping.












\ifstandalone
  \section*{Appendix}
  \section{Some distributions and their properties}

This is a reference relating to the multivariate normal, conically truncated multivariate normal, matrix normal, Wishart, and gamma distributions for that we collate for convenience.
Of interest are probability density functions, first and second moments, and entropy (as defined in \hltodo{Chapter 3}).

\subsection{Multivariate normal distribution}

Let $X\in\bbR^d$ be distributed according to a multivariate normal (Gaussian) distribution with mean $\mu \in \bbR^d$ and covariance matrix $\Sigma \in\bbR^d$ (a square, symmetric, positive-definite matrix).
We say that $X\sim\N_d(\mu,\Sigma)$.
Then,
\begin{itemize}
  \item \textbf{Pdf}. $p(X|\mu,\Sigma) = (2\pi)^{-d/2}|\Sigma|^{-1/2}\exp\big(-\half (X-\mu)^\top\Sigma^{-1}(X-\mu)\big)$.
  \item \textbf{Moments}. $\E X=\mu$, $\E [XX^\top] = \Sigma + \mu \mu^\top$.
  \item \textbf{Entropy}. $H(p) = \half \log \abs{2\pi e \Sigma} = \half[d](1 + \log 2\pi) + \half\log\abs{\Sigma}$.
\end{itemize}

\begin{lemma}[Properties of multivariate normal]
  Assume that $X \sim \N_d(\mu,\Sigma)$ and $Y \sim \N_{d}(\nu,\Psi)$, where
  \[
    X = 
    \begin{pmatrix}
      X_a \\ X_b
    \end{pmatrix},
    \hspace{0.5cm}
    \mu = 
    \begin{pmatrix}
      \mu_a \\ \mu_b
    \end{pmatrix},
    \hspace{0.25cm}\text{and}\hspace{0.25cm}
    \Sigma = 
    \begin{pmatrix}
      \Sigma_a    &\Sigma_{ab} \\
      \Sigma_{ab}^\top &\Sigma_b \\
    \end{pmatrix}.
  \]
  Then,
  \begin{itemize}
    \item \textbf{Marginal distributions}.
    \[
      X_a \sim \N_{\dim X_a}(\mu_a,\Sigma_a)
      \hspace{0.5cm}\text{and}\hspace{0.5cm}
      X_b \sim \N_{\dim X_b}(\mu_b,\Sigma_b).
    \]
    \item \textbf{Conditional distributions}.
    \[
      X_a|X_b \sim \N_{\dim X_a}(\tilde\mu_a,\tilde\Sigma_a)
      \hspace{0.5cm}\text{and}\hspace{0.5cm}
      X_b \sim \N_{\dim X_b}(\tilde\mu_b,\tilde\Sigma_b),
    \]
    where
    \begin{alignat*}{5}
      & \tilde\mu_a 
      &&= \mu_a + \Sigma_{ab}\Sigma_b^{-1}(X_b-\mu_b)
      &&\hspace{1cm}
      &&\tilde\mu_b 
      &&= \mu_b + \Sigma_{ab}^\top\Sigma_a^{-1}(X_a-\mu_a) \\
      &\tilde\Sigma_a 
      &&= \Sigma_a -  \Sigma_{ab}\Sigma_b^{-1}\Sigma_{ab}^\top
      &&\hspace{1cm}
      &&\tilde\Sigma_b 
      &&= \Sigma_b -  \Sigma_{ab}^\top\Sigma_a^{-1}\Sigma_{ab} 
    \end{alignat*}
    \item \textbf{Linear combinations}. 
    \[
      AX + BY + C \sim \N_{d}(A\mu + B\nu + C, A\Sigma A^\top + B\Psi B^\top)
    \]
    where $A$ and $B$ are appropriately sized matrices, and $C\in\bbR^d$.
    \item \textbf{Product of Gaussian densities}. 
    \[
      p(X|\mu,\Sigma)p(Y|\nu,\Psi) \propto p(Z|m,S)
    \]
    where $p(Z)$ is a Gaussian density, $m = S(\Sigma^{-1}\mu + \Psi^{-1}\nu)$ and $S= (\Sigma^{-1} + \Psi^{-1})^{-1}$.
    The normalising constant is equal to the density of $\mu\sim\N(\nu,\Sigma + \Psi)$.
  \end{itemize}
\end{lemma}

\begin{proof}
  Omitted---see \citet[§8]{petersen2008matrix}.
\end{proof}

Frequently, in Bayesian statistics especially, the following identities will be useful in deriving posterior distributions involving multivariate normals.

\begin{lemma}
  Let $x,b\in\bbR^d$ be a vector, $X,B\in\bbR^{n\times d}$ a matrix, and $A \in \bbR^{d \times d}$ a symmetric, invertible matrix.
  Then,
  \begin{align*}
    -\half x^\top A x + b^\top x 
    &= -\half (x - A^{-1}b)^\top A (x - A^{-1}b) + \half b^\top A^{-1} b \\
    -\half \tr (X^\top A X) + \tr(B^\top X)
    &= -\half \tr\big((X - A^{-1}B)^\top A(X - A^{-1}B) \big) + \half\tr(B^\top A^{-1} B).
  \end{align*}
\end{lemma}

\begin{proof}
  Omitted---see \citet[§8.1.6]{petersen2008matrix}.
\end{proof}

\subsection{Matrix normal distribution}

The matrix normal distribution is an extension of the Gaussian distribution to matrices.
Let $X\in\bbR^{n \times m}$ matrix, and let $X$ follow a matrix normal distribution with mean $\mu\in\bbR^{n \times m}$ and row and column variances $\Sigma \in \bbR^{n \times n}$ and $\Psi \in \bbR^{m \times m}$ respectively, which we denote by $X\sim\MN_{n,m}(\mu,\Sigma,\Psi)$.
Then,
\begin{itemize}
  \item \textbf{Pdf}. $p(X|\mu,\Sigma,\Psi) = (2\pi)^{-nm/2}|\Sigma|^{-m/2}|\Psi|^{-n/2} e^{-\half \tr \big(\Psi^{-1}(X-\mu)^\top\Sigma^{-1}(X-\mu)\big)}$.
  \item \textbf{Moments}. $\E X=\mu$, $\Var(X_{i \bigcdot }) = \Psi$ for $i=1,\dots,n$, and $\Var(X_{\bigcdot j}) = \Sigma$ for $j=1,\dots,m$. 
  \item \textbf{Entropy}. $H(p) = \half \log \abs{2\pi e (\Psi \otimes \Sigma)} = \half[nm](1 + \log 2\pi) + \half\log\abs{\Sigma}^m\abs{\Psi}^n$.
\end{itemize}

In the above, `$\otimes$' denotes the Kronecker matrix product defined by
\[
  \Psi \otimes \Sigma = 
  \begin{pmatrix}
    \Psi_{11}\Sigma &\Psi_{12}\Sigma &\cdots &\Psi_{1m}\Sigma \\
    \Psi_{21}\Sigma &\Psi_{22}\Sigma &\cdots &\Psi_{2m}\Sigma \\    
    \vdots & \vdots &\ddots  &\vdots \\
    \Psi_{m1}\Sigma &\Psi_{m2}\Sigma &\cdots &\Psi_{mm}\Sigma \\
  \end{pmatrix} \in \bbR^{nm\times nm}.
\]
Of use will be these properties of the Kronecker product \citep{zhang2013kronecker}.
\begin{itemize}
  \item \textbf{Bilinearity and associativity}. For appropriately sized matrices $A$, $B$ and $C$, and a scalar $\lambda$,
  \begin{align*}
    A \otimes (B + C) &= A \otimes B + A \otimes C \\
    (A + B) \otimes C &= A \otimes C + B \otimes C \\
    \lambda A \otimes B &= A \otimes \lambda B = \lambda(A \otimes B) \\
    (A \otimes B) \otimes C &= A \otimes (B \otimes C)
  \end{align*}
  \item \textbf{Non-commutative}. In general, $A \otimes B \neq B \otimes A$, but they are \emph{permutation equivalent}, i.e. $A \otimes B \neq P(B \otimes A)Q$ for some permutation matrices $P$ and $Q$.
  \item \textbf{The mixed product property}. $(A \otimes B)(C \otimes D) = AC \otimes BD$.
  \item \textbf{Inverse}. $A \otimes B$ is invertible if and only if $A$ and $B$ are both invertible, and $(A \otimes B)^{-1} = A^{-1} \otimes B^{-1}$.
  \item \textbf{Transpose}. $(A \otimes B)^\top = A^\top \otimes B^\top$.
  \item \textbf{Determinant}. If $A$ is $n\times n$ and $B$ is $m \times m$, then $\abs{A \otimes B} = \abs{A}^m \abs{B}^n$. Note that the exponent of $\abs{A}$ is the order of $B$ and vice versa.
  \item \textbf{Trace}. Suppose $A$ and $B$ are square matrices. Then $\tr (A \otimes B) = \tr A \tr B$.
  \item \textbf{Rank}. $\rank (A \otimes B) = \rank A \rank B$.
  \item \textbf{Matrix equations}. $AXB = C \Leftrightarrow (B^\top \otimes A) \vecc X = \vecc (AXB) = \vecc C$.
\end{itemize}
The vectorisation operation `$\vecc$' stacks the columns of the matrices into one long vector, for instance,
\[
  \vecc \Psi = (\Psi_{11},\dots,\Psi_{m1},\Psi_{12},\dots,\Psi_{m2},\dots,\Psi_{1m},\dots,\Psi_{mm})^\top \in \bbR^{m \times m}.
\]

\begin{lemma}[Equivalence between matrix and multivariate normal]
  $X\sim\MN_{n,m}(\mu,\Sigma,\Psi)$ if and only if $\vecc X\sim \N_{nm}(\vecc\mu,\Psi \otimes \Sigma)$.
\end{lemma}

\begin{proof}
  In the exponent of the matrix normal pdf, we have
  \begin{align*}
    -\half \tr \big(\Psi^{-1}(X-\mu)^\top &\Sigma^{-1}(X-\mu)\big) \\
    &= -\half \vecc(X-\mu)^\top \vecc(\Sigma^{-1}(X-\mu)\Psi^{-1}) \\
    &= -\half \vecc(X-\mu)^\top (\Psi^{-1} \otimes \Sigma^{-1}) \vecc(X-\mu) \\
    &= -\half (\vecc X- \vecc \mu)^\top (\Psi \otimes \Sigma)^{-1} (\vecc X- \vecc \mu).     
  \end{align*} 
  Also, $|\Sigma|^{-m/2}|\Psi|^{-n/2} = |\Psi \otimes \Sigma|^{-1/2}$.
  This converts the matrix normal pdf to that of a multivariate normal pdf.
\end{proof}

Some useful properties of the matrix normal distribution are listed:
\begin{itemize}
  \item \textbf{Expected values}.
  \begin{align*}
    \E (X-\mu)(X-\mu)^\top &= \tr(\Psi)\Sigma \in \bbR^{n\times n} \\
    \E (X-\mu)^\top(X-\mu) &= \tr(\Sigma)\Psi \in \bbR^{m\times m} \\
    \E XAX^\top &= \tr(A^\top\Psi)\Sigma + \mu A\mu^\top \\
    \E X^\top BX &= \tr(\Sigma B^\top)\Psi + \mu^\top B\mu \\   
    \E X CX &=  \Sigma C^\top\Psi  + \mu C \mu \\    
  \end{align*} 
  \item \textbf{Transpose}. $X^\top \sim \MN_{m,n}(\mu^\top, \Psi, \Sigma)$.
  \item \textbf{Linear transformation}. Let $A \in \bbR^{a \times n}$ be of full-rank $a \leq n$ and $B \in \bbR^{m \times b}$ be of full-rank $b\leq m$. Then $AXB  \sim \MN_{a,b}(\mu^\top, A\Sigma A^\top, B^\top \Psi B)$.
  \item \textbf{Iid}. If $X_i \iid \N_m(\mu,\Psi)$ for $i=1,\dots,n$, and we arranged these vectors row-wise into the matrix $X = (X_1^\top,\dots,X_n^\top)^\top \in \bbR^{n\times m}$, then $X \sim \MN(1_n \mu^\top, I_n, \Psi)$.
\end{itemize}

\subsection{Truncated univariate normal distribution}

Let $X \sim \N(\mu,\sigma^2)$ with $X$ lying in the interval $(a,b)$.
Then we say that $X$ follows a truncated normal distribution, and we denote this by $X\sim\tN(\mu,\sigma^2,a,b)$.
Let $\alpha = (a-\mu)/\sigma$, $\beta = (b-\mu)/\sigma$, and $C = \Phi(\beta) - \Phi(\alpha)$.
Then,
\begin{itemize}
  \item \textbf{Pdf}. $p(X|\mu,\sigma,a,b) = C^{-1} (2\pi\sigma^2)^{-1/2}e^{-\frac{1}{2\sigma^2} (X-\mu)^2} = \sigma C^{-1} \phi(\frac{x-\mu}{\sigma})$.
  \item \textbf{Moments}. 
  \vspace{-1.2em}
  \begin{gather*}
    \E X = \mu + \sigma \frac{\phi(\alpha) - \phi(\beta)}{C} \\
    \E X^2 = \sigma^2 + \mu^2 + \sigma^2  \frac{\alpha\phi(\alpha) - \beta\phi(\beta)}{C}   + 2\mu\sigma \frac{\phi(\alpha) - \phi(\beta)}{C} \\
    \Var X = \sigma^2 \left[ 1 +  \frac{\alpha\phi(\alpha) - \beta\phi(\beta)}{C} - \left(\frac{\phi(\alpha) - \phi(\beta)}{C}\right)^2 \right]
  \end{gather*}
  \item \textbf{Entropy}.
  \begin{align*}
    H(p) 
    &= \half\log 2\pi e\sigma^2 + \log C + \frac{\alpha\phi(\alpha) + \beta\phi(\beta)}{2C} \\
    &= \half\log 2\pi e\sigma^2 + \log C + \frac{1}{2\sigma^2}\cdot \greyoverbrace{\sigma^2\frac{\alpha\phi(\alpha) + \beta\phi(\beta)}{C}}{\Var X -\sigma^2 + (\E X - \mu)^2} \\
    &= \half\log 2\pi \sigma^2 + \log C + \frac{1}{2\sigma^2}\E [X - \mu]^2 
  \end{align*}
  because $\Var X + (\E X - \mu)^2 = \E X^2 - \cancel{(\E X)^2} + \cancel{(\E X)^2} + \mu^2 - 2\mu\E X.$
\end{itemize}

For binary probit models, the distributions that come up are one-sided truncations at zero, i.e. $\tN(\mu,\sigma^2,0,+\infty)$ (upper tail/positive part) and $\tN(\mu,\sigma^2,-\infty,0)$ (lower tail/negative part), for which their moments are of interest.
As an aside, if $\mu = 0$ then the truncation $\tN(0,\sigma^2,0,+\infty)$ is called the \emph{half-normal} distribution.
For the positive one-sided truncation at zero, $C = \Phi(+\infty) - \Phi(-\mu/\sigma) = 1 - \Phi(-\mu/\sigma) = \Phi(\mu/\sigma)$, and for the negative one-sided truncation at zero, $C = \Phi(-\mu/\sigma) - \Phi(-\infty) = 1 - \Phi(\mu/\sigma)$.

One may simulate random draws from a truncated normal distribution by drawing from $\N(\mu,\sigma^2)$ and discarding samples that fall outside $(a,b)$.
Alternatively, the inverse-transform method using
\[
  X = \mu + \sigma\Phi^{-1}\left( \Phi(\alpha) + UC \right)
\]
with $U\sim\Unif(0,1)$ will work too.
Either of these methods will work reasonably well as long as the truncation region is not too far away from $\mu$, but neither is particularly fast.
Efficient algorithms have been explored which are along the lines of either accept/reject algorithms \citep{robert1995simulation}, Gibbs sampling \citep{damien2001sampling}, or pseudo-random number generation algorithms \citep{chopin2011fast}.
The latter algorithm is inspired by the Ziggurat algorithm \citep{marsaglia2000ziggurat} which is considered to be the fastest Gaussian random number generator.


\subsection{Truncated multivariate normal distribution}

Consider the restriction of $X\sim \N_d(\mu,\Sigma)$ to a convex subset\footnote{A convex subset is a subset of a space that is closed under convex combinations. In Euclidean space, for every pair of points in a convex set, all the points that lie on the straight line segment which joins the pair of points are also in the set.}~$\cA \subset \bbR^d$.
Call this distribution the truncated multivariate normal distribution, and denote it $X \sim \tN_d(\mu,\Sigma,\cA)$.
The pdf is $p(X|\mu,\Sigma,\cA) = C^{-1}\phi(X|\mu,\Sigma)\ind[X\in\cA]$, where
\[
  C = \int_\cA \phi(x|\mu,\Sigma) \dint x = \Prob(X \in \cA).
\] 

Generally speaking, there are no closed-form expressions for $\E g(X)$ for any well-defined functions $g$ on $X$.
One strategy to obtain values such as $\E X$ (mean), $\E X^2$ (second moment) and $E \log p(X)$ (entropy) would be Monte Carlo integration.
If $X^{(1)},\dots,X^{(T)}$ are samples from $X\sim\tN_d(\mu,\Sigma,\cA)$, then $\widehat{\E g(X)} = \frac{1}{T} \sum_{i=1}^T g(X^{(i)})$.

Sampling from a truncated multivariate normal distribution is described by \citet{robert1995simulation} and \citet{damien2001sampling}.
In the latter, the authors explore a simple Gibbs-based approach that is easy to implement in practice.
Assume that the one-dimensional slices of $\cA$ 
\[
  \cA_k(X_{-j}) = \{X_j | (X_1,\dots,X_{j-1},X_j, X_{j+1},\dots,X_d) \in \cA \}
\]
are readily available so that the bounds or anti-truncation region of $X_j$ given the rest of the components $X_{-j}$ are known to be $(x_j^-, x_j^+)$.
Using properties of the normal distribution, the full conditionals of $X_j$ given $X_{-j}$ is
\begin{gather*}
  X_j \sim \tN(\tilde\mu_j,\tilde\sigma_j^2, x_j^-, x_j^+) \\
  \tilde\mu_j = \mu_{j} + \Sigma_{j,-j}^\top\Sigma_{-j,-j}(x_{-j} - \mu_{-j}) \\
  \tilde\sigma_j^2 = \Sigma_{11} - \Sigma_{j,-j}^\top \Sigma_{-j,-j} \Sigma_{j,-j}.
\end{gather*}
According to \citet{robert1995simulation}, if $\Psi = \Sigma^{-1}$, then 
\[
  \Sigma_{-j,-j}^{-1} = \Psi_{-j,-j} - \Psi_{j,-j}\Psi_{-j,-j}^\top / \Psi_{jj}
\]
which means that we need only compute one global inverse $\Sigma^{-1}$.
Introduce a latent variable $Y \in \bbR$ such that the joint pdf of $X$ and $Y$ is
\[
  p(X_1,\dots,X_d,Y) \propto \exp(-Y/2) \ind[y > (x-\mu)^\top\Sigma^{-1}(x-\mu)]\ind[X\in\cA].
\]
Now, the Gibbs conditional densities for the $X_k$'s are given by
\[
  p(X_j|X_{-j},Y) \propto \ind[X_j \in \cB_j]
\]
where
\[
  \cB_j \in (x_j^-, x_j^+) \cap \{X_j | (X-\mu)^\top\Sigma^{-1}(X-\mu) < Y \}.
\]
The Gibbs conditional density for $Y|X$ is a shifted exponential distribution, which can be sampled using the inverse-transform method.
Thus, both $X$ and $Y$ can be sampled directly from uniform variates.

For probit models, we are interested in the conical truncations $\cC_j = \{ X_j > X_k | k\neq j, \text{and } k=1,\dots,m  \}$ for which the $j$'th component of $X$ is largest.
These truncations form cones in $d$-dimensional space such that $\cC_1 \cup \cdots \cup \cC_d = \bbR^d$, and hence the name.



\subsection{Wishart distribution}

\subsection{Gamma distribution}














  \section{Proofs related to conically truncated multivariate normal distribution}

\subsection{Proof of Lemma}

$X_i \sim \tN(\mu_i,\sigma_i^2,-\infty,x_j)$
\begin{align*}
  \int g(x_i) \ind[x_i < x_j] \phi(x_i) \dint x_i = \E g(X_i)
\end{align*}

\begin{align*}
  \E g(X_i)
  &= C^{-1} \idotsint  g(x_i) \ind[x_k < x_j, \forall k \neq j]  \prod_{k=1}^d p(x_k) \dint x_1 \cdots \dint x_d \\
  &= C^{-1} \iint g(x_i) \ind[x_i < x_j] p(x_i) p(x_j) \mathop{\prod_{k=1}^d}_{k \neq i,j} \Phi \left( \frac{x_j - \mu_k}{\sigma_k} \right)  \dint x_i \dint x_j \\
  &= C^{-1} \int \E_{X_i\sim\tN(\mu_i,\sigma_i^2,-\infty,x_j)} g(X_i) p(x_j) \mathop{\prod_{k=1}^d}_{k \neq i,j} \Phi \left( \frac{x_j - \mu_k}{\sigma_k} \right)  \dint x_j
  \end{align*}

\begin{align*}
  \E X_i 
  &= C^{-1} \int  \left(\mu_i \Phi \left( \frac{x_j-\mu_i}{\sigma_i}\right) - \sigma_i  \phi \left( \frac{x_j-\mu_i}{\sigma_i} \right) \right) p(x_j) \mathop{\prod_{k=1}^d}_{k \neq i,j} \Phi \left( \frac{x_j - \mu_k}{\sigma_k} \right)   \dint x_j \\
  &= \mu_i C^{-1} \int p(x_j) \mathop{\prod_{k=1}^d}_{k \neq j} \Phi \left( \frac{x_j - \mu_k}{\sigma_k} \right)   \dint x_j \\
  &\phantom{==} - \sigma_i C^{-1} \int p(x_j)  \phi \left( \frac{x_j-\mu_i}{\sigma_i} \right)  \mathop{\prod_{k=1}^d}_{k \neq j} \Phi \left( \frac{x_j - \mu_k}{\sigma_k} \right)   \dint x_j \\
  &= \mu_i - \sigma_i C^{-1} \int \phi(z)  \phi \left( \frac{\sigma_j z + \mu_j -\mu_i}{\sigma_i} \right)  \mathop{\prod_{k=1}^d}_{k \neq j} \Phi \left( \frac{\sigma_j z + \mu_j - \mu_k}{\sigma_k} \right)   \dint z \\
  &= \mu_i - \sigma_i C^{-1} \E_Z\bigg[ \phi \left( \frac{\sigma_j Z + \mu_j -\mu_i}{\sigma_i} \right)  \mathop{\prod_{k=1}^d}_{k \neq j} \Phi \left( \frac{\sigma_j Z + \mu_j - \mu_k}{\sigma_k} \right) \bigg]
\end{align*}

\begin{proof}
\begin{enumerate}[label=(\roman*)]
  \item Due to the independence structure in the pdf of $\bX$, it is easy to consider the expectations of each of the components separately and marginalising out the rest of the components. For $i \neq j$, we have
  \begin{align*}
    \E[x_i] 
    &= C^{-1} \idotsint \ind[x_k < x_j, \forall k \neq j] \cdot x_i  \prod_{k=1}^d \frac{1}{\sigma_k}\phi \left( \frac{x_k - \mu_k}{\sigma_k} \right) \d x_1 \cdots \d x_d \\
    &= C^{-1} \iint \ind[x_i < x_j] \frac{x_i}{\sigma_i} \, \phi \left( \frac{x_i - \mu_i}{\sigma_i} \right)  \prod_{k \neq i,j} \Phi \left( \frac{x_j - \mu_k}{\sigma_k} \right) \frac{1}{\sigma_j}\phi \left( \frac{x_j - \mu_j}{\sigma_j} \right) \d x_i \d x_j \\
    &= C^{-1} \iint \ind[\sigma_i z_i + \mu_i < \sigma_j z_j + \mu_j] (\sigma_i z_i + \mu_i) \phi (z_i)  \prod_{k \neq i,j} \Phi \left( \frac{\sigma_j z_j + \mu_j - \mu_k}{\sigma_k} \right) \phi (z_j) \d z_i \d z_j \\
    &= \mu_i C^{-1} \iint \ind[ z_i < (\sigma_j z_j + \mu_j - \mu_i) / \sigma_i] \phi (z_i)  \prod_{k \neq i,j} \Phi \left( \frac{\sigma_j z_j + \mu_j - \mu_k}{\sigma_k} \right) \phi (z_j) \d z_i \d z_j \\*    
    &\phantom{==} + \sigma_i C^{-1} \iint \ind[z_i < (\sigma_j z_j + \mu_j - \mu_i) / \sigma_i] z_i \phi (z_i)  \prod_{k \neq i,j} \Phi \left( \frac{\sigma_j z_j + \mu_j - \mu_k}{\sigma_k} \right) \phi (z_j) \d z_i \d z_j \\
    &= \mu_i C^{-1} 
    \overbrace{
    \int  \prod_{k \neq j} \Phi \left( \frac{\sigma_j z_j + \mu_j - \mu_k}{\sigma_k} \right) \phi (z_j) \d z_j
    }^{C} \\  
    &\phantom{==} + \sigma_i C^{-1} \int \ind[ z_i < (\sigma_j z_j + \mu_j - \mu_i) / \sigma_i] z_i \phi (z_i) \prod_{k \neq i,j} \Phi \left( \frac{\sigma_j z_j + \mu_j - \mu_k}{\sigma_k} \right) \phi (z_j) \d z_i \d z_j \\
  \end{align*}
  The integral involving $z_i$ in the second part of the sum is recognised as the (unnormalised) expectation of the lower-tail of a univariate standard normal distribution truncated at $\tau_{ij} = (\sigma_j z_j + \mu_j - \mu_i) / \sigma_i$. That is,
  \[
    \E[Z_i | Z_i < \tau_{ij}] 
    = \big[\Phi(\tau_{ij})\big]^{-1} \int \ind [z_i < \tau_{ij}] z_i \phi(z_i) \d z_i 
    = - \frac{\phi(\tau_{ij})}{\Phi(\tau_{ij})}
  \] Plugging this expression back into the derivation of this expectation, we get
  \begin{align*}
  \E[X_i] 
  &= \mu_i -  \sigma_i C^{-1} \int 
  \phi \left( \frac{\sigma_j z_j + \mu_j - \mu_i}{\sigma_i} \right)
  \prod_{k \neq i,j} \Phi \left( \frac{\sigma_j z_j + \mu_j - \mu_k}{\sigma_k} \right) \phi (z_j) \d z_j \\
  &= \mu_i - \sigma_i C^{-1} \E \left[ \phi \left( \frac{\sigma_j Z_j + \mu_j - \mu_i}{\sigma_i} \right)
  \prod_{k \neq i,j} \Phi \left( \frac{\sigma_j Z_j + \mu_j - \mu_k}{\sigma_k} \right) \right].
  \end{align*}
  
  The expectation for the $j$th component is
  \begin{align*}
    \E[X_j] 
    &= C^{-1} \idotsint \ind[x_k < x_j, \forall k \neq j] \cdot x_j  \prod_{k=1}^d \frac{1}{\sigma_k}\phi \left( \frac{x_k - \mu_k}{\sigma_k} \right) \d x_1 \cdots \d x_d \\
    &= C^{-1} \int x_j  \prod_{k \neq j} \Phi \left( \frac{x_j - \mu_k}{\sigma_k} \right) 
    \cdot \frac{1}{\sigma_j} \phi \left( \frac{x_j - \mu_j}{\sigma_j} \right) \d x_j  \\    
    &= C^{-1} \int (\sigma_j z_j + \mu_j)  \prod_{k \neq j} \Phi \left( \frac{\sigma_j z_j + \mu_j - \mu_k}{\sigma_k} \right) \cdot \phi (z_j) \d z_j  \\   
    &= \mu_j C^{-1} 
    \overbrace{
    \int  \prod_{k \neq j} \Phi \left( \frac{\sigma_j z_j + \mu_j - \mu_k}{\sigma_k} \right) \cdot \phi (z_j) \d z_j
    }^{C}  \\   
    &\phantom{==} + \sigma_j C^{-1} \int  \prod_{k \neq j} \Phi \left( \frac{\sigma_j z_j + \mu_j - \mu_k}{\sigma_k} \right) \cdot z_j \phi (z_j) \d z_j \\
    &= \mu_j + \sigma_j C^{-1} \E \left[ Z_j \prod_{k \neq j} \Phi \left( \frac{\sigma_j Z_j + \mu_j - \mu_k}{\sigma_k} \right) \right] \\
    &= \mu_j + \sigma_j  \mathop{\sum_{i=1}^d}_{i \neq j} \sigma_i C^{-1} \E \Bigg[ \phi \left( \frac{\sigma_j Z_j + \mu_j - \mu_i}{\sigma_i} \right) \mathop{\prod_{k=1}^d}_{k \neq i,j} \Phi \left( \frac{\sigma_j Z_j + \mu_j - \mu_k}{\sigma_k} \right) \Bigg] \\
    &= \mu_j - \sigma_j \sum_{i \neq j} \big(\E[X_i] - \mu_i \big)
  \end{align*}
  where we have made use of Lemma \ref{lem:EZgZ} in the second last step of the above.

  \item The differential entropy is given by
  \begin{align*}
    \cH(p) &= -\int p(\bx) \log p(\bx) \d \bx = -\E \left[ \log p(\bx) \right] \\
    &= -\E \left[-\log C - \half[d] \log 2\pi - \half \sum_{i=1}^d \log \sigma_i^2 - \half \sum_{i=1}^d \left( \frac{x_i - \mu_i}{\sigma_i} \right)^2 \right] \\
    &= \log C + \half[d] \log 2\pi + \half \sum_{i=1}^d \log \sigma_i^2 + \half \sum_{i=1}^d \frac{1}{\sigma_i^2} \E [ x_i - \mu_i ]^2.
  \end{align*}
\end{enumerate}  
\end{proof}

\begin{lemma}\label{lem:EZgZ}
  Let $Z \sim \N(0,1)$. Then for all $m \in \{\bbN \, | \, m > 1\}$ and $(\mu, \sigma) \in \bbR \times \bbR^+$, 
  \[
    \E \Bigg[ Z \mathop{\prod_{k=1}^m}_{k \neq j} \Phi(\sigma_k Z + \mu_k) \Bigg]
    = \mathop{\sum_{i=1}^m}_{i \neq j} \E \Bigg[ \sigma_i \phi(\sigma_i Z + \mu_i) \mathop{\prod_{k=1}^m}_{k \neq i,j} \Phi (\sigma_k Z + \mu_k) \Bigg]
  \]
  for some $j \in \{1, \dots, m\}$.
\end{lemma}

\begin{proof}
  Use the fact that for any differentiable function $g$, $\E[Zg(Z)] = \E[g'(Z)]$, and apply the result with the function $g_m:z \mapsto \prod_{k \neq j} \Phi(\sigma_k z + \mu_k)$. All that is left is to derive the derivative of $g$, and we use an inductive proof to do this. 
  
  We adopt the following notation for convenience:
  \begin{align*}
    \phi_i = \phi(\sigma_i z + \mu_i) \\
    \Phi_i = \Phi(\sigma_i z + \mu_i) 
  \end{align*}
  
  The simplest case is when $m=2$, which can be trivially shown to be true. Without loss of generality, let $j=1$. Then
  \begin{align*}
    g_2(z) &= \Phi_2 \\
    \Rightarrow g_2'(z) &= \sigma_2 \phi_2 = \mathop{\sum_{i=1}^2}_{i \neq 1} \Bigg[ \sigma_i \phi_i \mathop{\sum_{k=1}^2}_{k \neq 1,2} \Phi_k \Bigg].
  \end{align*}
  
  Now assume that the inductive hypothesis holds for some $m \in \{\bbN \, | \, m > 1\}$. That is, the derivative of
  \[
    g_m(z) = \mathop{\prod_{k=1}^m}_{k \neq j} \Phi_k
  \]
  which is
  \[
    g_m'(z) = \mathop{\sum_{i=1}^m}_{i \neq j} \bigg[  \sigma_i \phi_i \mathop{\prod_{k=1}^m}_{k \neq i,j} \Phi_k \bigg],
  \]
  is assumed to be true. Assume that without loss of generality, $j \neq m+1$. Then the derivative of
  \[
    g_{m+1}(z) = \mathop{\prod_{k=1}^{m+1}}_{k \neq j} \Phi_k = g_m(z) \Phi_{m+1}
  \]
  is found to be
  \begin{align*}
    g_{m+1}'(z) &= \sigma_{m+1} \phi_{m+1} g_m(z) + g_m'(z) \Phi_{m+1} \\
    &= \sigma_{m+1} \phi_{m+1} \mathop{\prod_{k=1}^m}_{k \neq j} \Phi_k + \mathop{\sum_{i=1}^m}_{i \neq j} \bigg[  \sigma_i \phi_i \mathop{\prod_{k=1}^m}_{k \neq i,j} \Phi_k \bigg] \Phi_{m+1} \\
    &= \sigma_{m+1} \phi_{m+1} \mathop{\prod_{k=1}^{m+1}}_{k \neq j, m+1} \Phi_k + \mathop{\sum_{i=1}^m}_{i \neq j} \bigg[  \sigma_i \phi_i \mathop{\prod_{k=1}^{m+1}}_{k \neq i,j} \Phi_k \bigg] \\
    &= \mathop{\sum_{i=1}^{m+1}}_{i \neq j} \bigg[  \sigma_i \phi_i \mathop{\prod_{k=1}^{m+1}}_{k \neq i,j} \Phi_k \bigg] \\
    &= g_{m+1}'(z).
  \end{align*}
  Thus, by induction and linearity of expectations, the proof is complete.
\end{proof}

  \section{Derivation of the CAVI algorithm}

Let $\cZ = \{\by^*,\bw,\balpha,\eta,\bPsi \}$.
Approximate the posterior for $\cZ$ by a mean-field variational distribution
\begin{align*}
  p(\by^*, \bw, \alpha, \eta, \bPsi | \by) 
  &\approx q(\by^*)q(\bw)q(\balpha)q(\eta)q(\bPsi) \\
  &= \prod_{i=1}^n q(\by_{i}^*)q(\bw)q(\balpha)q(\eta)q(\bPsi).
\end{align*}
The first line is by assumption, while the second line follows from an induced factorisation on the latent propensities, as we will see later. 
If needed, we also assume that $q(\eta)$ factorises into its constituents components.
Recall that, for each $\xi\in\cZ$, the optimal mean-field variational density $\tilde q$ for $\xi$ satisfies
\[
  \log \tilde q(\xi) = \E_{-\xi} [\log p(\by, \cZ)] + \const \tag{\ref{eq:qtilde}}
\]
Write $\bff = \bH_\eta \bw \in  \bbR^{n\times m}$.
The joint likelihood $p(\by, \cZ)$ is given by
\begin{align*}
  p(\by, \cZ) 
  &= p(\by|\cZ)p(\cZ) \\
  &= p(\by|\by^*) p(\by^* | \balpha,\bw,\eta,\bPsi) p(\bw|\bPsi) p(\eta) p(\bPsi)p(\balpha).
\end{align*}
For reference, the relevant distributions are listed below.

\begin{itemize}
  \item {\boldmath$p(\by|\by^*)$}. For each observation $i\in\{1,\dots,n\}$, given the corresponding latent propensities $\by^*_i = (y_{i1}^*,\dots,y_{im}^*)$, the distribution for $y_i$ is a degenerate distribution which depends on the $j$'th component of $\by^*_i$ being largest, where the value observed for $y_i$ was $j$. Since each of the $y_i$'s are independent, everything is multiplicative.
  \begin{align*}
    p(\by|\by^*) 
    &= \prod_{i=1}^n \prod_{j=1}^m p_{ij}^{[y_i = j]} 
    = \prod_{i=1}^n \prod_{j=1}^m \ind[y_{ij}^* 
    = \max_k y_{ik}^*]^{\ind[y_i = j]}.
  \end{align*}
  
  \item {\boldmath$p(\by^*|\balpha,\bw,\eta,\bPsi)$}. Given values for the parameters and I-prior random effects, the distribution of the latent propensities is matrix normal
  \[
    \by^*|\balpha,\bw,\eta,\bPsi \sim \MN_{n,m}(\bone_n\balpha^\top + \bH_\eta\bw, \bI_n, \bPsi^{-1}).
  \]
%  Equivalently, 
%  \[
%    \vecc \by^* | \balpha,\bw,\eta,\bPsi \sim \N_{nm}\big(\vecc(\bone_n\balpha^\top + \bH_\eta\bw), \bPsi^{-1} \otimes \bI_n \big).
%  \]
%  The $n$ rows of $\by^* = (\by_1^{*\top},\dots,\by_n^{*\top})^\top$ are independent of each other, with each row following a $m$-variate normal distribution $\by_i^* \sim \N_{m}(\balpha + \bff(x_i), \bPsi^{-1})$.
  Write $\bmu = \bone_n\balpha^\top + \bH_\eta\bw$.
  Its pdf is
  \begin{align*}
    p(\by^*|\balpha,\bw,\eta,\bPsi)
%    &= \prod_{i=1}^n \phi(\by_i^*|\balpha + \bff(x_i), \bPsi^{-1}) \\
    &= \exp \left[-\half[nm]\log 2\pi + \half[n]\log\abs{\bPsi} - \half\tr \big((\by^* - \bmu) \bPsi (\by^* - \bmu)^\top  \big)  \right] \\
    &= \exp \left[-\half[nm]\log 2\pi + \half[n]\log\abs{\bPsi} - \half\sum_{i=1}^n (\by^*_{i \bigcdot} - \bmu_{i \bigcdot})^\top \bPsi (\by^*_{i \bigcdot} - \bmu_{i \bigcdot})   \right],
  \end{align*}
  where $\by^*_i \in\bbR^m$ and $\bmu_i \in\bbR^m$ are the rows of $\by^*$ and $\bmu$ respectively.
  The second line follows directly from the definition of the trace, but  also emanates from the fact that $\by_i^*$ are independent multivariate normal with mean $\bmu_i$ and variance $\bPsi^{-1}$.
  
  \item {\boldmath$p(\bw|\bPsi)$}. The $\bw$'s are normal random matrices $\bw \sim \MN_{n,m}(\bzero, \bI_n,\bPsi)$ with pdf
  \begin{align*}
    p(\bw|\bPsi) 
    &= \exp \left[-\half[nm]\log 2\pi - \half[n]\log\abs{\bPsi} - \half\tr \big( \bw \bPsi^{-1} \bw^\top \big)  \right] \\
    &= \exp \left[-\half[nm]\log 2\pi - \half[n]\log\abs{\bPsi} - \half\sum_{i=1}^n  \bw_{i \bigcdot}^\top \bPsi^{-1} \bw_{i \bigcdot}   \right].
  \end{align*}
  
  \item {\boldmath$p(\eta)$}. The most common scenario would be $\eta = \{\lambda_1,\dots,\lambda_p\}$ only. In this case, choose independent normal priors for each $\lambda_k \sim \N(m_k,v_k)$, $k=1,\dots,p$, whose pdf is
  \begin{align*}
    p(\eta) = \prod_{k=1}^p \exp\left[-\half\log 2\pi - \half\log v_k - \frac{1}{2v_k} (\lambda_k - m_k)^2 \right].
  \end{align*} 
  An improper prior $p(\eta) \propto \const$ can be used as well, and this is the same as letting $m_k \to 0$ and $v_k\to 0$.
  The resulting posterior will be proper.
  If $\eta$ contains other parameters as well, such as the Hurst coefficient $\gamma \in (0,1)$, SE lengthscale $l >0$ or polynomial offset $c>0$, then appropriate priors should be used to match the support of the parameter.
  Choices include $p(\gamma) = \ind\big(\gamma \in (0,1)\big)$ and $l,c \sim \Gamma(a,b)$.
  
  \item {\boldmath$p(\bPsi)$}. Our analysis shows that regardless of prior choice of $\bPsi$, be it in the full or independent I-probit model, the posterior for $\bPsi$ will not be of a recognisable form. Without giving too much thought, assume an improper prior on $\bPsi$, i.e. $p(\bPsi) \propto \const$
%  For the precision matrix, a Wishart prior with scale matrix $\bG^{-1}$ and $g$ degrees of freedom, denoted $\bPsi \sim \Wis_m(\bG^{-1},g)$, is convenient. It has pdf
%  \[
%    p(\bPsi) = \exp\left[\const + \half[g-m-1]\log \abs{\bPsi} - \half\tr(\bG\bPsi)  \right].
%  \]
%  For the independent I-probit model, $\bPsi = \diag(\psi_1,\dots,\psi_m)$, and we choose independent Gamma distributions for each precision $\sigma_j^{-2} \sim \Gamma(s_j,r_j)$, where $s_j$ and $r_j$ are the shape and rate parameters.
%  Then,
%  \[
%    p(\bPsi) = \prod_{j=1}^m \exp\left[\const + (s_j - 1)\log \psi_j - r_j\psi_j \right].
%  \] 
  
  \item {\boldmath$p(\balpha)$}. Choose independent normal priors for the intercept, $\alpha_j \sim \N(a_j,A_j)$ for $j=1,\dots,m$. The pdf is
  \[
    p(\balpha) = \prod_{j=1}^m \exp\left[
    -\half\log 2\pi - \half\log A_j - \frac{1}{2A_j} (\alpha_j - a_j)^2  
    \right].
  \]
\end{itemize}

\begin{remark}
  The priors on the parameters $\{ \balpha,\eta \}$ can be set to very vague or even improper priors, and the resulting posterior will still yield a proper distribution.
  Using improper priors eases the algebra slightly.
  For the precision matrix $\bPsi$, it is best to stick with the Wishart prior to avoid positive-definite issues, unless the independent I-probit model is used, in which case Jeffreys' prior for the precisions $p(\sigma_j^{-2})\propto \sigma_j^2$ is a convenient choice.
\end{remark}

\subsection{Derivation of \texorpdfstring{$\tilde q(\by^*)$}{$\tilde q(y^*)$}}
% [Derivation of q ystar]

The rows of $\by^*$ are independent, and thus we can consider the variational density for each $\by_i^*$ separately.
Consider the case where $y_i$ takes one particular value $j \in \{1,\dots,m\}$. The mean-field density $q(\by_{i}^*)$ for each $i=1,\dots,n$ is found to be
\begin{align*}
  \log \tilde q(\by_{i}^*) 
  &=  \ind[y_{ij}^* = \max_k y_{ik}^*] \, \E_{\cZ\backslash\{\by^*\}\sim q} \left[ - \half (\by^*_i - \bmu_i)^\top \bPsi (\by^*_i - \bmu_i)  \right] + \const \\
  &= \ind[y_{ij}^* = \max_k y_{ik}^*] \, \left[ - \half (\by^*_i - \tilde\bmu_i)^\top \tilde\bPsi (\by^*_i - \tilde\bmu_i)  \right] + \const \tag{$\star$} \\
%  &\equiv
%  \begin{cases}
%    \prod_{k=1}^m \N(\tilde f_{ik}, 1) & \text{ if } y_{ij}^* > y_{ik}^*, \forall k \neq j \\
%    0 & \text{ otherwise} \\
%  \end{cases}
  &\equiv
  \begin{cases}
    \phi(\by_i^*|\tilde\bmu_i,\tilde\bPsi) & \text{ if } y_{ij}^* > y_{ik}^*, \forall k \neq j \\
    0 & \text{ otherwise} \\
  \end{cases}
\end{align*}
where $\tilde\bmu_i = \E\balpha + (\E\bH_\eta \E\bw)_i$, and expectations are taken under the optimal mean-field distribution $\tilde q$. 
The distribution $q(\by_i^*)$ is a truncated $m$-variate normal distribution such that the $j$'th component is always largest. 
Unfortunately, the expectation of this distribution cannot be found in closed-form, and must be approximated by techniques such as Monte Carlo integration.
If, however, the independent I-probit model is used and $\tilde\bPsi$ is diagonal, then \hltodo{Lemma X} provides a simplification.

\begin{remark}
  In ($\star$)  above,  we needn't consider the second order terms in the expectations because they do not involve $\by^*$ and can be absorbed into the constant.
  To see this,
  \begin{align*}
    \E[(\by^*_i - \bmu_i)^\top\bPsi(\by^*_i - \bmu_i)]
    &= \E[\by^{*\top}_i\bPsi \by^*_i + \bmu_i^\top\bPsi\bmu_i - 2\bmu_i^\top\bPsi\by^*_i] \\
    &= \by^{*\top}_i\bPsi \by^*_i - 2 \E[\bmu_i^\top] \E[\bPsi]\by^*_i + \const \\
    &= \by^{*\top}_i\bPsi \by^*_i - 2 \tilde\bmu_i^\top\tilde\bPsi\by^*_i + \const \\
    &= (\by^*_i - \tilde\bmu_i)^\top \tilde\bPsi (\by^*_i - \tilde\bmu_i) + \const
  \end{align*}
  We will see this occurring a lot later on and we shall take note of this fact.
\end{remark}


\subsection{Derivation of \texorpdfstring{$\tilde q(\bw)$}{$\tilde q(w)$}}

The terms involving $\bw$ in \cref{eq:qtilde} are the $p(\by^*|\balpha,\bw,\eta,\bPsi)$ and $p(\bw|\bPsi)$ terms, and the rest are absorbed into the constant.
The easiest way to derive $\tilde q(\bw)$ is to vectorise $\by^*$ and $\bw$.
We know that
\begin{gather*}
  \vecc \by^* |  \balpha,\bw,\eta,\bPsi \sim \N_{nm}\big(\vecc (\bone_n\balpha^\top + \bH_\eta\bw), \bPsi^{-1}\otimes\bI_n\big) \\
  \text{and}\\
  \vecc \bw | \bPsi \sim \N_{nm} (\bzero, \bPsi\otimes\bI_n)
\end{gather*}
using properties of matrix normal distributions.
We also use the fact that $\vecc (\bH_\eta\bw) = (\bI_m \otimes \bH_\eta)\vecc\bw$.  %\diag(\bH_\eta,\dots,\bH_\eta)\vecc\bw = 
For simplicity, write $\bar\by^* = \vecc(\by^* - \bone_n\balpha^\top)$, and $\bM = (\bI_m \otimes \bH_\eta)$.
Thus,
\begin{align*}
  \log \tilde q(\bw) 
  &= \E_{\cZ\backslash\{\bw \}\sim q} \left[ 
  -\half (\bar\by^* - \bM\vecc\bw )^\top(\bPsi^{-1} \otimes \bI_n)^{-1} (\bar\by^* - \bM\vecc\bw )
  \right] \\
  &\phantom{==} + \E_{\cZ\backslash\{\bw \}\sim q} \left[ 
  -\half (\vecc \bw )^\top(\bPsi \otimes \bI_n)^{-1} \vecc (\bw ) \right] + \const \\
  &= -\half\E_{\cZ\backslash\{\bw \}\sim q} \left[ 
  (\vecc\bw)^\top \Big( \,
  \greyoverbrace{\bM^\top(\bPsi \otimes \bI_n)\bM + (\bPsi^{-1} \otimes \bI_n)}{\bA} 
  \, \Big) \vecc (\bw )
  \right] \\
  &\phantom{==} + \E_{\cZ\backslash\{\bw \}\sim q} \Big[ 
  \greyoverbrace{\bar\by^{*\top} (\bPsi \otimes \bI_n) \bM}{\ba^\top} \vecc (\bw )
  \Big] + \const \\
  &= -\half\E_{\cZ\backslash\{\bw \}\sim q} \left[
  (\vecc\bw - \bA^{-1}\ba)^\top \bA (\vecc\bw - \bA^{-1}\ba)
  \right] + \const
\end{align*}
This is recognised as a multivariate normal of dimension $nm$ with mean and precision given by $\vecc \tilde\bw = \E[\bA^{-1}\ba]$ and $\tilde\bV_w ^{-1}= \E[\bA]$ respectively.
With a little algebra, we find that
\begin{align*}
  \bV_w^{-1} 
  &= \E_{\cZ\backslash\{\bw \}\sim q} [\bA] \\
  &= \E_{\cZ\backslash\{\bw \}\sim q} \left[(\bI_m \otimes \bH_\eta)^\top(\bPsi \otimes \bI_n)(\bI_m \otimes \bH_\eta) + (\bPsi^{-1} \otimes \bI_n) \right] \\
  &= \E_{\cZ\backslash\{\bw \}\sim q} \left[(\bPsi \otimes \bH_\eta^2) + (\bPsi^{-1} \otimes \bI_n) \right] \\
  &= (\tilde\bPsi \otimes \tilde\bH_\eta^2) + (\tilde\bPsi^{-1} \otimes \bI_n) 
\end{align*}
and making a first-order approximation $(\E \bA )^{-1} \approx \E [\bA^{-1}]$\footnotemark,
\begin{align*}
  \vecc \wtilde 
  &= \E_{\cZ\backslash\{\bw \}\sim q} [\bA^{-1}\ba] \\
  &= \tilde\bV_w \E_{\cZ\backslash\{\bw \}\sim q} \big[(\bI_m \otimes \bH_\eta) (\bPsi \otimes \bI_n) \vecc(\by^* - \bone_n\balpha^\top)  \big] \\
  &= \tilde\bV_w \E_{\cZ\backslash\{\bw \}\sim q} \big[(\bPsi \otimes \bH_\eta) \vecc(\by^* - \bone_n\balpha^\top)  \big] \\
  &= \tilde\bV_w (\tilde\bPsi \otimes \tilde\bH_\eta) \vecc(\tilde\by^* - \bone_n\tilde\balpha^\top).
\end{align*}
Ideally, we do not want to work with the $nm \times nm$ matrix $\bV_w$, since its inverse is expensive to compute.
Refer to \cref{sec:complxiprobit} for details.

\footnotetext{
  \citet{groves1969note} show that $\E [\bA^{-1}] = (\E \bA )^{-1} + \bB$, where $\bB$ is a positive-definite matrix. This approximation has been used also by \citet{girolami2006variational} in their work.
}

In the case of the I-probit model, where $\bPsi = \diag(\psi_1,\dots,\psi_m)$, then the covariance matrix takes a simpler form.
Specifically, it has the block diagonal structure:
\begin{align*}
  \tilde\bV_w
  &= \E \big[\diag(\psi_1,\dots,\psi_m) \otimes \bH_\eta^2 + \diag(\psi_1,\dots,\psi_m) \otimes \bI_n \big]^{-1} \\
  &= \diag\Big(
  \E\big( \psi_1\bH_\eta^2 + \psi_1^{-1}\bI_n\big)^{-1},
  \cdots,
  \E\big(\psi_m\bH_\eta^2 + \psi_m^{-1}\bI_n\big)^{-1}
  \Big) \\
  &\approx \diag\Big(
  \big( \tilde\psi_1\tilde\bH_\eta^2 + \tilde\psi_1^{-1}\bI_n\big)^{-1},
  \cdots,
  \big(\tilde\psi_m\tilde\bH_\eta^2 + \tilde\psi_m^{-1}\bI_n\big)^{-1}
  \Big) \\
  &=: \diag(\tilde\bV_{w_1},\dots,\tilde\bV_{w_m}).
  \end{align*}
The mean $\vecc \tilde\bw$ is
\begin{align*}
  \vecc \tilde\bw 
  &= \tilde\bV_w (\diag(\tilde\psi_1,\dots,\tilde\psi_m) \otimes \tilde\bH_\eta) \vecc(\tilde\by^* - \bone_n\tilde\balpha^\top) \\
  &= \diag(\tilde\bV_{w_1},\dots,\tilde\bV_{w_m})
  \diag(\tilde\psi_1\tilde\bH_\eta,\dots,\tilde\psi_m\tilde\bH_\eta)  
  \vecc(\tilde\by^* - \bone_n\tilde\balpha^\top) \\
  &= \diag(\tilde\psi_1\tilde\bV_{w_1}\tilde\bH_\eta,\dots,\tilde\psi_m\tilde\bV_{w_m}\tilde\bH_\eta)  
  (\tilde\by^* - \bone_n\tilde\balpha^\top) \\
  &= 
  \bordermatrix{
  &\color{gray}\tilde\bw_{\bigcdot 1} 
  &\color{gray}\cdots 
  &\color{gray}\tilde\bw_{\bigcdot m} \cr
  &\tilde\psi_1\tilde\bV_{w_{1}}\tilde\bH_\eta(\tilde\by^*_{\bigcdot 1} - \tilde\alpha_1\bone_n)      
  &\cdots 
  &\tilde\psi_m\tilde\bV_{w_{m}}\tilde\bH_\eta(\tilde\by^*_{\bigcdot m} - \tilde\alpha_m\bone_n) 
  }{}^\top.
\end{align*}
Therefore, we can consider the distribution of $\bw = (\bw_{\bigcdot 1},\dots,\bw_{\bigcdot m})$ columnwise, and each are normally distributed with mean and variance
\[
  \tilde\bw_{\bigcdot j} = \tilde\sigma_j^{-2}\tilde\bV_{w_{j}}\tilde\bH_\eta(\tilde\by^*_{\bigcdot j} - \tilde\alpha_j\bone_n) 
  \hspace{0.5cm}\text{and}\hspace{0.5cm}
  \tilde\bV_{w_{j}} = \big(\tilde\sigma_j^{-2}\tilde\bH_\eta^2 + \tilde\sigma_j^2\bI_n\big)^{-1}.
\]

A quantity that we will be requiring time and again will be $\tr(\bC\E[\bw^\top\bD\bw ])$, where $\bC \in \bbR^{m \times m}$ and $\bD \in \bbR^{n \times n}$ are both square and symmetric matrices.
Using the definition of the trace directly, we get
\begin{align}
  \tr(\bC\E[\bw^\top\bD\bw ])
  &= \sum_{i,j=1}^m \bC_{ij} \E[\bw^\top\bD\bw ]_{ij} \nonumber \\
  &= \sum_{i,j=1}^m \bC_{ij} \E[\bw_{\bigcdot i}^\top\bD\bw_{\bigcdot j} ]. \label{eq:trCEwDw}
\end{align}
The expectation of the univariate quantity $\bw_{\bigcdot i}^\top\bD\bw_{\bigcdot j}$ is inspected below:
\begin{align*}
  \E[\bw_{\bigcdot i}^\top\bD\bw_{\bigcdot j}]
  &= \tr(\bD \E[\bw_{\bigcdot j}\bw_{\bigcdot i}^\top]) \\
  &= \tr\big(\bD (\Cov(\bw_{\bigcdot j},\bw_{\bigcdot i}) + \E[\bw_{\bigcdot j}]\E[\bw_{\bigcdot i}]^\top) \big) \\
  &= \tr\big(\bD (\bV_w[i,j]  + \tilde\bw_{\bigcdot j}\tilde\bw_{\bigcdot i}^\top)\big).
\end{align*}
where $\bV_w[i,j] \in \bbR^{n\times n}$ refers to the $(i,j)$'th submatrix block of $\bV_w$.
Of course, in the independent the I-probit model, this is equal to 
\[
  \bV_w[i,j] = \delta_{ij}(\psi_j\bH_\eta^2 + \psi_j^{-1}\bI_n)^{-1}
\]
where $\delta$ is the Kronecker delta. 
Continuing on \cref{eq:trCEwDw} leads us to
\begin{align*}
  \tr(\bC\E[\bw^\top\bD\bw ])
  &= \sum_{i,j=1}^m \bC_{ij} \left( 
  \tr\big(\bD (\delta_{ij}\bV_{w_j}  + \tilde\bw_{\bigcdot j}\tilde\bw_{\bigcdot i}^\top)\big).
  \right).
\end{align*}
If $\bC = \diag(c_1,\dots,c_m)$, then
\begin{align*}
  \tr(\bC\E[\bw^\top\bD\bw ]) 
  &= \sum_{j=1}^m c_j\left( 
  \tr \big( \bD \tilde\bV_{w_j} \big)  +
  \tilde\bw_{\bigcdot j}^\top \bD \tilde\bw_{\bigcdot j}
  \right) \\
  &= \sum_{j=1}^m c_j
  \tr \big( \bD (\tilde\bV_{w_j} + \tilde\bw_{\bigcdot j} \tilde\bw_{\bigcdot j}^\top) \big)
\end{align*}

\subsection{Derivation of $\tilde q(\eta)$}


By looking at only the terms involving $\eta$ in \cref{eq:qtilde}, we deduce that $\tilde q$ for $\eta$ satisfies
\begin{align*}
  \log\tilde q(\eta) 
  &=  -\half\tr\E_{\cZ\backslash\{\eta\}\sim q} \Big[ 
  (\by^* - \bone_n\balpha^\top - \bH_\eta\bw) \bPsi (\by^* - \bone_n\balpha^\top - \bH_\eta\bw)^\top \Big] + \log p(\eta) \\
  &\phantom{==} + \const \\
  &=  -\half\tr \E_{\cZ\backslash\{\eta\}\sim q} \Big(
  \bPsi\bw^\top\bH_\eta^2\bw - 2\bPsi\bw^\top\bH_\eta(\by^*-\balpha)
  \Big) + \log p(\eta)  + \const \\
  &= -\half \tr \Big( 
   \tilde\bPsi \E[\bw^\top\bH_\eta^2\bw] - 2\tilde\bPsi\tilde\bw^\top\bH_\eta(\tilde\by^* - \tilde\balpha)
   \Big) + \log p(\eta) + \const
\end{align*}
with some appropriate prior $p(\eta)$.
In general, this does not have a recognisable form in $\eta$, especially when it is not linearly dependent on the kernel matrix.
This happens when considering parameters other than the scales of the RKHSs.
Our interest would be to obtain $\tilde\bH_\eta := \E_{\eta\sim q}\bH_\eta$ and $\tilde\bH_\eta^2 := \E_{\eta\sim q}\bH_\eta^2$.
We use a Metropolis random-walk algorithm to obtain these quantities, as detailed in the algorithm below.

\begin{algorithm}[hbt]
\caption{Metropolis random-walk to sample $\eta$}
\begin{algorithmic}[1]
  \State \textbf{inputs} $\tilde\balpha$, $\tilde\bw$, $\tilde\bPsi$, and  $s$ Metropolis sampling s.d.
  \State \textbf{initialise} $\eta^{(0)} \in \bbR^q$ and $t \gets 0$
  \For{$t=1,\dots,T$}
    \State Draw $\eta^* \sim \N_q(\eta^{(t)}, s^2)$
    \State Accept/reject proposal state, i.e.
    \[
      \eta^{(t+1)} \gets 
      \begin{cases}
        \eta^* &\text{if } u\sim\Unif(0,1) < \pi_{\text{acc}} \\
        \eta^{(t)} &\text{otherwise}       
      \end{cases}
    \]
    \hspace{1.8em}where
    \[
      \pi_{\text{acc}} = \min\left(1, 
%      \frac{\tilde q(\eta^*)}{\tilde q(\eta^{(t)})}
      \exp\big(\log \tilde q(\eta^*) - \log \tilde q(\eta^{(t)})\big)
      \right).
    \]
  \EndFor
  \State $\tilde\bH_\eta \gets \frac{1}{T}\sum_{i=1}^T \bH_{\eta^{(t)}}$ and $\tilde\bH_\eta^2 \gets \frac{1}{T}\sum_{i=1}^T \bH_{\eta^{(t)}}^2$
\end{algorithmic}
\end{algorithm}

%In calculating the acceptance probabilities, we need the matrix $\E[\bw^\top\bH_\eta^2\bw]$.
%This is a bit awkward to compute---it requires sampling $\vecc \bw^{(t)}$ from the distribution $\tilde q(\vecc \bw)$, an $nm$-variate normal distribution, and the sample mean of $\bw^{(t)\top}\bH_\eta^2\bw^{(t)}$ calculated.
%If the independent I-probit model is considered, then 
%\[
%  \tr(\tilde\bPsi \E[\bw^\top\bH_\eta^2\bw])
%  = \sum_{i,j=1}^m \tilde\bPsi_{ij}\E[\bw_{\bigcdot i}^\top\bH_\eta^2\bw_{\bigcdot j}]
%\]
%where each $\bw_{\bigcdot j} \sim \N_n(\tilde\bw_{\bigcdot j},\bV_{w_j})$ and are independent of each other, so
%$\E[\bw_{\bigcdot i}^\top\bH_\eta^2\bw_{\bigcdot j}] = \tilde\bw_{\bigcdot i} \bH_\eta^2 \tilde\bw_{\bigcdot j}$.


%
%\begin{align*}
%  \vecc \bH_\eta\bw \sim \N_{nm}(\vecc \bH_\eta\wtilde, (\bI_m\otimes\bH_\eta)\bV_w(\bI_m\otimes\bH_\eta))
%\end{align*}
%
%\begin{align*}
%  \E[(\vecc \bH_\eta\bw)(\vecc \bH_\eta\bw)^\top] &=
%  (\bI_m\otimes\bH_\eta)\bV_w(\bI_m\otimes\bH_\eta) + \vecc \bH_\eta\wtilde (\vecc \bH_\eta\wtilde)^\top \\
%  \E[\bw^\top\bH_\eta^2\bw] &=
%  (\bI_m\otimes\bH_\eta)\bV_w(\bI_m\otimes\bH_\eta) + \vecc \bH_\eta\wtilde (\vecc \bH_\eta\wtilde)^\top  
%\end{align*}

Now consider the case where $\eta = \{\lambda_1,\dots,\lambda_p \}$ (RKHS scale parameters only), and the scenario described in the exponential family EM algorithm of \hltodo{Section 4.3.3} applies.
In particular, for $k=1,\dots,p$, we can decompose the kernel matrix as $\bH_\eta = \lambda_k \bR_k + \bS_k$ and its square as $\bH_\eta^2 = \lambda_k^2 \bR_k^2 + \lambda_k \bU_k + \bS_k^2$.
Then, for $j = 1,\dots,m$, assuming each of the $q(\lambda_k)$ densities are independent of each other, we find that
\begin{align*}
  \log \tilde q(\lambda_k) 
  &= \E_{\cZ\backslash\{\eta\}\sim q} \left[ 
  - \half\tr \big((\by^* - \bmu) \bPsi (\by^* - \bmu)^\top  \big)
  \right] 
  - \frac{1}{2v_k^2} (\lambda_k - m_k)^2 + \const \\
  &= - \half\tr \E_{\cZ\backslash\{\eta\}\sim q} \left[ \bPsi\bw^\top \bH_\eta^2 \bw - 2\bPsi(\by^* - \bone\balpha^\top)^\top\bH_\eta \bw \right] \\
  &\phantom{==} - \frac{1}{2v_k^2} (\lambda_k - m_k)^2 + \const \\
  &= - \half\tr \E_{\cZ\backslash\{\eta\}\sim q} \left[ \bPsi\bw^\top (\lambda_k^2\bR_k^2 + \lambda_k\bU_k) \bw - 2\bPsi(\by^* - \bone\balpha^\top)^\top(\lambda_k\bR_k) \bw \right] \\
  &\phantom{==} - \frac{1}{2v_k^2} (\lambda_k^2 - 2m_k\lambda_k) + \const \\
  &= - \half\tr \E_{\cZ\backslash\{\eta\}\sim q} \left[ 
  \lambda_k^2\bPsi\bw^\top \bR_k^2 \bw 
  -2 \lambda_k \Big(
  \bPsi(\by^* - \bone\balpha^\top)^\top\bR_k \bw 
  -\half\bPsi\bw^\top \bU_k \bw
  \Big) \right] \\
  &\phantom{==} - \half\left(\frac{1}{v_k^2}\lambda_k^2  - 2 \frac{m_k}{v_k^2} \lambda_k \right) + \const \\
  &= -\half \Big[ 
 \lambda_k^2 \big( 
 \greyoverbrace{\tr(\tilde\bPsi\E[\bw^\top\bR_k^2\bw]) + v_k^{-2}}{c_k}
 \big) \\ 
  &\hspace{2cm} - 2\lambda_k \Big( 
  \greyoverbrace{
  \tr\Big( \tilde\bPsi(\tilde\by^* - \bone_n\tilde\balpha^\top)^\top\bR_k \tilde\bw 
  - \half\tilde\bPsi\E[\bw^\top \bU_k \bw] \Big) 
  + m_k v_k^{-2}}{d_k}
  \Big) \Big]
\end{align*}
By completing the squares, we recognise this is as the kernel of a univariate normal density. 
Specifically, $\lambda_k \sim \N(d_k/c_k,1/c_k)$.
The quantity $\tilde\bH_\eta$ can be obtained by substituting $\lambda_k \mapsto \E_{\lambda_k\sim q}[\lambda_k]$ in the \hltodo{expression XXX}.
However, in the calculation of $\tilde\bH_\eta^2$, we must replace $\lambda_k^2 \mapsto \E_{\lambda_k\sim q}[\lambda_k]^2 +  \Var_{\lambda_k\sim q}[\lambda_k]$ in all occurrences of square terms.
This can be cumbersome, so if felt necessary, use the approximation $\lambda_k^2 \mapsto \E_{\lambda_k\sim q}[\lambda_k]^2$ instead.

\begin{example}
  Suppose $k=1$, and we only have $\lambda$ to estimate.
  Then, $\bH_\eta = \lambda \bH$, $\bR_k = \bH$, $\bR_k^2 = \bH^2$, and $\bU_k = \bzero$.
  Suppose also we use an improper prior $\lambda_k \propto \const$, which is the same as having $v_k^2 \to 0$ and $m_k v_k^{-2} \to 0$.
  The mean field distribution for $\lambda$ is then
  \[
    \lambda \sim \N \left( 
    \frac{\tr\big(\tilde\bPsi(\tilde\by^* - \bone\tilde\balpha^\top)^\top \bH \tilde\bw \big)}{\tr(\tilde\bPsi\E[\bw^\top\bH^2\bw])},
    \frac{1}{\tr(\tilde\bPsi\E[\bw^\top\bH^2\bw])}
    \right)
  \]
  Further, if $\tilde\bPsi = \tilde\psi\bI_m$, then
  \[
    \lambda \sim \N \left( 
    \frac{\sum_{j=1}^m (\tilde\by^*_{\bigcdot j} - \tilde\alpha_j\bone)^\top \bH \tilde\bw_{\bigcdot j} }{\sum_{j=1}^m
  \tr \big( \bH^2 \E [\bw_{\bigcdot j} \bw_{\bigcdot j}^\top] \big)},
    \frac{1}{\sum_{j=1}^m
  \tr \big( \bH^2 \E [\bw_{\bigcdot j} \bw_{\bigcdot j}^\top] \big)}
    \right)
  \]
  which bears a resemblance to the exponential family EM algorithm solutions described in Chapter 4.
  Now, $\tilde \bH_\eta = \E[\lambda \bH] = \tilde\lambda \bH$, and $\tilde \bH_\eta^2 = \E[\lambda^2 \bH^2] = (\Var\lambda + \tilde\lambda^2) \bH^2$.
\end{example}

\subsection{Derivation of \texorpdfstring{$\tilde q(\bPsi)$}{$\tilde q(\Psi)$}}

%Introduce the transformed random matrix $\bu = \bw\bPsi^{-1} \in \bbR^{n\times m}$.
%Since we have that  $\vecc \bu = (\vecc \bw)^\top (\bPsi^{-1} \otimes \bI_n)$, the optimal mean-field distribution for $\bu$ is normal with mean $\vecc \tilde\bu = \vecc (\tilde\bw\tilde\bPsi^{-1} )$ and variance
%\begin{align*}
%  \tilde \bV_u
%  &= (\tilde\bPsi^{-1} \otimes \bI_n) \tilde \bV_w (\tilde\bPsi^{-1} \otimes \bI_n). 
%\end{align*}
%In the case of the independent model, its mean is $\tilde\bu_{\bigcdot j} = \tilde\psi_j^{-1} \tilde\bu_{\bigcdot j}$ for $j=1,\dots,m$ and its variance is
%\begin{align*}
%  \tilde \bV_u 
%  &= \diag(\tilde\psi_1^{-2}\tilde\bV_{w_1},\dots,\tilde\psi_m^{-2}\tilde\bV_{w_m}).
%\end{align*}

We find that $q(\bPsi)$ satisfies
\begin{align*}
  \log q(\bPsi)
  &= \E_{\cZ\backslash\{\bPsi\}\sim q} \Big[ 
%  \half[n]\log\abs{\bPsi} 
  - \half\tr \big((\by^* - \bmu)^\top (\by^* - \bmu)\bPsi  \big)
%  - \half[n]\log\abs{\bPsi} 
  - \half\tr \big( \bw^\top\bw\bPsi^{-1} \big)
  \Big] \\
  &\phantom{==} 
  + \log p(\bPsi)
%  + \half[g-m-1]\log \abs{\bPsi} - \half\tr(\bG\bPsi) 
  + \const \\
  &= -\half \tr \Big(
  \big( %\bG + 
  \greyoverbrace{\E[(\by^* - \bmu)^\top (\by^* - \bmu)]}{\bG_1}
  \big)\bPsi +
  \greyoverbrace{\E[\bw^\top\bw]}{\bG_2} \bPsi^{-1}
  \Big) \\
  &\phantom{==} 
  + \log p(\bPsi)
%  + \half[g-m-1]\log \abs{\bPsi} - \half\tr(\bG\bPsi) 
  + \const 
\end{align*}
This seems to be the pdf of $\Wis(\bG+\bG_1,g)$ plus the pdf of a distribution which almost resembles an inverse Wishart pdf.
Unfortunately, the properties such as its moments and entropy are unknown.
%However, we can compute the posterior mode:
%\begin{align*}
%  \frac{\partial}{\partial\bPsi} \log q(\bPsi)
%  &= -\half \tr \left(\frac{\partial}{\partial\bPsi} (\bG_1\bPsi) + \frac{\partial}{\partial\bPsi}(\bG_2\bPsi^{-1}) \right) \\
%  &= -\half \tr \left( \bG_1 - \bG_2\bPsi^{-2} \right)
%\end{align*}
%equated to zero means solving $m$ quadratic equations 

The matrix $\bG_1$ is 
\begin{align*}
  \bG_1 
  &= \E[(\by^* - \bmu)^\top (\by^* - \bmu)] \\
  &= \E \big[\by^{*\top}\by^* + \balpha\bone_n^\top \bone_n\balpha^\top + \bw^\top\bH_\eta^2\bw -2\by^{*\top}\bone_n\balpha^\top -2\by^{*\top}\bH_\eta\bw -2 \balpha\bone_n^\top\bH_\eta\bw \big] \\
  &= \E \big[\by^{*\top}\by^*] + n\E[\balpha\balpha^\top] + \E[\bw^\top\bH_\eta\bw] -2(\tilde\by^{*\top}\bone_n\tilde\balpha^\top + \tilde\by^{*\top}\tilde\bH_\eta\tilde\bw + \tilde\balpha\bone_n^\top\tilde\bH_\eta\tilde\bw),
\end{align*}
and this involves second order moments of a conically truncated multivariate normal distribution, which needs to be obtained via simulation.
Meanwhile,
\begin{align*}
  \bG_{2,ij}
  &= \E[\bw^\top\bw]_{ij} \\
  &= \E[\bw_{\bigcdot i}^\top \bw_{\bigcdot j}] \\
  &= \tilde\bV_w[i,j] + \tilde\bw_{\bigcdot i}^\top  \tilde\bw_{\bigcdot j}.
\end{align*}

In the case of the independent I-probit model, we use a gamma prior on each of the precisions in the diagonal entries of $\bPsi=\diag(\psi_1,\dots,\psi_m)$.
Then, the variational density for each $\psi_j$ is found to be
\begin{align*}
  \log  q(\psi_j)
  &= \E_{\cZ\backslash\{\bPsi\}\sim q} \Big[ 
  \half[n]\log (\psi_1\cdots\psi_m) - \half\sum_{j=1}^m \sum_{i=1}^n \psi_j(\by^*_{ij} - \bmu_{ij})^2 
  \Big] \\
  &\phantom{==} +\E_{\cZ\backslash\{\bPsi\}\sim q} \Big[ 
  -\half[n]\log (\psi_1\cdots\psi_m) - \half\sum_{j=1}^m \sum_{i=1}^n \psi_j^{-1} \bw_{ij}^2 
  \Big] \\
  &\phantom{==} + \sum_{j=1}^m \big( (s_j - 1)\log \psi_j - r_j\psi_j \big) + \const \\
  &= (s_j - 1)\log \psi_j 
  - \psi_j \left( \half \E\norm{\by_{\bigcdot j}^* -  \bmu_{\bigcdot j}}^2 +  r_j \right) \\
  &\phantom{==}
  - \psi_j^{-1} \left( \half\E\norm{\bw_{\bigcdot j}}^2  \right) + \const
\end{align*}
which again, is a pdf of an unknown distribution.
However, its posterior mode can be computed.
Write $a = -\left( \half \E\norm{\by_{\bigcdot j}^* -  \bmu_{\bigcdot j}}^2 + r_j \right)$,
$b = s_j - 1$, and $c=\left( \half\E\norm{\bw_{\bigcdot j}}^2  \right)$.
Then,
\begin{align*}
  \frac{\partial}{\partial\psi_j} \log q(\psi_j)
  = \frac{\partial}{\partial\psi_j} \left( a\psi_j + b\log \psi_j - c\psi_j^{-1} \right) 
  = a +b\psi_j^{-1} + c\psi_j^{-2} 
\end{align*}
equated to zero means solving a quadratic equation in $\psi_j$.
Suppose that $p(\psi_j) \propto \const$, then $s_j=1$ and $r_j = 0$ so $\tilde\psi_j$ can be solved directly to be
\[
  \hat\psi_j = \sqrt{\frac{ \E\norm{\by_{\bigcdot j}^* -  \bmu_{\bigcdot j}}^2 }{\E\norm{\bw_{\bigcdot j}}^2}}.
\]
%The mean is given by $\tilde\psi_j = (s_j + n)(\half \E\norm{\by_{\bigcdot j}^* - \bmu_{\bigcdot j}}^2 + \half \E\norm{\bu_{\bigcdot j}}^2  + r_j)^{-1}$.
If the posterior mean is close to its mode, then $\hat\psi_j$ is a good approximation for $\tilde\psi_j$.

To calculate $\E\norm{y_{\bigcdot j}^* -  \bmu_{\bigcdot j}}^2 = \E \sum_{i=1}^n (\by_{ij}^* - \mu_{ij})^2$, one first needs $\E (y_{ij}^* - \alpha_j - \bw_{\bigcdot j}^\top\bh_\eta(x_i))^2$.
This, in itself, presents a challenge to compute analytically, because it requires, among other things, the second moments $\E y_{ij}^{*2}$ and $\E [\bw_{\bigcdot j}^\top\bh_\eta(x_i)\bh_\eta(x_i)^\top \bw_{\bigcdot j}]$.
Although not entirely accurate, it is simpler to use the approximation
\[
  \E\norm{y_{\bigcdot j}^* -  \bmu_{\bigcdot j}}^2 \approx 
  \norm{\tilde y_{\bigcdot j}^* -  \tilde \bmu_{\bigcdot j}}^2.
\]
% from the distribution of $\by_{i \bigcdot} \sim \tN(\bmu_{i \cdot}, \tilde\bPsi^{-1}, \cC_{y_i})$.
%Because of the independence structure, these can be computed componentwise.
(see note on \hltodo{page}).
Also, we have $\bw_{\bigcdot j} \sim \N_n(\tilde\bw_{\bigcdot j}, \tilde\bV_{w_j})$, and so $  \E\norm{\bw_{\bigcdot j}}^2 = \tr(\tilde\bV_{w_j} + \tilde\bw_{\bigcdot j}\tilde\bw_{\bigcdot j}^\top)$.

\subsection{Derivation of \texorpdfstring{$\tilde q(\balpha)$}{$\tilde q(\alpha)$}}

Let $\bA = \diag(A_1,\dots,A_m)$ and $\ba = (a_1,\dots,a_m)^\top$.
The terms involving $\alpha_j$ in \cref{eq:qtilde} are
\begin{align*}
  \log q(\balpha)
  &= \E_{\cZ\backslash\{\balpha\}\sim q} \left[ 
  -\half\sum_{i=1}^n 
  \big(\by^*_{i \bigcdot} - \balpha - \bw^\top\bh_\eta(x_i)\big)^\top 
  \bPsi \big(\by^*_{i \bigcdot} - \balpha - \bw^\top\bh_\eta(x_i)\big)
  \right] \\
  &\phantom{==}
  - \half (\balpha - \ba)^\top \bA^{-1} (\balpha - \ba) + \const \\
  &= -\half \Bigg[ \balpha^\top(
  \greyoverbrace{n\bPsi + \bA^{-1}}{\tilde \bA}
  )\balpha -2\Bigg( \, 
   \greyoverbrace{\sum_{i=1}^n \bPsi\big( \tilde\by^*_{i \bigcdot} - \tilde\bw^\top\tilde\bh_\eta(x_i) \big) + \bA^{-1} \ba}{\tilde \ba} 
  \, \Bigg)^\top \balpha \Bigg]
\end{align*}
which implies a normal mean-field distribution for $\balpha$ whose mean and variance are $\tilde\balpha = \tilde \bA^{-1}\tilde \ba$ and $\tilde \bA^{-1}$ respectively.
If $\bPsi$ is diagonal, the components of $\balpha$ would be independent.

As a remark, due to identifiability, only $m-1$ of these intercept are estimable.
We can either put a constraint that one of the intercepts is fixed at zero, or the sum of the intercepts equals zero.
The latter constraint is implemented in this thesis, and this is realised by estimating all the intercepts and then centring them.


  \section{Deriving the ELBO expression}

The evidence lower bound (ELBO) expression involves the following calculation:
\begin{align*}
  \cL &= \idotsint q(\by^*,\bw,\theta) 
  \log \frac{p(\by,\by^*,\bw,\theta)}{q(\by^*,\bw,\theta)}
  \dint\by^* \dint\bw \dint\theta \\
  &= \E \log \greyoverbrace{p(\by,\by^*,\bw,\theta)}{\text{joint likelihood}}
  +
  (\greyoverbrace{-\E \log q(\by^*,\bw,\theta)}{\text{entropy}}) \\
  &= \E\bigg[
  \cancel{\sum_{i=1}^n \sum_{j=1}^m \log  p(y_{i}|y_{ij}^*)} + 
  \sum_{i=1}^n \log  p(\by_{i \bigcdot}^*|\balpha,\bw,\bPsi,\eta) +
  \log p(\bw|\bPsi) +
  \log p(\bPsi) \\ 
  &\hspace{2cm} +
  \log p(\eta) +
  \log p(\balpha)
  \bigg] \\
  &\phantom{==} 
  + \sum_{i=1}^n H\big[q(\by^*_{i \bigcdot}) \big]
  + H\big[q(\bw) \big]
  + H\big[q(\bPsi) \big]
  + H\big[q(\eta) \big]
  + H\big[q(\balpha) \big].
\end{align*}

\begin{remark}
  As discussed, given the latent propensities $\by^*$, the pdf of $\by$ is degenerate and hence can be disregarded.  
\end{remark}

\begin{remark}
  When using improper priors for the hyperparameters, i.e. $p(\bPsi,\eta,\balpha) \propto \const$, then these terms can be disregarded.  
\end{remark}

\subsection{Terms involving distributions of \texorpdfstring{$\by^*$}{$y^*$}}

\begin{align*}
  \sum_{i=1}^n  \bigg( &
  \E  \log p(\by_{i \bigcdot}^*|\balpha,\bw,\bPsi,\eta) 
  + H \big[q(\by^*_{i \bigcdot}) \big] 
  \bigg) \\
  &=  -\half[nm]\log 2\pi + \half[n]\E \log \abs{\bPsi} - \half \E \sum_{i=1}^n (\by^*_{i \bigcdot} - \bmu_{i \bigcdot})^\top \bPsi (\by^*_{i \bigcdot} - \bmu_{i \bigcdot}) \\
  &\phantom{==} +\half[nm]\log 2\pi - \half[n] \log \abs{\tilde\bPsi} + \half \E \sum_{i=1}^n (\by^*_{i \bigcdot} - \tilde\bmu_{i \bigcdot})^\top \tilde\bPsi (\by^*_{i \bigcdot} - \tilde\bmu_{i \bigcdot}) + \log C_i  \\
%  &= \const + n \sum_{j=1}^m \tilde\bPsi_{jj} \tilde v_{\alpha_j} + 
%  \sum_{i=1}^n \left\{ \log C_i 
%  +  \tr\big(\tilde\bPsi\bOmega_i \big) 
%  \right\}.
   &=\const + \sum_{i=1}^n \log C_i 
\end{align*}
where $\bOmega_i = \Var \bw^\top\bh_\eta(x_i)$, and $C_i$ is the normalising constant for the distribution of multivariate truncated normal $\by_{i \bigcdot}$.

Notes:
\begin{enumerate}
  \item $p(\by_{i \bigcdot}^*)$ is the pdf of $\N(\bmu_{i \bigcdot}, \bPsi^{-1})$, and $q(\by_{i \bigcdot}^*)$ is the pdf of $\tN(\tilde\bmu_{i \bigcdot}, \tilde\bPsi^{-1}, \cC_{y_i})$, where $\bmu_{i \bigcdot} = \balpha + \bw^\top\bh_\eta(x_i) \in \bbR^m$.
  \item For $\bPsi\sim \Wis(\cdot,\cdot)$ with mean $\tilde\bPsi$, $\E \log \abs{\bPsi} = \log \tilde\bPsi + \const$ \citep[§10.2]{bishop2006pattern}.
  \item It is simpler to use the approximation
  \begin{align}\label{eq:elboyaprx}
    \E (\by^*_{i \bigcdot} - \bmu_{i \bigcdot})^\top \bPsi (\by^*_{i \bigcdot} - \bmu_{i \bigcdot})
    \approx \E (\by^*_{i \bigcdot} - \tilde\bmu_{i \bigcdot})^\top \tilde\bPsi (\by^*_{i \bigcdot} - \tilde\bmu_{i \bigcdot}).   
  \end{align}
  rather than work out the actual quantity, which is
  \begin{align}\label{eq:elboyact}
    \E (\by^*_{i \bigcdot} - \bmu_{i \bigcdot})^\top \bPsi (\by^*_{i \bigcdot} - \bmu_{i \bigcdot})
    &= \E (\by^*_{i \bigcdot} - \tilde\bmu_{i \bigcdot})^\top \tilde\bPsi (\by^*_{i \bigcdot} - \tilde\bmu_{i \bigcdot}) + \tr(\tilde\bPsi\Var \bmu_{i \bigcdot})
  \end{align}
  where $\Var \bmu_{i \bigcdot} = \Var \balpha + \Var \bw^\top\bh_\eta(x_i)$, obtained by taking expectations with respect to everything except $\by^*_{i \bigcdot}$.
  The first term is a diagonal matrix of the posterior variances of the intercepts.
  The second term is where things get complicated 
  Let $\bOmega_i = \Var \bw^\top\bh_\eta(x_i)$. 
  Then $\bOmega_{i,kj} \approx \Cov(\bw^\top_{\bigcdot k}\bh_\eta(x_i), \bw^\top_{\bigcdot j}\bh_\eta(x_i)) = \bh_\eta(x_i)^\top \tilde\bV_w[k,j]\bh_\eta(x_i)$. 
  So
  \[
    \tr(\tilde\bPsi \bOmega_i) \approx \sum_{k,j=1}^m \tilde\bPsi_{kj} \bh_\eta(x_i)^\top \tilde\bV_w[k,j]\bh_\eta(x_i)
  \]
  However, we know that $\Var XY = \E X^2Y^2 - (\E XY)^2 = \Var X \Var Y + \Var X (\E Y)^2 + \Var Y (\E X)^2$, so there is actually some covariance terms which need to be considered, and these are not so easily computed.
  In practice, we find that using \cref{eq:elboyaprx} gives satisfactory results as far as determining convergence for the variational algorithm goes. 
\end{enumerate}

\subsection{Terms involving distributions of $\bw$}

\begin{align*}
  \E \log p(\bw|\bPsi) + H \big[q(\bw) \big] 
  &= \cancel{-\half[nm]\log 2\pi} - \half[n]\E\log\abs{\bPsi} - \half\E\tr \big( \bw \bPsi^{-1} \bw^\top \big) \\
  &\phantom{==} + \half[nm](1 + \cancel{\log 2\pi}) + \half\log\abs{\tilde\bV_w} \\
  &= \const - \half[n]\log\tilde\bPsi 
  - \half \sum_{j=1}^m \tr \big(\tilde\bPsi^{-1} (\tilde\bV_w[j,j]  + \tilde\bw_{\bigcdot j}\tilde\bw_{\bigcdot j}^\top)\big)
\end{align*}

Notes:
\begin{enumerate}
  \item $p(\bw)$ is the pdf of $\MN(\bzero,\bI_n,\bPsi)$, and $q(\bw)$ is the pdf of $\N(\vecc \tilde\bw,\tilde\bV_w)$.
  \item We used the first order approximation $\E \bPsi^{-1} \approx (\E \bPsi)^{-1} = \tilde\bPsi^{-1}$.
  \item $\tilde\bV_w[j,j]$ are the $n \times n$ sub matrices along the diagonal of $\tilde\bV_w$.
\end{enumerate}

\subsection{Terms involving distributions of $\eta$}

If no closed-form expression for $q(\eta)$ is found, then the expression $\E [\log p(\eta) - q(\eta)]$ must be obtained by sampling methods.
Otherwise, consider the case where $\eta =\{\lambda_1,\dots,\lambda_p\}$.
Then, the contribution to the ELBO is
\begin{align*}
  \E \log p(\lambda_1 & ,\dots,\lambda_p) + H\big[q(\lambda_1,\dots,\lambda_p) \big] \\
  &=  -\half[p]\log 2\pi - \half\log v_1\cdots v_k - \half \sum_{k=1}^p  \frac{\E(\lambda_k - m_k)^2}{v_k} \\
  &\phantom{==} + \half[p](1 + \log 2\pi) + \half\log \tilde v_1\cdots \tilde v_p \\
  &= \const + \half \sum_{k=1}^p \log \tilde v_k - \half \sum_{k=1}^p  \frac{\tilde v_k + \tilde\lambda_k^2 - 2\tilde\lambda_k m_k}{v_k}
\end{align*}

Notes:
\begin{enumerate}
  \item The priors on the $\lambda_k$'s are $\N(m_k,v_k)$, and  $q(\lambda_k)$ is the density of $\N(\tilde\lambda_k, v_{\lambda_k})$.
  \item When using improper priors $\lambda_k \propto \const$, then we need only consider the middle term involving the sums of $\log \tilde v_{\lambda_k}$.
\end{enumerate}

\subsection{Terms involving distributions of \texorpdfstring{$\bPsi$}{$\Psi$}}

The terms involving $\bPsi$ are $\E \log p(\bPsi) + H\big[q(\bPsi)\big]$.
In the case of the full I-probit model, this becomes
\begin{align*}
  \const & + \half[g-m-1]\E \log \abs{\bPsi} - \half \E \tr(\bG\bPsi) 
  + \log B - \half[\tilde g-m-1]\E \log \abs{\bPsi} + \half[\tilde g m] \\
  &= \const  + \log B + \half[\tilde g m] + \half[g-\tilde g] \log \abs{\tilde\bPsi} - \half \tr(\bG\tilde\bPsi)
%  + \frac{m+1}{2}\log\abs{\bG + \bG_1 + \bG_2} + \half m(m+1)\log 2 + \log \Gamma_p \left(\half[s] \right) - \half[s-m-1] \psi_m \left(\half[s] \right) + \half[sm]
\end{align*}
where $B = \half[\tilde g] \log \abs{\tilde G} + \half[\tilde g m]\log 2 + \log \Gamma_m(\tilde g / 2)$.
In the case of the independent I-probit model, we have
\begin{align*}
  \const & + \sum_{j=1}^m \Big\{ (s_j - 1)\E \log \psi_j -  r_j \E\psi_j \Big\}
  - \sum_{j=1}^m \log \tilde r_j \\
  &= \const + \sum_{j=1}^m \Big\{ (s_j - 1)\log \tilde \psi_j   -  r_j \tilde\psi_j \Big\}
  - \sum_{j=1}^m \log \tilde r_j 
\end{align*}

Notes:
\begin{enumerate}
  \item The priors on the $\bPsi$ is $\Wis(\bG,g)$, or if $\bPsi = \diag(\psi_1,\dots,\psi_m)$, then each $\psi_j\sim \Gamma(s_j,r_j)$. $q(\bPsi)$ is the density of $\Wis(\tilde\bG,\tilde g)$ or in the case of the independent model, each $q(\psi_j)$ is the density of $\Gamma(s_j,\tilde r_j)$.
  \item Use the first order Taylor expansion about $\E\psi_j$ to approximate $\E \log \psi_j \approx \log \E \psi_j = \log \tilde \psi_j$, as per \citet{teh2007collapsed}.
\end{enumerate}

\subsection{Terms involving distribution of \texorpdfstring{$\balpha$}{$\alpha$}}

For the intercepts, consider only
\begin{align*}
  \E \log p(\balpha) + H \big[q(\balpha) \big] 
  &=  \const  - \half \E \sum_{j=1}^m \frac{(\alpha_j - a_j)^2}{A_j}    + \half \log \tilde v_{\alpha_1}\cdots\tilde v_{\alpha_m} \\ 
  &= \const + \half \sum_{j=1}^m \log \tilde v_{\alpha_j} - \half \sum_{j=1}^m \frac{v_{\alpha_j} + \tilde\alpha_j^2  -2a_j \tilde\alpha_j}{A_j} 
\end{align*}

Notes:
\begin{enumerate}
  \item $p(\balpha)$ is $\prod_{j=1}^m \phi(\alpha_j|a_j,A_j)$, and $q(\balpha)$ $\prod_{j=1}^m \phi(\alpha_j|\tilde \alpha_j,\tilde v_{\alpha_j})$.
\end{enumerate}

\subsection{ELBO summarised}

In the example section of Chapter 5, we considered  only 1) the independent I-probit model; 2) fixed $\bSigma = \bI_m$; 3) only RKHS scale parameters to estimate; and 4) and improper priors on the hyperparameters.
In such situations, the ELBO expression is simply
\begin{align*}
  \cL 
  &= \const + \sum_{i=1}^n  \log C_i 
  - \half \sum_{j=1}^m \tr \big(\tilde\bV_{w_j}  + \tilde\bw_{\bigcdot j}\tilde\bw_{\bigcdot j}^\top \big) + \half \sum_{k=1}^p \log \tilde v_k.
\end{align*}

As a final remark, often times the ELBO is treated as a proxy for the (penalised) marginal likelihood of the model, in which case it must be noted that the ELBO as we had derived is correct up to a constant.
We find that keeping track of the constants is slightly tedious, and hence decided not to do so.
When comparing ELBOs of two or more models, the comparison is still valid as only differences between the ELBOs matter, in which case the constants would cancel out.

\fi

\hClosingStuffStandalone
\end{document}