\documentclass[a4paper,showframe,11pt]{report}
\usepackage{standalone}
\standalonetrue
\ifstandalone
  \usepackage{../../haziq_thesis}  
  \usepackage{../../haziq_maths}
  \usepackage{../../haziq_glossary}
  \usepackage{../../knitr}
  \usepackage{../../matrix_fig}
  \addbibresource{../../bib/haziq.bib}
  \externaldocument{../01/.texpadtmp/chapter1}
  \externaldocument{../02/.texpadtmp/chapter2}
  \externaldocument{../03/.texpadtmp/chapter3}
  \externaldocument{../04/.texpadtmp/chapter4}
  \externaldocument{../appendix/.texpadtmp/appendix}
\fi

\begin{document}
\hChapterStandalone[5]{I-priors for categorical responses}
\label{chapter5}

Consider polytomous response variables $\by = \{y_1,\dots,y_n\}$, where each $y_i$ takes on exactly one of the values from the set of $m$ possible choices $\{1,\dots,m\}$.
Modelling categorical response variables is of profound interest in statistics, econometrics and machine learning, with applications aplenty. 
In the social sciences, categorical variables often arise from survey responses, and one may be interested in studying correlations between  explanatory variables and the categorical response of interest.
Economists are often interested in discrete choice models to explain and predict choices between several alternatives, such as consumer choice of goods or modes of transport.
In this age of big data, machine learning algorithms are used for classification of observations based on what is usually a large set of variables or features.

The model \cref{eq:model1} subject to normality assumptions \cref{eq:model1ass} is not entirely appropriate for polytomous variables $\by$.
As an extension to the I-prior methodology, we propose a flexible modelling framework suitable for regression of categorical response variables.
In the spirit of generalised linear models \citep{mccullagh1989}, we relate class probabilities of the observations to a normal I-prior regression model via a link function.
Perhaps though, it is more intuitive to view it as machine learners do: since the regression function is ranged on the entire real line, it is necessary to ``squash'' it through some sigmoid function to conform it to the interval $[0,1]$ suitable for probability ranges.

Expanding on this idea further, assume that the $y_i$'s follow a categorical distribution, $i=1,\dots,n$, denoted by
\[
  y_i \sim \Cat(p_{i1},\dots,p_{im}),
\]
with the class probabilities satisfying $p_{ij} \geq 0, \forall j=1,\dots,m$ and $\sum_{j=1}^m p_{ij} = 1$. 
The probability mass function (pmf) of $y_i$ is given by
\begin{align*}%\label{eq:catdist}
  p(y_i) = p_{i1}^{[y_i = 1]} \cdots p_{im}^{[y_i = m]}
\end{align*}
where the notation $[\cdot]$ refers to the Iverson bracket\footnote{$[A]$ returns 1 if the proposition $A$ is true, and 0 otherwise. The Iverson bracket is a generalisation of the Kronecker delta.}. 
The dependence of the class probabilities on the covariates is specified through the relationship
\[
  g(p_{i1},\dots,p_{im}) = \big(\alpha_1 + f_1(x_i), \dots, \alpha_m + f_m(x_i)\big)
\]
where $g:[0,1]^m\to\bbR^m$ is some specified link function.
As we will see later, an underlying normal regression model as in \cref{eq:model1} subject to \cref{eq:model1ass} naturally implies a \emph{probit} link function.
%, that is, $g$ is the inverse cumulative distribution function (cdf) of a standard normal distribution (or more precisely, a function that \emph{involves} the standard normal cdf).
With an I-prior assumed on the $f_j$'s, we call this method of probit regression using I-priors the \emph{I-probit} regression model.

%Implicitly in the above, there are $m$ regression curves which model the $m$ class probabilities.
%Often times, it is of interest that the model captures the correlations between choices, which are vital in choice models.
%Disregarding the correlations violates the independence of irrelevant alternatives (IIA) assumption, but this is usually fine for classification tasks where the alternatives are assumed not to be correlated.

Due to the nature of the model, unfortunately, the posterior distribution of the regression functions cannot be found in closed form.
In particular, marginalising the I-prior from the joint likelihood involves a high-dimensional intractable integral.
Similar problems are encountered in mixed logistic or probit multinomial models \citep{breslow1993approximate,mcculloch2000bayesian} and also Gaussian process classification \citep{neal1999,rasmussen2006gaussian} .
In these models, Laplace approximation for maximum likelihood estimation or Markov chain Monte Carlo (MCMC) methods for Bayesian estimation are used. 
We instead explore a \emph{variational approximation} to the marginal log-likelihood, and thus, to the posterior density of the regression function.
The main idea is to replace the difficult posterior distribution with an approximation that is tractable to be used within an EM framework.
As such, the computational work derived in the previous section is applicable for estimation of I-probit models as well.

As in the normal I-prior model, the I-probit model estimated using a \emph{variational EM} algorithm is seen as an empirical Bayes method of estimation, since the model parameters are replaced with their ML estimates.
It is emphasised again, that working in such a semi-Bayesian framework allows fast estimation of the model in comparison to traditional MCMC, yet provides us with the conveniences that come with Bayesian machinery.
For example, inferences around log-odds is usually cumbersome for probit models, but a credibility interval can easily be obtained by resampling methods from the posterior distribution of the regression function, which as we shall see, is approximated to be normally distributed.

By choosing appropriate RKHSs/RKKSs for the regression functions, we are able to fit a multitude of binary and multinomial models, including multilevel or random-effects models, linear and non-linear classification models, and even spatio-temporal models.
Examples of these models applied to real-world data is shown in  \cref{sec:iprobiteg}.
We find that the many advantages of the normal I-prior methodology  transfer over quite well to the I-probit model for binary and multinomial regression.

%\section{A naïve model}\label{sec:iprobitnaive}
%We describe a naïve classification model using I-priors.
Here, the responses are categorical $y_i \in \{ 1,\dots,m \} =: \cM$, and additionally, write $\by_{i \bigcdot} = (y_{i1},\dots,y_{im})^\top$ where the class responses $y_{ij}$ equal one if individual $i$'s response category is $y_i = j$, and zero otherwise.
In other words, there is exactly a single `1' at the $j$'th position in the vector $\by_{i \bigcdot}$, and zeroes everywhere else.
For $j=1,\dots,m$, we model 
\begin{equation}\label{eq:naiveclassmod}
  \begin{gathered}
    y_{ij} = \alpha + 
    \myoverbrace{\alpha_j + f_j(x_i)}{f(x_i,j)}
    + \epsilon_{ij}  \\
    (\epsilon_{i1},\dots,\epsilon_{im})^\top \iid \N_m(\bzero,\bPsi^{-1}).
  \end{gathered}
\end{equation}
The idea here being that we attempt to model the class responses $y_{ij}$ using class-specific regression functions $f_j$, and the class responses are assumed to be independent among individuals, but may or may not be correlated among classes for each individual.
The class correlations are manifest themselves in the variance of the errors $\bPsi^{-1}$, which is an $m\times m$ matrix.

Denote the regression function $f$ in \cref{eq:naiveclassmod} on the set $\cX\times\cM$ as $f(x_i,j) = \alpha_j + f_j(x_i)$.
This regression function corresponds to an ANOVA decomposition of the spaces $\cF_\cM$ and $\cF_\cX$ of functions over $\cM$ and $\cX$ respectively. 
That is, $\cF = \cF_\cM \oplus (\cF_\cM \otimes \cF_\cX)$ is a decomposition into the main effects of `class', and an interaction effect of the covariates for each class.
Let $\cF_\cM$ and $\cF_\cX$ be RKHSs respectively with kernels $a:\cM\times\cM\to\bbR$ and $b_\eta:\cX\times\cX\to\bbR$.
Then, the ANOVA RKKS $\cF$ possesses the reproducing kernel $h_\eta:(\cX \times \cM)^2 \to \bbR$ as defined by
\begin{align}\label{eq:anovaclass}
  h_\eta\big( (x,j), (x',j') \big) = a(j,j') + a(j,j')b_\eta(x,x').  
\end{align}
The kernel $b_\eta$ may be any of the kernels described in this thesis, ranging from the linear kernel, to the fBm kernel, or even an ANOVA kernel.
Choices for $a:\cM \times \cM \to \bbR$ include 
\begin{enumerate}
  \item \textbf{The Pearson kernel} (as defined in \cref{def:pearson}, \mypageref{def:pearson}). With $J\sim\Prob$, a probability measure over $\cM$,
  \[
    a(j,j') = \frac{\delta_{jj'}}{\Prob(J=j)} - 1.
  \]
  \item \textbf{The centred identity kernel}. With $\delta$ denoting the Kronecker delta function,
  \[
    a(j,j') = \delta_{jj'} - 1 / m.
  \]
\end{enumerate}
The purpose of either of these kernels is to contribute to the class intercepts $\alpha_j$, and to associate a regression function in each class.
The only difference between the two is the inverse probability weighting per class that is applied in the Pearson kernel, but not in the identity kernel.

With $f \in \cF$ (the RKKS with kernel $h_\eta$), it is straightforward to assign an I-prior on $f$. 
It is in fact
\begin{align}\label{eq:naiveclassiprior}
  \begin{gathered}
    f(x_i,j) = \sum_{j'=1}^m\sum_{i'=1}^n a(j,j')\big(1 + b_\eta(x_i,x_{i'})\big) w_{i'j'} \\
    (w_{i'1},\dots,w_{i'm})^\top \iid \N_m(\bzero,\bPsi)
  \end{gathered}
\end{align}
assuming a zero prior mean $f_0(x,j) = 0$.
The model then classifies the $i$'th data point to class $j$ if $\hat y_{ij} = \max(\hat y_{i1},\dots,\hat y_{im})$, where $\hat y_{ik} = \hat\alpha + \hat f(x_i,k)$, the prediction for the $k$'th component of $y_i$.

There are several drawbacks to using the model described above.
Unlike in the case of continuous response variables, the normal I-prior model is highly inappropriate for categorical responses.
For one, it violates the normality and homoscedasticity assumptions of the errors.
For another, predicted values may be out of the range $[0,m]$ and thus poorly calibrated.
Furthermore, it would be more suitable if the class probabilities---the probability of an observation belonging to a particular class---were also part of the model.
In \cref{chapter5}, we propose an improvement to this naïve I-prior classification model by considering a probit-like transformation of the regression functions.

%\begin{proof}
%  \[
%    f(x_i,j) = \sum_{j'=1}^m\sum_{i'=1}^n a(j,j')\big(1 + b_\eta(x_i,x_{i'})\big) w_{i'j'}
%  \]
%  
%  \[
%    \alpha_j = \sum_{j'=1}^m\sum_{i'=1}^n a(j,j') w_{i'j'}
%  \]
%  
%  \begin{align*}
%    \sum_{j=1}^m \alpha_j
%    &= \sum_{j=1}^m \sum_{j'=1}^m\sum_{i'=1}^n a(j,j') w_{i'j'} \\
%    &= \sum_{j=1}^m \sum_{j'=1}^m\sum_{i'=1}^n \delta_{jj'} w_{i'j'} \\
%    &= \sum_{j=1}^m\sum_{i'=1}^n  w_{i'j}
%  \end{align*}
%  
%  \begin{align*}
%    \sum_{j=1}^m f_j(x_i)
%    &= \sum_{j=1}^m \sum_{j'=1}^m\sum_{i'=1}^n a(j,j')b_\eta(x_i,x_{i'}) w_{i'j'} \\
%    &= \sum_{j=1}^m \sum_{j'=1}^m\sum_{i'=1}^n \delta_{jj'} b_\eta(x_i,x_{i'}) w_{i'j'} \\
%    &= \sum_{j=1}^m\sum_{i'=1}^n b_\eta(x_i,x_{i'}) w_{i'j}
%  \end{align*}
%  
%  \begin{align*}
%    \sum_{i=1}^n f_j(x_i)
%    &= \sum_{i=1}^n \sum_{j'=1}^m\sum_{i'=1}^n a(j,j')b_\eta(x_i,x_{i'}) w_{i'j'} \\
%    &= \sum_{i=1}^n \sum_{j'=1}^m\sum_{i'=1}^n \delta_{jj'} b_\eta(x_i,x_{i'}) w_{i'j'} 
%  \end{align*}
%\end{proof}

%Rearrange the $n$ observations per class.
%Let $\bff_j = \big(f_j(x_1),\dots,f_j(x_n)\big)^\top \in \bbR^n$.
%We can write the I-prior as $\bff_j = \bA_{jj} \cdot \bH\bw_j$
%Therefore, $\bff_j \sim \N_n(\bzero, \bPsi_{jj}\bA_{jj} \bH^2)$, and
%\begin{align*}
%  \Cov(\bff_j,\bff_k) &= \Cov(\bA_{jj} \cdot \bH\bw_j, \bA_{kk} \cdot \bH\bw_k) \\
%  &= \bA_{jj}\bA_{kk} \cdot \bH \Cov(\bw_j,\bw_k) \bH \\
%  &= \bA_{jj}\bA_{kk}\bPsi_{jk} \bH^2.
%\end{align*}


\section{A latent variable motivation: the I-probit model}
It is convenient, as we did in the previous subsection, to again think of the responses $y_i \in \{1,\dots,m\} = \cM$ as comprising of a binary vector $(y_{i1},\dots,y_{im})$, with a single `1' at the position corresponding to the value that $y_i$ takes. 
%That is, $y_i = (y_{i1}, \dots, y_{im})$ with
%\[
%  y_{ik} =
%  \begin{cases}
%    1 &\text{ if } y_i = k \\
%    0 &\text{ if } y_i \neq k.
%  \end{cases}
%\]
In this formulation, each $y_{ij}$ is distributed as Bernoulli with probability $p_{ij}$. Now, assume that, for each $y_i = (y_{i1}, \dots, y_{im})$, there exists corresponding \emph{continuous, underlying, latent variables} $y_{i1}^*, \dots, y_{im}^*$ such that
\begin{align}\label{eq:latentmodel}
  y_i =
  \begin{cases}
    1 &\text{ if } y_{i1}^* \geq y_{i2}^*, y_{i3}^*, \dots, y_{im}^* \\
    2 &\text{ if } y_{i2}^* \geq y_{i1}^*, y_{i3}^*, \dots, y_{im}^* \\
    \,\vdots \\
    m &\text{ if } y_{im}^* \geq y_{i2}^*, y_{i3}^*, \dots, y_{i\,m-1}^*. \\
  \end{cases}  
\end{align}
In other words, 
%$y_{ij} = [y_{ij}^* = \max_k y_{ik}^*]$.
$y_{ij} = \argmax_{k=1}^m y_{ik}^*$.
Such a formulation is common in economic choice models, and is rationalised by a utility-maximisation argument: an agent faced with a choice from a set of alternatives will choose the one which benefits them most.

Instead of modelling the observed $y_{ij}$'s directly, we model instead the $n$ latent variables in each class $j=1,\dots,m$ according to the regression problem
\begin{equation}\label{eq:multinomial-latent}
  \begin{gathered}
    y_{ij}^* = \alpha_j + f_j(x_i) + \epsilon_{ij} \\
    \bepsilon_{i} = (\epsilon_{i1}, \dots, \epsilon_{im})^\top  \iid \N_m(\bzero, \bPsi^{-1}). 
  \end{gathered}
\end{equation}
%with $\alpha_j$ being an intercept, and $f_j:\cX \to \bbR$ a regression function belonging to some RKHS/RKKS of functions $\cF$.
%having the reproducing kernel $h_{\eta_j}: \cX \times \cX \to \bbR$. 
We can see some semblance of this model with the one in \cref{eq:naiveclassiprior}, and ultimately the aim is to assign I-priors to the regression function of these latent variables, and we will describe this shortly.
For now, realise that each $\by_i^* := (y_{i1}^*, \dots, y_{im}^*)^\top$ has the distribution $\N_m(\balpha + \bff(x_i), \bPsi^{-1})$, conditional on the data $x_i$,  the intercepts $\balpha = (\alpha_1,\dots,\alpha_m)^\top$, the evaluations of the functions at $x_i$ for each class $\bff(x_i) = \big(f_1(x_i), \dots, f_m(x_i)\big)^\top$, and the error covariance matrix $\bPsi^{-1}$.

\newcommand{\intset}{\{y_{ij}^* > y_{ik}^* \,|\, \forall k \neq j\}}
The probability of belonging to class $j$ for observation $i$, i.e. $p_{ij}$, is calculated as
\begin{align}
  p_{ij} 
  &= \Prob(y_i = j) \nonumber \\
  &= \Prob\big(\intset\big) \nonumber \\
  &= \int\displaylimits_{\intset} \phi(\by_i^*|\balpha + \bff(x_i), \bPsi^{-1}) \dint\by^* \label{eq:pij},
  %&=: g_{\bSigma}^{-1} \big( \alpha_j + f_j(x_i) \big)_{j=1}^m \nonumber
\end{align}
where $\phi(\cdot|\mu,\Sigma)$ is the density of the multivariate normal with mean $\mu$ and variance $\Sigma$.
This is the probability that the normal random variable $\by_i^*$ belongs to the set $\intset$, which are cones in $\bbR^m$.
Since the union of these cones is the entire $m$-dimensional space of reals, the probabilities add up to one and hence they represent a proper probability mass function of the classes.
%Upon knowing all values for $\alpha_j$, $f_j(x_i)$, and $\bSigma$, one is able to calculate $p_{ij}$ through the relationship \eqref{eq:pij}, which we denote as $g^{-1}$.
While this does not have a closed-form expression and highlights one of the difficulties of working with probit models, the integral is by no means impossible to compute---see \cref{misc:mnint} for a note regarding this matter.

Note that the dimension of the integral \eqref{eq:pij} is $m-1$, since the $j$'th coordinates is fixed relative to the others.
Alternatively, we could have specified the model in terms of \emph{relative differences} of the latent variables.
Choosing the first category as the reference category, define new random variables $z_{ij} = y_{ij}^* - y_{i1}^*$, for $j = 2,\dots,m-1$. 
The model \cref{eq:latentmodel} is equivalently represented by
\begin{equation}
  y_i = 
  \begin{cases}
    1 & \text{if } \max (z_{i2},\dots,z_{im}) < 0 \\
    j & \text{if } \max (z_{i2},\dots,z_{im}) = z_{ij} \geq 0.
  \end{cases}
\end{equation} 
Write $\bz_i = (z_{i2},\dots,z_{im})^\top \in \bbR^{m-1}$.
Then $\bz_i = \bQ\by^*_i$, where $\bQ \in \bbR^{(m-1)\times m}$  is the $(m-1)$ identity matrix pre-augmented with a column vector of minus ones.
We have that $\bz_i \iid \N_{m-1}\big(\bQ(\balpha + \bff(x_i)), \bQ\bPsi^{-1}\bQ^\top)\big)$.
Thus, the class probabilities for $j=2,\dots,m$ are
\begin{align}
  p_{ij} = 
  \int\displaylimits_{\{z_{ik} < 0 \,|\, \forall k \neq j \}} \ind(z_{ij} \geq 0) \,\phi(\bz_i) \dint\bz_i \label{eq:pij2},
\end{align}
with $\phi(\bz_i)$ representing the $(m-1)$-variate normal density for $\bz_i$.
The class probability $p_{i1}$ is simply
\[
  p_{i1} = \int\displaylimits_{\{ z_{ik} <0 \}}  \phi(\bz_i) \dint\bz_i = 1 - \sum_{k\neq 1} p_{ik}.
\]
From this representation of the model, with $m=2$ (binary outcomes) we see that
\[
  p_{i1} = 
%  \int \ind(z_{i2} < 0) \phi(z_{i2}) \dint z_{i2} = 
  \Phi \left( \frac{z_{i2} - \mu}{\sigma} \right)
  \hspace{0.5cm}\text{and}\hspace{0.5cm}
  p_{i2} = 
%  \int \ind(z_{i2} \geq 0) \phi(z_{i2}) \dint z_{i2} = 
  1 - \Phi \left( \frac{z_{i2} - \mu}{\sigma} \right),
\]
where $\Phi(\cdot)$ is the CDF of the standard normal univariate distribution, and $\mu$ and $\sigma$ are the mean and standard deviation of the random variable $z_{i2}$.

%\begin{figure}[t]
%  \centering
%  \begin{tikzpicture}[scale=1.1, transform shape]
%    \tikzstyle{main}=[circle, minimum size=10mm, thick, draw=black!80, node distance=16mm]
%    \tikzstyle{connect}=[-latex, thick]
%    \tikzstyle{box}=[rectangle, draw=black!100]
%      \node[main, draw=black!0] (blank) [xshift=-0.55cm] {};  % pushes image to right slightly
%      \node[main, fill=black!10] (H) [] {$x_i$};
%      \node[main] (Sigma) [below=of H, yshift=-1.2cm, xshift=0.6cm] {$\bSigma$};
%      \node[main, double, double distance=0.6mm] (f) [right=of H, xshift=0.5cm] {$f_{ij}$};
%      \node[main, double, double distance=0.6mm] (ystar) [right=of f, xshift=0cm] {$y_{ij}^*$};
%      \node[main, double, double distance=0.6mm] (pij) [right=of ystar, xshift=0cm] {$p_{ij}$};
%      \node[main] (lambda) [above=of f, xshift=0cm, yshift=-0.3cm] {$\eta_j$};        
%      \node[main] (alpha) [above=of ystar, xshift=0cm, yshift=-0.3cm] {$\alpha_j$};  
%      \node[main, fill = black!10] (y) [right=of pij, xshift=0.2cm] {$y_{i}$};
%      \node[main] (w) [below=of f, yshift=0.3cm] {$w_{ij}$};  
%      \path (alpha) edge [connect] (ystar)
%            (lambda) edge [connect] (f)
%            (H) edge [connect] node [above] {$h \ \ $} (f)
%    		(f) edge [connect] (ystar)
%    		(ystar) edge [connect] node [above] {$g^{-1}$}  (pij)
%            (pij) edge [connect] (y)
%            (Sigma) edge [connect] (w)
%    		(w) edge [connect] (f);
%      \node[rectangle, draw=black!100, fit={($(H.north west) + (0.2,0cm)$) ($(y.north east) + (-0.2,0.4cm)$) (w)}] {}; 
%      \node[rectangle, fit= (w) (y), label=below right:{$i=1,\dots,n$}, xshift=0.95cm, yshift=0.5cm] {};  % the label
%      \node[rectangle, draw=black!100, fit={(lambda) ($(pij.north east) + (0.5cm,0.7cm)$) ($(w.south west) + (-0.5,-0.7cm)$)}] {}; 
%      \node[rectangle, fit={(f) ($(ystar.north east) + (0.5cm,0.7cm)$) ($(w.south west) + (-0.5,-0.7cm)$)}, label=below right:{$j=1,\dots,m$}, xshift=0.63cm, yshift=0.37cm] {}; 
%    \end{tikzpicture}
%    \caption{A DAG of the probit I-prior  model. Observed nodes are shaded, while double-lined nodes represented known or calculable quantities. There are at most $m-1$ sets of intercept ($\alpha_j$) and RKHS parameters ($\eta_j$)  to estimate due to identifiability. Depending on the specification of $\bSigma$, this may need to be estimated too.}
%\end{figure}

Now we'll see how to specify an I-prior on the regression problem \cref{eq:multinomial-latent}.
In the naïve I-prior model, we wrote $f(x_i,j) = \alpha_j + f_j(x_i)$, and specified for $f$ to belong to an ANOVA RKKS with kernel defined in \cref{eq:anovaclass}.
Instead of doing the same, we take a different approach.
Treat the $\alpha_j$'s in \cref{eq:multinomial-latent} as intercept parameters to estimate with the additional requirement that $\sum_{j=1}^m \alpha_j = 0$.
Further, let $\cF$ be a (centred) RKHS/RKKS of functions over $\cX$ with reproducing kernel $h_\eta$.
Now, consider putting an I-prior on the regression functions $f_j \in \cF$, $j=1\dots,m$, defined by
\[
  f_j(x_i) = \sum_{k=1}^n h_\eta(x_i,x_k)w_{ik}
\]
with $\bw_i := (w_{i1},\dots,w_{im})^\top \iid \N(0,\bPsi)$.
This is similar to the naïve I-prior specification \cref{eq:naiveclassiprior}, except that the intercept  have been treated as parameters rather than accounting for them using an RKHS of constant functions.
In particular, the overall regression relationship still satisfies the ANOVA functional decomposition.
We find that this method bodes well down the line computationally.

We call the multinomial probit regression model of \cref{eq:latentmodel} subject to \cref{eq:multinomial-latent} and I-priors on $f_j \in \cF$, the \emph{I-probit model}.
For completeness, this is stated again: for $i=1,\dots,n$, $y_i = \argmax_{k=1}^m y_{ik}^* \in \{1,\dots,m\}$, where, for $j=1,\dots,m$,
\begin{align}
  \begin{gathered}
    y_{ij}^* = \alpha_j + 
    \greyoverbrace{\sum_{k=1}^n h_\eta(x_i,x_k)w_{ik}}{f_j(x_i)}
    + \epsilon_{ij} \\
    \bepsilon_{i} = (\epsilon_{i1}, \dots, \epsilon_{im})^\top  \iid \N_m(\bzero, \bPsi^{-1}) \\
    \bw_i := (w_{i1},\dots,w_{im})^\top \iid \N(0,\bPsi).
  \end{gathered}
\end{align}
The parameters of the I-probit model are denoted by $\theta = \{\alpha_1,\dots,\alpha_m,\eta,\bPsi \}$.
Let $\by^* \in \bbR^{n \times m}$ denote the matrix containing $(i,j)$ entries $y_{ij}^*$.
Using the results in Chapter 4, the marginal distribution of the latent variables is
\[
  \vecc \by^* \sim \N_{nm}\big(\balpha, (\bPsi \otimes \bH_\eta^2) + (\bPsi^{-1} \otimes \bI_n)\big).
\]

\subsection{IIA}

In decision theory, the independence axiom states that an agent's choice between a set of alternatives should not be affected by the introduction or elimination of a (new) choice option.
The probit model is suitable for modelling multinomial data where the independence axiom, which is also known as the \emph{independence of irrelevant alternatives} (IIA) assumption, is not desired. 
Such cases arise frequently in economics and social science, and the famous Red-Bus-Blue-Bus example is often used to illustrate IIA.
Suppose commuters face the decision between taking cars and red busses. 
The addition of blue busses to commuters' choice should in theory be more likely chosen by those who prefer taking the bus over cars.
That is, assuming commuters are indifferent about the colour of the bus, commuters who are predisposed to taking the red bus would see the blue bus as an identical alternative.
 Yet, if IIA is imposed, then the three choices are distinct, and the fact that red and blue busses are substitutable is ignored.

In the I-probit model, the choice dependency is controlled by the error precision matrix $\bPsi$.
Specifically, the off-diagonal elements $\bPsi_{jk}$ capture the correlation between choices $j$ and $k$.
Allowing all $m(m+1)/2$ covariance elements of $\bPsi$ leads to the \emph{full I-probit model}, and would not assume an IIA position.

%Although crucial in choice models, it is not so much necessary for classification tasks when the alternatives under consideration are distinctly different.
%In such cases, one may choose to abandon the IIA

While it is an advantage to be able to model the correlations across choices (unlike in logistic models), it would be a major simplification algorithmically to consider all covariances in $\bPsi$ to be zero.
This would trigger the IIA assumption in the I-probit model.
There are applications where the IIA assumption would not adversely affect the analysis, such as when all the choices are mutually exclusive and exhaustive.
In these situations, it would be beneficial to reduce the I-probit model to a simpler version by assuming $\bPsi = \diag(\psi_1,\dots,\psi_m)$.

The independence assumption causes the distribution of the latent variables to be $y_{ij}^* \sim \N(\alpha_j + f_j(x_i), \sigma_j^2)$ for $j=1,\dots,m$.
As a continuation of line \eqref{eq:pij}, we can show the class probability $p_{ij}$ to be
\begin{align*}
  p_{ij} 
  &= \idotsint\displaylimits_{\{y_{ij}^* > y_{ik}^* | \forall k \neq j\}} 
  \prod_{k=1}^m \Big\{ p(y_{ik}^*|\alpha_j + f_k(x_i), \sigma_j^2) \d y_k^* \Big\} \\
  &= \int \mathop{\prod_{k=1}^m}_{k\neq j} 
  \Phi \left( \frac{y_{ij}^* - \alpha_k - f_{ik}}{\sigma_k} \right) \cdot
  \frac{1}{\sigma_j} \phi \left( \frac{y_{ij}^* - \alpha_j - f_{ij}}{\sigma_j} \right)  \d y_{ij}^* \\
  &= \E_Z \Bigg[ \mathop{\prod_{k=1}^m}_{k\neq j} 
  \Phi \left(\frac{\sigma_j}{\sigma_k} Z + \frac{\alpha_j + f_{ij} - \alpha_k - f_{ik}}{\sigma_k} \right) \Bigg] \nonumber
\end{align*}
where $Z\sim\N(0,1)$, and $\phi(\cdot)$ and $\Phi(\cdot)$ are its PDF and CDF respectively.
%In the binary case where $m=2$, we set $\sigma_1^2 = 2$ and fix $f_{i2} = 0$, and we get that 
%\[
%  p_{i1} = 1 - \Phi(\alpha_1 + f_{i1}) \ \text{ and } \ p_{i2} =  \Phi(\alpha_1 + f_{i1}),
%\]
%which clearly shows the probit relationship between the class probabilities and the latent regression.
The proof of this fact is included in the Appendix.
With the exception of the binary case, these probabilities still do not have a closed-form expression (per se) and numerical methods are required to calculate them.
In this simplified version of the I-probit model, the integral is unidimensional and involves the Gaussian PDF, and this can be efficiently obtained using quadrature methods.


%\begin{table}[]
%\centering
%\caption{My caption}
%\label{my-label}
%\begin{tabular}{@{}lp{5cm}p{5cm}@{}}
%\toprule
%                    & Naïve model                                                                              & I-probit model                                            \\ \midrule
%Grand intercept     & Fixed at $\alpha = 1/m$, equivalently $y_{ij}\mapsto y_{ij} - 1/m$ and set $\alpha = 0$. & $\alpha=0$ (otherwise, model is not location identified). \\
%Intercepts          & $\alpha_j \in \cF_\cM$, a centred RKHS with kernel $a$                                                  & $\alpha_j\in\bbR$, $\sum_{j=1}^m \alpha_j = 0$            \\
%Interaction effects & $f_j \in \cF_\cM \otimes \cF_\cX$, tensor product interaction space with kernel $ah_\eta$                     & $f_j\in\cF$ with kernel $h_\eta$                                                \\ \bottomrule
%\end{tabular}
%\end{table}




%A rearrangement of the error terms is beneficial for the process of deriving the required I-priors.
%Recall that since each $\bepsilon_i \iid \N_m(\bzero,\bSigma)$ then concatenating all of the error vectors result in $\bepsilon = (\bepsilon_1^\top,\dots,\bepsilon_n^\top)^\top$, and this has distribution $\N_{nm}(\bzero, \bOmega)$, where $\bOmega = \diag(\bSigma,\dots,\bSigma)$.
%By rearranging the entries of $\bepsilon$ according to class (instead of by observation) and defining $\bepsilon_j'=(\epsilon_{1j},\dots,\epsilon_{nj})^\top$ for $j=1,\dots,m$, we observe that
%\[
%  \bepsilon' = 
%  \begin{pmatrix}
%    \bepsilon_1' \\
%    \vdots \\
%    \bepsilon_m'
%  \end{pmatrix}
%  \sim \N_{nm}(\bzero,\bOmega')
%\]
%where $\bOmega'$ contains the appropriately rearranged entries of $\bOmega$. 
%For illustration, an example of this rearrangement is shown in Figure \ref{fig:covrearrange} for $m = 2$ and $n=5$.
%The subtle difference is that whereas $\bOmega$ is a block diagonal matrix, $\bOmega'$ is comprised of $m^2$ equally sized $n \times n$ partitions of diagonal matrices.
%However, the number of unique elements in $\bOmega$ and $\bOmega'$ is the same as in the $m \times m$ matrix $\bSigma$, i.e. $m(m+1)/2$.
%We can also express these matrices in terms of Kronecker products $\bOmega = \bI_n \otimes \bSigma$ and $\bOmega = \bSigma \otimes \bI_n$.
%
%Denote $f_{ij} = f_j(x_i)$ as the evaluation of the function $f_j(\cdot)$ at $x_i$, and also $\bff_j = (f_{1j}, \dots, f_{nj})^\top$ as the vector containing all $n$ evaluations pertaining to the $j$th class.
%Concatenate all of the $\bff_j$'s into one long vector of length $nm$: $\bff = (\bff_1^\top, \dots, \bff_m^\top)^\top$.
%The I-prior is $\bff \sim \N_{nm}(\bzero,\bV_f)$, where $\bzero$ is an vector of length $nm$ containing zeroes and the covariance matrix $\bV_f$ is has the block matrix structure
%\begin{align}\label{eq:vf}
%  \bV_f = \begin{pmatrix}
%    \bV_{11} & \bV_{12} & \dots  & \bV_{1m} \\
%    \bV_{21} & \bV_{22} & \dots  & \bV_{2m} \\
%    \vdots   & \vdots   & \ddots & \vdots \\
%    \bV_{m1} & \bV_{m2} & \dots  & \bV_{mm} \\
%  \end{pmatrix},
%\end{align}
%and each block entry above is given by
%\begin{align}\label{eq:fisherinformation}
%  \bV_{jk}(r,s) = \cI\big( f_j(x_r), f_k(x_s) \big) = \bPsi_{jk} \sum_{r'=1}^n \sum_{s'=1}^n  h_{\eta_j}(x_r, x_{r'}) h_{\eta_k}(x_{s'}, x_s),
%\end{align}
%where $\bPsi = \bSigma^{-1}$. 
%Equation \eqref{eq:fisherinformation} gives the Fisher information for the regression functions $f_j$ and $f_k$, $j,k\in\{1,\dots,m\}$, evaluated at two points $x_r$ and $x_s$, $r,s\in\{1,\dots,n\}$---see \cite{bergsma2017} for details.
%
%One may also write $\bff = \bG\bw$ where $\bw \sim \N_{nm}(\bzero, \bOmega'^{-1})$, $\bG = \diag(\bH_{\eta_1},\dots,\bH_{\eta_m})$, and $\bH_{\eta_j}$ are the $n\times n$ kernel matrices with $(r,s)$ entries $h_{\eta_j}(x_r,x_s)$.
%Realise that $\bOmega^{-1}$ is obtainable by repeating $\bSigma^{-1}$ diagonally, and that $\bOmega'^{-1}$ is then obtained using a similar rearrangement of its rows and columns to that of $\bOmega$.
%Substituting equation \eqref{eq:fisherinformation} into the model \eqref{eq:multinomial-latent}, we get that
%\begin{align}
%  \begin{gathered}\label{eq:ipriorw}
%    y_{ij}^* = \alpha_j + \sum_{r=1}^n h_{\eta_j}(x_i,x_r)w_{rj} + \epsilon_{ij} \\
%    (w_{i1},\dots,w_{im}) \iid \N_m(\bzero, \bSigma^{-1}) \\
%    (\epsilon_{i1},\dots,\epsilon_{im}) \iid \N_m(\bzero, \bSigma)
%  \end{gathered}
%\end{align}
%This formulation will be useful in the variational algorithm later on.
%





\section{Identifiability and IIA}\label{sec:iia}
The parameters in a linear multinomial probit model is well known to be unidentified \citep{Keane1992,train2009discrete}, and the reason for this is two-fold.
Firstly, an addition of a constant to the latent variables $y_{ij}^*$'s in \cref{eq:latentmodel} will not change which latent variable is maximal, and therefore leaves the model unchanged.
Secondly, all latent variables can be scaled by some positive constant without changing which latent variable is largest.
Therefore, a \emph{linear parameterisation} for the multinomial probit model is not identified as there can be more than one set of parameters for which the class probabilities are the same.
To fix this issue, constraints are imposed on location and scale of the latent variables.

However, for the I-probit model, this is not the case, because the model is not related to the parameters $\theta = \{\alpha_1,\dots,\alpha_m,\eta, \bPsi \}$ linearly.
One cannot simply add to or multiply $\theta$ by a constant and expect the model to be left unchanged.
Thus, the I-probit model is identified in the parameter set $\theta$ without having to impose any restrictions, particularly on the precision matrix $\bPsi$ (if this is to be estimated).

To see how the I-probit model is location identified, suppose assumptions \ref{ass:A4} and \ref{ass:A5} hold, and consider a constant $a$ added to the latent propensities.
This would then imply the relationship
\[
  a + y_{ij}^* = 
  \greyoverbrace{a + \alpha_j}{\alpha_j^*}  + f_j(x_i) + \epsilon_{ij},
\]
which is similar to adding the constant $a$ to all of the intercept parameters $\alpha_j$---denote these new intercepts by $\alpha_j^*$.
As a requirement of the functional ANOVA decomposition, the $\alpha_j^*$'s need to sum to zero, but we already have that $\sum_{j=1}^m \alpha_j=0$, so it must be that $a =0$.
This also highlights the reason behind assumption \ref{ass:A4} and \ref{ass:A5} for fixing the grand intercept $\alpha$ to zero.

As for identification in scale, consider multiplying the latent variables by $c>0$.
Denote by $\bV_y^*(\omega) \in \bbR^{nm \times nm}$ the marginal covariance matrix of the latent propensities, which depends on the scale parameters $\omega = \{\eta, \bPsi\}$.
The scaled latent variables $\{c^{1/2}y^*_{ij} \,|\, \forall i,j = 1,\dots \}$, which collectively has (marginal) variance and covariances given by the matrix $c \bV_y^*(\omega)$, is expected to have been generated from the model with parameters $c\omega$.
However, we have that 
\begin{align*}
  c\bV_y^*(\omega)
  &= c (\bPsi \otimes \bH_\eta^2) + c (\bPsi^{-1} \otimes \bI_n) \\
  &= (c \bPsi \otimes \bH_\eta^2) + (c \bPsi^{-1} \otimes \bI_n) \\
  &\neq \bV_y^*(c\omega).
\end{align*}

Now, we turn to a discussion of the role of $\bPsi$ in the model.
In decision theory, the independence axiom states that an agent's choice between a set of alternatives should not be affected by the introduction or elimination of a choice option.
The probit model is suitable for modelling multinomial data where the independence axiom, which is also known as the \emph{independence of irrelevant alternatives} (IIA) assumption, is not desired. 
Such cases arise frequently in economics and social science, and the famous Red-Bus-Blue-Bus example is often used to illustrate IIA:
suppose commuters face the decision between taking cars and red busses. 
The addition of blue busses to commuters' choices should, in theory, be more likely chosen by those who prefer taking the bus over cars.
That is, assuming commuters are indifferent about the colour of the bus, commuters who are predisposed to taking the red bus would see the blue bus as an identical alternative.
 Yet, if IIA is imposed, then the three choices are distinct, and the fact that red and blue busses are substitutable is ignored.

To put it simply, the model is IIA if choice probabilities depend only on the choice in consideration, and not on any other alternatives.
In the I-probit model, or rather, in probit models in general, choice dependency is controlled by the error precision matrix $\bPsi$.
Specifically, the off-diagonal elements $\bPsi_{jk}$ capture the correlation between alternatives $j$ and $k$.
Allowing all $m(m+1)/2$ covariance elements of $\bPsi$ to be non-zero leads to the \emph{full I-probit model}, and would not assume an IIA position.

\begin{figure}[hbt]
\centering\hspace{-13pt}
\begin{blockmatrixtabular}
\valignbox{
\begin{blockmatrixtabular}
&
\mblockmatrix{0.55in}{0in}{\footnotesize $j=1$}&
\mblockmatrix{0.55in}{0in}{\footnotesize $j=2$}&
\mblockmatrix{0.55in}{0in}{$\cdots$}&
\mblockmatrix{0.55in}{0in}{\footnotesize $j=m$}& \\
\mblockmatrix{0in}{0.55in}{\footnotesize $j=1$}&
\fblockmatrix[colblu!39]{0.55in}{0.55in}{\footnotesize $\bV[1,1]$}& 
\fblockmatrix[colblu!22]{0.55in}{0.55in}{\footnotesize $\bV[1,2]$}&
\fblockmatrix[colblu!24]{0.55in}{0.55in}{\footnotesize $\cdots$}& 
\fblockmatrix[colblu!46]{0.55in}{0.55in}{\footnotesize $\bV[1,m]$}\\
\mblockmatrix{0in}{0.55in}{\footnotesize $j=2$}&
\fblockmatrix[colblu!22]{0.55in}{0.55in}{\footnotesize $\bV[2,1]$}& 
\fblockmatrix[colblu!20]{0.55in}{0.55in}{\footnotesize $\bV[2,2]$}&
\fblockmatrix[colblu!42]{0.55in}{0.55in}{\footnotesize $\cdots$}& 
\fblockmatrix[colblu!40]{0.55in}{0.55in}{\footnotesize $\bV[2,m]$}\\
\mblockmatrix{0in}{0.55in}{\hspace{10pt}$\vdots$}&
\fblockmatrix[colblu!24]{0.55in}{0.55in}{\footnotesize $\vdots$}& 
\fblockmatrix[colblu!42]{0.55in}{0.55in}{\footnotesize $\vdots$}&
\fblockmatrix[colblu!33]{0.55in}{0.55in}{\footnotesize $\ddots$}& 
\fblockmatrix[colblu!30]{0.55in}{0.55in}{\footnotesize $\vdots$}\\
\mblockmatrix{0in}{0.55in}{\footnotesize $j=m$}&
\fblockmatrix[colblu!46]{0.55in}{0.55in}{\footnotesize $\bV[m,1]$}& 
\fblockmatrix[colblu!40]{0.55in}{0.55in}{\footnotesize $\bV[m,2]$}&
\fblockmatrix[colblu!30]{0.55in}{0.55in}{\footnotesize $\cdots$}& 
\fblockmatrix[colblu!20]{0.55in}{0.55in}{\footnotesize $\bV[m,m]$}\\
\end{blockmatrixtabular}
}&
\valignbox{\mblockmatrix{0.31in}{2.8in}{}}&
\valignbox{
\begin{blockmatrixtabular}
%&
\mblockmatrix{0.55in}{0in}{\footnotesize $j=1$}&
\mblockmatrix{0.55in}{0in}{\footnotesize $j=2$}&
\mblockmatrix{0.55in}{0in}{$\cdots$}&
\mblockmatrix{0.55in}{0in}{\footnotesize $j=m$}& \\
%\mblockmatrix{0in}{0.55in}{\footnotesize $j=1$}&
\fblockmatrix[colblu!39]{0.55in}{0.55in}{\footnotesize $\bV[1,1]$}& 
\fblockmatrix[none]{0.55in}{0.55in}{}&
\fblockmatrix[none]{0.55in}{0.55in}{}& 
\fblockmatrix[none]{0.55in}{0.55in}{}\\
%\mblockmatrix{0in}{0.55in}{\footnotesize $j=2$}&
\fblockmatrix[none]{0.55in}{0.55in}{}& 
\fblockmatrix[colblu!20]{0.55in}{0.55in}{\footnotesize $\bV[2,2]$}&
\fblockmatrix[none]{0.55in}{0.55in}{}& 
\fblockmatrix[none]{0.55in}{0.55in}{}\\
%\mblockmatrix{0in}{0.55in}{\hspace{10pt}$\vdots$}&
\fblockmatrix[none]{0.55in}{0.55in}{}& 
\fblockmatrix[none]{0.55in}{0.55in}{}&
\fblockmatrix[colblu!33]{0.55in}{0.55in}{\footnotesize $\ddots$}& 
\fblockmatrix[none]{0.55in}{0.55in}{}\\
%\mblockmatrix{0in}{0.55in}{\footnotesize $j=m$}&
\fblockmatrix[none]{0.55in}{0.55in}{}& 
\fblockmatrix[none]{0.55in}{0.55in}{}&
\fblockmatrix[none]{0.55in}{0.55in}{}& 
\fblockmatrix[colblu!20]{0.55in}{0.55in}{\footnotesize $\bV[m,m]$}\\
\end{blockmatrixtabular}
}&
\end{blockmatrixtabular}\\ 
\caption[Illustration of the covariance structure of the full I-probit model and the independent I-probit model.]{Illustration of the covariance structure of the full I-probit model (left) and the independent I-probit model (right). The full model has  $m^2$ blocks of $n \times n$ symmetric matrices, and the blocks themselves are arranged symmetrically about the diagonal. The independent model, on the other hand, has a block diagonal structure, and its sparsity induces simpler computational methods for estimation.}
\label{fig:iprobcovstr}
\end{figure}

%most economics articles prefer to estimate scaled probit models. in fact, it is an advantage of it! but do we care about the scale? maybe care more about IIA, which can't do without scales i suppose.

While it is an advantage to be able to model the correlations across choices (unlike in logistic models), there are applications where the IIA assumption would not adversely affect the analysis, such as classification tasks.
Some analyses might also be indifferent as to whether or not choice dependency exists.
In these situations, it would be beneficial, algorithmically speaking, to reduce the I-probit model to a simpler version by assuming $\bPsi = \diag(\psi_1,\dots,\psi_m)$, which would trigger the IIA assumption in the I-probit model.
We refer to this model as the \emph{independent I-probit model}.

The independence assumption causes the distribution of the latent variables to be $y_{ij}^* \sim \N(\mu_k(x_i), \sigma_j^2)$ for $j=1,\dots,m$, where $\sigma_j^2 = \psi_j^{-1}$.
As a continuation of line \cref{eq:pij}, we can show the class probability $p_{ij}$ to be
\begin{align}
  p_{ij} 
  &= \idotsint\displaylimits_{\{y_{ij}^* > y_{ik}^* | \forall k \neq j\}} 
  \prod_{k=1}^m \Big\{ \phi(y_{ik}^*|\mu_k(x_i), \sigma_k^2) \dint y_{ik}^* \Big\} \nonumber \\
  &= \int \mathop{\prod_{k=1}^m}_{k\neq j} 
  \Phi \left( \frac{y_{ij}^* - \mu_k(x_i)}{\sigma_k} \right) \cdot
   \phi(y_{ij}^*|\mu_j(x_i), \sigma_j^2)  \dint y_{ij}^* \nonumber \\
  &= \E_Z \Bigg[ \mathop{\prod_{k=1}^m}_{k\neq j} 
  \Phi \left(\frac{\sigma_j}{\sigma_k} Z + \frac{\mu_j(x_i) - \mu_k(x_i)}{\sigma_k} \right) \Bigg] \label{eq:pij2}
\end{align}
where $Z\sim\N(0,1)$, $\Phi(\cdot)$ its cdf, and $\phi(\cdot|\mu,\sigma^2)$ is the pdf of $X\sim\N(\mu,\sigma^2)$.
The equation \cref{eq:pij} is thus simplified to a unidimensional integral involving the Gaussian pdf and cdf, which can be computed fairly efficiently using quadrature methods.
The probit link function is evidently seen in the above equation.
%Moreover, in the binary case where $m=2$ and fixed error precision $\psi_1 = \psi_2 = 1/2$, we get
%\[
%  p_{i1} = 1 - \Phi\big(\mu_1(x_i) - \mu_2(x_i)\big) \ \text{ and } \ p_{i2} =  \Phi\big(\mu_1(x_i) - \mu_2(x_i)\big),
%\]
%which clearly shows the probit relationship between the class probabilities and the latent regression.
%The proof of this fact is included in the Appendix.
%With the exception of the binary case, these probabilities still do not have a closed-form expression (per se) and numerical methods are required to calculate them.


\section{Estimation}
%The challenge of estimation is then to first overcome this intractability by means of a suitable approximation of the integral.
%Several methods may be employed, namely quadrature methods, Laplace approximation, and Markov chain Monte Carlo (MCMC) methods, but these all fail in the face of high dimensionality when the sample size $n$ is large.

As with the normal I-prior model, an estimate of the posterior regression function with optimised hyperparameters is sought.
The log likelihood function $L(\cdot)$ for $\theta$ using all $n$ observations $\{(y_1,x_1),\dots,(y_n,x_n)\}$ is obtained by integrating out the I-prior from the categorical likelihood, as follows:
\begin{align}
  L(\theta) 
  &= \log \int p(\by | \bw, \theta) \, p(\bw|\theta) \dint \bw \nonumber \\
  &= \log \int \prod_{i=1}^n \prod_{j=1}^m \Big( g^{-1}\big(\alpha_k + 
  \greyoverbrace{f_k(x_i)}{\hidewidth\sum_{i'=1}^n h_\eta(x_i,x_{i'})w_{i'}\hidewidth}
  \,\big)_{k=1}^m \Big)^{[y_i=j]} \cdot \phi(\bw|\bzero,  \bPsi \otimes \bI_n ) \dint \bw \label{eq:intractablelikelihood}
\end{align}
where we have denoted the probit relationship from \eqref{eq:pij} using the function $g^{-1}:\bbR^m \to [0,1]$.
Unlike in the continuous response models, the integral does not present itself in closed form due to the conditional categorical PMF of the $y_i$'s, which they themselves involve integrals of normal densities.
Furthermore, the posterior distribution of the regression function, which requires the density of $\bw|\by$, depends on the marginalisation provided by \cref{eq:intractablelikelihood}.
The challenge of estimation is then to first overcome this intractability by means of a suitable approximation of the integral.
We present three possible avenues to achieve this aim, namely the Laplace approximation, Markov chain Monte Carlo (MCMC) methods, and  variational Bayes.

%Methods of approximating the integral in \eqref{eq:intractablelikelihood} such as quadrature methods, Laplace approximation and MCMC tend to fail or are unsatisfactorily slow to accomplish.
%The main reason for this is the dimension of this integral, which is $nm$, and in particular, for large sample sizes and/or number of classes, is unfeasible for such methods.

\subsection{Laplace approximation}

One is interested in the posterior density $p(\bw|\by) \propto p(\by|\bw)p(\bw) =: e^{Q(\bw)}$, with normalising constant equal to the marginal density of $\by$, $p(\by) = \int e^{Q(\bw)} \dint \bw$, and we have established that the calculation of this marginal density is intractable.
Laplace's method \citep[§4.1.1, pp. 777--778]{kass1995bayes} entails expanding a Taylor series for $Q$ about its posterior mode $\hat\bw = \argmax_\bw p(\by|\bw)p(\bw)$, and this gives the relationship
\begin{align*}
  Q(\bw) 
  &= Q(\hat\bw) + 
  \cancelto{0}{(\bw - \hat\bw)^\top \nabla Q(\hat\bw)} 
  - \half (\bw - \hat\bw)^\top \bOmega (\bw - \hat\bw) + \cdots \\
  &\approx Q(\hat\bw) + 
  - \half (\bw - \hat\bw)^\top \bOmega (\bw - \hat\bw)
\end{align*}
because, assuming that $Q$ has a unique maxima, $\nabla Q$ evaluated at its mode is zero.
This is recognised as the logarithm of an unnormalised Gaussian density, implying $\bw|\by \sim \N_n(\hat\bw,\bOmega^{-1})$.
Here, $\bOmega = -\nabla^2 Q(\bw)|_{\bw=\hat\bw}$ is the negative Hessian of $Q$ evaluated at the posterior mode.

The marginal distribution is then approximated by
\begin{align*}
  p(\by) 
  &\approx \int \exp
  \greyoverbrace{Q(\bw)}{\hidewidth Q(\hat\bw) - \half (\bw - \hat\bw)^\top \bOmega (\bw - \hat\bw)\hidewidth}
   \dint \bw \\
  &= (2\pi)^{n/2} \abs{\bOmega}^{-1/2} e^{Q(\hat\bw)} 
  \int (2\pi)^{-n/2} \abs{\bOmega}^{1/2} \exp \left(- \half (\bw - \hat\bw)^\top \bOmega (\bw - \hat\bw) \right) \dint\bw \\
  &= (2\pi)^{n/2} \abs{\bOmega}^{-1/2} p(\by|\hat\bw)p(\hat\bw).
%  &= \cancelto{1}{\int \phi(\bw|\hat\bw,\bOmega^{-1})}
\end{align*} 
The log marginal density of course depends on the parameters $\theta$, which becomes the objective function to maximise in a likelihood maximising approach.
Note that, should a fully Bayesian approach be undertaken, i.e. priors prescribed on the model parameters using $\theta \sim p(\theta)$, then this approach is viewed as a maximum a posteriori approach.

In fact, under an EM algorithm approach, using the approximate posterior density which is normally distributed is simply using the posterior mode in lieu of the actual posterior means.
\hltodo[Expand on this further.]{}

In any case, each evaluation of the objective function $L(\theta) = \log p(\by|\btheta)$ involves finding the posterior modes $\hat\bw$.
This is a slow and difficult undertaking, especially for large sample sizes $n$, because the dimension of this integral is exactly the sample size.
Furthermore, standard errors for the parameters are cumbersome to calculate as well.
Lastly, as a comment, Laplace's method only approximates the true marginal likelihood well if the true function is small far away from the mode.

% IT TURNS OUT THE EM IS DIFFICULT! BECAUSE THERE IS NO INHERENT ASSUMPTION OF INDEPENDENCE BETWEEN YSTAR AND W, SO THE DISTRIBUTIONS ARE DIFFICULT TO ASCERTAIN. 
% IN THE VARIATIONAL ALGORITHM, THIS IS ASSUMED IN A MEAN-FIELD APPROXIMATION
%\subsection{Expectation-maximisation algorithm}
%
%An EM algorithm similar to the one seen in the previous Chapter can be employed, with a slight modification.
%This time, treat both the latent propensities $\by^*$ and the I-prior random effects $\bw$ as `missing', so the complete data is $\{\by,\by^*,\bw\}$.
%Now, due to the independence of the observations $i=1,\dots,n$, the complete data log-likelihood is
%\begin{align*}
%  \log p&(\by,\by^*,\bw) \\
%  &= \sum_{i=1}^n \Big\{ 
%  \log p(y_i|\by^*_{i \bigcdot}) + \log p(\by^*_{i \bigcdot}|\bw_{i \bigcdot}) + \log p(\bw_{i \bigcdot}) 
%  \Big\} \\
%  &= - \half \sum_{i=1}^n \ind[y_{ij}^* = \max_k y_{ik}^*] \Bigg[
%%   \cancel{\half\log\abs{\bPsi}} 
%    (\by^*_{i \bigcdot} - \balpha - \bw_{i \bigcdot}^\top\bH_\eta)^\top \bPsi (\by^*_{i \bigcdot} - \balpha - \bw_{i \bigcdot}^\top\bH_\eta)
%   +  \bw_{i \bigcdot}^\top \bPsi^{-1} \bw_{i \bigcdot} \Bigg] \\
%  &\phantom{==} 
%%  \cancel{- \half\log\abs{\bPsi}} 
%   + \const
%\end{align*}
%which looks like the complete data log-likelihood seen previously in \cref{eq:QfnEstep}, except that here, together with the $\bw_{i \bigcdot}$'s, the $\by^*_{i \bigcdot}$'s are never observed.
%
%For the E-step, it is of interest to determine the posterior density $p(\by^*,\bw|\by) = p(\by^*|\bw,\by)p(\bw|\by) = p(\bw|\by^*,\by)p(\by^*|\by)$.
%For the latent propensities, the conditional posterior mean of a normally distributed subject to a conical truncation $\cC_j = \{y_{ij}* > y_{ik}^* \,|\, \forall k \neq j \}$, i.e. $\by^*_{i \bigcdot}|\bw_{i \bigcdot},\{y_i=j\} \iid \tN_m(\balpha + \hat\bw_{i \bigcdot}^\top\bH_\eta, \bPsi^{-1},\cC_j)$, for each $i=1,\dots,n$, and $\hat\bw$ is the posterior mean of $\bw$.
%The distribution for $\bw|\by^*$ is found to be similar to the posterior distribution of $\bw$ in the normal case as in \cref{eq:posteriorw}, except with $\by$ replaced with $\by^*$.
%To be specific, $\vecc \bw | \by^* \sim \N_{nm}(\vecc \tilde\bw, \tilde \bV_w)$, where
%\begin{align*}
%  \vecc \tilde\bw = \tilde\bV_w (\bPsi \otimes \bH_\eta) \vecc(\by^* - \bone_n\balpha^\top)
%  \hspace{0.5cm}\text{and}\hspace{0.5cm}
%  \tilde \bV_w^{-1} = (\bPsi \otimes \bH_\eta^2) + (\bPsi^{-1} \otimes \bI_n).
%\end{align*}
%To obtain the first and second posterior moments for the I-prior random effects, use the law of total expectations:
%\begin{gather*}
%  \E[\vecc \bw|\by]
%  = \E \big[ \E[\vecc \bw | \by^*] \big| \by \big] =: \hat\bw \\
%  \text{and}\\
%  \E[\vecc \bw (\vecc \bw)^\top|\by]
%  = \E \big[ \E[\vecc \bw (\vecc \bw)^\top| \by^*] \big| \by \big] =: \hat\bW .
%\end{gather*}
%The $Q$ function, whose argument is $\theta = \{\balpha,\eta,\bPsi\}$, is then 
%\begin{align*}
%  Q(\theta) 
%  &= \E_{\by^*,\bw}\big[ \log p&(\by,\by^*,\bw) | \by,\theta^{(t)}\big] \\
%  &= \const - \half 
%\end{align*}
%

\subsection{Markov chain Monte Carlo methods}

\citet{albert1993bayesian} showed that the latent variable approach to probit models can be analysed using exact Bayesian methods, due to the underlying normality structure.
Paired with corresponding conjugate prior choices, sampling from the posterior is very simple using a Gibbs sampling approach.
On the other hand, this data augmentation scheme enlarges the variable space to $n+q$ dimensions, where $q$ is the number of parameters to estimate, which is inefficient and computationally challenging especially when $n$ is large.
It is no longer possible to marginalise the normal latent variables from the model, as this is intractable, just as we discussed previously.

Hamiltonian Monte Carlo is another possibility, since it does not require conjugacy.
For binary models, this is a feasible approach because the class probabilities are a function of the normal CDF, which means that it is doable in off-the-shelf software such as \proglang{Stan}.
Things get out of hand with multinomial responses, because the intractability of computing class probabilities is not addressed.

In summary, the computational challenge here stems from two sources: 1) integrating out the random effects $\bw$; and 2) evaluating class probabilities.
Point 1) is addressed using a Gibbs sampling data augmentation scheme (latent variable approach), but this is not feasible with large $n$.
Point 2) remains regardless whether Gibbs sampling or HMC is used.



\subsection{Variational inference}

We turn to variational inference as a method of estimation. Variational methods are widely discussed in the machine learning literature, and there have been efforts to popularise it in statistics \citep{blei2017variational}.

By factorising appropriately, we can obtain approximated posteriors for the regression function and the parameters of the I-prior model.
The algorithm itself typically condenses to that of a simple, sequential updating scheme, akin to the expectation-maximisation (EM) algorithm for exponential families we saw in Chapter 4, which is very fast to implement compared to the other methods described in the previous subsections.
A full derivation of the variational algorithm used by us will be described in \cref{sec:iprobitvar}.
%\citep{mclachlan2007algorithm}

\subsection{Comparison of estimation methods}

\hltodo{Compare: Laplace, variational and HMC.}

The three estimation methods described aim to overcome the intractable integral by means of either a deterministic approximation (Laplace and variational inference) or a stochastic approximation (MCMC).
In the Laplace and variational method, the posterior distribution of $\bw$ ends up being approximated by a Gaussian distribution, although the mean and variance is different in each method.
In essence, once $\bw|\by$ is approximately normal, then estimation of the parameters $\theta$ using a direct optimisation approach or an EM-type approach is straightforward.
On the other hand, MCMC approximates the density $p(\bw|\by)$ using samples generated via Gibbs sampling or HMC, and these samples would asymptotically be representative of draws from the true posterior.

Consider the data set...
Plot the data. Explain priors for HMC and variational. Compare.





\section{The variational EM algorithm for I-probit models}\label{sec:iprobitvar}
We present a variational inference algorithm to estimate the parameters of interest and values for the latent variables.
Begin by assuming some prior distribution on the parameters $p(\theta) = p(\balpha)p(\eta)p(\bSigma)$. 
Although one may devote more attention to the prior specification of these parameters, for our purposes it suffices that they are independent component-wise, and the PDFs belong to the exponential family of distributions with known hyperparameters.
The exponential family requirement greatly eases the complexity of deriving the variational algorithm later on\footnote{
Of interest, one may even opt to assign improper priors on $\theta$ and the algorithm would still work.
This is akin to obtaining empirical Bayes estimate of the $\theta$ if seen from an EM algorithm standpoint.
}.

Recall the fact that $\bff = \bG\bw$ from \eqref{eq:ipriorw}, where $\bw \sim \N_{nm}(\bzero, \bSigma^{-1} \otimes \bI_n)$.
The required posterior distribution is then $
  p(\by^*,\bw,\theta|\by) \propto p(\by|\by^*)p(\by^*|\bw,\theta)p(\bw|\theta)p(\theta)
$.
This is approximated by a mean-field distribution of the form $q(\by^*,\bw,\theta) \equiv q(\by^*)q(\bw)q(\theta)$, and also $q(\theta) = q(\balpha)q(\eta)q(\bSigma)$.
Denote by $\tilde q$ the distributions which minimise the Kullbeck-Leibler divergence (maximise the variational lower bound).
By appealing to \cite[equation 10.9, p.466]{bishop2006pattern}, we find that for each $\xi \in \{ \by^*,\bw,\theta \}$, $\tilde q$ satisfies
\begin{align}\label{eq:qtilde}
  \log \tilde q(\xi) = \E_{-\xi} [\log p(\by, \by^*, \bw, \theta)] + \const
\end{align}
where expectation of the log joint density of $(\by, \by^*, \bw, \theta)$ is taken with respect to all of the parameters except the one currently in consideration.
Estimates of the parameters are then obtained by taking the mean of this approximate posterior distribution.
In practice, rather than an explicit calculation of the normalising constant, one simply needs to inspect the equation \eqref{eq:qtilde} to recognise it as a known log-density function, which is the case when exponential family distributions are considered.
In other situations, it is possibly to perform some form of sampling method (such as a Metropolis random walk) to obtain quantities of interest, for example $\E[\xi]$.

\begin{figure}[t]
  \centering
  \begin{tikzpicture}[scale=1.1, transform shape]
    \tikzstyle{main}=[circle, minimum size=10mm, thick, draw=black!80, node distance=16mm]
    \tikzstyle{connect}=[-latex, thick]
    \tikzstyle{box}=[rectangle, draw=black!100]
      \node[main, draw=black!0] (blank) [xshift=-0.55cm] {};  % pushes image to right slightly
      \node[main, fill=black!10] (H) [] {$x_i$};
      \node[main] (Sigma) [below=of H, yshift=-1.2cm, xshift=0.6cm] {$\bSigma$};
      \node[main, double, double distance=0.6mm] (f) [right=of H, xshift=0.5cm] {$f_{ij}$};
      \node[main, double, double distance=0.6mm] (ystar) [right=of f, xshift=0cm] {$y_{ij}^*$};
      \node[main, double, double distance=0.6mm] (pij) [right=of ystar, xshift=0cm] {$p_{ij}$};
      \node[main] (lambda) [above=of f, xshift=0cm, yshift=-0.3cm] {$\eta_j$};        
      \node[main] (alpha) [above=of ystar, xshift=0cm, yshift=-0.3cm] {$\alpha_j$};  
      \node[main, fill = black!10] (y) [right=of pij, xshift=0.2cm] {$y_{i}$};
      \node[main] (w) [below=of f, yshift=0.3cm] {$w_{ij}$};  
      \path (alpha) edge [connect] (ystar)
            (lambda) edge [connect] (f)
            (H) edge [connect] node [above] {$h \ \ $} (f)
    		(f) edge [connect] (ystar)
    		(ystar) edge [connect] node [above] {$g^{-1}$}  (pij)
            (pij) edge [connect] (y)
            (Sigma) edge [connect] (w)
    		(w) edge [connect] (f);
      \node[rectangle, draw=black!100, fit={($(H.north west) + (0.2,0cm)$) ($(y.north east) + (-0.2,0.4cm)$) (w)}] {}; 
      \node[rectangle, fit= (w) (y), label=below right:{$i=1,\dots,n$}, xshift=0.95cm, yshift=0.5cm] {};  % the label
      \node[rectangle, draw=black!100, fit={(lambda) ($(pij.north east) + (0.5cm,0.7cm)$) ($(w.south west) + (-0.5,-0.7cm)$)}] {}; 
      \node[rectangle, fit={(f) ($(ystar.north east) + (0.5cm,0.7cm)$) ($(w.south west) + (-0.5,-0.7cm)$)}, label=below right:{$j=1,\dots,m$}, xshift=0.63cm, yshift=0.37cm] {}; 
    \end{tikzpicture}
    \caption{A DAG of the probit I-prior  model. Observed nodes are shaded, while double-lined nodes represented known or calculable quantities. There are at most $m-1$ sets of intercept ($\alpha_j$) and RKHS parameters ($\eta_j$)  to estimate due to identifiability. Depending on the specification of $\bSigma$, this may need to be estimated too.}
\end{figure}

We now present the $\tilde q$ distributions, which we call the posteriors, instead of the mean-field variational densities, when it is unambiguous to do so.
A note on notation: We will typically refer to posterior means of the parameters $\by^*$, $\bw$, $\theta$ and so on by use of a tilde.
For instance, we write $\tilde\bw$ to mean $\E_{\tilde q}[\bw]$, the expected value of $\bw$ under the pdf $\tilde q(\bw)$.

Write $\by_i^* = (y_{i1}^*,\dots,y_{im}^*)^\top$.
Due to the independence of each $\by_{i}^*|\theta,x_i \sim \N_m(\balpha + \bff(x_i),\bSigma)$, we have an induced factorisation of the posterior $q(\by^*) = \prod_{i=1}^n q(\by_i^*)$.
Each $q(\by_i^*)$ follows a \emph{conically truncated multivariate normal disribution}, i.e., for $i=1,\dots,n$, $\by_i^*$ is distributed according to
\begin{align}\label{eq:ystardist}
  \by_i^* \iid
  \begin{cases}
    \N_m(\tilde \alpha + \tilde \bff(x_i), \tilde\bSigma) & \text{ if } y_{ij}^* > y_{ik}^*, \forall k \neq j \\
    0 & \text{ otherwise}. \\
  \end{cases}
\end{align}
The posterior mean for the latent variables, $\tilde\by_i^* = (\tilde y_{i1}^*,\dots,\tilde y_{im}^*)$ depends on the value observed for $y_i \in \{1,\dots,m\}$. 
The expected value $\tilde\by_i^*$ for this truncated multinormal variable is tricky to compute.
One strategy might be Monte Carlo integration---using samples from $\N_m(\tilde \alpha + \tilde \bff(x_i), \tilde\bSigma)$, zero out those that do not satisfy the condition $y_{ij}^* > y_{ik}^*, \forall k \neq j$, then take the sample average.
If the independent I-probit is considered, the expected value can be considered component-wise, where each component of this expectation is given by
\begin{align}\label{eq:ystarupdate}
  \tilde y_{ik}^* =
  \begin{cases}
    \tilde\alpha_k + \tilde f_{ik} - \sigma_k C_i^{-1} \displaystyle{  \int \phi_{ik}(z) \prod_{l \neq k,j} \Phi_{il}(z) \phi(z) \d z }
    &\text{ if } k \neq y_i \\[1.5em]
    \tilde\alpha_{y_i} + \tilde f_{iy_i} - \sigma_{y_i} \sum_{k \neq y_i} \big(\tilde y_{ik}^* - \tilde f_{ik} \big) 
    &\text{ if } k = y_i \\
  \end{cases}
\end{align}
with 
\begin{align*}
  \phi_{ik}(Z) &= \phi \left(\frac{\sigma_{y_i}}{\sigma_k} Z + \frac{\tilde\alpha_{y_i} + \tilde f_{iy_i} - \tilde\alpha_k - \tilde f_{ik}}{\sigma_k} \right) \\
  \Phi_{ik}(Z) &= \Phi \left(\frac{\sigma_{y_i}}{\sigma_k} Z + \frac{\tilde\alpha_{y_i} + \tilde f_{iy_i} - \tilde\alpha_k - \tilde f_{ik}}{\sigma_k} \right) \\
  C_i &= \int \prod_{l \neq j} \Phi_{il}(z) \phi(z) \d z
\end{align*}
and $Z \sim \N(0,1)$ with PDF and CDF $\phi(\cdot)$ and $\Phi(\cdot)$ respectively. 
The integrals that appear above are functions of a unidimensional Gaussian pdf, and these can be computed rather efficiently using quadrature methods.

This time, collect all $n$ latent observations for each class and write ${\by_j^*}' = (y_{1j}^*,\dots,y_{nj}^*)^\top$ and $\bw_j = (w_{1j},\dots,w_{nj})$, $j=1,\dots,m$.
Denote also $\by^*$ and $\bw$ as the vectors concatenating all $j$ vector components which results in a vector of length $nm$ for each.
By doing so, we have that $\bff = \bG\bw$ from \eqref{eq:ipriorw}, where $\bw \sim \N_{nm}(\bzero, \bSigma^{-1} \otimes \bI_n)$ and thus $\by^* \sim \N_{nm}(\balpha + \bG\bw, \bSigma \otimes \bI_n)$.
With these two Gaussian densities, we find that the posterior for $\bw$ is also Gaussian with $\tilde q(\bw) \equiv \N_{nm}(\tilde \bw, \tilde\bV_w)$, where
\[
  \tilde \bw = \tilde \bV_w\tilde\bG (\tilde\bSigma^{-1} \otimes \bI_n) (\tilde\by^* - \tilde\balpha) \ \text{ and } \ \tilde \bV_w = \big(\tilde\bG(\tilde\bSigma^{-1} \otimes \bI_n)\tilde\bG + (\tilde\bSigma \otimes \bI_n) \big)^{-1}
\]
This multivariate normal of dimension $nm$ could present a challenge to work with, in particular the matrix inverse process required in calculating the posterior covariance matrix $\bV_w$.
Note that $\bG(\bSigma^{-1} \otimes \bI_n)\bG$ is in fact the covariance matrix for the I-prior, and has the structure of $\bV_f$ in equation \eqref{eq:vf}.
If the independent I-probit model is assumed, then the posterior covariance matrix $\bV_w$ has a simpler structure as the covariance terms disappear, which implies that the components $\bw_j$ would be independently distributed. 
This results in $m$ of these $n$-variate Gaussian distributions with mean and covariance matrix given by
\[
  \tilde \bw_j = \sigma_j^{-2}\tilde \bV_{w_j}\tilde\bH_{\eta_j} (\tilde\by^*_j - \tilde\alpha_j\bone_n) \ \text{ and } \ \tilde \bV_w = \big(\sigma_j^{-2}\tilde\bH_{\eta_j}^2 + \sigma_j^{2}\bI_n \big)^{-1}.
\]
Each of these covariance matrices require $O(n^3)$ to compute and there are $m$ of them, so in total $O(mn^3)$ computational time is required.
This is much less time than the $O(m^3n^3)$ required than the full I-probit model.\hltodo[I think this can be improved by exploiting matrix normal distributions and Kronecker products, but don't know how yet.]{}

%This time, collect all values of $w_{ij}$ into a matrix $\bw$ of dimensions $n \times m$.
%The posterior for $\bw$ is said to follow a \emph{matrix normal} distribution with mean $\tilde \bw$ and scale matrices $\bS_1$ and $\bS_2$, with dimensions $n \times n$ and $m \times m$ respectively.
%This is written as $\bw \sim \MN_{n,m}(\tilde \bw, \bS_1, \bS_2)$.
%The values of this posterior distribution are found to be
%\begin{align*}
%  \begin{gathered}
%    \tilde \bw = \bS_2^{-1} 
%    \tilde\bSigma^{-1} (\tilde\by^* - \tilde\balpha) \bH_{\eta_j}
%%    \begin{pmatrix}
%%      \bH_{\eta_1} \cdots \bH_{\eta_m}
%%    \end{pmatrix} 
%     \\
%    \bS_1 = \bI_n \\
%    \bS_2 =
%  \end{gathered}
%\end{align*}
%where $\bG = 
%\begin{pmatrix}
%  \bH_{\eta_1} \cdots \bH_{\eta_m}
%\end{pmatrix}_{n \times nm} 
%\times 
%\begin{pmatrix}
%  \bI_n \cdots \bI_n
%\end{pmatrix}_{nm \times n}^\top = \sum_{k=1}^m \bH_{\eta_k}$ is a $n \times n$ matrix.

The posterior density $\tilde q$ involving the RKHS parameters is of the form
\[
  \log\tilde q(\eta) =  -\half\E_{-\eta} \Big[ (\by^* - \balpha - \bG_\eta\bw)^\top (\bSigma^{-1} \otimes \bI_n) (\by^* - \balpha - \bG_\eta\bw) \Big] + \log p(\eta) + \const,
\]
where $p(\eta)$ is a prior distribution for $\eta$.
The RKHS parameters are contained in the kernel matrices within the $\bG_\eta$ matrix, and the subscript $\eta$ emphasises this fact.
The relevance of exponential family distributions for the priors are seen here---if the prior is normal, for example, then the PDF $\tilde q(\eta)$ is also normal.
Alternatively, samples $\eta^{(1)},\dots,\eta^{(T)}$ from $\tilde q(\eta)$ may be obtained using a Metropolis algorithm, and quantities such as $\tilde\bH_{\eta_j} = \E_q[\bH_{\eta_j}]$ may be approximated using $\frac{1}{T}\sum_{t=1}^T \tilde\bH_{\eta_j^{(t)}}$.
We note also that the RKHS parameters may be considered by class if the independent I-probit model is considered.

Moving on to the estimation of the covariance matrix $\bSigma$.
For this part, we revert back to considering the IID observations $\by_i^* = (y_{i1}^*,\dots,y_{im}^*)^\top \sim \N_m(\balpha - \bff(x_i), \bSigma)$. 
Create new random variables $\bu_1,\dots,\bu_n$ defined by $\bu_i = \bSigma\bP\bw$, where $\bw^\top = (\bw_1^\top,\dots,\bw_m^\top)$ is as before.
Since the vector $\bw$ is sorted according to class instead of observations, an appropriate permutation matrix $\bP$ is required to match this with the $\bu_i$'s.
Specifically, the permutation matrix is such that if $\bu \sim \N_{nm}(\bzero, \bSigma^{-1} \otimes \bI_n)$, then $\bP\bu \sim \N_{nm}(\bzero, \bI_n \otimes\bSigma^{-1})$.
Therefore, $\bu_i \iid \N_m(\bzero, \bSigma)$, and the posterior mean and variance for $\bu_i$ may be given as $\tilde \bu_i = \tilde\bSigma\bP_i\tilde\bw$ and $\tilde\bV_{u_i} = \tilde\bSigma\bP_i\tilde\bV_w\bP_i^\top\tilde\bSigma \vphantom{\iid}$.
This step is required so that the resulting posterior for $\bSigma$ is conjugate with an inverse Wishart prior on $\bSigma \sim \invWis(\bA, a)$.
We obtain an inverse Wishart distribution for $\bSigma$, i.e. $\tilde q(\bSigma) \equiv \invWis(\bA_1 + \bA_2 + \bA, 2nm + a)$, where
\begin{align*}
  \begin{gathered}
  \bA_1 = \E_{\by^*,\bw,\balpha} \left[ \sum_{i=1}^n (\by^*_i - \balpha - \bff(x_i))(\tilde\by^*_i - \balpha - \bff(x_i))^\top \right]   \\
  \bA_2 = \sum_{i=1}^n \left[\bu_i\bu_i^\top + \tilde\bV_{u_i} \right].
  \end{gathered}
\end{align*}
The challenge here is that it involves the second posterior moment of the conically truncated multivariate normal distribution for $\by^*$, which may be obtained by sampling or numerical integration as described earlier.

As a remark on identifiability, we might require that one of the components of $\bSigma$ be fixed, e.g. $\bSigma_{11} = 1$.
\cite{mcculloch2000bayesian} gives a Gibbs sampling algorithm which we find useful for our variational algorithm as well.
Partition $\bSigma$ as follows:
\[
  \bSigma = \begin{pmatrix}
    1       &\bzeta^\top \\
    \bzeta  &\bZ + \bzeta\bzeta^\top
  \end{pmatrix}.
\]
We can choose the priors $\bzeta \sim \N_{m-1}(\bzero,\bB)$ and $\bZ \sim \invWis(\bA,a)$, independent of each other with appropriately chosen hyperparameters.
Define $\tilde\bepsilon_i = \by^*_i - \tilde\balpha - \tilde\bff(x_i) = (\tilde\epsilon_{i1},\dots,\tilde\epsilon_{im})^\top$, and let $\upsilon_i = \surd 2 \tilde\epsilon_{i1}$ and $\bnu_i = \surd 2(\tilde\epsilon_{i2},\dots,\tilde\epsilon_{im})^\top$ such that $\surd 2\tilde\epsilon_i^\top = (\upsilon_i, \bnu_i^\top)^\top$.
The posterior distribution $\tilde q$ for $\bZ$ is inverse Wishart with scale equal to $\bA_3 + \bA$, where\hltodo[These equations can be simplified further.]{}
\begin{align*}
  \begin{gathered}
    \bA_3 = \E_{\bnu,\upsilon_i,\zeta} \left[ \sum_{i=1}^n (\bnu_i - \upsilon_i \bzeta)(\bnu_i - \upsilon_i \bzeta)^\top \right]   \\
    \end{gathered}
\end{align*}
and $2n(m-1)+a$ degrees of freedom.
The posterior distribution $\tilde q$ for $\bzeta$ is normal with mean and variance $\tilde\bzeta = \frac{1}{n}\sum_{i=1}^n \tilde\upsilon_i \tilde\bV_\zeta \tilde\bZ\tilde\bnu_i$ and $\tilde\bV_\zeta = (\tilde\bupsilon\tilde\bupsilon^\top\tilde\bZ^{-1} +\bB)^{-1}$.

If the independent I-probit model is considered, then $\bSigma = \diag(\sigma_1^2,\dots,\sigma_m^2)$, class independence holds so we can use independent inverse gamma distributions as a prior for $\bSigma$, i.e. $ p(\bSigma) = \prod_{j=1}^m  p(\sigma_j^2)$, where each $p(\sigma_j) \equiv \Gamma^{-1}(r,s)$.
The posterior for $\bSigma$ will also be of a similar factorised form , namely $\tilde q(\bSigma) = \prod_{j=1}^m \tilde q(\sigma_j^2)$, where $\tilde q(\sigma_j^2)$ is the PDF of an inverse gamma distribution with shape and scale parameters $\tilde r = 2n+r-1$ and $\tilde s = \half\Vert \tilde\by^*_j - \tilde\alpha_j - \tilde\bff_j \Vert^2 + \half \Vert \tilde\bu_j \Vert^2 + s$ respectively.

Finally, the posterior distribution for the intercepts follow a normal distribution should the prior specified on the intercepts also be a normal distribution, e.g. $\balpha \sim \N_m(\bzero,\bA)$.
The posterior mean and variance for the intercepts are given by
\[
  \tilde\balpha = \tilde\bV_\alpha \tilde\bSigma^{-1}\big(\tilde\by_i^* - \tilde\bff(x_i)\big) \ \text{ and } \ \tilde\bV_\alpha = \big(n\tilde\bSigma^{-1} + \bA^{-1}\big)^{-1}.
\]

Note that the evaluation of each of the component of the posterior depends on some of the components itself, and so this circular dependence is dealt with by using some arbitrary starting values and after which an iterative updating scheme of the components ensues.
The updating scheme is performed until a maximum number of iterations is reached, or ideally until some of convergence criterion is met.
In variational inference, the \emph{variational lower bound} is typically used to asses convergence.
The lower bound is given by
\begin{align*}
  \cL 
  &= \int q(\by^*,\bw,\theta) \log \left[ \frac{p(\by,\by^*,\bw,\theta)}{q(\by^*,\bw,\theta)} \right] \d\by^* \d\bw \d\theta \\
  &= \E[\log p(\by,\by^*,\bw,\theta)] - \E[\log q(\by^*,\bw,\theta)].
\end{align*}
These are calculable once the posterior distributions $\tilde q$ are known---the first term is the expectation of the logarithm of the joint density, whereas the second term factorises into the entropy of its individual components.
Similar to the EM algorithm, this quantity is\hltodo[Proof?]{expected to increase with every iteration.}


The following pseudocode summarises the variational algorithm for I-probit models.

\algrenewcommand{\algorithmiccomment}[1]{{\color{gray}\hskip2em$\triangleright$ #1}}
\begin{algorithm}[H]
\caption{VB-EM algorithm for the probit I-prior model}\label{alg:VBEM}
\begin{algorithmic}[1]
\Procedure{Initialise}{}
  \State $\bSigma^{(0)} \gets \bI_m$
  \For{$j=1,\dots,m$}
    \State Randomise $\alpha_j^{(0)}$, $\eta_j^{(0)}$, $\bw_j^{(0)}$
%    \State $\bw_j^{(0)} \gets \bzero_{n}$ %\Comment{or draw $w_i^{(0)} \ \sim \N(0,1)$ for $i=1,\dots,n$.}
    \State Calculate $\bH_{\eta_j}$ as per kernels chosen
  \EndFor
%  \State $\bW^{(t)} \gets \big(\bw_1^{(t+1)} \cdots  \bw_m^{(t+1)} \big)$
  \State $\bG^{(t)} \gets \diag(\bH_{\eta_1}, \dots, \bH_{\eta_m})$
\EndProcedure
\Statex
%\Procedure{Update for $\bff$ } {time $t$}
%  \For{$j=1,\dots,m$}
%    \State $\bff_j^{(t+1)} \gets \alpha_j^{(t)}\bone_{n} + \bH_{\eta_j}\bw_j^{(t)}$
%  \EndFor
%  \State $\bF^{(t+1)} \gets \big(\bff_1^{(t+1)} \cdots  \bff_m^{(t+1)} \big)$
%\EndProcedure
%\Statex
\Procedure{Update for $\by^*$ }{time $t$}
  \For{$i=1,\dots,n$}
    \State $j \gets y_i$
    \If{Independent I-probit model}
      \State $(y_{i1}^{*(t+1)},\dots,y_{im}^{*(t+1)}) \gets \E[\by_i^*]$ as per \eqref{eq:ystarupdate}
    \Else
      \State Sample from truncated normal as per \eqref{eq:ystardist}
      \State $(y_{i1}^{*(t+1)},\dots,y_{im}^{*(t+1)}) \gets$ sample mean
    \EndIf  
  \EndFor
\EndProcedure	
\Statex
\Procedure{Update for $\bw$ }{time $t$}
  \State $\bV_w^{(t+1)} \gets \big(\bG(\bSigma^{-1} \otimes \bI_n)\bG + (\bSigma \otimes \bI_n) \big)^{-1}$ 
  \Comment{Simpler if independent I-probit}
  \State $\bw^{(t+1)} \gets \bV_w\bG^{(t+1)} (\bSigma^{-1} \otimes \bI_n) (\by^* - \balpha)$ 
\EndProcedure	
\Statex
\Procedure{Update for $\eta$ }{time $t$}
  \State Metropolis sampling from density\vspace{-0.5em} \Comment{Simpler calculations if only RKHS scales}
  \[
    \tilde q(\eta) \propto \exp\left[ (\by^{*(t)} - \balpha^{(t)} - \bG^{(t)}\bw^{(t)})^\top (\bSigma^{-1} \otimes \bI_n) (\by^{*(t)} - \balpha^{(t)} - \bG^{(t)}\bw^{(t)}) \right] \vspace{-0.7em}
  \]
  \State $\eta^{(t+1)} \text{ and } \bG^{(t+1)} \gets$ sample mean
\EndProcedure	
\algstore{VBEMbreak1}	
\end{algorithmic}
\end{algorithm}


\begin{algorithm}[H]
\begin{algorithmic}[1]
\algrestore{VBEMbreak1}
\Procedure{Update for $\bSigma$ }{time $t$}
  \State $\bu^{(t)} \gets \bSigma^{(t)}\bP\bw$ and $\bV_u^{(t)} \gets \bSigma^{(t)}\bP\bV_w^{(t)}\bP^\top\bSigma^{(t)}$
  \State $\bA_1 \gets \sum_{i=1}^n  \E_{\by^*} \left[ (\by^*_i - \balpha^{(t)} - \bff_i^{(t)})(\by^*_i - \balpha^{(t)} - \bff_i^{(t)})^\top \right]$  
  \State $\bA_2 \gets \sum_{i=1}^n \left(\bu_i\bu_i^\top + \bV_{u_i} \right)$
  \State $\bSigma^{(t+1)} \gets (\bA_1 + \bA_2)/(2n)$ \Comment{Simpler if independent I-probit}
\EndProcedure	
\Statex
\Procedure{Update for $\balpha$ }{time $t$}
  \State $\balpha^{(t+1)} \gets \sum_{i=1}^n\big(\by_i^* - \bff^{(t)}(x_i)\big)/n $
\EndProcedure	
\Procedure{The VB-EM algorithm}{}
  \State $t \gets 0$ and initialise $\cL^{(0)}$
  \While{$\cL^{(t+1)} - \cL^{(t)} > \delta$ \textbf{or} $t < t_{max}$}{}
    \State \textbf{call} \Call{Update for $\by^*$}{}
    \State \textbf{call} \Call{Update for $\bw$}{}
    \State \textbf{call} \Call{Update for $\eta$}{}
    \State \textbf{call} \Call{Update for $\bSigma$}{}
    \State \textbf{call} \Call{Update for $\balpha$}{}
    \State \textbf{call} Calculate variational lower bound $\cL^{(t+1)}$
    \State $t \gets t + 1$
  \EndWhile
\EndProcedure
\end{algorithmic}
\end{algorithm}

\section{Post-estimation}\label{sec:iprobitpostest}
For a new data point $x_{\text{new}}$, we calculate the predicted latent values $\tilde f_{\text{new}} = (\tilde f_{\text{new},1}, \dots, \tilde f_{\text{new},m})$ for each of the classes, using the variational estimates of the posterior means for the unknown quantities (denoted with tildes), as follows:
\[
  \tilde f_{\text{new},j} = \sum_{k=1}^n \tilde h_{\eta_j}(x_{\text{new}},x_k) \tilde w_{kj}, \hspace{0.5cm} j = 1,\dots,m.
\]
This is in fact the posterior mean for the regression functions evaluated at the new point $x_{\text{new}}$, which stems from a posterior normal distribution.
Denote $\bg(x_{\text{new}}) = \diag\big(\bh_{\eta_j}^\top(x_{\text{new}})\big)_{j=1}^m$, an $m \times nm$ matrix containing the kernel entries relating to this new data point.
From Gaussian process regression theory, we know that the prediction at $x_{\text{new}}$ for the latent variable $\by^*(x_{\text{new}})$ has $m$ components equal to $\tilde y_{\text{new},j}^* = \tilde\alpha_j + \tilde f_{\text{new},j}$ for $j=1,\dots,m$, and variance 
\[
  \bV_y(x_{\text{new}}) = \bg^\top(x_{\text{new}}) \tilde\bV_w \bg(x_{\text{new}}) + \tilde\bSigma.
\]
Thus, information relating to the class and its corresponding probabilities are contained within the normal distribution $\N_{m}\big(\by^*(x_{\text{new}}), \bV_y(x_{\text{new}}) \big)$, subject to the truncation $\cA_j := \{ y_{\text{new},j}^* > y_{\text{new},k}^* | \forall k \neq j\}$ if the new observation belongs to class $j \in \{1,\dots, m\}$.
The predicted class is inferred from the latent variables via
\[
  \hat y_{\text{new}} = \argmax_j \tilde y_{\text{new},j}^*, 
\]
while the probabilities for each class are once again obtained using the integral stated in \eqref{eq:pij}, stated here for convenience:
\begin{align*}
  \tilde p_{\text{new},j} 
  &= \int_{\cA_j} \N(\by^*(x_{\text{new}}), \bV_y(x_{\text{new}})) \d\by^*_{\text{new}}.
\end{align*}

For the independent I-probit model, each component of the latent variables $\by^*(x_{\text{new}})$ are calculated in a similar manner, with the difference being that the components would be independent of each other.
This is expressed in the form for the predictive covariance matrix $\bV_y(x_{\text{new}}) = \diag \big( v_1^2(x_{\text{new}}),\dots, v_m^2(x_{\text{new}})\big)$, where each variance component is given by
\[
  v_j^2(x_{\text{new}}) = \bh_{\eta_j}^\top(x_{\text{new}}) \tilde\bV_w \bh_{\eta_j}(x_{\text{new}}) + \tilde\sigma_{j}^2.
\]
Class prediction is the same as before, but class probabilities are obtained in a more compact manner via
\[
  \tilde p_{\text{new},j} 
  = \E_Z \Bigg[ \mathop{\prod_{k=1}^m}_{k\neq j} 
  \Phi \left(\frac{v_j}{v_k} Z + \frac{\tilde y_{\text{new},j}^* - \tilde y_{\text{new},k}^*}{v_k} \right) \Bigg],
\]
where $Z\sim\N(0,1)$ and $\Phi(\cdot)$ its CDF.

Working in a Bayesian framework means that we are able to perform inferences on any quantity of interest through posterior sampling.
In the I-probit model, we use the approximate mean-field densities in lieu of the true posterior densities, and these are, for the most part, easy to sample from.
Take for example the class probabilities. 
We can obtain posterior samples for the $p_{ij}$'s by firstly sampling from the underlying normal latent variables, and then passing it through the probit link function.
Taking the empirical lower 2.5th and upper 97.5th percentile of this sample would give the upper and lower values for a 95\% credibility interval.

\section{Computational considerations}
Computational challenges for the I-probit model stems from two sources: 1) calculation of the class probabilities \cref{eq:pij}; and 2) storage and time requirements for the variational EM algorithm.
Ways in which to overcome these challenges are discussed.
In addition, we also discuss considerations to take into account if estimation of the error precision $\bPsi$ is desired, and thus pave the way for future work.

\subsection{Efficient computation of class probabilities}
\label{sec:mnint}

The issue at hand here is that for $m>4$, the evaluation of the class probabilities in \cref{eq:pij} is computationally burdensome using classical methods such as quadrature methods \citet{geweke1994alternative}.
As such, simulation techniques (Monte Carlo integration) are employed instead.
The simplest strategy to overcome this is a frequency simulator (otherwise known as Monte Carlo integration): obtain random samples from $\N_{m}\big(\bmu(x_i), \bPsi^{-1}\big)$, and calculate how many of these samples fall within the required  region.
This method is fast and yields unbiased estimates of the class probabilities.
However, in an extensive comparative study of various probability simulators, \citet{hajivassiliou1996simulation} concluded that the Geweke-Hajivassiliou-Keane (GHK) probability simulator \citep{geweke1989bayesian,hajivassiliou1998method,keane1994solution} is the most reliable under a multitude of scenarios.
This is now described, and for clarity, we drop the subscript $i$ denoting individuals. 

Suppose that an observation $y=j$ has been made.
Reformulate $\by^*$ in \cref{eq:latentmodel} by anchoring on the $j$'th latent variable $y_j^*$ to obtain
\[
  \bz := (
  \greyoverbrace{y_1^* - y_j^*}{ z_1},
  \dots,
  \greyoverbrace{y_{j-1}^* - y_j^*}{ z_{j-1}},
  \greyoverbrace{y_{j+1}^* - y_j^*}{ z_{j}},
  \dots, 
  \greyoverbrace{y_m^* - y_j^*}{ z_{m-1}},
  )^\top \in \bbR^{m-1}.
\]
Note that we have indexed the vector $\bz$ using $j' = k$ if $k < j$, and $j' = k -1$ if $k > j$ for $k=1,\dots,m$, so that the index $j'$ runs from $1$ to $m-1$.
Let $\bQ \in \bbR^{(m-1)\times m}$ be a matrix formed by inserting a column of minus ones at the $j$'th position in an $(m-1)$ identity matrix.
We can then write $\bz = \bQ\by^*$, and thus we have that $\bz\sim\N_{m-1}(\bnu_{(j)}, \bOmega_{(j)})$, where $\bnu_{(j)} = \bQ\bmu(x_i)$ and $\bOmega_{(j)} = \bQ\bPsi^{-1}\bQ^\top$.
These are indexed by `$(j)$' because the transformation is dependent on which latent variable the $\bz$'s are anchored on.

\begin{remark}
  Incidentally, the probit model in \cref{eq:latentmodel} is equivalently represented by 
  \begin{equation}\label{eq:latentmodel2}
    y_i = 
    \begin{cases}
      1 & \text{if } \max (y_{i2}^* - y_{i1}^*,\dots,y_{im}^* - y_{i1}^*) < 0 \\
      j & \text{if } \max (y_{i2}^* - y_{i1}^*,\dots,y_{im}^* - y_{i1}^*) = y_{ij}^* - y_{i1}^* \geq 0,
    \end{cases}
  \end{equation}   
  which is obtained by anchoring on the first latent variable (referred to as the reference category), although the choice of reference category is arbitrary.
  This is similar to fixing the latent variables of the reference category to zero, and thus, as discussed previously in \cref{sec:iia}, full identification is achieved by fixing one more element of the covariance matrix.
\end{remark}

For the symmetric and positive definite covariance matrix $\bOmega_{(j)}$, obtain its Cholesky decomposition as $\bOmega_{(j)} = \bL\bL^\top$, where $\bL$ is a lower triangular matrix.
Then, $\bz = \bnu_{(j)} + \bL\bzeta$, where $\bzeta \sim\N_{m-1}(\bzero,\bI_{m-1})$.
That is,
\begin{align*}
  \begin{pmatrix}
    z_1 \\
    z_2 \\
    \vdots \\
    z_{m-1}
  \end{pmatrix}  
  &=
  \begin{pmatrix}
    \nu_{(j)1} \\
    \nu_{(j)2} \\    
    \vdots \\
    \nu_{(j)m-1}
  \end{pmatrix}  
  + 
  \begin{pmatrix}
    L_{11} &       &       & \\
    L_{21} &L_{22} &       & \\
    \vdots &\vdots &\ddots & \\
    L_{m-1,1} &L_{m-1,2} &\cdots &L_{m-1,m-1} \\
  \end{pmatrix} 
  \begin{pmatrix}
    \zeta_1 \\
    \zeta_2 \\    
    \vdots \\
    \zeta_{m-1}
  \end{pmatrix} \\ %\displaybreak
  &=
  \begin{pmatrix}
    \nu_{(j)1} + L_{11}\zeta_1 \\
    \nu_{(j)2} + \sum_{k=1}^2 L_{k2} \zeta_k \\    
    \vdots \\
    \nu_{(j)m-1} + \sum_{k=1}^{m-1} L_{k,m-1} \zeta_k
  \end{pmatrix}.  
\end{align*}

With this setup, the probability $p_{j}$ of an observation belonging to class $j$, which is equivalent to the probability that each $ z_{j'} < 0$, $j'=1,\dots,m-1$, can be expressed as
\begin{align*}
  p_j 
  ={}& \Prob( z_1 < 0,\dots, z_{m-1} < 0) \\
  ={}& 
  \Prob(\zeta_1 < u_1,\dots,\zeta_{m-1}<u_{m-1}) \\
  ={}& 
  \Prob(\zeta_1 < u_1)
  \Prob(\zeta_2 < u_2|\zeta_1 < u_1)
  \Prob(\zeta_3 < u_3|\zeta_1 < u_1,\zeta_2 < u_2)
  \cdots \\
  &\cdots
  \Prob(\zeta_{m-1}<u_{m-1}|\zeta_1 < u_1,\dots,\zeta_{m-2}<u_{m-2}),
\end{align*}
where 
\begin{equation*}
  u_{j'} = 
  u_{j'}(\zeta_1,\dots,\zeta_{j'-1}) =
  \begin{cases}
    - \nu_{(j)1} / L_{11} &\text{for } j' = 1 \\
    - \big(\nu_{(j)j'} + \sum_{k=1}^{j'-1} L_{kj'}\zeta_k \big) / L_{j'j'} &\text{for } j' = 2,\dots,m-1
  \end{cases}
\end{equation*}
The GHK algorithm entails making draws from one-sided right truncated standard normal distributions (for instance, using an inverse transform method detailed in \cref{apx:truncuninorm},  \mypageref{apx:truncuninorm}):
\begin{itemize}
  \item Draw $\tilde\zeta_1 \sim \tN(0,1,-\infty,u_1)$.
  \item Draw $\tilde\zeta_2 \sim \tN(0,1,-\infty,\tilde u_2)$, where $\tilde u_2 = u_2(\tilde\zeta_1)$.
  \item Draw $\tilde\zeta_3 \sim \tN(0,1,-\infty,\tilde u_3)$, where $\tilde u_3 = u_3(\tilde\zeta_1,\tilde\zeta_2)$. 
  \item $\cdots$
  \item Draw $\tilde\zeta_{m-1} \sim \tN(0,1,-\infty,\tilde u_{m-2})$, where $\tilde u_{m-1} = u_m(\tilde\zeta_1,\dots,\tilde\zeta_{m-2})$.
\end{itemize}
These values are then used in the following manner:
\begin{itemize}
  \item Use $\tilde\zeta_1$ to obtain a ``draw'' of $\Prob(\zeta_2 < u_2|\zeta_1 < \zeta_1)$,
  \begin{align*}
    \widetilde\Prob(\zeta_2 < u_2|\zeta_1 < \zeta_1)
    &=\Prob(\zeta_2 < u_2|\zeta_1 = \tilde\zeta_1) \\
    &= \Phi \Big( - \big(\nu_{(j)2} +  L_{12}\tilde\zeta_1 \big) / L_{22} \Big)
  \end{align*}
  \item Use $\tilde\zeta_1$ and $\tilde\zeta_2$ to obtain a ``draw'' of $\Prob(\zeta_3 < u_3|\zeta_1 < u_1,\zeta_2 < u_2)$,
  \begin{align*}
    \widetilde\Prob(\zeta_3 < u_3|\zeta_1 < u_1,\zeta_2 < u_2)
    &=\Prob(\zeta_3 < u_3|\zeta_1 = \tilde\zeta_1,\zeta_2 = \tilde\zeta_2)\\
    &= \Phi \Big( - \big(\nu_{(j)3} + L_{13}\tilde\zeta_1 + L_{23}\tilde\zeta_2 \big) / L_{33} \Big)
  \end{align*} 
  \item And so on. 
\end{itemize}
Therefore, a simulated probability for $p_j$ (denoted with a tilde) is obtained as
\begin{equation}\label{eq:tildepj}
  \tilde p_j = \Phi\left( -\nu_{(j)1}/ L_{11} \right) 
  \prod_{j'=2}^{m-1} \Phi \left( 
  - \big(\nu_{(j)j'} + \textstyle\sum_{k=1}^{j'-1} L_{kj'}\tilde\zeta_k \big) / L_{j'j'} 
  \right).
\end{equation}
By performing the above scheme $T$ number of times to obtain $T$ such simulated probabilities $\{p_j^{(1)},\dots,p_j^{(T)} \}$, the actual probability of interest $p_j$ is then approximated by the sample mean of the draws,
\[
  \hat p_j = \frac{1}{T} \sum_{t=1}^T p_j^{(t)}.
\]

If it so happens that one of the standard normal cdfs in \cref{eq:tildepj} is extremely small, this can cause loss of significance due to floating-point errors (catastrophic cancellation).
It is better to work on a log-probability scale, so the products in \cref{eq:tildepj} turn into sums, and revert back by exponentiating.

\begin{remark}
  The GHK algorithm provides reasonably fast and accurate calculations of class probabilities when $\bPsi$ is dense.
  As we alluded to earlier in the chapter, the class probabilities condense to a unidimensional integral involving products of normal cdfs (c.f. equation \ref{eq:pij2}) if $\bPsi$ is diagonal.
  Note that if $\bPsi$ is diagonal, then the transformed $\bOmega_{(j)} = \bQ\bPsi^{-1}\bQ^\top$ is certainly not: the components of $\bz$ are correlated because they are all anchored on the same random variable.
  Thus, direct evaluation of \cref{eq:pij2} using quadrature methods avoids the Cholesky step and random sampling employed by the GHK method.
\end{remark}

%\begin{remark}
%  The GHK probability simulator relates to the \emph{method of simulated likelihood}.  
%\end{remark}

\subsection{Efficient Kronecker product inverse}
\label{sec:complxiprobit}

As with the normal I-prior model, the time complexity of the variational inference algorithm for I-probit models is dominated by the step involving the posterior evaluation of the I-prior random effects $\bw$, which essentially is the inversion of an $nm \times nm$ matrix.
The matrix in question is %(from \cref{eq:varipostw})
\begin{align}
  \bV_w = \big[ (\bPsi \otimes \bH_\eta^2) + (\bPsi^{-1} \otimes \bI_n) \big]^{-1}. \tag{from \ref{eq:varipostw}}
\end{align}
We can actually exploit the Kronekcer product structure to compute the inverse efficiently.
Perform an orthogonal eigendecomposition of $\bH_\eta$ to obtain $\bH_\eta = \bV\bU\bV^\top$ and of $\bPsi$ to obtain $\bPsi = \bQ\bP\bQ^\top$.
This process takes $O(n^3 + m^3) \approx O(n^3)$ time if $m\ll n$ or if done in parallel, and needs to be performed once per CAVI iteration.
Then, manipulate $\bV_w^{-1}$ as follows:
\begin{align*}
  \bV_w^{-1} 
  &= (\bPsi \otimes \bH_\eta^2) + (\bPsi^{-1} \otimes \bI_n) \\
  &= (\bQ\bP\bQ^\top \otimes \bV\bU^2\bV^\top) + (\bQ\bP^{-1}\bQ^\top \otimes \bV\bV^\top) \\
  &= (\bQ \otimes \bV)(\bP \otimes \bU^2)(\bQ^\top \otimes \bV^\top) + 
  (\bQ \otimes \bV)(\bP^{-1} \otimes \bI_n)(\bQ^\top \otimes \bV^\top) \\
  &= (\bQ \otimes \bV)(\bP \otimes \bU^2 + \bP^{-1} \otimes \bI_n)(\bQ^\top \otimes \bV^\top) 
\end{align*}
Its inverse is 
\begin{align*}
  \bV_w 
  &=  (\bQ^\top \otimes \bV^\top)^{-1}(\bP \otimes \bU^2 + \bP^{-1} \otimes \bI_n)^{-1} (\bQ \otimes \bV)^{-1} \\
  &= (\bQ \otimes \bV)(\bP \otimes \bU^2 + \bP^{-1} \otimes \bI_n)^{-1}(\bQ^\top \otimes \bV^\top)
\end{align*}
which is easy to compute since the middle term is an inverse of diagonal matrices.
This brings time complexity of the variational EM algorithm down to a similar requirement as if $\bPsi$ were diagonal.
Unfortunately, storage requirements remain at $O(n^2m^2)$ when $\bPsi$ is dense, because the entire $nm \times nm$ matrix $\bV_w$ is needed to evaluate the posterior mean of $\vecc\bw$.

\subsection{Estimation of \texorpdfstring{$\bPsi$}{$\Psi$} in future work}
\label{sec:difficultPsi}

Suppose that $\bPsi \in \bbR^{m\times m}$ is a free parameter to be estimated, bearing in mind that only $m(m-1)/2 - 1$ variance components are identified in the I-probit model (see \cref{sec:iia}).
If so, the $Q$ function from \cref{eq:iprobitQEstep} conditional on the rest of the parameters can be written as
\vspace{-1.3em}
\begin{align*}
  Q(\bPsi|\balpha,\eta)
  &= \const 
  -\half \tr 
  \big( 
  \bPsi
  \greyoverbrace{\E\big( (\by^* - \bmu)^\top (\by^* - \bmu) \big)}{\bG_1}
  +
  \bPsi^{-1}
  \greyoverbrace{\E(\bw^\top\bw)}{\bG_2} 
  \Big)
\end{align*}
with $\bmu = \bone_n\balpha^\top + \bH_\eta\bw$.
This can be solved using numerical methods, though it must be ensured that the identifiability constraints and positive-definiteness are satisfied.
Specifically in the case where $\bPsi$ is a diagonal matrix $\diag(\psi_1,\dots,\psi_m)$, then
\begin{align*}
  \vspace{-0.5em}
  Q(\bPsi|\balpha,\eta)
  ={}& \const -\half \sum_{j=1}^m \psi_j \tr
  \E\big( (\by^*_{\bigcdot j} - \bmu_{\bigcdot j})(\by^*_{\bigcdot j} - \bmu_{\bigcdot j})^\top \big) \\
  & -\half \sum_{j=1}^m \psi_j^{-1} \tr
  \E(\bw_{\bigcdot j}\bw_{\bigcdot j}^\top)
\end{align*}
is maximised, for $j=1,\dots,m$, at
\[
  \hat\psi_j = \left( 
  \frac{\E( \bw_{\bigcdot j}^\top\bw_{\bigcdot j} ) }{\E\big( (\by^*_{\bigcdot j} - \bmu_{\bigcdot j})^\top (\by^*_{\bigcdot j} - \bmu_{\bigcdot j}) \big) }
  \right)^{\half},
\] 
independently of the rest of the other $\psi_k$'s, $k\neq j$.
As per the derivations in \cref{apx:qw} \colp{\mypageref{eq:trCEwDw}}, the numerator of this expression is equal to $\tr(\tilde\bV_{w_j} + \tilde\bw_{\bigcdot j}\tilde\bw_{\bigcdot j}^\top) = \tr (\tilde\bW_{jj})$.
The denominator on the other hand is
\[
  \E(\by_{\bigcdot j}^{*\top}\by_{\bigcdot j}^*) - 
  n\alpha_j^2 - \tr( \bH_\eta^2 \tilde\bW_{jj}) 
  - 2\by^{*\top}_{\bigcdot j}\bH_\eta\tilde\bw_{\bigcdot j}
  - 2\alpha_j \sum_{i=1}^n\sum_{i'=1}^n (y_{ij}^* - h_\eta(x_{i},x_{i'}) \tilde w_{ij}).
\]

In either the full or I-probit model, solving $\bPsi$ involves the second moments of a truncated normal distribution.
In the case where $\bPsi$ is dense, this is obtained by Monte Carlo methods, where samples from a truncated multivariate normal distribution are obtained using Gibbs sampling.
Although this strategy can be used when $\bPsi$ is diagonal, we show that the form for the second moments  involve integration of standard normal cdfs and pdfs \colp{\cref{thm:contruncn}, \mypageref{thm:contruncn}}, much like the formula for the first moments.


\section{Examples}
\label{sec:iprobiteg}
\documentclass[showframe,11pt]{report}\usepackage[]{graphicx}\usepackage[]{color}
%% maxwidth is the original width if it is less than linewidth
%% otherwise use linewidth (to make sure the graphics do not exceed the margin)
\makeatletter
\def\maxwidth{ %
  \ifdim\Gin@nat@width>\linewidth
    \linewidth
  \else
    \Gin@nat@width
  \fi
}
\makeatother

\definecolor{fgcolor}{rgb}{0.196, 0.196, 0.196}
\newcommand{\hlnum}[1]{\textcolor[rgb]{0.063,0.58,0.627}{#1}}%
\newcommand{\hlstr}[1]{\textcolor[rgb]{0.063,0.58,0.627}{#1}}%
\newcommand{\hlcom}[1]{\textcolor[rgb]{0.588,0.588,0.588}{#1}}%
\newcommand{\hlopt}[1]{\textcolor[rgb]{0.196,0.196,0.196}{#1}}%
\newcommand{\hlstd}[1]{\textcolor[rgb]{0.196,0.196,0.196}{#1}}%
\newcommand{\hlkwa}[1]{\textcolor[rgb]{0.231,0.416,0.784}{#1}}%
\newcommand{\hlkwb}[1]{\textcolor[rgb]{0.627,0,0.314}{#1}}%
\newcommand{\hlkwc}[1]{\textcolor[rgb]{0,0.631,0.314}{#1}}%
\newcommand{\hlkwd}[1]{\textcolor[rgb]{0.78,0.227,0.412}{#1}}%
\let\hlipl\hlkwb

\usepackage{framed}
\makeatletter
\newenvironment{kframe}{%
 \def\at@end@of@kframe{}%
 \ifinner\ifhmode%
  \def\at@end@of@kframe{\end{minipage}}%
  \begin{minipage}{\columnwidth}%
 \fi\fi%
 \def\FrameCommand##1{\hskip\@totalleftmargin \hskip-\fboxsep
 \colorbox{shadecolor}{##1}\hskip-\fboxsep
     % There is no \\@totalrightmargin, so:
     \hskip-\linewidth \hskip-\@totalleftmargin \hskip\columnwidth}%
 \MakeFramed {\advance\hsize-\width
   \@totalleftmargin\z@ \linewidth\hsize
   \@setminipage}}%
 {\par\unskip\endMakeFramed%
 \at@end@of@kframe}
\makeatother

\definecolor{shadecolor}{rgb}{.97, .97, .97}
\definecolor{messagecolor}{rgb}{0, 0, 0}
\definecolor{warningcolor}{rgb}{1, 0, 1}
\definecolor{errorcolor}{rgb}{1, 0, 0}
\newenvironment{knitrout}{}{} % an empty environment to be redefined in TeX

\usepackage{alltt}
\usepackage{standalone}
\standalonetrue
\ifstandalone
  \usepackage{../../haziq_thesis}
  \usepackage{../../haziq_maths}
  \usepackage{../../haziq_glossary}
  \addbibresource{../../bib/haziq.bib}
  \externaldocument{../01/.texpadtmp/introduction}
  \externaldocument{../02/.texpadtmp/chapter2}
  \externaldocument{../03/.texpadtmp/chapter3}
  \externaldocument{../04/.texpadtmp/chapter4}
  \externaldocument{../05/.texpadtmp/chapter5}
  \externaldocument{../appendix/.texpadtmp/appendix}
\fi




\IfFileExists{upquote.sty}{\usepackage{upquote}}{}
\begin{document}

We present analyses of real-data examples using the I-probit model for a variety of applicaitons, namely binary and multiclass classification, meta-analysis, and spatio-temporal modelling of point processes.
Examples in this section have been analysed using in \proglang{R} using the in-development \pkg{iprobit} package written by us.
Code for replication is provided at \url{http://myphdcode.haziqj.ml}.
All of these examples had assumed a fixed error precision $\bPsi = \bI_m$.

\subsection{Predicting cardiac arrhythmia}

Statistical learning tools are being used in the field of medicine as a means to aid medical diagnosis of diseases.
In this example, factors determining the presence or absence of heart diseses are studied.
Traditionally, cardiologists inspect patients' cardiac activity (ECG data) in order to reach a diagnosis, which remains the ``gold standard'' method of obtaining diagnoses.
The study by \citet{guvenir1997supervised} aimed to predict cardiac abnormalities by way of machine learning, and minimise the difference between the gold standard and computer-based classifications.

The data set\footnote{Data is made publicly available at \url{https://archive.ics.uci.edu/ml/datasets/arrhythmia}.} at hand contains a myriad of ECG readings and other patient attributes such as age, height, and weight.
Altogether, there are $n=451$ observations and $p=279$ predictors.
In order for a valid comparison to be made to other studies, we excluded nominal covariates, leaving us with $p=194$ continuous predictors, which we then standardised.
In the original data set, there are 13 distinct classes of cardiac arrhythmia---again, following the lead of other studies, we had combined all forms of cardiac diseases to form a single class, thus reducing the problem to a binary classification task (normal vs. arrhythmia).

Following \cref{eq:iprobitbin}, the relationship between patient $i$'s probability of having a form of cardiac arrhthmia $p_i$ and the predictors $x_i\in\cX \equiv \bbR^{194}$ is modelled as
\begin{gather*}
  \Phi(p_i) = \alpha + f(x_i).
\end{gather*}
Further, assuming $f\in\cF$ a suitable RKHS with kernel $h_\lambda$, we may assign an I-prior on the (latent) regression function $f$.
We consider three RKHSs: the canonical (linear) RKHS, the fBm-0.5 RKHS and the SE RKHS.
The first of these three assumes an underlying linear relationship of the covariates and the probabilities, while the other two assumes a smooth relationship.
As all covariates had been standardised, it is sufficient to assign a single scale parameter $\lambda$ for the I-probit model.

For reference, fitting an I-probit model on the full data set takes about 4 seconds only, with convergence reached in at most 15 iterations.
\cref{fig:cardiac.mod.full.plot} plots the variational lower bound value over time and iterations for the cardiac arrhythmia data set.
As expected, the lower bound value increases over time until a convergence criterion is reached.




\begin{knitrout}
\definecolor{shadecolor}{rgb}{1, 1, 1}\color{fgcolor}\begin{figure}[htb]

{\centering \includegraphics[width=0.785\linewidth]{figure/05-cardiac_mod_full_plot-1} 
\includegraphics[width=0.785\linewidth]{figure/05-cardiac_mod_full_plot-2} 

}

\caption[Plot of variational lower bound over time (left), and plot of training error rate and Brier scores over time (right)]{Plot of variational lower bound over time (left), and plot of training error rate and Brier scores over time (right).}\label{fig:cardiac.mod.full.plot}
\end{figure}


\end{knitrout}

To measure predictive ability, we fit the I-probit models on a random subset of the data and obtain the out-of-sample test error rates from the remaining held-out observations.
We then compare the results against popular machine learning classifiers, namely: 1) linear and quadratic discriminant analysis (LDA/QDA); 2) $k$-nearest neighbours; 3) support vector machines (SVM) \citep{steinwart2008support}; 4) Gaussian process classification \citep{rasmussen2006gaussian}; 5) random forests \citep{breiman2001random}; 6) nearest shrunken centroids (NSC) \citep{tibshirani2002diagnosis}; and 7) L-1 penalised logistic regression \citep{friedman2001elements}.
The experiment is set up as follows:
\begin{enumerate}
  \item Form a training set by sub-sampling $s \in \{50, 100, 200\}$ observations.
  \item The remaining unsampled data is used as the test set.
  \item Fit model on training set, and obtain test error rates defined as
  \[
    \text{test error rate} = \frac{1}{s} \sum_{i=1}^n [y^{\text{pred}}_i \neq y^{\text{test}}_i] \times 100 \%.
  \]
  \item Repeat steps 1-3 100 times to obtain the \emph{average} test error rates and standard errors.
\end{enumerate}
Results for the methods listed above were extracted from the in-depth study by \citet{cannings2017random}, who also conducted identical experiments using their random projection (RP) ensemble classification method.
These are all tabulated in \cref{tab:cardiac}.

\begin{table}[htb]
  \caption{Mean out-of-sample misclassification rates and standard errors in parantheses for 100 runs of various training ($s$) and test ($451-s$) sizes for the cardiac arrhythmia binary classification task.}
  \label{tab:cardiac}
  \centering
  \begin{tabular}{l r r r}
  \toprule
  \Bot&\multicolumn{3}{ c }{{Misclassification rate (\%)}} \\
  \cline{2-4}
  \Top Method
  & {$s = 50$}
  & {$s = 100$}
  & {$s = 200$} \\
  \midrule
  \emph{I-probit} \\
  \hspace{0.5em} Linear            & 35.52 (0.44) & 31.35 (0.33) & 29.45 (0.38) \\
  \hspace{0.5em} Smooth (fBm-0.5)  & 33.64 (0.66) & 28.12 (0.34) & 24.33 (0.24) \\
  \hspace{0.5em} Smooth (SE-1.0)   & 48.26 (0.40) & 48.32 (0.43) & 47.11 (0.37) \\
  \\
  \emph{Others} \\
  \hspace{0.5em} RP-LDA            & 33.24 (0.42) & 30.19 (0.35) & 27.49 (0.30) \\
  \hspace{0.5em} RP-QDA            & 30.47 (0.33) & 28.28 (0.26) & 26.31 (0.28) \\
  \hspace{0.5em} RP-$k$-NN         & 33.49 (0.40) & 30.18 (0.33) & 27.09 (0.31)
  \\[0.5em]
  \hspace{0.5em} Random forests    & 31.65 (0.39) & 26.72 (0.29) & 22.40 (0.31) \\
  \hspace{0.5em} SVM (linear)      & 36.16 (0.47) & 35.61 (0.39) & 35.20 (0.35) \\
  \hspace{0.5em} SVM (Gaussian)    & 48.39 (0.49) & 47.24 (0.46) & 46.85 (0.43)
  \\[0.5em]
  \hspace{0.5em} GP (Gaussian)     & 37.28 (0.42) & 33.80 (0.40) & 29.31 (0.35) \\
  \hspace{0.5em} NSC               & 34.98 (0.46) & 33.00 (0.40) & 31.08 (0.41) \\
  \hspace{0.5em} L-1 logistic      & 34.92 (0.42) & 30.48 (0.34) & 26.12 (0.27) \\
  \end{tabular}
\end{table}

Of the three I-probit models, the fBm model performed the best.
That it performed better than the canonical linear I-probit model is unsurprising, since an underlying smooth function to model the latent variables is expected to generalise better than a rigid straight line function.
The poor performance of the SE I-probit model may be due to the fact that the lengthscale parameter was not estimated (fixed at $l=1$), but then again, we notice reliable performance of the fBm even with fixed Hurst index ($\gamma = 0.5$).
It can be seen that the fBm I-probit model also outperform the more popular machine learning algorithms out there including $k$-nearest neighbours, support vector machines and Gaussian process classification.
It came second only to random forests, an ensemble learning method, which is also generally faster to train than Gaussian process-like regressions such as I-prior models.
The time complexity of a random forest algorithm is $O(pqn\log(n))$ \citep{louppe2014understanding}, where $p$ is the number of variables used for training, $q$ is the number of random decision trees, and $n$ is the number of observations.

\subsection{Meta-analysis of smoking cessation}

Conider the smoking cessation data set, as described in \citet{skrondal2004generalized}.
It contains observations from 27 separate smoking cessation studies in which participants are subjected to either a nicotine gum treatment or a placebo.
The interest is to estimate the treatment effect size, and whether it is statistically significant, i.e. whether or not nicotine gum is an effective treatment to quit smoking.
The studies are conducted at different times and due to various reasons such as funding and cultural effects, the results from all of the studies may not be in agreement.
The number of effective participants plays a major role in determining the power of the statistical tests performed in individual studies.
The question then becomes how do we meaningfully aggregate all the data to come up with one summary measure?

Several methods exist to analyse such data sets.
One may consider a fixed-effects model, similar to a classical one-way ANOVA model to establish whether or not the effect size is significant.
Because of the study-specific characteristics, it is natural to consider multilevel or random-effects models as a means to estimate the effect size.
Regardless of method, the approach of analysing study-level treatment effects instead of patient-level data only is the paradigm for meta-analysis, and our I-prior model takes this approach as well.
% However, analysing study-level estimates of effect size can be problematic for various reasons, such as small group samples or rare occurences.

\begin{knitrout}
\definecolor{shadecolor}{rgb}{1, 1, 1}\color{fgcolor}\begin{figure}[htb]

{\centering \includegraphics[width=\maxwidth]{figure/05-plot_data_smoke-1} 

}

\caption[Comparative box-plots of the distribution of patients who successfully quit smoking and those who remained smokers, in the two treatment groups]{Comparative box-plots of the distribution of patients who successfully quit smoking and those who remained smokers, in the two treatment groups. It is evident that there are more successful patients quitting smoking in the treatment group than in the control group. The raw odds ratio of quitting smoking (treatment vs. control) is 1.66.}\label{fig:plot.data.smoke}
\end{figure}


\end{knitrout}
A summary of the data is displayed by the box-plot in \cref{fig:plot.data.smoke}.
On the whole, there are a total of 5,908
% study.size[length(study.size)]
patients, and they are distributed roughly equally among the control and treatment groups (46.33\% and 53.67\% respectively, on average).
From the box-plots, it is evident that there are more patients who quit smoking in the treatment group as compared to the placebo control group.
There are various measures of treatment effect size, such as risk ratio or risk differences, but we shall concentrate on \emph{odds ratios} as defined by
\[
  \text{odds ratio} = \frac{\text{odds of quitting smoking in \emph{treatment} group}}
  {\text{odds of quitting smoking in \emph{control} group}}.
\]
The odds of quitting smoking in either group is defined as
\[
  \text{odds} = \frac{\Prob(\text{quit smoking})}{1 - \Prob(\text{quit smoking})},
\]
and these probabilities, odds and ultimately the odds ratio can be estimated from sample proportions.
This raw odds ratio for all study groups is calculated as $1.66 = e^{0.50}$.
It is also common for the odds ratio to be reported on the log scale (usually as a remnant of logistic models).
A value greater than one for the odds ratio (or equivalently, greater than zero for the log-odds ratio) indicates a significant treatment effect.

A random-effects analysis using a multilevel logistic model has been considered by \citet{agresti2000tutorial}.
Let $i=1,\dots,n_k$ index the patients in study group $k \in \{1,\dots,27\}$.
For patient $i$ in study $j$, $p_{ik}$ denotes the probability that the patient has successfully quit smoking.
Additionally, $x_{ik}$ is the centred dummy variable indicating patient $i$'s treatment group in study $k$.
These take on two values: 0.5 for treated patients and -0.5 for control patients.
The logistic random-effects model is
\begin{gather*}
  \log \left( \frac{p_{ij}}{1-p_{ij}} \right) = \beta_{0j} + \beta_{1j}x_{ij} \\
  \text{with} \\
  \begin{pmatrix} \beta_{0j} \\ \beta_{1j} \end{pmatrix}
  \sim \N \left(
  \begin{pmatrix} \beta_0 \\ \beta_1 \end{pmatrix},
  \begin{pmatrix} \sigma_0^2 & \sigma_{01} \\ \sigma_{01} & \sigma_1^2 \\ \end{pmatrix}
  \right)
\end{gather*}
\citet{agresti2000tutorial} also made the additional assumption $\sigma_{01} = 0$, so that, coupled with the contrast coding used for $x_{ik}$, the total variance $\Var(\beta_{0k} + \beta_{1j}x_{ik})$ would be constant in both treatment groups.
The overall log odds ratio is represented by $\beta_1$, and this is estimated as $0.57 = \log 1.76$.

In an I-prior model, the Bernoulli probabilities $p_{ik}$ are regressed against the treatment group indicators $x_{ik}$ and also the patients' study group $k$ via the regression function $f$ and a probit link:
\begin{align*}
  \Phi^{-1}(p_{ik})
  &= f(x_{ik}, k) \\
  &= f_1(x_{ik}) + f_2(k) + f_{12}(x_{ik}, j).
\end{align*}
We have decomposed our function $f$ into three parts: $f_1$ represents the treatment effect, $f_2$ represents the effect of the study groups, and $f_{12}$ represents the interaction effect between the treatment and study group on the modelled probabilities.
As both $x_{ik}$ and $k$ are nominal variables, the functions $f_1$ and $f_2$ both lie in the Pearson RKHS of functions $\cF_1$ and $\cF_2$, each with RKHS scale parameters $\lambda_1$ and $\lambda_2$.
As such, it does not matter how the $x_{ik}$ variables are coded (dummy coding 0, 1 vs. centred coding -0.5, 0.5) as the scaling of the function is determined by the RKHS scale parameters.
The interaction effect $f_{12}$ lies in the RKHS tensor product $\cF_1 \otimes \cF_2$.
In I-prior modelling, there are only two parameters to estimate, while in the standard logistic random-effects model, there are six.
The results of the I-prior fit are summarised in the table below.


\newcolumntype{R}[1]{>{\raggedleft\arraybackslash}p{#1}}
\begin{table}[hbt]
  \centering
  \caption{Results of the I-probit model fit for three models.}
  \label{tab:mod.compare.smoke}
  \begin{tabular}{lrrrR{2cm}}
  \toprule
  Model & ELBO & Error rate (\%) & Brier score & No. of parameters \\
  \midrule
  $f_1$
  &-3210.76
  &23.65
  &0.179 & 1 \\
  $f_1+f_2$
  &-3142.24
  &29.30
  &0.206 & 2 \\
  $f_1+f_2+f_{12}$
  &-3091.20
  &23.48
  &0.168 & 2 \\
  \bottomrule
  \end{tabular}
\end{table}

The approximated marginal log-likelihood value for the I-prior model (i.e. variational lower bound), the Brier score for each model and the number of RKHS scale parameters estimated in the model are reported in \cref{tab:mod.compare.smoke}. Three models were fitted:
1) A model with only the treatment effect;
2) A model with a treatment effect and a study group effect; and
3) Model 2 with the additional assumption that treatment effect varies across study groups.
Model 1 disregards the study group effects, while Model 2 assumes that the effectiveness of the nicotine gum treatment does not vary across study groups (akin to a varying-intercept model).
A model comparison using the evidence lower bound indicates that Model 3 has the highest value, and the difference is significant from a Bayes factor standpoint---$\BF(M_3,M_1)$ and $\BF(M_3,M_2)$ are both greater than 150.
% with both Bayes factor comparing Model 3 against Models 1 and 2 being greater than 150.
% $\BF(M_3, M_1)iprior::dec_plac(exp(l[3] - l[1]), 2)$ and $\BF(M_3, M_2)iprior::dec_plac(exp(l[3] - l[2]), 2)$.
% Although not soundly based in theory, we may compare variational lower bounds of the three models for model selection as a proxy to using the true log-likelihood value.
% In this case, Model 3 has the highest lower bound value.
The misclassification rate and Brier score indicates good predictive performance of the models, and there is not much to distinguish between the three although Model 3 is the best out of the three models.

\begin{knitrout}
\definecolor{shadecolor}{rgb}{1, 1, 1}\color{fgcolor}\begin{figure}[p]

{\centering \includegraphics[width=\linewidth]{figure/05-smoke_forest_plot-1} 

}

\caption[Forest plot of effect sizes (log odds ratios) in each group as well as the overall effect size together with their 95\% confidence bands]{Forest plot of effect sizes (log odds ratios) in each group as well as the overall effect size together with their 95\% confidence bands. The plot compares the raw log odds ratios, the logistic random-effect estimates, and the I-prior estimates. Sizes of the points indicate the relative sample sizes per study group.}\label{fig:smoke.forest.plot}
\end{figure}


\end{knitrout}

Unlike in the logistic random-effects model, where the log odds ratio can be read off directly from the coefficients, with an I-prior probit model the log odds ratio needs to be calculated manually from the fitted probabilities.
The probabilities of interest are the probabilities of quitting smoking under each treatment group for each study group $k$---call these $p_k(\text{treatment})$ and $p_k(\text{control})$.
That is,
\begin{align*}
  p_k(\text{treatment}) &= \Phi\big( \hat\nu(\text{treatment}, k) \big) \\
  p_k(\text{control})   &= \Phi\big( \hat\nu(\text{control}, k) \big),
\end{align*}
where $\hat \nu$ represents the standardised posterior mean estimate for the regression functions which are distributed according to
\[
  f(x_{ik},k)|\by,\hat\theta \sim \N\big(\hat\mu(x_{ik}, k), \hat\sigma^2(x_{ij}, k) \big),
\]
with $x_{ik}\in\{\text{treatment}, \text{control} \}$ and $k \in \{1,\dots,27\}$  (see details in \cref{sec:iprobitpostest}).
The log odds ratio for each study group can then be calculated as usual.
For the overall log odds ratio, the probabilities that are used are the averaged probabilities weighted according to the sample sizes in each group.
This has been calculated as $0.51 = \log 1.66$, slightly lower than both the raw log odds ratio and the log odds ratio estimated by the logistic random-effects model.
This can perhaps be attributed to some shrinkage of the estimated probabilities due to placing a prior with zero mean on the regression functions.

The credibility intervals for the log odds ratios in the forest plot of \cref{fig:smoke.forest.plot} are also noticeably narrower under an I-prior compared to the fitted multilevel model.
One explanation is that empirical Bayes estimates, such as the I-probit estimates under a variational EM framework, tend to underestimate the variability in the estimates because the variability in the parameters are ignored when point estimates are used, compared to distributions in a true Bayesian estimation framework.

\subsection{Multiclass classification: Vowel recognition data set}





We illustrate multiclass classification using I-priors on a speech recognition data set\footnotemark~with $m = 11$ classes to be predicted from digitized low pass filtered signals generated from voice recordings.
Each class corresponds to a vowel sound made when pronouncing a specific word.
The words that make up the vowel sounds are shown in \cref{tab:vowel}. Each word was uttered once by multiple speakers, and the data are split into a training and a test set.
Four males and four female speakers contributed to the training set, while four male and three female speakers contributed to the test set.
The recordings were manipulated using speech processing techniques, such that each speaker yielded six frames of speech from the eleven vowels, each with a corresponding 10-dimensional numerical input vector (the predictors).
This means that the size of the training set is 528, while 462 data points are available for testing the predictive performance of the models.
This data set is also known as Deterding's vowel recognition data (after the original collector, \cite{deterding1989speaker}).
Machine learning methods such as neural networks and nearest neighbour methods were analysed by \citet{robinson1989dynamic}.

\footnotetext{Data is publicaly available from the UCI Machine Learning Repository, URL: \url{https://archive.ics.uci.edu/ml/datasets/Connectionist+Bench+(Vowel+Recognition+-+Deterding+Data)}.}

\begin{table}[htb]
\centering
\caption{The eleven words that make up the classes of vowels.}
\label{tab:vowel}
\begin{tabular}{llllllllll}
\toprule
Class & Label          & Vowel & Word &  && Class & Label          & Vowel & Word  \\
\midrule
1     & \texttt{hid} & \dsil{iː}    & heed &&  & 7     & \texttt{hOd} & \dsil{ɒ}    & hod   \\
2     & \texttt{hId} & \dsil{ɪ}     & hid  &&  & 8     & \texttt{hod} & \dsil{ɔː}   & hoard \\
3     & \texttt{hEd} & \dsil{ɛ}     & head &&  & 9     & \texttt{hUd} & \dsil{ʊ}    & hood  \\
4     & \texttt{hAd} & \dsil{a}     & had  &&  & 10    & \texttt{hud} & \dsil{uː}   & who'd \\
5     & \texttt{hYd} & \dsil{ʌ}     & hud  &&  & 11    & \texttt{hed} & \dsil{əː}   & heard \\
6     & \texttt{had} & \dsil{ɑː}    & hard &&  &       &              &             &       \\
\bottomrule
\end{tabular}
\end{table}

We will fit the data using an I-probit model with the canonical linear kernel, fBm-0.5 kernel, and the SE kernel with lengthscale $l=1$.
% We assume $m = 11$ distinct I-priors corresponding to the latent variables in each class, thus there are 11 unique intercepts and 11 RKHS scale parameters to estimate in each model.
Each model took roughly 13 seconds per iteration in fitting the training data set ($n=528$).
In particular, the canonical kernel model took a long time to converge, with each variational inference iteration improving the lower bound only slighly each time.
In contrast, both the fBm-0.5 and SE model were quicker to converge.
Multiple restarts from different random seeds were conducted, and we found that they all converged to a similar lower bound value.
This alleviates any concerns that the model might have converged to different multiple local optima.

\begin{knitrout}
\definecolor{shadecolor}{rgb}{1, 1, 1}\color{fgcolor}\begin{figure}[htb]

{\centering \subfloat[Canonical kernel\label{fig:vowel.confusion.matrix1}]{\includegraphics[width=0.485\linewidth]{figure/05-vowel_confusion_matrix-1} }
\subfloat[fBm-0.5 kernel\label{fig:vowel.confusion.matrix2}]{\includegraphics[width=0.485\linewidth]{figure/05-vowel_confusion_matrix-2} }
\subfloat[SE kernel\label{fig:vowel.confusion.matrix3}]{\includegraphics[width=0.485\linewidth]{figure/05-vowel_confusion_matrix-3} }

}

\caption[Confusion matrices for the vowel classification problem in which predicted values were obtained from the I-probit models]{Confusion matrices for the vowel classification problem in which predicted values were obtained from the I-probit models. The maximum value for any cell is 42 (seven speakers delivered six frames of speech per vowel). Blank cells indicate nil values.}\label{fig:vowel.confusion.matrix}
\end{figure}


\end{knitrout}

A good way to visualise the performance of model predictions is through a confusion matrix, as shown in \cref{fig:vowel.confusion.matrix}.
The numbers in each row indicate the instances of a predicted class, while the numbers in the column indicate instances of the actual classes, while nil values are indicated by blank cells.
% A quick glance of the plots seem to favour the fBm-0.5 kernel as having better predictions.
% There are a lot more misclassifications when using the canonical kernel.
% Under the fBm-0.5 model, the model makes understandable mistakes - confusing very similar words, especially `hod' and `hud'.

Comparisons to other methods that had been used to analyse this data set is given in \cref{tab:vowel.tab}.
In particular, the I-probit model is compared against 1) linear regression; 2) logistic linear regression; 3) linear and quadratic discriminant analysis; 4) decision trees; 5) neural networks; 6) $k$-nearest neighbours; and 7) flexible discriminant analysis.
All of these methods are described in further detail in \citet[Ch.4 \& 12, Table 12.3]{friedman2001elements}.
The I-probit model using both the fBm-0.5 and SE kernel offers one of the best out-of-sample classification error rates (34.4\%) of all the methods compared.
The linear I-probit model is seen to be comparable to logistic regression, linear and quadratic discrimant analysis, and also decision trees.
It also provides significant improvement over multiple linear regression.

\begin{table}
  \caption{Results of various classification methods for the vowel data set.}
  \label{tab:vowel.tab}
  \centering
  \begin{tabular}{l r r}
  \toprule
  \Bot &\multicolumn{2}{c}{Error rate (\%)} \\
  \cline{2-3}
  \Top Model & Train & Test \\
  \midrule
  \emph{I-probit} \\
  \hspace{0.5em} Linear
  & 29
  & 54 \\
  \hspace{0.5em} Smooth (fBm-0.5)
  & 22
  & 40 \\
  \hspace{0.5em} Smooth (SE-1.0)
  & 7
  & 34 \\
  \\
  \emph{Others} \\
  \hspace{0.5em} Linear regression               & 48 & 67 \\
  \hspace{0.5em} Logistic regression             & 22 & 51
  \\[0.5em]
  \hspace{0.5em} Linear discriminant analysis    & 32 & 56 \\
  \hspace{0.5em} Quadratic discriminant analysis & 1  & 53
  \\[0.5em]
  \hspace{0.5em} Decision trees                  & 5  & 54 \\
  \hspace{0.5em} Neural networks                 &    & 45 \\
  \hspace{0.5em} $k$-nearest neighbours          &    & 44
  \\[0.5em]
  \hspace{0.5em} FDA/BRUTO                       & 6  & 44 \\
  \hspace{0.5em} FDA/MARS                        & 13 & 39 \\
  \bottomrule
  \end{tabular}
\end{table}

\subsection{Spatio-temporal modelling of bovine tuberculosis in Cornwall}



Data containing the number of breakdows of bovine tubercolosis (BTB) in Cornwall, the locations of the infected animals, and the year of occurence is analysed.
The interest, as motivated by veterinary epidimiology, is to understand whether or not there is spatial segregation between the herds, and whether there is a time-element to presence or absence of this spatial segregation.
There has been previous work done to analyse this data set.
\citet{diggle2005nonparametric} developed a non-parametric method to estimate spatial segregation using a multivariate point process.
The occurrences are modelled as Poisson point processes, and spatial segregation is said to have occured if the model-estimated type-specific breakdown probabilities at any given location are not significantly different from the sample proportions.
The authors estimated the probabilities via kernel regression, and the test statistic of interest had to be estimated via Monte Carlo methods.
Other works include \citet{diggle2013spatial}, who used a fully Bayesian approach for spatio-temporal multivariate log-Gaussian Cox processes, which is implemented in the \proglang{R} package \pkg{lgcp} \citep{taylor2013lgcp}.

\begin{knitrout}
\definecolor{shadecolor}{rgb}{1, 1, 1}\color{fgcolor}\begin{figure}[htb]

{\centering \includegraphics[width=\linewidth]{figure/05-plot_cow-1} 

}

\caption[Distribution of the different types (Spoligotypes) of bovine tubercolosis affecting herds in Cornwall over the period 1989 to 2002]{Distribution of the different types (Spoligotypes) of bovine tubercolosis affecting herds in Cornwall over the period 1989 to 2002.}\label{fig:plot.cow}
\end{figure}


\end{knitrout}

The data set contains $n=919$ recorded cases over a span of 14 years.
For each of the cases, spatial data pertaining to the location of the farm (Northings and Eastings, measured in kilometres) are available.
Originally, 11 unique spoligotypes were recorded in the data, with the four most common spoligotypes being Sp9 ($m=1$), Sp12 ($m=2$), Sp15 ($m=3$) and Sp20 ($m=4$), as shown by the histogram in \cref{fig:plot.cow}.
We had grouped the remaining seven spoligotypes into an `Others' category ($m=5$), so that the problem becomes a multinomial regression with five distinct outcomes.

\begin{knitrout}
\definecolor{shadecolor}{rgb}{1, 1, 1}\color{fgcolor}\begin{figure}[htb]

{\centering \includegraphics[width=\linewidth]{figure/05-plot_cornwall-1} 

}

\caption[Spatial distribution of all cases over the 14 years]{Spatial distribution of all cases over the 14 years.}\label{fig:plot.cornwall}
\end{figure}


\end{knitrout}

We are able to investigate any spatio-temporal patterns of infection using I-priors rather simply.
Let $p_{ij}$ denote the probability that a particular farm $i$ is infected with a BTB disease with spoligotype $j \in \{1,\dots,5\}$.
We model the transformed probabilities $g_j(p_{ij})$ as following a function which takes two covariates, i.e. the spatial data $x_1 \in \bbR^2$, and the temporal data $x_2$ (year of infection):
\begin{align*}
  p_{ij}
  &= g_j^{-1} \big( f_k(x_1, x_2) \big)_{k=1}^m \\
  &= g_j^{-1} \big( f_{1k}(x_1) + f_{2k}(x_2) + f_{12k}(x_1, x_2) \big)_{k=1}^m,
\end{align*}
where the function $g_j^{-1}:\bbR^m \to [0,1]$ is the same squashing function used in equation \cref{eq:intractablelikelihood2}.
We assume a smooth effect of space and time on the probabilities, and appropriate RKHSs for the functions $f_1 \in \cF_1$ and $f_2 \in \cF_2$ are the fBm-0.5 RKHS.
Alternatively, as per \citet{diggle2005nonparametric}, divide the data into four distinct time periods:
1) 1996 and earlier;
2) 1997 to 1998;
3) 1999 to 2000;
and finally 4) 2001 to 2002.
In this case, $x_2$ would indicate which period the infection took place in, and thus would have a nominal effect on the probabilities.
An appropriate RKHS for $f_2$ in such a case would be the Pearson RKHS.
In either case, the function $f_{12} \in \cF_1 \otimes \cF_2$ would be the ``interaction effect'', meaning that with such an effect present, the spatial distribution of the diseases are assumed to vary across the years.

\begin{sidewaystable}[p]
\caption{\label{tab:table.btb}Results of the fitted I-probit models. Estimates of the class intercepts and scale parameters, together with their respective bootstrap standard errors, are presented. For model comparison, we can look at ELBOs, error misclassification rates, and Brier scores.}
\centering
\begin{tabular}[t]{lrrrrrrrrrrr}
\toprule
\multicolumn{1}{c}{ } & \multicolumn{2}{r}{$M_0$: Intercepts only} & \multicolumn{1}{c}{ } & \multicolumn{2}{r}{$M_1$: Spatial only} & \multicolumn{1}{c}{ } & \multicolumn{2}{r}{$M_2$: Spatio-temporal} & \multicolumn{1}{c}{ } & \multicolumn{2}{r}{$M_3$: Spatio-period} \\
\cmidrule(l{2pt}r{2pt}){2-3} \cmidrule(l{2pt}r{2pt}){5-6} \cmidrule(l{2pt}r{2pt}){8-9} \cmidrule(l{2pt}r{2pt}){11-12}
  & Estimate & S.E. &   & Estimate & S.E. &   & Estimate & S.E. &   & Estimate & S.E.\\
\midrule
Intercept (Sp9) & 0.948 & 0.000 &  & 1.364 & 0.015 &  & 1.401 & 0.079 &  & 1.395 & 0.103\\
Intercept (Sp12) & -0.173 & 0.000 &  & -0.435 & 0.013 &  & -0.506 & 0.017 &  & -0.463 & 0.045\\
Intercept (Sp15) & 0.103 & 0.000 &  & -0.020 & 0.011 &  & -0.008 & 0.059 &  & -0.010 & 0.094\\
Intercept (Sp20) & -0.202 & 0.000 &  & -0.775 & 0.051 &  & -0.795 & 0.223 &  & -0.783 & 0.343\\
Intercept (Others) & -0.676 & 0.000 &  & -0.134 & 0.016 &  & -0.091 & 0.077 &  & -0.139 & 0.104\\[0.5em]
Scale (spatial) &  &  &  & 0.194 & 0.008 &  & -0.176 & 0.178 &  & 0.172 & 0.169\\
Scale (temporal) &  &  &  &  &  &  & -0.006 & 0.003 &  & -0.004 & 0.006\\
\\
ELBO & -1187.47 &  &  & -564.33 &  &  & -537.23 &  &  & -543.94 & \\
Error rate (\%) & 46.25 &  &  & 19.26 &  &  & 18.06 &  &  & 18.50 & \\
Brier score & 0.249 &  &  & 0.143 &  &  & 0.136 &  &  & 0.138 & \\
\bottomrule
\end{tabular}
\end{sidewaystable}

We fitted four different models:
\begin{itemize}
  \item \textbf{\boldmath$M_0$: Intercept only}.
  \[
    p_{ij} = g^{-1}_j\big( \alpha_k \big)_{k=1}^m
  \]
  \item \textbf{\boldmath$M_1$: Spatial segregation}.
  \[
    p_{ij} = g^{-1}_j\big(\alpha_k + f_{1k}(x_i) \big)_{k=1}^m
  \]
  $f_{1k} \in \cF_1$ Pearson RKHS.
  \item \textbf{\boldmath$M_2$: Spatio-temporal}.
  \[
    p_{ij} = g^{-1}_j\big(\alpha_k + f_{1k}(x_i) + f_{2k}(t_i) + f_{12k}(x_i,t_i) \big)_{k=1}^m
  \]
  $f_{1k} \in \cF_1$ Pearson RKHS, $f_{2k} \in \cF_2$ fBm-0.5 RKHS, and $f_{12k} \in \cF_1\otimes\cF_2$
  \item \textbf{\boldmath$M_3$: Spatio-period}.
  \[
    p_{ij} = g^{-1}_j\big(\alpha_k + f_{1k}(x_i) + f_{2k}(t_i) + f_{12k}(x_i,t_i) \big)_{k=1}^m
  \]
  $f_{1k} \in \cF_1$ Pearson RKHS, $f_{2k} \in \cF_2$ Pearson RKHS, and $f_{12k} \in \cF_1\otimes\cF_2$
\end{itemize}
Model $M_0$ corresponds to a model which ignores any spatial or temporal effects (the baseline intercept only model).
Model $M_1$ takes into account only spatial effects.
Both models $M_2$ and $M_3$ account for spatio-temporal effects, but $M_2$ assumes a smooth effect of time, while $M_3$ segregates the points into four distinct time periods for analysis.
Model comparison is easily done, and \cref{tab:table.btb} indicates that model $M_2$ has the highest log-likelihood of the four models, making it the preferable model.

% Alternatively, spatio-temporal effects of the BTB breakdowns can easily be inferred through the RKHS scale parameters.
% Let $h_k$, $k \in \{1,2\}$ denote the reproducing kernel of the spatial and temporal RKHSs respectively.
% Then, an I-prior on $f_j = f_{1j} + f_{2j} + f_{12j}$, $j=1,\dots,5$, takes the form
% \[
%   f_j(x_1,x_2) = \sum_{i=1}^n \big(
%   \lambda_1 h_1(x_1, x_{i1}) + \lambda_2 h_2(x_2, x_{i2}) +
%   \lambda_1\lambda_2 h_1(x_1, x_{i1}) h_2(x_2, x_{i2})
%   \big)w_{ij}
% \]
% where it is assumed $(w_{i1},\dots,w_{i5})^\top \iid \N(\bzero,\bI_5)$.
% The hypothesis of temporal significance is the same as testing the significance of the $\lambda_2$ parameter, while the test of both spatial and temporal effects are conducted on $\lambda_1$ and $\lambda_2$ simultaneously.
% From \hltodo{Chapter X}, we know that these scale parameters follow a normal posterior distribution, so we can calculate the $Z$-scores by dividing the mean by its corresponding standard deviation.
% Absolute values greater than three would satisfy a Bayesian hypothesis test of significance at the 0.01 level.
% The conclusion from \cref{tab:table.btb} is that the data supports a hypothesis for a spatio-temporal or spatio-period model.

For a more visual approach, we can look at the plots of the surface probabilities.
To obtain these probabilities, we first determined the spatial points (Northings and Eastings) which fall inside the polygon which makes up Cornwall.
We then obtained predicted probabilities for each class of disease at each location.
\cref{fig:plot.btb} was obtained using the model with spatial covariates only, thus ignoring any temporal effects.
In the case of the spatio-temporal model, we used the model which had the period formulation for time (model $M_3$).
This way, we can display the surface probabilities of the time periods in four plots only, which is more economical to exhibit within the margins of this thesis.
Note that there is no issue with using the continuous time model---we have produced an animated gif image at \url{http://phd.haziqj.ml/examples/}, showing the evolution of the surface probabilities over time.

\begin{knitrout}
\definecolor{shadecolor}{rgb}{1, 1, 1}\color{fgcolor}\begin{figure}[p]

{\centering \includegraphics[width=\linewidth]{figure/05-plot_btb-1} 

}

\caption[Predicted probability surfaces for BTB contraction in Cornwall for the four largest spoligotypes of the bacterium \emph{Mycobacterium bovis} over the entire time period using model $M_1$]{Predicted probability surfaces for BTB contraction in Cornwall for the four largest spoligotypes of the bacterium \emph{Mycobacterium bovis} over the entire time period using model $M_1$.}\label{fig:plot.btb}
\end{figure}


\end{knitrout}
\begin{knitrout}
\definecolor{shadecolor}{rgb}{1, 1, 1}\color{fgcolor}\begin{figure}[p]

{\centering \includegraphics[width=\linewidth]{figure/05-plot_temporal_btb-1} 

}

\caption[Predicted probability surfaces for BTB contraction in Cornwall for the four largest spoligotypes of the bacterium \emph{Mycobacterium bovis} over four different time periods using model $M_3$]{Predicted probability surfaces for BTB contraction in Cornwall for the four largest spoligotypes of the bacterium \emph{Mycobacterium bovis} over four different time periods using model $M_3$.}\label{fig:plot.temporal.btb}
\end{figure}


\end{knitrout}

As the plots suggests, there is indeed spatial segregation for the four most common spoligotypes, and this is also very prominently seen from \cref{fig:plot.btb}.
In comparing the distribution of the spoligotypes over the years, we may refer to \cref{fig:plot.temporal.btb}, a series of predicted probability surface plots over the four time periods obtained from model $M_3$.
For each time period, we also superimposed the actual observations onto the predicted surface probabilities.
In addition, coloured dotted lines are displayed to indicate the ``decision boundaries'' for each of the four spoligotypes.
The most evident change is seen to the spatial distribution of spoligotype 12, with the decision boundary giving it a large area in years 1996 and earlier, but this steadily shrunk over the years.
Spoligotype 9, which is most commonly seen in the east of Cornwall, seems to have made its way down to the south-west over the years.
The other two spoligotypes seem to be rather constant over the years.

\end{document}




\section{Conclusion}

This work presents an extension of the normal I-prior methodology to fit categorical response models using probit link functions---a methodology we call the I-probit.
The main motivation behind this work is to overcome the drawbacks of modelling probabilities using the normal I-prior model.
We assumed latent variables that represent `class propensities' exist, modelled these using a normal I-prior, and simply transformed them via a probit link function.
In this way, all of the advantages of the I-prior methodology seen for the normal model are preserved for binary and multinomial regression as well.

The core of this work explores ways in which to overcome the intractable integral presented by the I-probit model in \cref{eq:intractablelikelihood}.
Techniques such as quadrature methods, Laplace approximation and MCMC tend to fail, or are unsatisfactorily slow to accomplish.
The main reason for this is the dimension of this integral, which is $nm$, and thus for large sample sizes and/or number of classes, is unfeasible with such methods.
We turned to variational inference in the face of an intractable posterior density that hampers an EM algorithm, and the result is a sequential updating scheme, similar in time and storage requirements to the EM algorithm.

In terms of similarity to other works, the generalised additive models (GAMs) of \citet{hastie1986} comes close.
The setup of GAMs is near identical to the I-probit model, although estimation is done differently. 
GAMs do not assume smooth functions from any RKHS, but instead estimates the $f$'s using a local scoring method or a local likelihood method.
Kernel methods for classification are extremely popular in computer science and machine learning; examples include support vector machines \citep{scholkopf2002learning} and Gaussian process classification \citep{rasmussen2006gaussian}, with the latter being more closely related to the I-probit method.
However, Gaussian process classification typically uses the logistic sigmoid function, and estimation most commonly performed using Laplace approximation, but other methods such as expectation propagation \citep{minka2001expectation} and MCMC \citep{neal1999} have been explored as well.
Variational inference for Gaussian process probit models have been studied by \citet{girolami2006variational}, with their work providing a close reference to the variational algorithm employed by us.

Suggestions for future work include:
\begin{enumerate}
  \item \textbf{Estimation of $\bPsi$}. 
  A limitation we had to face in this work was to treat $\bPsi$ as fixed.
  The discussion in \cref{sec:difficultPsi} shows that estimation of $\bPsi$ is possible, however, the specific nature of implementing this in computer code could not be explored in time.
  In particular, for the full I-probit model, the best method of imposing positive-definite constraints for $\bPsi$ in the M-step has not been fully researched.
  
  \item \textbf{Inclusion of class-specific covariates}. 
  Throughout the chapter, we assumed that covariates were unit-specific, rather than class-specific. 
  One such example is modelling the choice of travel mode between two destinations (car, coach, train or aeroplane) as a function of travel time. 
  Clearly, travel time depends on the mode of transport. 
  This would require a careful rethink of the appropriate RKHS/RKKS to which the regression function belongs.
  The regression on the latent propensities could be extended as such:
  \[
    y_{ij}^* = \alpha_j + f_j(x_i) + e(z_{ij}) + \epsilon_{ij}
  \]
  and $f_j\in \cF_\cX$, the RKHS with kernel $h:(\cX \times \{1,\dots,m \})^2\to\bbR$ defined by $\delta_{jj'}h(x,x')$, and $e \in \cF_\cZ$, the RKHS of functions of the form $e:\{z_{ij}|i=1,\dots,n, \ j=1,\dots,m \}\to\bbR$.
%  An I-prior would then be applied as usual, but the implications on the estimation would need to be considered as well.
  
  \item \textbf{Improving computational efficiency}. 
  The $O(n^3m)$ time requirement for estimating I-probit models hinder its use towards large-data applications.
  In a limited study, we did not obtain reliable improvements using low-rank approximations of the kernel matrix such as the Nyström method.
  The key to improving computational efficiency could lie in sparse variational methods, a suggestion that was made to improve normal I-prior models as well.
%  \item \textbf{Efficient calculation of normal integrals}.
\end{enumerate}

\begin{figure}[hbt]
  \centering
  \includegraphics[width=0.9\textwidth]{figure/05-iprobit_runtime}
  \caption[Time taken to complete a single variational inference iteration]{Time taken to complete a single variational inference iteration for varying sample sizes and number of classes $m$. The solid line represents actual timings, while the dotted lines are linear extrapolations.}
\end{figure}

As a final remark, we note that variational Bayes, which entails a fully Bayesian treatment of the model (setting priors on model parameters $\theta$), is a viable alternative to variational EM.
The output of such a variational inference algorithm would be approximate posterior densities for $\theta$, in addition to $q(\by^*)$ and $q(\bw)$, instead of point estimates for $\theta$.
Posterior inferences surrounding the parameters would then be possible, such as obtaining posterior standard deviations, credibility intervals, and so on.
However, a variational Bayes route has its cons:
\begin{enumerate}
  \item \textbf{Tedious derivations}. As the parameters now have a distribution $\theta = \{\balpha,\eta,\bPsi\} \sim q(\balpha,\eta,\bPsi)$, quantities such as
  \begin{itemize}
    \item $\E[\log \vert \bPsi \vert]$;
    \item $\E[\bH_\eta^2]$; and
    \item $\tr \E[(\by^*-\bone_n\balpha^\top - \bH_\eta\bw)\bPsi (\by^*-\bone_n\balpha^\top - \bH_\eta\bw)^\top ]$,
  \end{itemize}
  among others, will need to be derived for the variational inference algorithm, and these can be tricky to compute.
  
  \item \textbf{Suited only to conjugate exponential family models}. When conjugate exponential family models are considered, the approximate variational densities (under a mean-field assumption) are easily recognised, as they themselves belong to the same exponential family as the model/prior. However, I-prior does not always admit conjugacy for the kernel parameters $\eta$ (only for ANOVA RKKSs scale parameters), and most certainly not for $\bPsi$ (at least not in the current parameterisation). When this happens, techniques such as importance sampling or Metropolis algorithms need to be employed to obtain the posterior means required for the variational algorithm to proceed.
  
  \item \textbf{Prior specification and sensitivity}. It is not clear how best to specify prior information (from a subjectivist's standpoint) for the RKHS scale parameters, intercepts, and perhaps the error precision, because these are parameters relating to the latent propensities which are not very meaningful or interpretable. Of course, one could easily specify vague or even diffuse priors. The concern is that the model could be sensitive to prior choices.
\end{enumerate}

In consideration of the above, we opted to employ a variational EM algorithm for estimation of I-probit models, instead of a full variational Bayes estimation.
In any case, computational complexity is expected to be the same between the two methods.
An interesting point to note is that the RKHS scale parameters and intercept would admit a normal posterior under a variational Bayes scheme. 
This means that the posterior mode and the posterior mean coincide, so point estimates under a variational EM algorithm are exactly the same as the posterior mean estimates under a variational Bayes framework when a diffuse prior is used.

\hClosingStuffStandalone
\end{document}