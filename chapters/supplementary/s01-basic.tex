\index{frequentist}
Statistics concerns what can be learned from data \citep{davison2003statistical}.
A statistical model comprises of a probabilistic component which drives the data generative process, in addition to a systematic or deterministic component, which sets it apart from pure mathematical models.
Real-valued observations $\by := \{y_1,\dots,y_n\}$ are treated as realisations from an assumed probability distribution with parameters $\theta = (\theta_1,\dots,\theta_p)^\top \in \Theta \subseteq \bbR^p$.
The crux of statistical inference is to estimate $\theta$ given the observed values, so that this optimised value may be used in the model to make deductions.
We describe the \emph{frequentist} and \emph{Bayesian} paradigms for parameter estimation.

\section{Maximum likelihood estimation}

In the frequentist setting, the \emph{likelihood} function, or simply likelihood, is a function of the parameters $\theta$ which measures the plausibility of the parameter value given the observed data to fit a statistical model.
It is defined as the mapping $\theta \mapsto p(\by|\theta)$, where $p(\by|\theta)$ is the probability density function (or in the case of discrete observations, the probability mass function) of the modelled distribution of the observations.
 
\index{maximum likelihood}
\index{likelihood}
\index{log-likelihood}
It is logical to consider the parameter which provides the largest likelihood value,
\begin{equation}\label{eq:ml}
  \hat\theta_{\text{ML}} = \argmax_\theta p(\by|\theta).
\end{equation}
The value $\hat\theta_{\text{ML}}$ is referred to as the \emph{maximum likelihood estimate} for $\theta$.
For convenience, the \emph{log-likelihood} function $L(\theta) = \log p(\by|\theta)$ is maximised instead; as the logarithm is a monotonically increasing function, the maximiser of the log-likelihood function is exactly the maximiser of the likelihood function itself.

\index{gradient!method}
When ML estimates are unable to be found in closed-form, the maximisation problem of \cref{eq:ml} requires iterative, numerical methods to find the maximum.
These methods are often \emph{gradient based}, i.e. algorithms that make use of the gradient of the objective function to be optimised.
Examples include Newton's method, Fisher's scoring, quasi-Newton methods, gradient descent, and conjugate gradient methods.
As the name suggests, these methods require evaluation of gradients or approximate gradients, and in some cases, the Hessian.
Depending on the situation, gradients or Hessians can be expensive or inconvenient to compute or approximate.
In cases of multi-modality of the objective function, the algorithms can potentially converge to a local optima, as it is known that the algorithms are quite sensitive to starting locations.

\index{efficiency}
\index{Cramer-Rao lower bound@Cramér-Rao lower bound}
Besides invariance, the ML estimate comes with the attractive limiting property $\surd{n}(\hat\theta_{\text{ML}} - \theta_{\text{true}}) \xrightarrow{\text{dist.}} \N_p\big(0,\cI(\theta)^{-1}\big)$ \citep{casella2002statistical} as sample size $n\to\infty$, where $\cI(\theta)$ is the Fisher information for $\theta$.
Other asymptotic properties of the ML estimate include consistency, i.e. $\Prob(\Vert \hat\theta_{\text{ML}} - \theta_{\text{true}}\Vert > \epsilon) \xrightarrow{\text{prob.}} 0$ for any $\epsilon>0$, and efficiency, i.e. it achieves the Cramér-Rao lower bound $\Var(\theta_{\text{ML}}) \geq \cI(\theta)^{-1}$.

\index{likelihood ratio test}
As the likelihood measures the plausibility of a parameter value given the data, it can be used to compare two competing models.
Let $\Theta_0 = \{\theta \,|\, \theta_{d+1} = \theta_{d+1,0},\dots,\theta_{p} = \theta_{p,0} \}$ be the set of parameters with restrictions on the last $d$ components of $\theta$.
The \emph{likelihood ratio test} statistic for testing the null hypothesis $\text{H}_0: \theta \in \Theta_0$ against the alternative $\text{H}_1: \theta \notin \Theta_0$ is
\begin{equation}
  \lambda = -2 \log \frac{\sup_{\theta \in \Theta_0} L(\theta)}{\sup_{\theta \in \Theta} L(\theta)} = -2 \big( \log L(\hat\theta_0) - \log L(\hat\theta) \big),
\end{equation}
where $\hat\theta_0 = \argmax_{\theta \in \Theta_0} \log p(\by|\theta)$.
Wilks' theorem states that $\lambda$ has an asymptotic chi-squared distribution with degrees of freedom equal to the number of restrictions  imposed (or rather, the difference in dimensionality of $\Theta$ and $\Theta_0$).
This gives a convenient way of comparing nested models. 

As a remark, models with more parameters will always have higher, or similar, log-likelihood, than models with fewer parameters, because the model has a better ability to fit the data with more free parameters.
In a linear regression setting, this relates to overfitting: a linear model with as many explanatory variables as there are data points ($n=p$) will extrapolate every point in the data set.
Overfitting is an oft cited problem of maximum likelihood.

\section{Bayesian estimation}

\index{subjective Bayes}
\index{objective Bayes}
The \emph{Bayesian} approach to estimating $\theta$ takes a different outlook, in that it supplements what is already known from the data with additional information in the form of prior beliefs about the parameters.
This usually means treating the parameters as random, following some distribution dictated by a \emph{prior density} $p(\theta)$.
There are many ways of categorising different types of priors.
Broadly speaking, priors, and hence Bayesian analysis \citep{robert2007bayesian,kadane2011principles}, can be either \emph{subjective} or \emph{objective}, with the demonyms `subjectivists' and `objectivists' used to refer to those subscribing to each respective principle.
Subjectivists assert that probabilities are merely opinions, while objectivists, in contrast, view probabilities as an extension of logic.
In this regard, objective Bayes seek to minimise the statistician's contribution to inference and ``let data speak for itself'', while subjective Bayes does the opposite.

In either case, inference about the parameters are then performed using the \emph{posterior density}
\begin{equation}
  p(\theta|\by) \propto 
  \myoverbrace{p(\by|\theta)}{\text{likelihood}}
  \times
  \myoverbrace{p(\theta)}{\text{prior}},
\end{equation}
rather than through a single point estimate such as the ML estimate in the frequentist case.
The posterior density encapsulates the uncertainty surrounding the parameters $\theta$ after observing the data $\by$.
The \emph{posterior mean} 
\begin{equation}\label{eq:postmeanest}
  \tilde \theta = \int \theta p(\theta|\by) \dint\theta
\end{equation}
is normally taken to be the point estimate for $\theta$, with its uncertainty usually reported in the form of a \emph{credible interval}: if $\theta_k$ is the $k$'th component of $\theta$, then a $(1-\alpha) \times 100\%$ credible interval for $\theta_k$ is $(\theta_k^l, \theta_k^u)$, where  $\Prob( \theta_k^l \leq \theta_k \leq \theta_k^u ) = (1-\alpha) \times 100\%$.
Under a quadratic loss function, $\tilde\theta$ minimises the expected loss $\E[(\theta - \theta_{\text{true}})^2 ]$ \citep[Sec. 4.4.2, Result 3]{berger2013statistical}, and is hence also viewed as the \emph{minimum mean squared error} (MMSE) estimator.
\index{minimum mean squared error (MMSE) estimate}

\index{MCMC}
On a practical note, integration over the parameter space may be intractable, for instance, the model consists of a large number of parameters for which we would like the posterior mean of, or the marginalising integral cannot be found in closed form.
Markov chain Monte Carlo (MCMC) methods are the standard way of approximating such integrals, by way of random sampling from the posterior.
The sample $\{\theta^{(1)},\dots,\theta^{(T)} \}$ is then manipulated in a way to derive its approximation.
In the case of the posterior mean,
\begin{equation}
  \hat\E[\theta|\by] = \frac{1}{T}\sum_{i=1}^T \theta^{(t)}
\end{equation}
gives an approximation, and its $(1-\alpha) \times 100\%$ credible interval can be approximated using the lower $\alpha/2\times 100\%$ and upper $(1-\alpha/2)\times 100\%$ quantile of the sample.

\index{evidence|see{model evidence}}
\index{model evidence}
The normalising constant is the marginal likelihood over the distribution of the parameters, $p(\by) = \int p(\by|\theta) p(\theta) \dint \theta$.
The quantity $p(\by)$ is also known as the \emph{model evidence}, or simply, \emph{evidence}.
As its name suggests, model evidence is used as a measure of how much support there is for a particular model.
As such, it is used as a basis for model comparison.
Let $p(\by|M_0)$ and $p(\by|M_1)$ be the model evidence for two competing models $M_0$ and $M_1$ respectively.
Define the \emph{Bayes factor} for comparing model $M_0$ against an alternative model $M_1$ as\index{Bayes factor}
\begin{equation}
  \BF(M_0,M_1) = \frac{p(\by|M_0)}{p(\by|M_1)}.
\end{equation}
Values of $\BF(M_0,M_1) < 1$ would suggest that the data provides more evidence for model $M_1$ over $M_0$.

Note that the model evidence is free of $\theta$ because all of the parameters have been marginalised out, or put another way, considered in entirety and averaged over all possible values of $\theta$ drawn from its prior density.
Thus, model comparison using Bayes factors differs from the frequentist likelihood ratio comparison in that it does not depend on any one particular set of values for the parameters.

\section{Maximum a posteriori estimation}

\index{maximum a posteriori estimate|see{MAP estimate}}
\index{MAP estimate}
One may also find the value of $\theta$ which maximises the posterior,
\begin{equation}\label{eq:mapest}
  \hat\theta_{\text{MAP}} = \argmax_\theta p(\by|\theta)p(\theta),
\end{equation}
which is the mode of the posterior distribution.
This quantity is known as the \emph{maximum a posteriori} (MAP) estimate. 
It is different from the ML estimate in that the maximisation objective is augmented with with the prior density for $\theta$.
In this sense, MAP estimation can be seen as regularisation of the ML estimation procedure, whereby a ``penalty'' term is added to avoid overfitting.

MAP estimation is often criticised for not being representative of Bayesian methods.
That is, MAP estimation returns a point estimate with no apparent way of quantifying its uncertainty.
Furthermore, unlike ML estimators, MAP estimators are not invariant under reparameterisation.
If $\theta$ is a random variable with density $p(\theta)$, then the pdf of $\xi := g(\theta)$, where $g:\theta \mapsto g(\theta)$ is a one-to-one transformation, is
\begin{equation}\label{eq:pdftransform}
  p_\xi(\xi) = p_\theta\big(g^{-1}(\xi)\big) \left\vert \frac{\d }{\d \xi} g^{-1}(\xi) \right\vert.
\end{equation}
The second term in \cref{eq:pdftransform} is called the \emph{Jacobian (determinant)}.\index{Jacobian}
Therefore, a different parameterisation of $\theta$ will impact the location of the maximum because of the introduction of the Jacobian into the optimisation objective \cref{eq:mapest}.

\section{Empirical Bayes}

The term \emph{empirical Bayes} \citep{robbins1956empirical,casella1985introduction} refers to a procedure in which features of the prior is informed by the data.
This is realised by parameterising the prior by a hyperparameter $\eta$, i.e. $\theta \sim p(\theta|\eta)$.
%Consider a scenario in which the parameters to be estimated are the mean components of a multivariate normal distribution. 
%It is reasonable to assume that the means were drawn from an identical prior distribution, hyperparameterised by $\eta$.
%%That is, if $\theta = (\theta_1,\dots,\theta_p)^\top$, then $\theta_k|\eta \iid p(\theta_k|\eta)$ for each $k=1,\dots,p$.
%The posterior density for $\theta$ is now written
%\begin{equation}
%  p(\theta|\by) 
%  = \int p(\theta,\eta|\by)p(\eta|\by) \dint\eta
%  = \int \frac{p(\by|\theta)p(\theta|\eta)}{p(\by|\eta)}p(\eta|\by) \dint\eta.
%\end{equation}
%Assuming that the density $p(\eta|\by)$ is concentrated around a point estimate, $\hat\eta = \argmax_\eta p(\eta|\by)$ say, then we can assume that
%\begin{equation}
%  p(\theta|\by) \approx \frac{p(\by|\theta)p(\theta|\hat\eta)}{p(\by|\hat\eta)}.
%\end{equation}
%This approach of optimising the hyperparameters $\eta$ by maximising the 
Values for the hyperparameter are clearly important, because they appear in the posterior for $\theta$: 
\begin{equation}\label{eq:empbayes1}
  p(\theta|\by) = \frac{p(\by|\theta)p(\theta|\eta)}{p(\by|\eta)} 
\end{equation}
To avoid the subjectivist's approach of specifying values for $\eta$ a priori, one instead turns to the data for guidance.
Information concerning $\eta$ is contained in the marginal likelihood $p(\by|\eta) = \int p(\by|\theta)p(\theta|\eta) \dint \theta$.
This paves the way for using the \emph{maximum marginal likelihood} estimate
\begin{equation}
  \hat\eta = \argmax_\eta p(\by|\eta) 
\end{equation}
in place of $\eta$ in the equation of \cref{eq:empbayes1}.
This procedure is also coined \emph{maximum likelihood type-II} \citep{bishop2006pattern}, and is commonly referred to as such in the machine learning literature. \index{maximum likelihood!Type-II}
It is also commonplace in statistics, especially in random-effects or latent variable models which employ a maximum likelihood procedure such as EM algorithm.

As a remark, estimation of $\eta$ itself can be made to conform to Bayesian philosophy, i.e., by placing priors on it and inferring $\eta$ through its posterior.
Such a procedure is referred to as \emph{Bayesian hierarchical modelling}.
A motivation for doing this is because the ML estimate of $\eta$ ignores any uncertainty in it.
Of course, the hyperprior for $\eta$ could be parameterised by a hyper-hyperparameter, and itself have a prior, and so on and so forth.
Evidently the model is specified until such a point where there are parameters of the model which are left unoptimised and must be specified in subjective manner \citep{beal2003variational}.