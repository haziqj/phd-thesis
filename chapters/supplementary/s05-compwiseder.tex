

Let $\cX$ be a topological vector space, and $g:\cX\to\bbR$ a real-valued function over $\cX$.
Also suppose that $\{e_1,e_2,\dots\}$ is an (algebraic) orthonormal basis for $\cX$, such that any $x\in\cX$ can be represented as $x = x_1e_1 + x_2e_2 + \cdots = (x_1,x_2,\dots)$.
We define the partial derivative of $g$ with respect to the $k$'th component of $x$ as
\[
  \frac{\partial g(x)}{\partial x_k}  = \lim_{t\to 0} \frac{g(x_1,\dots,x_k+t,\dots) - g(x_1,x_2,\dots)}{t}.
\]
Incidentally, this definition also coincides with the Gâteaux derivative of $g$ at $x$ in the direction $e_k$, i.e., $\frac{\partial g(x)}{\partial x_k} = \partial_{e_k}g(x)$.
Denote by 
\[
  \nabla g(x) = \left(
  \frac{\partial g(x)}{\partial x_1}, 
  \frac{\partial g(x)}{\partial x_2},
  \dots
  \right)
\]
the total differential, or as it is more commonly known, the gradient of $g$ at $x$.
This is, in essence, a component-wise derivative of $g$ using the usual definition of derivatives.
The issue here is that the existence of the limits for each component of the differential does not guarantee existence of the gradient.
This is because component-wise convergence does not guarantee convergence with respect to the actual topology of the  space.

\begin{example}[Existence of partial derivatives does not imply differentiability]
Define $g:\bbR^2\to\bbR$ by
\[
  g(x) = 
  \begin{cases}
    \displaystyle\frac{x_1x_2}{x_1^2 + x_2^2} &\text{if } (x_1,x_2) \neq (0,0) \\
    0 &\text{if } x_1=x_2=0.
  \end{cases}
\]
We note that $\lim_{x\to 0} g(x) = \half \neq 0 = g(0,0)$ along the line $x_1 = x_2$.
Therefore, this function is discontinuous, and hence non-differentiable, at zero.
On the other hand, the partial derivatives of $g$ are 
\[
  \frac{\partial g(x)}{\partial x_1} = \frac{x_2(x_2^2-x_1^2)}{(x_1^2+x_2^2)^2}
  \hspace{0.5cm}\text{and}\hspace{0.5cm}
  \frac{\partial g(x)}{\partial x_2} = \frac{x_1(x_1^2-x_2^2)}{(x_1^2+x_2^2)^2},
\]
and thus along the $x_1$-axis, $\frac{\partial g(x_1,0)}{\partial x_1} = 0$, and similarly along the $x_2$-axis, $\frac{\partial g(0,x_2)}{\partial x_2} = 0$.
Existence of partial derivatives is not sufficient in this case for the existence of the gradient.

\end{example}
