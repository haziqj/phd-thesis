Computational complexity for estimating I-prior models (and in fact, for GPR in general) is dominated by the inversion of the $n \times n$ matrix $\bSigma_\theta = \psi\bH_\eta^2 + \psi^{-1}\bI_n$, which scales as $O(n^3)$ in time.
As mentioned earlier, the \pkg{iprior} package inverts this by way of the eigendecomposition of $\bH_\eta$, but this operation is also $O(n^3)$.
For the direct optimisation method, this matrix inversion is called when computing the log-likelihood, and thus must be computed at each Newton step.
For the EM algorithm, this matrix inversion appears when calculating $\tilde \bw$, the posterior mean of the I-prior random effects.
Furthermore, storage requirements for I-priors models are similar to that of GPR models, which is $O(n^2)$.

\subsection[The Nystrom approximation]{The Nyström approximation}

The machine learning literature is rich in ways to resolve this issue, as summarised by \cite{quinonero2005unifying}.
One such method is to use low-rank matrix approximations.
Let $\bQ$ be a matrix with rank $q < n$, and that $\bQ\bQ^\top$ can be used to approximate the kernel matrix $\bH_\eta$.
Then
%
\[
  (\psi\bH_\eta^2 + \psi^{-1}\bI_n)^{-1} \approx
  \psi\left[
  \bI_n -
  \bQ\left( \big(\psi^2\bQ^\top\bQ\big)^{-1} +\bQ^\top\bQ \right)^{-1} \bQ^\top
  \right],
\]
%
obtained via the Woodbury matrix identity, is a potentially much cheaper operation which scales $O(nq^2)$---$O(q^3)$ to do the inversion, and $O(nq)$ to do the multiplication (because typically the inverse is premultiplied to a vector).
When the kernel matrix itself is sufficiently low ranked (for instance, when using the linear kernel for a low-dimensional covariate) then the above method is exact.
However, other interesting kernels such as the fractional Brownian motion (fBm) kernel or the squared exponential kernel results in kernel matrices which are full rank.

Another method of approximating the kernel matrix, and the method implemented by our package, is the Nystr\"om method \citep{williams2001using}.
The theory has its roots in approximating eigenfunctions, but this has since been adopted to speed up kernel machines.
The main idea is to obtain an (approximation to the true) eigendecomposition of $\bH_\eta$ based on a small subset $m \ll n$ of the data points.
Reorder the rows and columns and partition the kernel matrix as
%
\[
  \bH_\eta =
  \begin{pmatrix}
    \bA_{m\times m}         & \bB_{m \times (n-m)} \\
    \bB_{m \times (n-m)}^\top  & \bC_{(n-m) \times (n-m)} \\
  \end{pmatrix}.
\]
%
The Nystr\"om method provides an approximation to the lower right block $\bC$ by manipulating the eigenvectors and eigenvalues of $\bA$, an $m \times m$ matrix, together with the matrix $\bB$ to give
%
\[
  \bH_\eta \approx
  \begin{pmatrix}
    \bV_m \\
    \bB^\top\bV_m\bU_m^{-1}
  \end{pmatrix}
  \bU_m
  \begin{pmatrix}
    \bV_m^\top & \bU_m^{-1}\bV_m^\top\bB
  \end{pmatrix}
\]
%
where $\bU_m$ is the diagonal matrix containing the $m$ eigenvalues of $\bA$, and $\bV_m$ is the corresponding matrix of eigenvectors.
An orthogonal version of this approximation is of interest, which has been studied by \cite{fowlkes2001efficient}, which allows us to easily calculate the inverse of $\bSigma_\theta$.
Estimating I-prior models using the Nystr\"om method takes $O(nm^2)$ times and $O(nm)$ storage.

\subsection{An efficient EM algorithm}

As a sacrifice: store all kernel matrices beforehand.
When lambdas are in closed form... what is the computational time complexity?
Describe lambda in closed form.
What happens if lambda not in closed form.
psi always in closed form.

\subsection{Convex EM}

Small speed up to the EM algorithm.
Adaptive?
