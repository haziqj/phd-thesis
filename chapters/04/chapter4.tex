\documentclass[showframe,11pt,twoside,openright]{report}
\usepackage{standalone}
\standalonetrue
\ifstandalone
  \usepackage{../../haziq_thesis}
  \usepackage{../../haziq_maths}
  \usepackage{../../haziq_glossary}
  \usepackage{../../knitr}
  \addbibresource{../../bib/haziq.bib}
  \externaldocument{../01/.texpadtmp/chapter1}
  \externaldocument{../02/.texpadtmp/chapter2}  
  \externaldocument{../03/.texpadtmp/chapter3}
%  \externaldocument{../04/.texpadtmp/chapter4}
  \externaldocument{../05/.texpadtmp/chapter5}
  \externaldocument{../06/.texpadtmp/chapter6}
  \externaldocument{../07/.texpadtmp/chapter7}
  \externaldocument{../appendix/.texpadtmp/appendix}  
\fi

\begin{document}
\hChapterStandalone[4]{Regression with I-priors}
\label{chapter4}
\thispagestyle{chapterfour}

In the previous chapter, we defined an I-prior for the normal regression model \cref{eq:model1} subject to \cref{eq:model1ass} and $f$ belonging to a reproducing kernel Hilbert or Kreĭn space of functions $\cF$, as a Gaussian distribution on $f$ with covariance function proportional to the Fisher information for $f$.
We also saw how new function spaces can be constructed via the polynomial and ANOVA reproducing kernel Kreĭn spaces (RKKSs).
In this chapter, we shall describe various regression models, and identify the regression function in each of these models as belonging to an appropriate RKKS, so that an I-prior may be defined.

Methods for estimating I-prior models are described in \cref{sec:ipriorestimation}.
Estimation here refers to obtaining the posterior distribution of the regression function under an I-prior, while optimising the kernel parameters of $\cF$ and the error precision $\bPsi$.
Likelihood based methods, namely direct optimisation of the likelihood and the expectation-maximisation (EM) algorithm, are the preferred estimation methods of choice.
Having said this, it is also possible to estimate I-prior models under a full Bayesian paradigm by employing Markov chain Monte Carlo methods to sample from the relevant posterior densities.
Once estimation is completed, post-estimation procedures such as inference and prediction for a new data point can be done.
This is described in \cref{sec:ipriorpostest}.

Careful considerations of the computational aspects are required to ensure efficient estimation of I-prior models, and these are discussed in \cref{sec:ipriorcompcons}.
The culmination of the computational work on I-prior estimation is the \pkg{iprior} package \citep{jamil2017iprior}, which is a publicly available \proglang{R} package that has been published to the Comprehensive \proglang{R} Archive Network (CRAN).

Finally, several examples of I-prior modelling are presented in  \cref{sec:ipriorexamples}: in particular, a multilevel data set, a longitudinal data set, and a data set involving a functional covariate, are analysed using the I-prior methodology.
Code for replication is available at \url{http://myphdcode.haziqj.ml}.

\section{Various regression models}
\label{sec:various-regression}
In the introductory chapter (Section 1.1), we described several interesting regression models.
The goal of this section is to formulate the I-prior model that describes each of these models.
This is done by carefully choosing the RKHS/RKKS $\cF$ of real functions over a set $\cX$ to which the regression function $f$ belongs.
Without loss of generality and for simplicity, assume a prior mean of zero for the I-prior distribution.

\subsection{Multiple linear regression}

Let $\cX\equiv \bbR^p$ be equipped with the regular Euclidean dot product, and $\cF_\lambda$ be the scaled canonical RKHS of functions over $\cX$ with kernel $h_\lambda(\bx,\bx') = \lambda \bx^\top\bx'$, for any two $\bx,\bx'\in\bbR^p$. 
Then, an I-prior on $f$ implies that 
\begin{align*}
  f(\bx_i) &= \sum_{j=1}^n \lambda \bx_i^\top\bx_j w_j \\
  &= \sum_{j=1}^n \lambda \left( \sum_{k=1}^p x_{ik}x_{jk} \right) w_j \\
  &= \beta_1 x_{i1} + \dots + \beta_p x_{ip},
\end{align*}
where each $\beta_k := \lambda \sum_{j=1}^n  x_{jk}w_j$.
This implies a multivariate normal prior distribution for the regression coefficients   
\begin{align}\label{eq:ipriorcanonical}
  \boldsymbol\beta := (\beta_1,\dots,\beta_p) \sim \N_p(\bzero, \lambda^2 \bX^\top \bPsi \bX),
\end{align}
where $\bX$ is the $n \times p$ design matrix for the covariates, excluding the column of ones at the beginning typically reserved for the intercept. 
As expected, the covariance matrix for $\boldsymbol\beta$ is recognised as the scaled Fisher information matrix for the regression coefficients.

%Functions in the canonical RKHS are of the form $f(\bx) = \bx^\top\boldsymbol\beta$.
If the covariates are not scaled similarly, then the values of $f$ are incoherent---if $x_1$  measures weight in kilograms and $x_2$ height in centimetres, what measurement does $\beta_1x_1 + \beta_2x_2$ represent?
To overcome this, one could decompose the regression function into
\[
  f(\bx_i) = f_1(x_{i1}) + \cdots + f_p(x_{ip}) 
\]
for which $f \in \cF_\lambda \equiv \cF_{\lambda_1}\oplus\cdots\oplus\cF_{\lambda_p}$, and $\cF_{\lambda_k}$, $k=1,\dots,p$ are unidimensional canonical RKHSs with kernels $h_{\lambda_k}(x_{ik},x_{jk}) = \lambda_k x_{ik} x_{jk}$.
%, where $\bar x_k = \sum_{i=1}^n x_{ik}/n$.
In effect, we now have $p$ scale parameters, one for each of the RKKSs associated with the $p$ covariates.
%Denote $\tilde x_{ik} = x_{ik}-\bar x_k$, the centred covariates.
The RKKS $\cF_\lambda$ therefore has kernel
\[
  h(\bx_i,\bx_j) = \sum_{k=1}^p \lambda_k  x_{ik} x_{jk},
\]
and hence each regression coefficient can now be written as $\beta_k =  \sum_{j=1}^n  \lambda_k x_{jk}w_j$.
Thus, the corresponding I-prior for $\boldsymbol{\beta}$ is
\[
  \boldsymbol{\beta} \sim \N_p(\bzero, \lambda^2 \bX^\top\bLambda \bPsi \bLambda \bX),
\]
with $\bLambda = \diag(\lambda_1,\dots,\lambda_p)$.
%Note that the overall effect $\beta_0$ can be treated in a number of ways, the simplest of which is as an intercept to be estimated.
Note that $\cF_\lambda$ can be seen as a special case of the ANOVA RKKS, in which only the main effects are considered, in which case the \emph{centred canonical RKHSs} should be considered instead.
This approach is disadvantageous when $p$ is large, in which case there would be numerous scale parameters to estimate.

\begin{remark}
  The I-prior for $\boldsymbol{\beta}$ in \eqref{eq:ipriorcanonical} bears resemblance to the $g$-prior \citep{zellner1986assessing}, and in fact, the $g$-prior can be interpreted as an I-prior if the inner product of $\cX$ is the Mahalonobis inner product.
  See Miscellanea \ref{misc:gprior} for a discussion.
\end{remark}

\subsection{Multilevel linear modelling}
\label{sec:multilevelmodels}

Let $\cX\equiv\bbR^p$, and suppose that alongside the covariates, there is information on group levels $\cM= \{1,\dots,m\}$ for each unit $i$.
That is, every observation for unit $i$ is known to belong to a specific group $j$, and we write $\bx_i^{(j)}$ to indicate this.
Let $n_j$ denote the sample size for cluster $j$, and the overall sample size be $n = \sum_{j=1}^m n_j$.
When modelled linearly with the responses $y_i^{(j)}$, the model is known as a multilevel (linear) model, although it is known by many other names: random-effects models, random coefficient models, hierarchical models, and so on.
As this model is seen as an extension of linear models, applications are plenty, especially in research designs for which the data varies at more than one level.

Consider a functional ANOVA decomposition of the regression function as follows:
\begin{align}\label{eq:anovamultilevel}
  f(\bx_i^{(j)}, j) = \alpha + f_1(\bx_i^{(j)}) + f_2(j) + f_{12}(\bx_i^{(j)}, j).  
\end{align}
To mimic the multilevel model, assume $f_1\in\cF_1$ the Pearson RKHS, $f_2\in\cF_2$ the centred canonical RKHS, and $f_{12} \in \cF_{12} = \cF_1 \otimes \cF_2$, the tensor product space of $\cF_1$ and $\cF_2$.
As we know, $\alpha$ is the overall intercept, and the varying intercepts are given by the function $f_2$.
While $f_1$ is the (main) linear effect of the covariates, $f_{12}$ provides the varying linear effect of the covariates by each group.
The I-prior for $f-\alpha$ is assumed to lie in the function space $\cF-\alpha$, which is an ANOVA RKKS with kernel
\[
  h_\lambda\big((\bx_i^{(j)},j),(\bx_i^{(j')},j')\big) = \lambda_1 h_1(\bx_i^{(j)},\bx_{i'}^{(j')}) + \lambda_2 h_2(j,j') + \lambda_1\lambda_2 h_1(\bx_i^{(j)},\bx_{i'}^{(j')})h_2(j,j'),
\]
with $h_1$ the centred canonical kernel and $h_2$ the Pearson kernel.
The reason for not including an RKHS of constant functions in $\cF$ is because the overall intercept is usually simpler to estimate as an external parameter (see \hltodo{Section X}).

We can show that the regression function \eqref{eq:anovamultilevel} corresponds to the standard way of writing the multilevel model, 
\[
  f(\bx_i^{(j)}, j) = \beta_0 + \bx_i^{(j)\top}\boldsymbol{\beta}_1 + \beta_{0j} + \bx_i^{(j)\top}\boldsymbol{\beta}_{1j}.   
\]
and determine the prior distributions on $(\beta_{0j},\boldsymbol{\beta}_{1j}^\top)^\top \in \bbR^{p+1}$.
For the interested reader, the details are in Miscellanea \ref{misc:multilevelmodels}.
The standard multilevel random effects assumption is that $(\beta_{0j},\boldsymbol{\beta}_{1j}^\top)^\top$ is normally distributed with mean zero and covariance matrix $\bPhi$.
In total, there are $p+1$ regression coefficients and $(p+1)(p+2)/2$ covariance parameters in $\Phi$ to be estimated.
In contrast, the I-prior model is parameterised by only two RKKS scale parameters---one for $\cF_1$ and one for $\cF_2$---and the error precision $\psi$.
While the estimation procedure for $\bPhi$ in the standard multilevel model can result in non-positive covariance matrices, the I-prior model has the advantage that positive definiteness is taken care of automatically\footnote{By virtue of the estimate of the regression function belonging to $\cF_n$, an RKHS with a positive definite kernel equal to the Fisher information for $f$.}.
%This is seen from the calculations for $\Var \beta_{0j}$, $\Var \boldsymbol\beta_{1j}$ and the respective covariances.
%An example of multilevel modelling using I-priors is given in \hltodo{Section 4.3.1}.

As a remark, the following regression functions are nested 
\begin{itemize}
  \item $f(\bx_i^{(j)}, j) = f_0 + f_1(\bx_i^{(j)}) + f_2(j)$ (random intercept model);  %\cF_0 \oplus \lambda_1\cF_1 \oplus \lambda_2\cF_2
  \item $f(\bx_i^{(j)}, j) = f_0 + f_1(\bx_i^{(j)})$ (linear regression model);
  \item $f(\bx_i^{(j)}, j) = f_0 + f_2(j)$ (ANOVA model);
  \item $f(\bx_i^{(j)}, j) = f_0 $ (intercept only model),
\end{itemize}
and thus one may compare likelihoods to ascertain the best fitting model.
In addition, one may add flexibility to the model in two possible ways:
\begin{enumerate}
  \item \textbf{More than two levels}. The model can be easily adjusted to reflect the fact that that the data is structured in a hierarchy containing three or more levels. For the three level case, let the indices $j\in\{1,\dots,m_1\}$ and $k\in\{1,\dots,m_2\}$ denote the two levels, and simply decompose the regression function accordingly:
  \begin{align*}
    \begin{gathered}
      f(\bx_i^{(j,k)}, j, k) = f_0 + f_1(\bx_i^{(j,k)}) + f_2(j) + f_3(k) + f_{12}(\bx_i^{(j,k)}, j) + f_{13}(\bx_i^{(j,k)}, k)\\ 
      + f_{23}(j, k) + f_{123}(\bx_i^{(j,k)}, j, k).
    \end{gathered}
  \end{align*}
  \item \textbf{Covariates not varying with levels}. Suppose now we would like to add covariates with a fixed effect to the model, i.e., covariates $\bz_i^{(j)}$ which are not assumed to affect the responses differently in each group. The regression function would be:
  \begin{align*}
    \begin{gathered}
      f(\bx_i^{(j)}, j, \bz_j) = f_0 + f_1(\bx_i^{(j)}) + f_2(j) + f_3(\bz_i^{(j)}) + f_{12}(\bx_i^{(j)}, j).
    \end{gathered}
  \end{align*}
  This can be seen as a limited functional ANOVA decomposition of $f$.
\end{enumerate}

\begin{remark}
  \setlength{\parindent}{0em}
  Indexing can be tricky, but we find the following helpful.
  Supposing $m=2$, and $n_1 = n_2 = 3$, then a typical panel data set looks like this:
  \begin{table}[H]
  \centering
  \begin{tabular}{lllrrr}
  \hline
  $y$ & $x$  & $z$ & $i$ & $j$ & $k$ \\ 
  \hline
  $y_{11}$      & $x_{11}$    & $z_{1}$ &1&1& 1 \\
  $y_{21}$      & $x_{21}$    & $z_{1}$ &2&1& 2 \\
  $y_{31}$      & $x_{31}$    & $z_{1}$ &3&1& 3 \\
  $y_{12}$      & $x_{12}$    & $z_{2}$ &1&2& 4 \\
  $y_{22}$      & $x_{22}$    & $z_{2}$ &2&2& 5 \\
  $y_{32}$      & $x_{32}$    & $z_{2}$ &3&2& 6 \\  
  \hline
  \end{tabular}
  \end{table}
  \vspace{-1.5em}
  The $y$'s are the responses, $x$'s covariates, and $z$'s group-level covariates.
  If $\iota:(i,j)\mapsto k$ is a function which maps the dual index set $(i,j)$ to the single index set $k\in\{1,\dots,n\}$, then the multilevel regression function can be expressed as the regression function in model \eqref{eq:model1}.
\end{remark}


\subsection{Longitudinal modelling}

Longitudinal or panel data observes covariate measurements $x_i\in\cX$ and responses $y_i(t)\in\bbR$ for individuals $i=1,\dots,n$ across a time period $t \in \{1,\dots,T\} =: \cT$. 
Often, the time indexing set $\cT$ may be unique to each individual $i$, so measurements for unit $i$ happens across a time period $\{t_{i1},\dots,t_{iT_i} \} =: \cT_i$---this is known as an unbalanced panel.
It is also possible that covariate measurements vary across time too, so appropriately they are denoted $x_i(t)$.
For example, $x_i(t)$ could be repeated measurements of the variable $x_i$ at time point $t\in\cT_i$.
The relationship between the response variables $y_i(t)$ at time $t\in\cT_i$ is captured through the equation
\[
  y_i(t) = f\big(x_i, t \big) + \epsilon_{i}(t)
\]
where the distribution of $\bepsilon_i = \big(\epsilon_i(t_{i1}),\dots,\epsilon_i(t_{iT_i}) \big)^\top$ is Gaussian with mean zero and covariance matrix $\bPsi_i$.
Assuming $\bPsi_i=\psi_i\bI_{T_i}$ or even $\bPsi_i=\psi\bI_{T_i}$ are perfectly valid choices, even though this seemingly ignores any time dependence between the observations.
In reality, the I-prior induces time dependence of the observations via the kernels in the prior covariance matrix for $f$.
Additionally, the random vectors $\bepsilon_i$ and $\bepsilon_{i'}$ are assumed to be independent for any two distinct $i,i'\in\{1,\dots,n\}$.
%, in which case, $y_i(t)$ are viewed as multidimensional responses.

Using the functional ANOVA decomposition on the regression function, we obtain
\begin{align}\label{eq:longitudinalanova}
  f(x_i,t) = f_0 + f_1(x_i) + f_2(t) + f_{12}(x_i,t),
\end{align}
where $f_0$ is an overall constant, $f_1\in\cF_1$, $f_2\in\cF_2$, and $f_{12}\in\cF_1\otimes\cF_2$.
Choices for $\cF_1$ and $\cF_2$ are plentiful.
In fact, any of the RKHS/RKKS described in Chapter 3 can be used to either model a linear dependence (canonical RKHS), nominal dependence (Pearson RKHS), polynomial dependence (polynomial RKKS) or smooth dependence (fBm or SE RKHS) on the $x_i$'s and $t$'s on $f$.

\begin{remark}
  Although \eqref{eq:longitudinalanova} is a special case of the multilevel model decomposition \eqref{eq:anovamultilevel} for which $x_i = x_i(t)$ (time-varying covariates), it is different to how longitudinal models are normally treated using a mixed effects model.
  As a multilevel model, longitudinal models treat the individuals as the groups or clusters (level two), and the time points as the various measurements within the clusters (level one).
\end{remark}

\subsection{Smoothing models}

Single- and multi-variable smoothing models can be fitted under the I-prior methodology using the fBm RKHS.
In standard kernel based smoothing methods, the squared exponential kernel is often used, and the corresponding RKHS contains analytic functions.
There are several attractive properties of using the fBm RKHS, and for one-dimensional smoothing, these are discussed below.

Assume that, up to a constant, the regression function lies in the scaled, centred fBm RKHS $\cF$ of functions over $\cX \equiv \bbR$ with Hurst index $1/2$.
Thus, with a centring with respect to the empirical distribution $\Prob_n$ of $\{x_1,\dots,x_n\}$ and using the absolute norm on $\bbR$, $\cF$ has kernel
\[
  h_\lambda(x,x') = \frac{\lambda}{2n^2} \sum_{i=1}^n\sum_{j=1}^n \big( \abs{x-x_i} + \abs{x'-x_j} - \abs{x-x'} - \abs{x_i-x_j} \big).
\]
According to \citet[Section 10]{van2008reproducing}, $\cF$ contains absolutely continuous functions possessing a square integrable weak derivative satisfying $f(0)=0$.
The norm is given by $\norm{f}^2_\cF = \int \dot f^2 \d x$.
The posterior mean of $f$ based on an I-prior is then a (one-dimensional) smoother for the data.
For $f$ of the form $f = \sum_{i=1}^n h(\cdot,x_i)w_i$, i.e., $f\in\cF_n$, the finite subspace of $\cF$ as in \hltodo{Section}, then \citet{bergsma2017} shows that $f$ can be represented as 
\begin{align}\label{eq:ipriorbrownianbridge}
  f(x) = \int_{-\infty}^x \beta(t) \dint t
\end{align}
where 
\begin{align}
  \beta(t) = \sum_{i:x_i \leq t} w_i =  \frac{f(x_{i_t + 1}) - f(x_{i_t})}{x_{i_t + 1} - x_{i_t}}
\end{align}
with $i_t = \max_{x_i \leq t} i$.
Under the I-prior with an iid assumption on the errors, the $w_i$'s are zero mean normal random variables with variance $\psi$, so that $\beta$ as defined above is an ordinary Brownian bridge with respect to the empirical distribution $\Prob_n$.
The I-prior for $f$ is piecewise linear with knots at $x_1,\dots,x_n$, and the same holds true for the posterior mean.
The implication is that the I-prior automatically adapts to irregularly spaced $x_i$: in any region where there are no observations, the resulting smoother is linear.
This is explained by the reduced Fisher information about the derivative of the regression curve in regions with no observation.

In \citep{bergsma2017}, it is stated that the covariance function for $\beta$ is 
\[
  \Cov\big(\beta(x),\beta(x') \big) = n\big( \min\{\Prob_n(X < x), \Prob_n(X_n < x') \} -  \Prob_n(X < x) \Prob_n(X_n < x') \big)
\]
From this, notice that $\Var \beta(x) = \Prob_n(X_n < x)\big(1 - \Prob_n(X_n < x) \big)$, which shows an automatic boundary correction: close to the boundary there is little Fisher information on the derivative of the regression function $\beta(x)$, so the prior variance is small.
This will lead to more shrinkage of the posterior derivative of $f$ towards the derivative of the prior mean $f_0$.

%\[
%  \norm{f}^2_\cF = \sum_{i=1}^{n-1} \frac{\big( f(x_{i+1}) - f(x_i) \big)^2}{x_{i+1} - x_i}
%\]
%because $\cF_n$ is the set of functions which integrate to zero and are piecewise linear with knots at $x_1,\dots,x_n$.
%Assuming $x_1\leq x_2\leq \cdots \leq x_n$, then it can be shown that
%\[
%  w_1 = \frac{f(x_2) - f(x_1)}{x_2 - x_1}
%\]

%With $w_k\iid\N(0,\psi)$, functions in $\cF$ are of the form
%\begin{align*}
%  f(x) 
%  &= \sum_{k=1}^n h(x,x_k)w_k \\
%  &= \sum_{k=1}^n \left( \frac{\lambda}{2n^2} \sum_{i=1}^n\sum_{j=1}^n \big( \abs{x-x_i} + \abs{x_k-x_j} - \abs{x-x_k} - \abs{x_i-x_j} \big) \right) w_k.
%\end{align*}
%The derivative of $f(x)$ with respect to $x$ is
%\begin{align*}
%  \frac{\d }{\d x}f(x) 
%  &= \frac{\lambda}{2n^2} \sum_{k=1}^n  \sum_{i=1}^n\sum_{j=1}^n \left( \frac{\d}{\d x}\abs{x-x_i}  - \frac{\d}{\d x}\abs{x-x_k}  \right) w_k \\
%  &= \frac{\lambda}{2n^2} \sum_{k=1}^n  \sum_{i=1}^n\sum_{j=1}^n \left( \sign(x-x_i)  - \sign(x-x_k)  \right) w_k \\
%  &= \frac{\lambda}{2n} \sum_{k=1}^n  \sum_{i=1}^n           \left( \sign(x-x_i)  - \sign(x-x_k)  \right) w_k \\
%\end{align*}

Another advantage of the I-prior methodology is the ability to fit single or multidimensional smoothing models with just two parameters to be estimated: the RKHS scale parameter $\lambda$ and the error precision $\bPsi$.
The Hurst parameter $\gamma \in (0,1)$ of the fBm RKHS can also be treated as a free parameter for added flexibility, but for most practical applications, we find that the default setting of $\gamma = 1/2$ performs sufficiently well.

\begin{remark}
  From \eqref{eq:ipriorbrownianbridge}, the prior process for $f$ is thus an integrated Brownian bridge. 
  This shows a close relation with cubic spline smoothers, which can be interpreted as the posterior mean when the prior is an integrated Wiener process \citep{wahba1990spline}.
  Unlike I-priors however, cubic spline smoothers do not have automatic boundary corrections, and typically the additional assumption is made that the smoothing curve is linear at the boundary knots.
\end{remark}

\subsection{Regression with functional covariates}
\label{sec:regfunctionalcov}

Suppose that we have functional covariates $x$ in the real domain, and that $\cX$ is a set of differentiable functions.
If so, it is reasonable to assume that $\cX$ is a Hilbert-Sobolev space with inner product
%
\[
  \langle x,x' \rangle_\cX = \int \dot{x}(t) \dot{x}'(t) \dint t,
\]
%
so that we may apply the linear, fBm or any other kernels which make use of inner products by making use of the polarisation identity.
Furthermore, let $z \in \bbR^T$ be the discretised realisation of the function $x \in \cX$ at regular intervals $t = 1,\dots,T$. Then
%
\[
  \langle x,x' \rangle_\cX \approx \sum_{t=1}^{T-1} (z_{t+1} - z_t)(z'_{t+1} - z_t').
\]
%
For discretised observations at non-regular intervals $\{t_1,\dots,t_T\}$ then a more general formula to the above one might be used, for instance,
\[
  \langle x,x' \rangle_\cX \approx \sum_{i=1}^{T-1} \frac{(z_{t_{i+1}} - z_{t_i})(z'_{t_{i+1}} - z_{t_i}')}{t_{i+1} - t_{i}}.
\]

%\subsection{Structural equation modelling}
%
%Let $y_i = (y_{i1},\dots,y_{ip})$, $i=1,\dots,n$ and $j=1,\dots,p$.
%Here, $y_{ij}$ denotes the $j$th item measurement for the $i$th individual.
%In a one factor model, this measurement is assumed to rely on an `item intercept', and individual level `factors', plus an error term which depends on the items.
%Specifically, the one factor model is
%\begin{align}
%  y_{ij} = \mu(j) + f(i,j) + \epsilon_{ij}  
%\end{align}
%with $\epsilon_{ij}\sim\N(0,\psi_j)$.







\section{Estimation}
\label{sec:ipriorestimation}
After choosing an appropriate function space...
What are the kernel parameters? 
State the model $y = \alpha + \sum h w + e$.
Goal is to estimate posterior regression function.

After imposing an I-prior on the regression function, the interest is then to obtain the posterior distribution of the regression function.
This has been described in Chapter 1, and repeated here for convenience.
In particular, the posterior distribution for the regression function of the form $f(x) = \sum_{i=1}^n h(x,x_i)w_i$ merely depends on the posterior distribution of $  \bw := (w_1,\dots,w_n)^\top$, which is $  \bw|\by \sim \N_n(\wtilde, )$, where
\begin{align*}
  \begin{gathered}
    \wtilde = 
  \end{gathered}
\end{align*}

explain package


\subsection{The intercept}

Given the regression model \eqref{eq:model1} subject to an I-prior \eqref{eq:model1ass}, the marginal likelihood of the intercept $\alpha$ (after integrating out the I-prior) can be maximised with respect to $\alpha$, which yields the sample mean for $y$ as the ML estimate for intercept.

\subsection{Direct optimisation}

The kernel parameter $\eta$ and the error precision $\psi$ (which we collectively refer to as the model hyperparameters of the covariance kernel $\theta$) can be estimated in several ways.
One of these is direct optimisation of the marginal log-likelihood---the most common method in the Gaussian process literature.
%
\begin{align*}
  \log L(\theta)
  &= \log \int p(\by|\bff)p(\bff) \d \bff \\
  &= -\half[n]\log 2\pi - \half\log\vert \bSigma_\theta \vert - \half \by^\top \bSigma_\theta^{-1} \by
\end{align*}
%
where $\bSigma_\theta = \psi\bH_\eta^2 + \psi^{-1}\bI_n$.
This is typically done using conjugate gradients with a Cholesky decomposition on the covariance kernel to maintain stability, but the \pkg{iprior} package opts for an eigendecomposition of the kernel matrix (Gram matrix) $\bH_{\eta} = \bV\cdot\text{diag}(u_1,\dots,u_n)\cdot\bV^\top$ instead.
Since $\bH_{\eta}$ is a symmetrix matrix, we have that $\bV\bV^\top = \bI_n$, and thus
%
\[
  \bSigma_\theta = \bV \cdot \text{diag} (\psi u_1^2 + \psi^{-1},\dots,\psi u_n^2 + \psi^{-1}) \cdot \bV^\top
\]
%
for which the inverse and log-determinant is easily obtainable.
This method is relatively robust to numerical instabilities and is better at ensuring positive definiteness of the covariance kernel.
The eigendecomposition is performed using the \pkg{Eigen} \proglang{C++} template library and linked to \pkg{iprior} using \pkg{Rcpp} \citep{eddelbuettel2011rcpp}.
The hyperparameters are transformed by the \pkg{iprior} package so that an unrestricted optimisation using the quasi-Newton L-BFGS algorithm provided by \code{optim()} in \proglang R.
Note that minimisation is done on the deviance scale, i.e., minus twice the log-likelihood.
The direct optimisation method can be prone to local optima, in which case repeating the optimisation at different starting points and choosing the one which yields the highest likelihood is one way around this.

\subsection{Expectation-maximisation algorithm}

Alternatively, the expectation-maximisation (EM) algorithm may be used to estimate the hyperparameters, in which case the I-prior formulation in \eqref{eq:ipriorre} is convenient.
Substituting this into \eqref{eq:linmod} we get something that resembles a random effects model.
By treating the $w_i$ as ``missing'', the $t$th iteration of the E-step entails computing
%
\begin{align}
  Q(\theta) = \E \left[ \log p(\by, \bw | \theta) \big\vert \by,\theta^{(t)} \right].
\end{align}
%
As a consequence of the properties of the normal distribution, the required joint and posterior distributions $p(\by, \bw)$ and $p(\bw | \by)$ are easily obtained.
The M-step then maximises the $Q$ function above, which boils down to solving the first order conditions
%
\begin{align}
  \frac{\partial Q}{\partial\eta}
  &= -\half \tr \left(\frac{\partial \bSigma_\theta}{\partial\eta} \tilde\bW^{(t)} \right) + \psi \cdot \by ^\top \frac{\partial \bH_\eta}{\partial\eta} \tilde\bw^{(t)} \label{eq:emtheta} \\
  \frac{\partial Q}{\partial\psi}
  &= -\half \by^\top\by - \tr \left(\frac{\partial \bSigma_\theta}{\partial\psi} \tilde\bW^{(t)} \right) + \by^\top \bH_\eta \tilde\bw^{(t)} \label{eq:empsi}
\end{align}
%
equated to zero.
Here, $\tilde\bw$ and $\tilde\bW$ are the first and second posterior moments of $\bw$.
The solution to \eqref{eq:empsi} can be found in closed-form, but not necessarily for \eqref{eq:emtheta}.
In cases where closed-form solutions exist, then it is just a matter of iterating the update equations until a suitable convergence criterion is met (e.g. no more sizeable increase in successive log-likelihood values).
In cases where closed-form solutions do not exist for $\theta$, the $Q$ function is again optimised with respect to $\theta$ using the L-BFGS algorithm.

The EM algorithm is more stable than direct maximization, and is especially suitable if there are many scale parameters. However, it is typically slow to converge.
The \pkg{iprior} package provides a method to automatically switch to the direct optimisation method after running several EM iterations.
This then combines the stability of the EM with the speed of direct optimisation.

\subsection{Markov chain Monte Carlo methods}

For completeness, it should be mentioned that a full Bayesian treatment of the model is possible, with additional priors on the hyperparameters set.
Markov chain Monte Carlo (MCMC) methods can then be employed to sample from the posteriors of the hyperparameters, with point estimates obtained using the posterior mean or mode, for instance.
Additionally, the posterior distribution encapsulates the uncertainty about the parameter, for which inference can be made.
Posterior sampling can be done using Gibbs-based methods in \pkg{WinBUGS} \citep{lunn2000winbugs} or \pkg{JAGS} \citep{plummer2003jags}, and both have interfaces to \proglang{R} via \pkg{R2WinBUGS} \citep{sturtz2005r2winbugs} and \pkg{runjags} \citep{denwood2016runjags} respectively.
Hamiltonian Monte Carlo (HMC) sampling is also a possibility, and the \proglang{Stan} project \citep{carpenter2016stan} together with the package \pkg{rstan} \citep{rstan}  makes this possible in \proglang{R}.
All of these MCMC packages require the user to code the model individually, and we are not aware of the existence of MCMC-based packages which are able to estimate GPR models.
This makes it inconvenient for GPR and I-prior models, because in addition to the model itself, the kernel functions need to be coded as well and ensuring computational efficiency would be a difficult task.
Note that this full Bayesian method is not implemented in \pkg{iprior}, but described here for completeness.

\subsection{Comparison of estimation methods}

Running example: smoothing in one dimension.
Data simulated, what are the true parameters?
Run three methods of estimation, compare solutions, bias, MSE of prediction.


\section{Computational considerations and implementation}
\label{sec:ipriorcompcons}
Computational complexity for estimating I-prior models (and in fact, for GPR in general) is dominated by the inversion (by way of eigendecomposition in our case) of the $n \times n$ matrix $\bSigma_\theta = \bH_\eta\bPsi\bH_\eta + \bPsi^{-1}$, which scales as $O(n^3)$ in time.
%Inversion by way of the eigendecomposition of $\bH_\eta$ is $O(n^3)$.
For the direct optimisation method, this matrix inversion is called when computing the log-likelihood, and thus must be computed at each Newton step.
For the EM algorithm, this matrix inversion appears when calculating $\tilde \bw$ and $\tilde \bW$, the first and second posterior moments of the I-prior random effects.
Furthermore, storage requirements for I-priors models are similar to that of GPR models, which is $O(n^2)$.
In what follows, assumptions \ref{ass:A1}--\ref{ass:A3} hold.

\subsection[The Nystrom approximation]{The Nyström approximation}

The shared computational issues of I-prior and GPR models allow us to delve into machine learning literature, which is rich in ways to resolve these issue, as summarised by \citet{quinonero2005unifying}.
One such method is to exploit low rank structures of kernel matrices.
The idea is as follows.
Let $\bQ$ be a matrix with rank $q < n$, and suppose that $\bQ\bQ^\top$ can be used sufficiently well to represent the kernel matrix $\bH_\eta$.
Then
%
\[
  (\psi\bH_\eta^2 + \psi^{-1}\bI_n)^{-1} \approx
  \psi\left[
  \bI_n -
  \bQ\left( \big(\psi^2\bQ^\top\bQ\big)^{-1} +\bQ^\top\bQ \right)^{-1} \bQ^\top
  \right],
\]
%
obtained via the Woodbury matrix identity, is potentially a much cheaper operation which scales $O(nq^2)$: $O(q^3)$ to do the inversion, and $O(nq)$ to do the multiplication (because typically the inverse is premultiplied to a vector).
When using the linear kernel for a low-dimensional covariate then the above method is exact.
This fact is clearly demonstrated by the equivalence of the $p$-dimensional linear model implied by \eqref{eq:ipriorcanonical} with the $n$-dimensional I-prior model using the canonical RKHS.
If $p \ll n$ then certainly using the linear representation is much more efficient.

However, other interesting kernels such as the fractional Brownian motion (fBm) kernel or the squared exponential kernel results in kernel matrices which are full rank.
An approximation to the kernel matrix using a low-rank matrix is the Nystr\"om method \citep{williams2001using}.
The theory has its roots in approximating eigenfunctions, but this has since been adopted to speed up kernel machines.
The main idea is to obtain an (approximation to the true) eigendecomposition of $\bH_\eta$ based on a small subset $m \ll n$ of the data points.

Let $\bH_\eta = \bV\bU\bV^\top = \sum_{i=1}^n u_i \bv_i \bv_i^\top$ be the (orthogonal) decomposition of the symmetric matrix $\bH_\eta$.
As mentioned, avoiding this expensive $O(n^3)$ eigendecomposition is desired, and this is achieved by selecting a subset $\cM$ of size $m$ of the $n$ data points $\{1,\dots,n \}$, so that $\bH_\eta$ may be approximated using the rank $m$ matrix $\bH_\eta \approx \sum_{i\in\cM} \tilde u_i \tilde\bv_i\tilde\bv_i^\top$.
Without loss of generality, reorder the rows and columns of $\bH_\eta$ so that the data points indexed by $\cM$ are used first:
%
\[
  \bH_\eta =
  \begin{pmatrix}
    \bA_{m\times m}         & \bB_{m \times (n-m)} \\
    \bB_{m \times (n-m)}^\top  & \bC_{(n-m) \times (n-m)} \\
  \end{pmatrix}.
\]
%
In other words, the data points indexed by $\cM$ forms the smaller $m\times m$ kernel matrix $\bA$. 
Let $\bA = \bV_m\bU_m\bV_m^\top = \sum_{i=1}^m u_i^{(m)}\bv_i^{(m)}\bv_i^{(m)\top}$ be the eigendeceomposition of $\bA$.
The Nyström method provides the formulae for $\tilde u_i$ and $\tilde\bv_i$ \citep[§8.1, equations 8.2 and 8.3]{rasmussen2006gaussian} as
\begin{align*}
  \tilde u_i &:= \frac{n}{m} u_i^{(m)} \in \bbR \\
  \tilde \bv_i &:= \sqrt{\frac{m}{n}} \frac{1}{u_i^{(m)}}
  \begin{pmatrix}
    \bA & \bB
  \end{pmatrix}^\top
  \bv_i^{(m)} \in \bbR^n.
\end{align*}
Denoting $\bU_m$ as the diagonal matrix of eigenvalues $u_1^{(m)},\dots,u_m^{(m)}$, and $\bV_m$ the corresponding matrix of eigenvectors $\bv_i^{(m)}$, we have
\[
  \bH_\eta \approx
  {\color{gray}
  \overbrace{\color{black}
  \begin{pmatrix}
    \bV_m \\
    \bB^\top\bV_m\bU_m^{-1}
  \end{pmatrix}
  }^{\bar\bV}
  }
  \bU_m
  {\color{gray}
  \overbrace{\color{black}
  \begin{pmatrix}
    \bV_m^\top & \bU_m^{-1}\bV_m^\top\bB
  \end{pmatrix}
  }^{\bar\bV^\top}
  }.
\]
Unfortunately, it may be the case that $\bar\bV\bar\bV^\top \neq \bI_n$, while orthogonality is crucial in order to easily calculate the inverse of $\bSigma_\theta$.
An additional step is required to obtain an orthogonal version of the Nyström decomposition, as studied by \citet{fowlkes2001efficient}.
 Let $\bK = \bA + \bA^{-\half}\bB^\top\bB\bA^{-\half}$, where $\bA^{-\half} = \bV_m\bU_m^{-\half}\bV_m$, and obtain the eigendecomposition of this $m\times m$ matrix $\bK = \bR\hat\bU\bR^\top$.
 Defining
 \[
   \hat\bV = 
   \begin{pmatrix}
     \bA \\
     \bB^\top
   \end{pmatrix}
   \bA^{-\half}\bR\hat\bU^{-\half} \in \bbR^n \times \bbR^m,
 \]
 then 
 \hltodo[Attempt to prove this.]{we have that $\bH_\eta \approx \hat\bV\hat\bU\hat\bV^\top$ such that $\hat\bV\hat\bV^\top = \bI_n$}.
 Estimating I-prior models with the Nystr\"om method including the orthogonalisation step takes roughly $O(nm^2)$ time and $O(nm)$ storage.
 
The issue of selecting the subset $\cM$ remains.
The simplest method, and that which is implemented in the \pkg{iprior} package, 
would be to uniformly sample a subset of size $m$ from the $n$ points.
Although this works well in practice, the quality of approximation might suffer if the points do not sufficiently represent the training set.
In this light, greedy approximations have been suggested to select the $m$ points, so as to reduce some error criterion relating to the quality of approximation.
For a brief review of more sophisticated methods of selecting $\cM$, see \citet[§8.1, pp. 173--174]{rasmussen2006gaussian}.

%\begin{align*}
%  \bH_\eta 
%  &\approx \sum_{i\in\cM} \tilde u_i \tilde\bv_i\tilde\bv_i^\top \\
%  &= \cancel{\frac{n/m}{\sqrt{n/m \times n / m}}}
%  \begin{pmatrix}
%    (\bV_m\bU\bV_m^\top)\bV_m\bU^{-1} \\
%    \bB^\top\bV_m\bU^{-1} \\
%  \end{pmatrix} 
%  \bU_m 
%  \begin{pmatrix}
%    (\bV_m\bU\bV_m^\top)\bV_m\bU^{-1} 
%    & \bB^\top\bV_m\bU^{-1} 
%  \end{pmatrix} \\
%  &=   
%  \begin{pmatrix}
%    \bV_m \\
%    \bB^\top\bV_m\bU_m^{-1}
%  \end{pmatrix}
%  \bU_m
%  \begin{pmatrix}
%    \bV_m^\top & \bU_m^{-1}\bV_m^\top\bB
%  \end{pmatrix}
%\end{align*}

\subsection{An efficient EM algorithm}
\label{sec:efficientEM1}

The evaluation of the $Q$ function in \eqref{eq:QfnEstep} is $O(n^3)$, because a change in the values of $\theta$ requires evaluating $\bSigma_\theta = \psi\bH_\eta^2 + \psi^{-1}\bI_n$, for which squaring $\bH_\eta$ takes the bulk of the computational time.
In this section, we describe an efficient method of evaluating $Q$ if the I-prior model only involves estimating the RKHS scale parameters and the error precision under assumptions \ref{ass:A1}--\ref{ass:A3}.

%Separate the RKHS scale parameters $\lambda$ from the other kernel parameters $\xi$ such as the Hurst index of the fBm RKHS, lengthscale of the SE RKHS, and offset parameter of the polynomial RKKS, and write $\theta = \{\lambda_1,\dots,\lambda_p,\psi,\xi \}$.
Corresponding to $p$ building block RKHSs $\cF_1,\dots,\cF_p$ of functions over $\cX_1,\dots,\cX_p$, there are $p$ scale parameters $\lambda_1,\dots,\lambda_p$ and reproducing kernels $h_1,\dots,h_p$.
Write $\theta = \{\lambda_1,\dots,\lambda_p,\psi\}$.
The most common modelling scenarios that will be encountered are listed below:
\begin{enumerate}
  \item \textbf{Single scale parameter}. With $p=1$, $f\in\cF\equiv \lambda_1\cF_1$ of functions over a set $\cX$. $\cF$ may be any of the building block RKHSs. Note that $\cX_1$ itself may be more than one-dimensional. The kernel over $\cX_1\times\cX_1$ is therefore
  \[
    h_\lambda = \lambda_1 h_1.
  \]
  \item \textbf{Multiple scale parameters}. Here, $\cF$ is a RKKS of functions $f:\cX_1\times\cdots\times\cX_p\to\bbR$, and thus $\cF\equiv \lambda_1\cF_1 \oplus \cdots \oplus \lambda_p\cF_p$, where each $\cF_k$ is one of the building block RKHSs. The kernel is
  \[
    h_\lambda = \lambda_1 h_1 + \cdots + \lambda_p h_p.
  \]
  \item \textbf{Multiple scale parameters with level-2 interactions}. This occurs commonly with multilevel and longitudinal models. Suppose that $\cX_1$ is the set of `levels' and there are $p-1$ covariate sets $\cX_k$, $k=2,\cdots,p$. The function space $\cF$ is a special case of the ANOVA RKKS containing only main and two-way interaction effects, and its kernel is
  \[
    h_\lambda = \sum_{j=1}^p \lambda_j h_j + \sum_{j < k} \lambda_j\lambda_k h_j h_k,
  \]
  where $\cF_1$ is the Pearson RKHS, and the remaining are any of the building block RKHSs.
  \item \textbf{Polynomial RKKS}. When using the polynomial RKKS of degree $d$ to incite a polynomial relationship of the covariate set $\cX_1$ on the function $f\in\cF$ (excluding an intercept), then the kernel of $\cF$ is
  \[
    h_\lambda = \sum_{k=1}^d b_k \lambda_1^k h_1^k.
  \]
  where $b_k = \frac{d!}{k!(d-k)!}$, $k=1,\dots,d$ are constants.
\end{enumerate}
Of course, many other models are possible, such as the ANOVA RKKS with all $p$ levels of interactions.
What we realise is that any of these scenarios are simply a sum-product of a manipulation of the set of scale parameters $\lambda = \{\lambda_1,\dots,\lambda_p\}$ and the set of kernel functions $h = \{h_1,\dots,h_p\}$.
%Furthermore, scenarios 1--3 are special cases of the ANOVA RKKS excluding the grand mean\footnote{As discussed, for simplicity the RKHS of constant functions is ignored and the model includes an intercept to be estimated instead.}: in 1. and 2., $\cF$ is the ANOVA RKKS of main effects only, and in 3., $\cF$ is the ANOVA RKKS of main effects and level-two interactions.

Let us be more concrete about what we mean by `manipulation' of the sets $\lambda$ and $h$.
Define an `instruction operator' which expands out both sets identically as required by the modelling scenario.
Computationally speaking, this instruction could be as simple as a list containing the indices to multiply out.
For the four scenarios above, the list $\cQ$ is
\begin{enumerate}
  \item $\cQ = \big\{ \{1\} \big\}$.
  \item $\cQ = \big\{ \{1\},\dots,\{p\}\big\}$.
  \item  $\cQ = \big\{ \{1\},\dots,\{p\},\{1,2\},\dots, \{p-1,p\}\big\}$.
  \item $\cQ = \big\{ \{1\},\{1,1\},\dots, \{\myoverbrace{1,\dots,1}{d}\}\big\}$.
\end{enumerate}
For the polynomial RKKS in the fourth example, one must also multiply the constants $b_k$ to the $\lambda$'s as appropriate.
Let $q$ be the cardinality of the set $\cQ$, which is the number of summands required to construct the kernel for $\cF$.
Denote the instructed sets as $\xi = \{\xi_1,\dots,\xi_q \}$ for $\lambda$ and $a = \{a_1,\dots,a_q\}$ for $h$.
We can write the kernel $h_\lambda$ as a linear combination of $\xi$ and $a$,
\[
  h_\lambda = \xi_1a_1 + \cdots + \xi_qa_q.
\]
The reason this is important is because changes in $\lambda$ for $h_\lambda$ only changes the $\xi_k$'s, but not the $a_k$'s.
This allows us to compute and store all of the required $n\times n$ kernel matrices $\bA_1,\dots,\bA_q$ from the application of instruction set on $h$ evaluated at all pairs of data points $(x_i,x_j)\in\cX\times\cX$.
This process of initialisation need only be done once prior to commencing the EM algorithm---a step we refer to as `kernel loading'.
In the \pkg{iprior} package, kernel loading is performed using the \code{kernL()} command.
%The application of the instruction set $\cQ$ to $\lambda$ to obtain $\xi$ is computationally effortless.

Notice that
\begin{align*}
  \tr\big(\bSigma_\theta \Wtilde^{(t)}\big)
  &=  \tr\big( (\psi\bH_\eta^2 + \psi^{-1}\bI_n ) \Wtilde^{(t)} \big) \\
  &= \psi \tr (\bH_\eta^2\Wtilde^{(t)}) + \psi^{-1}\tr \Wtilde^{(t)} \\
  &= \psi \tr \left(\sum_{j,k=1}^q \xi_j \xi_k \big(\bA_j \bA_k + (\bA_j \bA_k)^\top\big)
   \Wtilde^{(t)}\right) + \psi^{-1}\tr \Wtilde^{(t)} \\
   &= 2\psi \sum_{j,k=1}^q \xi_j \xi_k \tr \left(  \bA_j \bA_k 
   \Wtilde^{(t)}\right) + \psi^{-1}\tr \Wtilde^{(t)}.
\end{align*}
Provided that we have the matrices $\bA_{jk} = \bA_j\bA_k$, $j,k=1,\dots,q$ in addition to $\bA_1,\dots,\bA_q$ pre-calculated and stored, then evaluating $\tr\big(\bA_{jk} \Wtilde^{(t)} \big) = \vecc(\bA_{jk})^\top \vecc(\Wtilde^{(t)} )$ is $O(n^2)$, although this  only need to be done once per EM iteration.
Thus, with the kernels loaded, the overall time complexity to evaluate $Q$ is $O(n^2)$ at the beginning of each iteration, but roughly linear in $\xi$ thereafter.

As a remark, we have achieved efficiency at the expense of storage and a potentially long initialisation phase of kernel loading.
The storing of the kernel matrices $a$ can be very expensive, especially if the sample size is very large.
On the bright side, once the kernel matrices are stored in memory, the \pkg{iprior} package allows them to be reused again and again.
A practical situation where this might be useful is when we would like to repeat the EM at various initial values.

\subsection{The exponential family EM algorithm}

In the original EM paper by \citet{dempster1977maximum}, the EM algorithm was demonstrated to be easily administered to complete data likelihoods belonging to the exponential family for which the maximum likelihood estimates are easily computed.
If this is the case, then the M-step simply involves replacing the unknown sufficient statistics in the ML estimates with their \emph{conditional expectations} (see Appendix \ref{apx:expem} for details).
Certain I-prior models emit this property, namely regression functions belonging to the full or limited ANOVA RKKS, and we describe its estimation below.

Assume \ref{ass:A1}--\ref{ass:A3} applies, and that only the error precision $\psi$ and the RKHS scale parameters $\lambda_1,\dots,\lambda_p$ need to be estimated, i.e. all other kernel parameters are fixed---a similar situation was described in the previous subsection.
For the full ANOVA RKKS, the kernel is
\begin{align*}
  h_\lambda 
  &= \sum_{i=1}^p \lambda_i h_i + \sum_{i<j} \lambda_i \lambda_j h_i h_j + \cdots + \prod_{i=1}^p \lambda_i h_i \\
  &= \lambda_k 
  \myoverbrace{
  \Bigg(  
  {\color{colblu} h_k + \sum_{i} \lambda_i h_i h_k + \cdots + h_k \prod_{i\neq k} \lambda_i h_i}
  \Bigg)
  }{\text{terms of $\lambda_k$}} 
  + 
  \myoverbrace{\color{colred}
  \phantom{\Bigg(}
  \sum_{i\neq k} \lambda_i h_i + \sum_{i,j \neq k} \lambda_i \lambda_j h_i h_j + \cdots + 0
  }{\text{no $\lambda_k$ here}} \\
  &= \lambda_k {\color{colblu}r_k} + {\color{colred}s_k}
\end{align*}
where $r_k$ and $s_k$ are both functions over $\cX\times\cX$, defined respectively as the terms of the ANOVA kernel involving $\lambda_k$, and the terms not involving $\lambda_k$.
The reason for splitting $h_\lambda$ like this will become apparently momentarily.

Programmatically this looks complicated to implement in software, but in fact it is not.
Consider again the instruction list $\cQ$ for the ANOVA RKKS (Example 3, Section \ref{sec:efficientEM1}).
We can split this list into two: $\cR_k$ as those elements of $\cQ$ which involve the index $k$, and $\cS_k$ as those elements of $\cQ$ which do not involve the index $k$.
%Additionally, define $\cR_k^\lambda$ as the index set $\cR_k$ which removes occurrences of $k$ from its elements.
Let $\zeta_k$, $e_k$ be the sets of $\lambda$ and $h$ after applying the instructions of $\cR_k$ 
%and $\cR_k$ respectively
, and let $\xi_k$ and $a_k$ be the sets of $\lambda$ and $h$ after applying the instructions of $\cS_k$.
Now, we have 
\[
  r_k = \frac{1}{\lambda_k} \sum_{i=1}^{\abs{\cR_k}} \zeta_{ik} e_{ik} 
  \hspace{0.5cm}\text{and}\hspace{0.5cm}
  s_k = \sum_{i=1}^{\abs{\cS_k}} \xi_{ik} a_{ik}.   
\]
Defining $\bR_k$ and $\bS_k$ as the kernel matrices with $(i,j)$ entries $r_k(x_i,x_j)$ and $s_k(x_i,x_j)$ respectively, we have that
\[
  \bH_\eta^2 = \lambda_k^2\bR_k^2 + \lambda_k \myoverbrace{\big(\bR_k\bS_k + (\bR_k\bS_k)^\top \big)}{\bU_k} + \bS_k^2.
\]

Consider now the full data log-likelihood for $\lambda_k$, $k=1,\dots,p$, conditionally dependent on the rest of the unknown parameters $\psi$ and $\lambda_{-k} = \{\lambda_1,\dots,\lambda_p\} \backslash \{ \lambda_k \}$:
\begin{align}
  L(\lambda_k|\by,\bw,\lambda_{-k},\psi)
  &= \const 
  - \half \tr \Big( (
  \psi\bH_\eta^2 + \psi^{-1}\bI_n
  )\bw\bw^\top \Big)
  + \psi \tilde\by^\top \bH_\eta \bw \label{eq:logliklambdak} \\
  &= \const 
  - \lambda_k^2 \cdot \half[\psi] \tr(\bR_k^2 \bw\bw^\top)
  + \lambda_k \cdot \left( 
  \psi \tilde\by^\top \bR_k \bw - \half[\psi] \tr(\bU_k \bw\bw^\top)
  \right). \nonumber
\end{align}
Notice that the above likelihood is an exponential family distribution with the natural parameterisation $\beta = (-\lambda_k^2, \lambda_k)$ and sufficient statistics $T_1$ and $T_2$ defined by
\[
  T_1 = \half[\psi] \tr(\bR_k^2 \bw\bw^\top)
  \hspace{0.5cm}\text{and}\hspace{0.5cm}
  T_2 =  \psi\tilde\by^\top \bR_k \bw - \half[\psi]\tr(\bU_k^2 \bw\bw^\top).
\]
This likelihood is maximised at $\hat\lambda_k = T_2/2T_1$, but of course, the variables $w_1,\dots,w_n$ are never observed.
As per the exponential family EM routine, replace occurrences of $\bw$ and $\bw\bw^\top$ with their respective conditional expectations, i.e. $\bw\mapsto\E[\bw|\by] = \wtilde$ and $\bw\bw^\top\mapsto\E[\bw\bw^\top|\by] = \tilde\bV_w + \wtilde\wtilde^\top$ as defined in \eqref{eq:posteriorw}.
That the $\lambda_k$'s have closed-form expressions, together with the closed-form expression for $\psi$ in \eqref{eq:closedformpsi}, greatly simplifies the EM algorithm.
At the M-step, one simply updates the parameters in turn, and as such, there is no maximisation per se.
This form of the EM algorithm is known as the \emph{conditional expectation-maximisation} algorithm \citep{meng1993maximum}.

The algorithm is summarised in Algorithm \ref{alg:EM2}.
The exponential family EM for ANOVA-type I-prior models require $O(n^3)$ computational time at each step, which is spent on computing the matrix inverse in the E-step.
The M-step takes at most $O(n^2)$ time to compute.
As a remark, it is not necessary that $h_\lambda$ is the full ANOVA RKKS; any of the examples 1--3 in \autoref{sec:efficientEM1} can be estimated using this method, since they are seen as special cases of the ANOVA decomposition.
%This is also true if we decide to drop any of the terms in the ANOVA kernel.

\begin{algorithm}[hbt]
\caption{Exponential family EM for ANOVA-type I-prior models}\label{alg:EM2}
\begin{algorithmic}[1]
  \Procedure{Initialisation}{}
    \State Initialise $\lambda_1^{(0)},\dots,\lambda_p^{(0)}, \psi^{(0)}$
    \State Compute and store matrices as per $\cR_k$ and $\cS_k$.
    \State $t \gets 0$
  \EndProcedure 
  \Statex
  \While{not converged}{}
    \Procedure{E-step}{}
      \State $\wtilde \gets \psi^{(t)} \bH_{\eta^{(t)}} \big(\psi^{(t)} \bH_{\eta^{(t)}}^2 + \psi^{-(t)}\bI_n \big)^{-1} \tilde\by$
      \State $\Wtilde \gets \big(\psi^{(t)} \bH_{\eta^{(t)}}^2 + \psi^{-(t)}\bI_n \big)^{-1} + \wtilde\wtilde^{\top}$
    \EndProcedure
    \Statex
    \Procedure{M-step}{}
      \For{$k=1,\dots,p$}
        \State $T_{1k} \gets \half \tr(\bR_k^2 \Wtilde)$
        \State $T_{2k} \gets \tilde\by^\top \bR_k \wtilde - \half \tr(\bU_k^2 \Wtilde^\top)$
        \State $\lambda_k^{(t+1)} \gets T_{2k} / 2T_{1k}$
      \EndFor
      \State $T_3 \gets \tilde\by^\top\tilde\by + \tr(\bH_{\eta^{(t)}}^2\Wtilde^{(t)}) - 2\tilde\by^\top\bH_{\eta^{(t)}}\wtilde^{(t)}$
      \State $\psi^{(t+1)} \gets \tr \Wtilde^{(t)} / T_3$
    \EndProcedure
    \State $t \gets t+1$
  \EndWhile
\end{algorithmic}
\end{algorithm}

%\bPsi \bH_\eta \bV_y^{-1} (\by - \alpha\bone_n - \bff_0)
%    \hspace{0.5cm}\text{and}\hspace{0.5cm}
%    \tilde\bV_w = \big(\bH_\eta\bPsi\bH_\eta + \bPsi^{-1}\big)^{-1} = \bV_y^{-1},

%For these three examples, the specific form of the matrices $\bR_k$ and $\bS_k$ are given below
%\begin{enumerate}
%  \item \textbf{Single scale parameter}.
%  \[
%    \bR_1 = \bH_1 
%    \hspace{0.5cm}\text{and}\hspace{0.5cm}
%    \bS_1 = \bzero
%  \]
%  \item \textbf{Multiple scale parameters}.
%  \[
%    \bR_k = \bH_k
%    \hspace{0.5cm}\text{and}\hspace{0.5cm}
%    \bS_k = \sum_{j\neq k} \lambda_j \bH_j
%  \]
%  \item \textbf{Multiple scale parameters with level-2 interactions}.
%  \[
%    \bR_k = \bH_k + \sum_{j\neq k} \lambda_j (\bH_j \circ \bH_k)
%    \hspace{0.5cm}\text{and}\hspace{0.5cm}
%    \bS_k = \sum_{j\neq k} \lambda_j \bH_j + \sum_{j<j', j,j' \neq k} \lambda_j\lambda_{j'} (\bH_j \circ \bH_{j'})
%  \]
%\end{enumerate}

While the exponential family EM algorithm takes similar computational time as the efficient EM algorithm described in \autoref{sec:efficientEM1}, there is one compelling reason to consider Algorithm \ref{alg:EM2}: conjugacy of the exponential family of distributions.
Realise that $\lambda_k|(\by,\bw,\lambda_{-k},\psi)$ is in fact normally distributed, with mean and variance given by $T_2/2T_1$ and $1/2T_1$ respectively.
If we were so compelled to assign a normal prior on each of the $\lambda_k$'s, then the conditionally dependent log-likelihood of $\lambda_k$, $L(\lambda_k|\by,\bw,\lambda_{-k},\psi)$, would have a normal log-likelihood prior involving $\lambda_k$ added on.
Importantly, viewed as a posterior log-density for $\lambda_k$, the posterior density for $\lambda_k$ would also be a normal distribution.
The EM as a whole would then generate maximum a posteriori (MAP) estimates for the parameters.
Although not shown here, similar conjugacy benefits for the $\psi$ parameter can be argues, whereby the gamma distribution is the density in question.
The usual EM algorithm without using any priors can be viewed as using improper priors for the parameters, i.e. $p(\lambda_k) \propto \const$ and $p(\psi) \propto \const$.

In the next chapter on binary and multinomial regression using I-priors, the exponential family EM algorithm described here is especially relevant, as it is connected to the variational Bayesian algorithm \citep{bernardo2003variational} that will be used for estimating the models described therein.

\begin{remark}
  Earlier, we restricted attention to ANOVA RKKS. 
  Hopefully, it is now apparent that ANOVA kernels are a requirement for Algorithm \ref{alg:EM2} to work easily.
  As soon as higher degrees of the $\lambda_k$'s come into play, e.g. using the polynomial kernel, then the ML estimate for $\lambda_k$ involve solving a polynomial of degree $2d-1$ the FOC equations.
  Although this is not in itself hard to do, the elegance of the algorithm, especially viewed as having the normal conjugacy property for the $\lambda_k's$, is lost.
\end{remark}

\subsection{Accelerating the EM algorithm}

A criticism of the EM algorithm is that it may take many iterations to converge.
Several novel ideas have been looked at in a bid to `accelerate the EM algorithm', as it were.
One such approach, which does not require any amendment to the particular EM algorithm at hand, is called the \emph{monotonically over-relaxed EM algorithm} (MOEM) by \citet{yu2012monotonically}.

The idea of MOEM is as follows.
At every iteration of the MOEM, perform as usual the E-step and M-step to obtain an updated parameter value $\theta^{(t+1)}_\text{EM}$.
Instead of using this update value of the parameter, modify it instead, and use
\[
  \theta^{(t+1)} = (1 + \omega) \theta^{(t+1)}_\text{EM} - \omega \theta^{(t)},
\]
where $\omega$ is an \emph{over-relaxation} parameter.
Under mild conditions, among them that $Q(\theta^{(t+1)}) > Q(\theta^{(t)})$, the MOEM estimate does not decrease the log-likelihood at each step.
This condition is a slight inconvenience to check under the usual EM algorithm, but is a great companion to exponential family EM algorithm.
From \eqref{eq:logliklambdak}, we see that $Q(\lambda_k) = \E_\bw \big[ L(\lambda_k|\theta\backslash\{\lambda_k \} ) | \by,\theta^{(t)} \big]$ is quadratic in $\lambda_k$, therefore any $\omega \in [0,1]$ will maintain monotonicity of the EM algorithm.


\section{Post-estimation}
\label{sec:ipriorpostest}
One of the perks of a (semi-)Bayesian approach to regression modelling is that we are able to use Bayesian post-estimation machinery involving the relevant posterior distributions.
With the normal I-prior model, there is the added benefit that posterior distributions are easily obtained in closed form.
The plots that are shown in this subsection is a continuation from the example in \hltodo{Section X}.

Recall that for the regression function as specified in \eqref{eq:model2}, its posterior regression function is found to be $f(x) = \sum_{i=1}^n h_{\hat\eta}(x,x_i)\tilde w_i$, where $\hat\eta$ is the ML estimate for the kernel parameters, and the $\tilde w_i$'s are multivariate-normally distributed with mean and variance according to \eqref{eq:posteriorw}.
Denote by $\bh_{\hat\eta}(x)$ the $n$-vector with entries equal to $h_{\hat\eta}(x,x_i)$.
Therefore, the posterior density for the regression function is
\begin{align}
  p\big(f(x) | \by \big) \sim \N \Big( 
  \bh_{\hat\eta}(x)\hat\bw, 
  \bh_{\hat\eta}(x)^\top \big(\bH_{\hat\eta}\hat\bPsi\bH_{\hat\eta} + \hat\bPsi^{-1}\big)^{-1} \bh_{\hat\eta}(x) 
  \Big)
\end{align}
for any $x$ in the domain of the regression function.
Here, the hats on the parameters indicate the use of the optimised model parameters, i.e. the ML or MAP estimates.

\begin{figure}[p]
  \centering
  \includegraphics[width=0.95\textwidth]{figure/04-post_reg_prior_post}
  \caption{Prior (top) and posterior (bottom) sample path realisations of regression functions drawn from their respective distributions when $\cF$ is a fBm-0.5 RKHS. At the very top of the figure, a smoothed density estimate of the $x$'s is overlaid. In regions with few data points (near the centre), there is little Fisher information, and hence a conservative prior closer to zero, the prior mean, for this region.}
\end{figure}

Prediction of a new data point is also of interest.
A priori, assume that $y_\new = \hat\alpha + f(x_\new) + \epsilon_\new$, where $\epsilon_\new \sim \N(0,\psi^{-1}_\new)$, and $f\sim \text{I-prior}$.
Denote the covariance between $\epsilon_\new$ and $\bepsilon = (\epsilon_1,\dots,\epsilon_n)^\top$ by $\bsigma_\new^\top \in \bbR^n$.
Under an iid model (assumption A3), then $\psi_\new = \psi = \Var \epsilon_i$ for any $i\in\{1,\dots,n\}$, and $\bsigma_\new^\top=\bzero$, but otherwise, these extra parameters need to be dealt with somehow, either by specifying them a priori or estimating them again, which seems excessive.
In any case, using a linearity argument, the posterior distribution for $y_\new$ is normal, with mean and variance given by
\begin{gather}
  \E[y_\new|\by] = \hat\alpha + \E \big[ f(x_\new) |\by \big] + \text{mean correction term} \\
  \text{and} \nonumber \\
  \Var[y_\new|\by] 
  = \Var\big[f(x_\new)|\by\big] + \psi^{-1}_\new + \text{variance correction term}.
\end{gather}
A derivation is presented in Appendix \ref{apx:postpred}.
Note that the mean and variance correction term vanishes under an iid assumption A3.
The posterior distribution for $y_\new$ can be used in several ways. 
Among them, is to construct a $100(1 - \alpha/2)\%$ credibility interval for the (mean) predicted value $y_\new$ using
\[
  \E[y_\new|\by] \pm \Phi^{-1}(1 - \alpha/2) \cdot \Var[y_\new|\by]^{\half},
\]
where $\Phi(\cdot)$ is the standard normal cumulative distribution function.
One could also perform a posterior predictive density check of the data $\by$, by repeatedly sampling $n$ points from its posterior distribution.
This provides a visual check of whether there are any systematic deviances between what the model predicts, and what is observed from the data.

\begin{figure}[p]
  \centering
  \includegraphics[width=0.9\textwidth]{figure/04-post_reg_cred}
  \caption{The estimated regression line (solid black) is the posterior mean estimate of the regression function (shifted by the intercept), which also gives the posterior mean estimate for the responses $y$. The shaded region is the 95\% credibility interval for predictions. The true regression line (dashed red) is shown for comparison.}
  \vspace{1em}
  \includegraphics[width=0.9\textwidth]{figure/04-post_reg_ppc}
  \caption{Posterior predictive density checks of the responses: repeated sampling from the posterior density of the $y_i$'s and plotting their densities allows us to compare model predictions against observed samples.}
\end{figure}

Lastly, we discuss model comparison.
Recall that the marginal distribution for $\by$ after integrating out the I-prior for $f$ in model \eqref{eq:model2} is a normal distribution.
Suppose that we are interested in comparing two candidate models $M_1$ and $M_2$, each with the parameter set $\theta_1$ and $\theta_2$.
Commonly, we would like to test whether or not particular terms in the ANOVA RKKS are significant contributors in explaining the relationship between the responses and predictors.
A log-likelihood comparison is possible using an asymptotic chi-squared distribution, with degrees of freedom equal to the difference between the number of parameters in $\theta_2$ and $\theta_1$.
This is assuming model $M_1$ is nested within $M_2$, which is the case for ANOVA-type constructions.
Note that if two models have the same number of parameters, then the model with the higher likelihood is preferred.

As a remark, this method of comparing marginal likelihoods can be seen as Bayesian model selection using \emph{empirical Bayes factors}, where the Bayes factor of comparing model $M_1$ to model $M_2$ is defined as
\[
  \BF(M_1,M_2) = \frac{\int p(\by|\theta_1,\bff)p(\bff) \dint \bff }{\int p(\by|\theta_2,\bff)p(\bff) \dint \bff}
\]
The word ‘empirical’ stems from the fact that the parameters are estimated via an empirical Bayes approach (maximum marginal likelihood).
This approach is fine when the number of comparisons to be made is small, but can be computationally unfeasible when many marginal likelihoods need to be pairwise compared.
In Chapter 6, we explore a fully Bayesian approach to explore the entire model space for the special case of linear models.

%From \eqref{eq:marglogliky}, the (marginal) log-likelihood for $\theta$ is 
%\[
%  L(\theta|\by) = -\half[n]\log 2\pi - \half \log \bSigma_\theta 
%\]



\section{Examples}\label{sec:ipriorexamples}
\documentclass[showframe,11pt,twoside,openright]{report}\usepackage[]{graphicx}\usepackage{xcolor}
%% maxwidth is the original width if it is less than linewidth
%% otherwise use linewidth (to make sure the graphics do not exceed the margin)
\makeatletter
\def\maxwidth{ %
  \ifdim\Gin@nat@width>\linewidth
    \linewidth
  \else
    \Gin@nat@width
  \fi
}
\makeatother

\definecolor{fgcolor}{rgb}{0.196, 0.196, 0.196}
\newcommand{\hlnum}[1]{\textcolor[rgb]{0.063,0.58,0.627}{#1}}%
\newcommand{\hlstr}[1]{\textcolor[rgb]{0.063,0.58,0.627}{#1}}%
\newcommand{\hlcom}[1]{\textcolor[rgb]{0.588,0.588,0.588}{#1}}%
\newcommand{\hlopt}[1]{\textcolor[rgb]{0.196,0.196,0.196}{#1}}%
\newcommand{\hlstd}[1]{\textcolor[rgb]{0.196,0.196,0.196}{#1}}%
\newcommand{\hlkwa}[1]{\textcolor[rgb]{0.231,0.416,0.784}{#1}}%
\newcommand{\hlkwb}[1]{\textcolor[rgb]{0.627,0,0.314}{#1}}%
\newcommand{\hlkwc}[1]{\textcolor[rgb]{0,0.631,0.314}{#1}}%
\newcommand{\hlkwd}[1]{\textcolor[rgb]{0.78,0.227,0.412}{#1}}%
\let\hlipl\hlkwb

\usepackage{framed}
\makeatletter
\newenvironment{kframe}{%
 \def\at@end@of@kframe{}%
 \ifinner\ifhmode%
  \def\at@end@of@kframe{\end{minipage}}%
  \begin{minipage}{\columnwidth}%
 \fi\fi%
 \def\FrameCommand##1{\hskip\@totalleftmargin \hskip-\fboxsep
 \colorbox{shadecolor}{##1}\hskip-\fboxsep
     % There is no \\@totalrightmargin, so:
     \hskip-\linewidth \hskip-\@totalleftmargin \hskip\columnwidth}%
 \MakeFramed {\advance\hsize-\width
   \@totalleftmargin\z@ \linewidth\hsize
   \@setminipage}}%
 {\par\unskip\endMakeFramed%
 \at@end@of@kframe}
\makeatother

\definecolor{shadecolor}{rgb}{.97, .97, .97}
\definecolor{messagecolor}{rgb}{0, 0, 0}
\definecolor{warningcolor}{rgb}{1, 0, 1}
\definecolor{errorcolor}{rgb}{1, 0, 0}
\newenvironment{knitrout}{}{} % an empty environment to be redefined in TeX

\usepackage{alltt}
\usepackage{standalone}
\standalonetrue
\ifstandalone
  \usepackage{../../haziq_thesis}
  \usepackage{../../haziq_maths}
  \usepackage{../../haziq_glossary}
  \addbibresource{../../bib/haziq.bib}
  \externaldocument{../01/.texpadtmp/chapter1}
  \externaldocument{../02/.texpadtmp/chapter2}
  \externaldocument{../03/.texpadtmp/chapter3}
  \externaldocument{../04/.texpadtmp/chapter4}
  \externaldocument{../appendix/.texpadtmp/appendix}
\fi





\IfFileExists{upquote.sty}{\usepackage{upquote}}{}
\begin{document}

We demonstrate I-prior modelling on a toy data set to illustrate the Nyström method, as well as three other real-data examples.
All of the analyses were conducted in \proglang{R}, and I-prior model estimation was done using the \pkg{iprior} package \citep{jamil2017iprior}.
The \pkg{iprior} package comes documented with usage examples in the vignette.
The complete source code for replication is found at \url{http://myphdcode.haziqj.ml}.
Note that in all of these examples, \ref{ass:A1}--\ref{ass:A3} were assumed.

\pagebreak
\subsection{Random effects models}
\index{random-effects model|see{multilevel model}}
\index{multilevel model}
\index{IGF data set}
\index{ANOVA kernel/RKKS}
\index{canonical kernel/RKHS}
\index{Pearson kernel/RKHS}

In this section, a comparison between a standard random effects model and the I-prior approach for estimating varying intercept and slopes model is illustrated.
The example concerns control data\footnotemark\ from several runs of radioimmunoassays (RIA) for the protein insulin-like growth factor (IGF-I) (explained in further detail in \cite[sec. 3.2.1]{davidian1995nonlinear}).
RIA is an in vitro assay technique which is used to measure concentration of antigens---in our case, the IGF-I proteins.
When an RIA is run, control samples at known concentrations obtained from a particular lot are included for the purpose of assay quality control.
It is expected that the concentration of the control material remains stable as the machine is used, up to a maximum of about 50 days, at which point  control samples from a new batch is used to avoid degradation in assay performance.

\begin{knitrout}
\definecolor{shadecolor}{rgb}{1, 1, 1}\color{fgcolor}\begin{kframe}
\singlespacing\begin{alltt}
\hlstd{R> }\hlkwd{data}\hlstd{(IGF,} \hlkwc{package} \hlstd{=} \hlstr{"nlme"}\hlstd{)}
\hlstd{R> }\hlkwd{head}\hlstd{(IGF)}
\end{alltt}
\begin{verbatim}
## Grouped Data: conc ~ age | Lot
##   Lot age conc
## 1   1   7 4.90
## 2   1   7 5.68
## 3   1   8 5.32
## 4   1   8 5.50
## 5   1  13 4.94
## 6   1  13 5.19
\end{verbatim}
\end{kframe}
\end{knitrout}

\footnotetext{This data is available in the \proglang{R} package \pkg{nlme} \citep{nlme}.}

The data consists of IGF-I concentrations (\code{conc}) from control samples from 10 different lots measured at differing \code{age}s of the lot.
The data were collected with the aim of identifying possible trends in control values \code{conc} with \code{age}, ultimately investigating whether or not the usage protocol of maximum sample age of 50 days is justified.
\citet{pinheiro2000mixed} remarks that this is not considered a longitudinal problem because different samples were used at each measurement.

We shall model the IGF data set using the I-prior methodology using the ANOVA-decomposed regression function
\[
  f(\texttt{age}, \texttt{Lot}) = f_1(\texttt{age}) + f_2(\texttt{Lot}) + f_{12}(\texttt{age}, \texttt{Lot})
\]
where $f_1$ lies in the linear RKHS $\cF_1$, $f_2$ in the Pearson RKHS $\cF_2$ and $f_{12}$ in the tensor product space $\cF_{12} = \cF_1 \otimes \cF_2$.
The regression function $f$ then lies in the RKHS $\cF = \cF_1 \oplus \cF_2 \oplus \cF_{12}$ with kernel equal to the sum of the kernels from each of the RKHSs.
The explanation here is that the \code{conc} levels are assumed to be related to both \code{age} and \code{Lot}, and in particular, the contribution of \code{age} on \code{conc} varies with each individual \code{Lot}.
This gives the intended effect of a linear mixed-effects model, which is thought to be suitable in this case, in order to account for within-lot and between-lot variability.
We first fit the model using the \pkg{iprior} package, and then compare the results with the standard random effects model using the \proglang{R} command \code{lme4::lmer()}.
The command to fit the I-prior model using the EM algorithm is

\begin{knitrout}
\definecolor{shadecolor}{rgb}{1, 1, 1}\color{fgcolor}\begin{kframe}
\singlespacing\begin{alltt}
\hlstd{R> }\hlstd{mod.iprior} \hlkwb{<-} \hlkwd{iprior}\hlstd{(conc} \hlopt{~} \hlstd{age} \hlopt{*} \hlstd{Lot, IGF,} \hlkwc{method} \hlstd{=} \hlstr{"em"}\hlstd{)}
\end{alltt}
\begin{verbatim}
## =========================================
## Converged after 58 iterations.
\end{verbatim}
\begin{alltt}
\hlstd{R> }\hlkwd{summary}\hlstd{(mod.iprior)}
\end{alltt}
\begin{verbatim}
## Call:
## iprior(formula = conc ~ age * Lot, data = IGF, method = "em")
## 
## RKHS used:
## Linear (age)
## Pearson (Lot)
## 
## Residuals:
##    Min. 1st Qu.  Median 3rd Qu.    Max. 
## -4.4890 -0.3798 -0.0090  0.2563  4.3972 
## 
## Hyperparameters:
##           Estimate    S.E.      z P[|Z>z|]    
## lambda[1]   0.0000  0.0002  0.004    0.997    
## lambda[2]  -0.0007  0.0030 -0.239    0.811    
## psi         1.4577  0.1366 10.672   <2e-16 ***
## ---
## Signif. codes:  0 '***' 0.001 '**' 0.01 '*' 0.05 '.' 0.1 ' ' 1
## 
## Closed-form EM algorithm. Iterations: 58/100 
## Converged to within 1e-08 tolerance. Time taken: 3.553403 secs
## Log-likelihood value: -291.9033 
## RMSE of prediction: 0.8273565 (Training)
\end{verbatim}
\end{kframe}
\end{knitrout}
\begin{knitrout}
\definecolor{shadecolor}{rgb}{1, 1, 1}\color{fgcolor}\begin{kframe}
\singlespacing\end{kframe}\begin{figure}

{\centering \includegraphics[width=\maxwidth]{figure/04-IGF_mod_iprior_plot-1} 

}

\caption[Plot of fitted regression line for the I-prior model on the IGF data set]{Plot of fitted regression line for the I-prior model on the IGF data set, separated into each of the 10 lots.}\label{fig:IGF.mod.iprior.plot}
\end{figure}


\end{knitrout}

To make inference on the covariates, we look at the scale parameters \code{lambda}.
We see that both scale parameters for \code{age} and \code{Lot} are close to zero, and a test of significance is not able to reject the hypothesis that these parameters are indeed null.
We conclude that neither \code{age} nor \code{Lot} has a linear effect on the \code{conc} levels.
The plot of the fitted regression line in \cref{fig:IGF.mod.iprior.plot} does show an almost horizontal line for each \code{Lot}.

The standard random effects model, as explored by \citet{davidian1995nonlinear} and \citet{pinheiro2000mixed}, is
\begin{align*}
  \begin{gathered}
    \texttt{conc}_{ij} = \beta_{0j} + \beta_{1j}\texttt{age}_{ij} + \epsilon_{ij} \\
    \begin{pmatrix}
      \beta_{0j} \\
      \beta_{1j} \\
    \end{pmatrix}
    \sim \N \left(
      \begin{pmatrix}
        \beta_{0} \\
        \beta_{1} \\
      \end{pmatrix},
      \begin{pmatrix}
        \sigma_{0}^2 & \sigma_{01} \\
        \sigma_{01}  & \sigma_1^2 \\
      \end{pmatrix}
    \right) \\
    \epsilon_{ij} \sim \N(0, \sigma^2) \\
  \end{gathered}
\end{align*}
for $i=1,\dots,n_j$ and the index $j$ representing the 10 \code{Lots}.
Fitting this model using \code{lmer}, we can test for the significance of the fixed effect $\beta_0$, for which we find that it is not ($p$-value = 0.627), and arrive at the same conclusion as in the I-prior model.

\begin{knitrout}
\definecolor{shadecolor}{rgb}{1, 1, 1}\color{fgcolor}\begin{kframe}
\singlespacing\begin{alltt}
\hlstd{R> }\hlstd{(mod.lmer} \hlkwb{<-} \hlkwd{lmer}\hlstd{(conc} \hlopt{~} \hlstd{age} \hlopt{+} \hlstd{(age} \hlopt{|} \hlstd{Lot), IGF))}
\end{alltt}
\begin{verbatim}
## Linear mixed model fit by REML ['lmerModLmerTest']
## Formula: conc ~ age + (age | Lot)
##    Data: IGF
## REML criterion at convergence: 594.3662
## Random effects:
##  Groups   Name        Std.Dev. Corr 
##  Lot      (Intercept) 0.082507      
##           age         0.008092 -1.00
##  Residual             0.820628      
## Number of obs: 237, groups:  Lot, 10
## Fixed Effects:
## (Intercept)          age  
##    5.374974    -0.002535
\end{verbatim}
\begin{alltt}
\hlstd{R> }\hlkwd{round}\hlstd{(}\hlkwd{coef}\hlstd{(}\hlkwd{summary}\hlstd{(mod.lmer)),} \hlnum{4}\hlstd{)}
\end{alltt}
\begin{verbatim}
##             Estimate Std. Error      df t value Pr(>|t|)
## (Intercept)   5.3750     0.1075 41.5757 50.0053   0.0000
## age          -0.0025     0.0050  9.5136 -0.5025   0.6267
\end{verbatim}
\end{kframe}
\end{knitrout}

However, we notice that the package reports a perfect negative correlation between the random effects, $\sigma_{01}$.
This indicates a potential numerical issue when fitting the model---a value of exactly $-1$, $0$ or $1$ is typically imposed by the package to force through estimation in the event of non-positive definite covariance matrices arising.
We can inspect the eigenvalues of the covariance matrix for the random effects to check that they are indeed non-positive definite.
One of the eigenvalues was found to be negative, so the covariance matrix is non-positive definite.

\begin{knitrout}
\definecolor{shadecolor}{rgb}{1, 1, 1}\color{fgcolor}\begin{kframe}
\singlespacing\begin{alltt}
\hlstd{R> }\hlkwd{eigen}\hlstd{(}\hlkwd{VarCorr}\hlstd{(mod.lmer)}\hlopt{$}\hlstd{Lot)}
\end{alltt}
\begin{verbatim}
## eigen() decomposition
## $values
## [1]  6.872939e-03 -1.355253e-20
## 
## $vectors
##             [,1]        [,2]
## [1,] -0.99522490 -0.09760839
## [2,]  0.09760839 -0.99522490
\end{verbatim}
\end{kframe}
\end{knitrout}


\begin{knitrout}
\definecolor{shadecolor}{rgb}{1, 1, 1}\color{fgcolor}\begin{kframe}
\singlespacing\end{kframe}\begin{figure}[htbp]

{\centering \includegraphics[width=\maxwidth]{figure/04-IGF_plot_beta-1} 

}

\caption[A comparison of the estimates for random intercepts and slopes (denoted as points) using the I-prior model and the standard random effects model]{A comparison of the estimates for random intercepts and slopes (denoted as points) using the I-prior model and the standard random effects model. The dashed vertical lines indicate the fixed effect values.}\label{fig:IGF.plot.beta}
\end{figure}


\end{knitrout}

Degenerate covariance matrices often occur in models with a large number of random coefficients, and in cases where values of the variance components are estimated at the boundary.
These are typically solved by setting restrictions which then avoids overparameterising the model.
One advantage of the I-prior method for varying intercept/slopes model is that the positive-definiteness is automatically taken care of.
Furthermore, I-prior models typically require fewer parameters to fit a similar varying intercept/slopes model---in the above example, the I-prior model estimated only three parameters, while the standard random effects model estimated a total of six parameters.

It is also possible to ``recover'' the estimates of the standard random effects model from the I-prior model, albeit in a slighly manual fashion (refer to \cref{sec:multilevelmodels}).
Denote by $f^j$ the individual linear regression lines for each of the $j=1,\dots,10$ \code{Lots}.
Then, each of these $f^j$ has a slope and intercept for which we can estimate from the fitted values $\hat f^j(x_{ij})$, $i=1,\dots,n_j$.
This would give us the estimate of the posterior mean of the random intercepts and slopes; these would typically be obtained using empirical-Bayes methods in the case of the standard random effects model.

Furthermore, $\sigma_0^2$ and $\sigma_1^2$ gives a measure of variability of the intercepts and slopes of the different groups, and this can be calculated from the estimates of the random intercepts and slopes.
In the same spirit, $\rho_{01} = \sigma_{01} / (\sigma_0 \sigma_1)$, which is the correlation between the random intercept and slope, can be similarly calculated.
Finally, the fixed effects can be estimated from the intercept and slope of the best fit line running through the I-prior estimated \code{conc} values.
The intuition for this is that the fixed effects are essentially the ordinary least squares (OLS) of a linear model if the groupings are disregarded.
\cref{fig:IGF.plot.beta} illustrates the differences in the estimates for the random coefficients, while  \cref{tab:igf} illustrates the differences in the estimates for the covariance matrix.
Minor differences do exist, with the most noticeable one being that the slopes in the I-prior model are categorically estimated as zero, and the sign of the correlation $\rho_{01}$ being opposite in both models.
Even so, the conclusions from both models are similar.

\begin{table}[htb]
\centering
\caption[A comparison of the estimates for the IGF data set]{A comparison of the estimates for the covariance matrix of the random effects using the I-prior model and the standard random effects model.}
\label{tab:igf}
\begin{tabular}{lrr}
\toprule
Parameter     & \texttt{iprior} & \texttt{lmer} \\
\midrule
$\sigma_0$    & 0.012 & 0.083 \\
$\sigma_1$    & 0.000 & 0.008 \\
$\rho_{01}$   & -0.691& -1.000 \\
\bottomrule
\end{tabular}
\end{table}

\subsection{Longitudinal data analysis}
\label{sec:cows}
\index{longitudinal model}
\index{cow growth data set}
\index{ANOVA kernel/RKKS}
\index{fBm kernel/RKHS}
\index{Pearson kernel/RKHS}

We consider a balanced longitudinal data set consisting of weights in kilograms of 60 cows, 30 of which were randomly assigned to treatment group A, and the remaining 30 to treatment group B.
The animals were weighed 11 times over a 133-day period; the first 10 measurements for each animal were made at two-week intervals and the last measurement was made one week later.
This experiment was reported by \citet{kenward1987method}, and the data set is included as part of the package \pkg{jmcm} \citep{jmcm} in \proglang{R}.
The variable names have been renamed for convenience.

\begin{knitrout}
\definecolor{shadecolor}{rgb}{1, 1, 1}\color{fgcolor}\begin{kframe}
\singlespacing\begin{alltt}
\hlstd{R> }\hlkwd{data}\hlstd{(cattle,} \hlkwc{package} \hlstd{=} \hlstr{"jmcm"}\hlstd{)}
\hlstd{R> }\hlkwd{names}\hlstd{(cattle)} \hlkwb{<-} \hlkwd{c}\hlstd{(}\hlstr{"id"}\hlstd{,} \hlstr{"time"}\hlstd{,} \hlstr{"group"}\hlstd{,} \hlstr{"weight"}\hlstd{)}
\hlstd{R> }\hlstd{cattle}\hlopt{$}\hlstd{id} \hlkwb{<-} \hlkwd{as.factor}\hlstd{(cattle}\hlopt{$}\hlstd{id)}  \hlcom{# convert to factors}
\hlstd{R> }\hlkwd{levels}\hlstd{(cattle}\hlopt{$}\hlstd{group)} \hlkwb{<-} \hlkwd{c}\hlstd{(}\hlstr{"Treatment A"}\hlstd{,} \hlstr{"Treatment B"}\hlstd{)}
\hlstd{R> }\hlkwd{str}\hlstd{(cattle)}
\end{alltt}
\begin{verbatim}
## 'data.frame':	660 obs. of  4 variables:
##  $ id    : Factor w/ 60 levels "1","2","3","4",..: 1 1 1 1 1 1 1 1 1..
##  $ time  : num  0 14 28 42 56 70 84 98 112 126 ...
##  $ group : Factor w/ 2 levels "Treatment A",..: 1 1 1 1 1 1 1 1 1 1 ..
##  $ weight: int  233 224 245 258 271 287 287 287 290 293 ...
\end{verbatim}
\end{kframe}
\end{knitrout}

The response variable of interest are the \code{weight} growth curves, and the aim is to investigate whether a treatment effect is present.
The usual approach to analyse a longitudinal data set such as this one is to assume that the observed growth curves are realizations of a Gaussian process.
For example, \citet{kenward1987method} assumed a so-called ante-dependence structure of order $k$, which assumes an observation depends on the previous $k$ observations, but given these, is independent of any preceeding observations.

Using the I-prior, it is not necessary to assume the growth curves were drawn randomly.
Instead, it suffices to assume that they lie in an appropriate function class.
For this example, we assume that the function class is the fBm RKHS, i.e. we assume a smooth effect of time on weight.
The growth curves form a multidimensional (or functional) response equivalent to a ``wide'' format of representing repeated measures data. In our analysis using the \pkg{iprior} package, we used the ``long'' format and thus our (unidimensional) sample size $n$ is equal to $60$ cows $\times$ $11$ repeated measurements.
We also have two covariates potentially influencing growth, namely the cow subject \code{id} and also treatment \code{group}. The regression model can then be thought of as
\begin{align*}
  \begin{gathered}
    \text{\code{weight}} = \alpha + f(\text{\code{id}}, \, \text{\code{group}}, \, \text{\code{time}}) + \epsilon \\
    \epsilon \sim \N(0, \psi^{-1}).
  \end{gathered}
\end{align*}

\begin{table}[t!]
\centering
\caption[A brief description of the five models fitted for the cows data set]{A brief description of the five models fitted using I-priors.}
\label{tab:cowmodel}
\begin{tabular}{lp{6cm}l}
\toprule
Model & Explanation & Formula (\verb@weight ~ ...@) \\
\midrule
1     & Growth does not vary with treatment nor among cows
&\verb@time@ \\
2     & Growth varies among cows only
&\verb@id * time@ \\
3     & Growth varies with treatment only
&\verb@group * time@ \\
4     & Growth varies with treatment and among cows
&\verb@id * time + group * time@ \\
5     & Growth varies with treatment and among cows, with an interaction effect between treatment and cows
&\verb@id * group * time@ \\
\bottomrule
\end{tabular}
\end{table}

We assume iid errors, and in addition to a smooth effect of \code{time}, we further assume a nominal effect of both cow \code{id} and treatment \code{group} using the Pearson RKHS.
In the \pkg{iprior} package, factor type objects are treated with the Pearson kernel automatically, and the only \code{model} option we need to specify is the \code{kernel = "fbm"} option for the \code{time} variable.
We shall use a default Hurst coefficient of 1/2 for the fBm kernel.
\cref{tab:cowmodel} explains the five models we have fitted.

The simplest model fitted was one in which the growth curves do not depend on the treatment effect or individual cows.
We then added treatment effect and the cow \code{id} as covariates, separately first and then together at once.
We also assumed that both of these covariates are time-varying, and hence added also the interaction between these covariates and the \code{time} variable.
The final model was one in which an interaction between treatment effect and individual cows was assumed, which varied over time.

All models were fitted using the \code{mixed} estimation method.
Compared to the EM algorithm alone, we found that the combination of direct optimisation with the EM algorithm fits the model about six times faster for this data set due to slow convergence of EM algorithm.
Here is the code and output for fitting the first model:

\begin{knitrout}
\definecolor{shadecolor}{rgb}{1, 1, 1}\color{fgcolor}\begin{kframe}
\singlespacing\begin{alltt}
\hlstd{R> }\hlcom{# Model 1: weight ~ f(time)}
\hlstd{R> }\hlstd{(mod1} \hlkwb{<-} \hlkwd{iprior}\hlstd{(weight} \hlopt{~} \hlstd{time, cattle,} \hlkwc{kern} \hlstd{=} \hlstr{"fbm"}\hlstd{,} \hlkwc{method} \hlstd{=} \hlstr{"mixed"}\hlstd{))}
\end{alltt}
\begin{verbatim}
## Running 5 initial EM iterations
## ======================================================================
## Now switching to direct optimisation
## final  value 1394.615062 
## converged
## Log-likelihood value: -2789.231 
## 
##  lambda     psi 
## 0.83592 0.00375
\end{verbatim}
\end{kframe}
\end{knitrout}


\newcolumntype{R}[1]{>{\raggedleft\arraybackslash}p{#1}}
\begin{table}[htb]
\centering
\caption[Summary of the five I-prior models fitted to the cow data set.]{Summary of the five I-prior models fitted to the cow data set. Error S.D. refers to the inverse square root of the error precision, $\psi^{-1/2}$.}
\label{tab:cowresults}
\begin{tabular}{rp{4.9cm}R{2.3cm}R{1.9cm}R{2.2cm}}
\toprule
{\small Model}
& {\small{Formula \newline (}\verb@weight ~ ...@{)}}
& {\small{Log-likelihood}}
& {\small{Error S.D.}}
& {\small{Number of parameters}}  \\
\midrule
1 & \code{time}
& -2789.23
& 16.33
& 1 \\
2 & \code{id * time}
& -2789.26
& 16.31
& 2 \\
3 & \code{group * time}
& -2295.16
& 3.68
& 2 \\
4 & \code{id * time + group * time}
& -2270.85
& 3.39
& 3 \\
5 & \code{id * group * time}
& -2249.26
& 3.90
& 3 \\
\bottomrule
\end{tabular}
\end{table}

The results of the model fit are summarised in \cref{tab:cowresults}. We can test for a treatment effect by testing Model 4 against the alternative that Model 2 is true.
The log-likelihood ratio test statistic is
$D = -2(-2789.26 - (-2270.85)) = 1036.81$, which has an asymptotic chi-squared distribution with $3 - 2 = 1$ degree of freedom.
The $p$-value for this likelihood ratio test is less than $10^{-6}$, so we conclude that Model 4 is significantly better.

We can next investigate whether the treatment effect differs among cows by comparing Models 5 and 4.
As these models have the same number of parameters, we can simply choose the one with the higher likelihood, which is Model 5.
We conclude that treatment does indeed have an effect on growth, and that the treatment effect differs among cows.
A plot of the fitted regression curves onto the cow data set is shown in \cref{fig:cows.plot}.

\begin{knitrout}
\definecolor{shadecolor}{rgb}{1, 1, 1}\color{fgcolor}\begin{kframe}
\singlespacing\end{kframe}\begin{figure}[htb]

{\centering \includegraphics[width=\maxwidth]{figure/04-cows_plot-1} 

}

\caption[A plot of the I-prior fitted regression curves from Model 5]{A plot of the I-prior fitted regression curves from Model 5. In this model, growth curves differ among cows and by treatment effect (with an interaction between cows and treatment effect), thus producing these 60 individual lines, one for each cow, split between their respective treatment groups (A or B).}\label{fig:cows.plot}
\end{figure}


\end{knitrout}

\subsection{Regression with a functional covariate}
\index{functional regression}
\index{Tecator data set}
\index{ANOVA kernel/RKKS}
\index{canonical kernel/RKHS}
\index{Pearson kernel/RKHS}
\index{polynomial kernel/RKKS}
\index{SE kernel/RKHS}

We illustrate the prediction of a real valued response with a functional covariate using a widely analysed data set for quality control in the food industry.
The data\footnotemark~contain samples of spectrometric curve of absorbances of 215 pieces of finely chopped meat, along with their water, fat and protein content.
These data are recorded on a Tecator Infratec Food and Feed Analyzer working in the wavelength range 850--1050 nm by the Near Infrared Transmission (NIT) principle.
Absorption data has not been measured continuously, but instead 100 distinct wavelengths were obtained. \cref{fig:tecator.data} shows a sample of 10 such spectrometric curves.

\footnotetext{
Obtained from Tecator (see \url{http://lib.stat.cmu.edu/datasets/tecator} for details).
We used the version made available in the dataframe \code{tecator} from the \proglang{R} package \pkg{caret} \citep{caret}.
}

\begin{knitrout}
\definecolor{shadecolor}{rgb}{1, 1, 1}\color{fgcolor}\begin{kframe}
\singlespacing\end{kframe}\begin{figure}[htb]

{\centering \includegraphics[width=11cm]{figure/04-tecator_data-1} 

}

\caption[Sample of spectrometric curves used to predict fat content of meat]{Sample of spectrometric curves used to predict fat content of meat. For each meat sample the data consists of a 100 channel spectrum of absorbances and the contents of moisture, fat (numbers shown in boxes) and protein measured in percent. The absorbance is $-\log 10$ of the transmittance measured by the spectrometer. The three contents, measured in percent, are determined by analytic chemistry.}\label{fig:tecator.data}
\end{figure}


\end{knitrout}

\index{neural network}
\index{functional regression}
\index{additive model}
\index{least squares!partial}
\index{kernel smoothing}
For our analyses and many others' in the literature, the first 172 observations in the data set are used as a training sample for model fitting, and the remaining 43 observations as a test sample to evaluate the predictive performance of the fitted model.
The focus here is to use the \pkg{iprior} package to fit several I-prior models to the Tecator data set, and calculate out-of-sample predictive error rates.
We compare the predictive performance of I-prior models against Gaussian process regression and the many other different methods applied on this data set.
These methods include neural networks \citep{thodberg1996review}, kernel smoothing \citep{ferraty2006nonparametric}, single and multiple index functional regression models \citep{chen2011single}, sliced inverse regression (SIR) and sliced average variance estimation (SAVE), multivariate adaptive regression splines (MARS), partial least squares (PLS), and functional additive model with and without component selection (FAM \& CSEFAM).
An analysis of this data set using the SIR and SAVE methods were conducted by  \citet{lian2014series}, while the MARS, PLS and (CSE)FAM methods were studied by \citet{zhu2014structured}.
\cref{tab:tecator} tabulates the all of the results from these various references.

Assuming a regression model as in \cref{eq:model2}, we would like to model the \code{fat} content $y_i$ using the spectral curves $x_i$.
Let $x_i(t)$ denote the absorbance for wavelength $t = 1,\dots,100$.
From \cref{fig:tecator.data}, it appears that the curves are smooth enough to be differentiable, and therefore it is reasonable to assume that they lie in the Sobolev-Hilbert space as discussed in \cref{sec:regfunctionalcov}.
We take first differences of the 100-dimensional matrix, which leaves us with the 99-dimensional covariate saved in the object named \code{absorp}.
The \code{fat} and \code{absorp} data have been split into \code{*.train} and \code{*.test} samples, as mentioned earlier.
Our first modelling attempt is to fit a linear effect by regressing the responses \code{fat.train} against a single high-dimensional covariate \code{absorp.train} using the linear RKHS and the direct optimisation method.

\begin{knitrout}
\definecolor{shadecolor}{rgb}{1, 1, 1}\color{fgcolor}\begin{kframe}
\singlespacing\begin{alltt}
\hlstd{R> }\hlcom{# Model 1: Canonical RKHS (linear)}
\hlstd{R> }\hlstd{(mod1} \hlkwb{<-} \hlkwd{iprior}\hlstd{(}\hlkwc{y} \hlstd{= fat.train, absorp.train))}
\end{alltt}
\begin{verbatim}
## iter   10 value 222.653144
## final  value 222.642108 
## converged
## Log-likelihood value: -445.2844 
## 
##     lambda        psi 
## 4576.86595    0.11576
\end{verbatim}
\end{kframe}
\end{knitrout}

Our second and third model uses polynomial RKHSs of degrees two and three, which allows us to model quadratic and cubic terms of the spectral curves respectively.
We also opted to estimate a suitable offset parameter, and this is called to \code{iprior()} with the option \code{est.offset = TRUE}.
Each of the two models has a single scale parameter, an offset parameter, and an error precision to be estimated.
The direct optimisation method has been used, and while both models converged regularly, it was noticed that there were multiple local optima that hindered the estimation (output omitted).

\begin{knitrout}
\definecolor{shadecolor}{rgb}{1, 1, 1}\color{fgcolor}\begin{kframe}
\singlespacing\begin{alltt}
\hlstd{R> }\hlcom{# Model 2: Polynomial RKHS (quadratic)}
\hlstd{R> }\hlstd{mod2} \hlkwb{<-} \hlkwd{iprior}\hlstd{(}\hlkwc{y} \hlstd{= fat.train, absorp.train,} \hlkwc{kernel} \hlstd{=} \hlstr{"poly2"}\hlstd{,}
\hlstd{+  }               \hlkwc{est.offset} \hlstd{=} \hlnum{TRUE}\hlstd{)}
\hlstd{R> }\hlcom{# Model 3: Polynomial RKHS (cubic)}
\hlstd{R> }\hlstd{mod3} \hlkwb{<-} \hlkwd{iprior}\hlstd{(}\hlkwc{y} \hlstd{= fat.train, absorp.train,} \hlkwc{kernel} \hlstd{=} \hlstr{"poly3"}\hlstd{,}
\hlstd{+  }               \hlkwc{est.offset} \hlstd{=} \hlnum{TRUE}\hlstd{)}
\end{alltt}
\end{kframe}
\end{knitrout}

Next, we attempt to fit a smooth dependence of fat content on the spectrometric curves using the fBm RKHS.
By default, the Hurst coefficient for the fBm RKHS is set to be 0.5.
However, with the option \code{est.hurst = TRUE}, the Hurst coefficient is included in the estimation procedure.
We fit models with both a fixed value for Hurst (at 0.5) and an estimated value for Hurst.
For both of these models, we encountered numerical issues when using the direct optimisation method.
The L-BFGS algorithm \index{L-BFGS algorithm} kept on pulling the hyperparameter towards extremely high values, which in turn made the log-likelihood value greater than the machine's largest normalised floating-point number (\code{.Machine$double.xmax = 1.797693e+308}).
% Investigating further, it seems that estimates at these large values give poor training and test error rates, though likelihood values here are high (local optima).
To circumvent this issue, we used the EM algorithm to estimate the fixed Hurst model, and the \code{mixed} method for the estimated Hurst model.
For both models, the \code{stop.crit} was relaxed and set to \code{1e-3} for quicker convergence, though this did not affect the predictive abilities compared to a more stringent \code{stop.crit}.

\begin{knitrout}
\definecolor{shadecolor}{rgb}{1, 1, 1}\color{fgcolor}\begin{kframe}
\singlespacing\begin{alltt}
\hlstd{R> }\hlcom{# Model 4: fBm RKHS (default Hurst = 0.5)}
\hlstd{R> }\hlstd{(mod4} \hlkwb{<-} \hlkwd{iprior}\hlstd{(}\hlkwc{y} \hlstd{= fat.train, absorp.train,} \hlkwc{kernel} \hlstd{=} \hlstr{"fbm"}\hlstd{,}
\hlstd{+  }                \hlkwc{method} \hlstd{=} \hlstr{"em"}\hlstd{,} \hlkwc{control} \hlstd{=} \hlkwd{list}\hlstd{(}\hlkwc{stop.crit} \hlstd{=} \hlnum{1e-3}\hlstd{)))}
\end{alltt}
\begin{verbatim}
## ==============================================
## Converged after 65 iterations.
## Log-likelihood value: -204.4592 
## 
##     lambda        psi 
##    3.24112 1869.32897
\end{verbatim}
\end{kframe}
\end{knitrout}
\begin{knitrout}
\definecolor{shadecolor}{rgb}{1, 1, 1}\color{fgcolor}\begin{kframe}
\singlespacing\begin{alltt}
\hlstd{R> }\hlcom{# Model 5: fBm RKHS (estimate Hurst)}
\hlstd{R> }\hlstd{(mod5} \hlkwb{<-} \hlkwd{iprior}\hlstd{(fat.train, absorp.train,} \hlkwc{kernel} \hlstd{=} \hlstr{"fbm"}\hlstd{,} \hlkwc{method} \hlstd{=} \hlstr{"mixed"}\hlstd{,}
\hlstd{+  }                \hlkwc{est.hurst} \hlstd{=} \hlnum{TRUE}\hlstd{,} \hlkwc{control} \hlstd{=} \hlkwd{list}\hlstd{(}\hlkwc{stop.crit} \hlstd{=} \hlnum{1e-3}\hlstd{)))}
\end{alltt}
\begin{verbatim}
## Running 5 initial EM iterations
## ======================================================================
## Now switching to direct optimisation
## iter   10 value 115.648462
## final  value 115.645800 
## converged
## Log-likelihood value: -231.2923 
## 
##    lambda     hurst       psi 
## 204.97184   0.70382   9.96498
\end{verbatim}
\end{kframe}
\end{knitrout}

Finally, we fit an I-prior model using the SE RKHS with lengthscale estimated.
Here we illustrate the use of the \code{restarts} option, in which the model is fitted repeatedly from different starting points.
In this case, eight random initial parameter values were used and these jobs were parallelised across the eight available cores of the machine.
The additional \code{par.maxit} option in the \code{control} list is an option for the maximum number of iterations that each parallel job should do.
We have set it to 100, which is the same number for \code{maxit}, but if \code{par.maxit} is less than \code{maxit}, the estimation procedure continues from the model with the best likelihood value.
We see that starting from eight different initial values, direct optimisation leads to (at least) two log-likelihood optima sites, $-231.5$ and $-680.5$.

\begin{knitrout}
\definecolor{shadecolor}{rgb}{1, 1, 1}\color{fgcolor}\begin{kframe}
\singlespacing\begin{alltt}
\hlstd{R> }\hlcom{# Model 6: SE kernel}
\hlstd{R> }\hlstd{(mod6} \hlkwb{<-} \hlkwd{iprior}\hlstd{(fat.train, absorp.train,} \hlkwc{est.lengthscale} \hlstd{=} \hlnum{TRUE}\hlstd{,}
\hlstd{+  }                \hlkwc{kernel} \hlstd{=} \hlstr{"se"}\hlstd{,} \hlkwc{control} \hlstd{=} \hlkwd{list}\hlstd{(}\hlkwc{restarts} \hlstd{=} \hlnum{TRUE}\hlstd{,}
\hlstd{+  }                                              \hlkwc{par.maxit} \hlstd{=} \hlnum{100}\hlstd{)))}
\end{alltt}
\begin{verbatim}
## Performing 8 random restarts on 8 cores
## ======================================================================
## Log-likelihood from random starts:
##     Run 1     Run 2     Run 3     Run 4     Run 5     Run 6     Run 7 
## -231.5440 -680.4637 -680.4637 -231.5440 -231.5440 -231.5440 -231.5440 
##     Run 8 
## -231.5440 
## Continuing on Run 6 
## final  value 115.771932 
## converged
## Log-likelihood value: -231.544 
## 
##      lambda lengthscale         psi 
##    96.11708     0.09269     6.15432
\end{verbatim}
\end{kframe}
\end{knitrout}



Predicted values of the test data is obtained using \code{predict()}.
An example for obtaining the first model's predicted values is shown below.
The \code{predict()} method for \code{ipriorMod} objects also return the test MSE if the vector of test data is supplied.

\begin{knitrout}
\definecolor{shadecolor}{rgb}{1, 1, 1}\color{fgcolor}\begin{kframe}
\singlespacing\begin{alltt}
\hlstd{R> }\hlkwd{predict}\hlstd{(mod1,} \hlkwc{newdata} \hlstd{=} \hlkwd{list}\hlstd{(absorp.test),} \hlkwc{y.test} \hlstd{= fat.test)}
\end{alltt}
\begin{verbatim}
## Test RMSE: 2.890353 
## 
## Predicted values:
##  [1] 43.607 20.444  7.821  4.491  9.044  8.564  7.935 11.615 13.807
## [10] 17.359
## # ... with 33 more values
\end{verbatim}
\end{kframe}
\end{knitrout}

These results are summarised in \cref{tab:tecator}.
For the I-prior models, a linear effect of the functional covariate gives a training RMSE of 2.89, which is improved by both the qudratic and cubic model.
The training RMSE is improved further by assuming a smooth RKHS of functions for $f$, i.e. the fBm and SE RKHSs.
When it comes to out-of-sample test error rates, the cubic model gives the best RMSE out of the I-prior models for this particular data set, with an RMSE of 0.58.
This is followed closely by the fBm RKHS with estimated Hurst coefficient (fBm-0.70) and also the fBm RKHS with default Hurst coefficient (fBm-0.50).
The best performing I-prior model is only outclassed by the neural networks of \citet{thodberg1996review}, who also performed model selection using automatic relevance determination (ARD).
The I-prior models also give much better test RMSE than Gaussian process regression.
\index{Gaussian process!regression}

% \renewcommand{\TPTminimum}{0.5\linewidth}
\newcolumntype{R}[1]{>{\raggedleft\arraybackslash}p{#1}}
\begin{table}[htbp]
\centering
\caption[A summary of the RMSE of prediction for the Tecator data set]{A summary of the root mean squared error (RMSE) of prediction for the I-prior models and various other methods in literature conducted on the Tecator data set. Values for the methods under \emph{Others} were obtained from the corresponding references cited earlier.}
\label{tab:tecator}
\begin{threeparttable}
% \begin{tabular}{@{}p{\textwidth}@{}}
\begin{tabular}{p{7cm}rr}
\toprule
\Bot &\multicolumn{2}{c}{RMSE} \\
\cline{2-3}
\Top Model & Train & Test \\
\midrule
\emph{I-prior} \\
\hspace{0.5em} Linear
& 2.89
& 2.89 \\
\hspace{0.5em} Quadratic
& 0.72
& 0.97 \\
\hspace{0.5em} Cubic
& 0.37
& 0.58
\\[0.5em]
\hspace{0.5em} Smooth (fBm-0.50)
& 0.00
& 0.68 \\
\hspace{0.5em} Smooth (fBm-0.70)
& 0.19
& 0.63 \\
\hspace{0.5em} Smooth (SE-0.09)
& 0.35
& 1.85 \\
\\
\emph{Gaussian process regression}\tnote{a}  \\
\hspace{0.5em} Linear
& 0.18
& 2.36 \\
\hspace{0.5em} Smooth (SE-7.28)
& 0.17
& 2.07 \\
\\
\emph{Others} \\
%\hspace{0.5em} Linear functional regression\tnote{a}       && 2.78 \\
%\hspace{0.5em} Quadratic functional regression\tnote{a}    && 0.80 \\
\hspace{0.5em} Neural network\tnote{b}                     && 0.36 \\
\hspace{0.5em} Kernel smoothing\tnote{c}                   && 1.49 \\
\hspace{0.5em} Single/multiple indices model\tnote{d}      && 1.55
\\[0.5em]
\hspace{0.5em} Sliced inverse regression                   && 0.90 \\
\hspace{0.5em} Sliced average variance estimation          && 1.70 \\
\hspace{0.5em} MARS\tnote{e}                               && 0.88
\\[0.5em]
\hspace{0.5em} Partial least squares\tnote{e}              && 1.01 \\
\hspace{0.5em} FAM\tnote{e}                                && 0.92 \\
\hspace{0.5em} CSEFAM\tnote{e}                             && 0.85 \\
\bottomrule
\end{tabular}
% \end{tabular}
\begin{tablenotes}\footnotesize
\item [a] GPR models were fit using \texttt{gausspr()} in \pkg{kernlab}.
\item [b] Neural network best results with automatic relevance determination (ARD) quoted.
\item [c] Data set used was a 160/55 training/test split.
\item [d] These are results of a leave-one-out cross-validation scheme.
\item [e] Data set used was an extended version with $n=240$, and a random 185/55 training/test split.
\end{tablenotes}
\end{threeparttable}
\end{table}

\subsection{Using the Nyström method}
\index{Nystrom method@Nyström method}
\index{fBm kernel/RKHS}

We investigate the use of the Nystr\"om method of approximating the kernel matrix in estimating I-prior models.
Let us revisit the data set generated by \cref{eq:examplesmoothingdata} described in \cref{sec:compareestimation}.
The features of this regression function are two large bumps at the centres of the mixed Gaussian pdfs, and also a small bump right after $x>4.5$ caused by the additional exponential function.
The true regression function tends to positive infinity as $x$ increases, and to zero as $x$ decreases.
Samples of $(x_i,y_i)$, $i=1,\dots,2000$ have been generated by the built-in \code{gen_smooth()} function, of which the first few lines of the data are shown below.

\begin{knitrout}
\definecolor{shadecolor}{rgb}{1, 1, 1}\color{fgcolor}\begin{kframe}
\singlespacing\begin{alltt}
\hlstd{R> }\hlstd{dat} \hlkwb{<-} \hlkwd{gen_smooth}\hlstd{(}\hlkwc{n} \hlstd{=} \hlnum{2000}\hlstd{,} \hlkwc{xlim} \hlstd{=} \hlkwd{c}\hlstd{(}\hlopt{-}\hlnum{1}\hlstd{,} \hlnum{5.5}\hlstd{),} \hlkwc{seed} \hlstd{=} \hlnum{1}\hlstd{)}
\hlstd{R> }\hlkwd{head}\hlstd{(dat)}
\end{alltt}
\begin{verbatim}
##            y         X
## 1  0.6803514 -2.608953
## 2  3.6747031 -2.554039
## 3 -1.1563508 -2.381275
## 4  2.2657657 -2.280259
## 5  2.5398243 -2.214122
## 6  1.2929592 -2.170532
\end{verbatim}
\end{kframe}
\end{knitrout}

One could fit the regression model using all available data points, with an I-prior from the fBm-0.5 RKHS of functions as follows (note that the \code{silent} option is used to suppress the output from the \code{iprior()} function):

\begin{knitrout}
\definecolor{shadecolor}{rgb}{1, 1, 1}\color{fgcolor}\begin{kframe}
\singlespacing\begin{alltt}
\hlstd{R> }\hlstd{(mod.full} \hlkwb{<-} \hlkwd{iprior}\hlstd{(y} \hlopt{~} \hlstd{X, dat,} \hlkwc{kernel} \hlstd{=} \hlstr{"fbm"}\hlstd{,}
\hlstd{+  }                    \hlkwc{control} \hlstd{=} \hlkwd{list}\hlstd{(}\hlkwc{silent} \hlstd{=} \hlnum{TRUE}\hlstd{)))}
\end{alltt}
\begin{verbatim}
## Log-likelihood value: -4355.075 
## 
##  lambda     psi 
## 2.30244 0.23306
\end{verbatim}
\end{kframe}
\end{knitrout}

To implement the Nystr\"om method, the option \code{nystrom = 50} was added to the  function call, which uses 50 randomly selected data points for the Nystr\"om approximation.

\begin{knitrout}
\definecolor{shadecolor}{rgb}{1, 1, 1}\color{fgcolor}\begin{kframe}
\singlespacing\begin{alltt}
\hlstd{R> }\hlstd{(mod.nys} \hlkwb{<-} \hlkwd{iprior}\hlstd{(y} \hlopt{~} \hlstd{X, dat,} \hlkwc{kernel} \hlstd{=} \hlstr{"fbm"}\hlstd{,} \hlkwc{nystrom} \hlstd{=} \hlnum{50}\hlstd{,}
\hlstd{+  }                   \hlkwc{control} \hlstd{=} \hlkwd{list}\hlstd{(}\hlkwc{silent} \hlstd{=} \hlnum{TRUE}\hlstd{)))}
\end{alltt}
\begin{verbatim}
## Log-likelihood value: -1945.33 
## 
##  lambda     psi 
## 1.64833 0.13538
\end{verbatim}
\end{kframe}
\end{knitrout}

\begin{knitrout}
\definecolor{shadecolor}{rgb}{1, 1, 1}\color{fgcolor}\begin{kframe}
\singlespacing\end{kframe}\begin{figure}[htbp]

{\centering \includegraphics[width=0.785\textwidth]{figure/04-nystrom_plot-1} 
\includegraphics[width=0.785\textwidth]{figure/04-nystrom_plot-2} 

}

\caption[Plot of predicted regression function for the full model (top) and the Nystr\"om approximated method (bottom)]{Plot of predicted regression function for the full model (top) and the Nystr\"om approximated method (bottom). For the Nystr\"om plot, the data points that were active are shown by circles with bold outlines.}\label{fig:nystrom.plot}
\end{figure}


\end{knitrout}
\begin{knitrout}
\definecolor{shadecolor}{rgb}{1, 1, 1}\color{fgcolor}\begin{kframe}
\singlespacing\begin{alltt}
\hlstd{R> }\hlkwd{get_time}\hlstd{(mod.full);} \hlkwd{get_size}\hlstd{(mod.full,} \hlstr{"MB"}\hlstd{);} \hlkwd{get_prederror}\hlstd{(mod.full)}
\end{alltt}
\begin{verbatim}
## 14.75346 mins
## 128.2 MB
## Training RMSE 
##      2.054232
\end{verbatim}
\begin{alltt}
\hlstd{R> }\hlkwd{get_time}\hlstd{(mod.nys);} \hlkwd{get_size}\hlstd{(mod.nys);} \hlkwd{get_prederror}\hlstd{(mod.nys)}
\end{alltt}
\begin{verbatim}
## 1.312222 secs
## 982.2 kB
## Training RMSE 
##      2.171928
\end{verbatim}
\end{kframe}
\end{knitrout}

The hyperparameters estimated for both models are slightly different.
The log-likelihood is also different, but this is attributed to information loss due to the approximation procedure.
Nevertheless, we see from \cref{fig:nystrom.plot} that the estimated regression functions are quite similar in both the full model and the approximated model.
The main difference is that the the Nystr\"om method was not able to extrapolate the right hand side of the plot well, because it turns out that there were no data points used from this region.
This can certainly be improved by using a more intelligent sampling scheme.
The full model took a little over 14 minutes to converge, while the Nystr\"om method took seconds without compromising too much on root mean squared error of predictions.
Storage savings is significantly higher with the Nystr\"om method as well.

\end{document}



\section{Conclusion}

The steps for I-prior modelling are essentially three-fold:
\begin{enumerate}
  \item Select an appropriate function space (equivalently, kernels) for which specific effects are desired on the covariates. 
%  Choices included a linear effect (canonical RKHS), a polynomial effect (polynomial RKKS), smoothing effect (fBm or SE RKHS), 
  \item Estimate the posterior regression function and optimise the hyperparameters, which include the RKHS scale parameter(s), error precision, and any other kernel parameters such as the Hurst index.
  \item Perform post-estimation procedures such as
  \begin{itemize}
    \item Posterior predictive checks;
    \item Model comparison via log-likelihood ratio tests/empirical Bayes factors; and
    \item Prediction of new data point.
  \end{itemize}
\end{enumerate}

\begin{figure}[t]
  \centering
  \includegraphics[width=0.85\textwidth]{figure/04-iprior_runtime}
  \caption[Average time taken to complete the estimation of an I-prior model.]{Average time taken to complete the estimation of an I-prior model (EM algorithm and direct optimisation) of varying sample sizes. The solid line represents actual timings, while the dotted lines are linear extrapolations.}
  \label{fig:ipriortime}
\end{figure}

The main sticking point with the estimation procedure is the involvement of the $n\times n$ kernel matrix, for which an inverse is needed.
This requires $O(n^2)$ storage and $O(n^3)$ computational time.
%The Nyström method of approximating the kernel matrix reduces complexity to $O(nm)$ storage and approximately $O(nm^2)$, and is highly advantageous if $m \ll n$.
The computational issue faced by I-priors are mirrored in GPR, so the methods to overcome these computational challenges in GPR can be explored further.
However, most efficient computational solutions exploit the nature of the SE kernel structure, which is the most common kernel used in GPR.
Nonetheless, we suggest the following as considerations for future work:
\begin{enumerate}
  \item \textbf{Sparse variational approximations}. \index{variational inference} Variational methods have seen an active development in recent times. By using inducing points \citep{titsias2009variational} or stochastic variational inference \citep{hensman2013gaussian}, such methods can greatly reduce computational storage and speed requirements. A recent paper by \citet{cheng2017variational} also suggests a variational algorithm with linear complexity for GPR-type models.
  \item \textbf{Accelerating the EM algorithm}. \index{EM algorithm!accelerating} Two methods can be explored. The first is called parameter-expansion EM algorithm (PXEM) by \citet{liu1998parameter}, which has been shown to be promising for random-effects type models. It involves correcting the M-step by a ``covariance adjustment'', so that extra information can be capitalised on to improve convergence rates. The second is a quasi-Newton acceleration of the EM algorithm as proposed by \citet{lange1995quasi}. A slight change to the EM gradient algorithm in the M-step steers the EM algorithm to the Newton-Raphson algorithm, thus exploiting the benefits of the EM algorithm in the early stages (monotonic increase in likelihood) and avoiding the pitfalls of Newton-Raphson (getting stuck in local optima). Both algorithms require an in-depth reassessment of the EM algorithm to be tailored to I-prior models.
\end{enumerate}

\hClosingStuffStandalone
\end{document}