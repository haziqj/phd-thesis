\documentclass[a4paper,showframe,11pt]{report}
\usepackage{standalone}
\standalonetrue
\ifstandalone
  \usepackage{../../haziq_thesis}
  \usepackage{../../haziq_maths}
  \usepackage{../../haziq_glossary}
  \usepackage{../../knitr}
  \addbibresource{../../bib/haziq.bib}
  \externaldocument{../01/.texpadtmp/introduction}
  \externaldocument{../02/.texpadtmp/chapter2}
  \externaldocument{../03/.texpadtmp/chapter3}
\fi

\begin{document}
\hChapterStandalone[4]{Regression modelling using I-priors}

In the previous chapter, we defined an I-prior for the normal regression model \eqref{eq:model1} subject to \eqref{eq:model1ass} and $f$ belonging to a reproducing kernel Hilbert or Krein space of functions.
We also saw how new function spaces can be constructed via the polynomial and ANOVA RKKS.
In this chapter, we shall describe various regression models, and connect them to an appropriate RKKS, so that an I-prior may be defined on it.
Methods for estimating I-prior models will also be described.
Finally, several examples of I-prior modelling are presented.

\section{Various regression models}\label{sec:various-regression}
%In the introductory chapter (Section 1.1), we described several interesting regression models.
The goal of this section is to formulate the I-prior model that describes each of these models.
This is done by carefully choosing the RKHS/RKKS $\cF$ of real functions over a set $\cX$ to which the regression function $f$ belongs.
Without loss of generality and for simplicity, assume a prior mean of zero for the I-prior distribution.

\subsection{Multiple linear regression}

Let $\cX\equiv \bbR^p$ be equipped with the regular Euclidean dot product, and $\cF_\lambda$ be the scaled canonical RKHS of functions over $\cX$ with kernel $h_\lambda(\bx,\bx') = \lambda \bx^\top\bx'$, for any two $\bx,\bx'\in\bbR^p$. 
Then, an I-prior on $f$ implies that 
\begin{align*}
  f(\bx_i) &= \sum_{j=1}^n \lambda \bx_i^\top\bx_j w_j \\
  &= \sum_{j=1}^n \lambda \left( \sum_{k=1}^p x_{ik}x_{jk} \right) w_j \\
  &= \beta_1 x_{i1} + \dots + \beta_p x_{ip},
\end{align*}
where each $\beta_k := \lambda \sum_{j=1}^n  x_{jk}w_j$.
This implies a multivariate normal prior distribution for the regression coefficients   
\begin{align}\label{eq:ipriorcanonical}
  \boldsymbol\beta := (\beta_1,\dots,\beta_p) \sim \N_p(\bzero, \lambda^2 \bX^\top \bPsi \bX),
\end{align}
where $\bX$ is the $n \times p$ design matrix for the covariates, excluding the column of ones at the beginning typically reserved for the intercept. 
As expected, the covariance matrix for $\boldsymbol\beta$ is recognised as the scaled Fisher information matrix for the regression coefficients.

%Functions in the canonical RKHS are of the form $f(\bx) = \bx^\top\boldsymbol\beta$.
If the covariates are not scaled similarly, then the values of $f$ are incoherent---if $x_1$  measures weight in kilograms and $x_2$ height in centimetres, what measurement does $\beta_1x_1 + \beta_2x_2$ represent?
To overcome this, one could decompose the regression function into
\[
  f(\bx_i) = f_1(x_{i1}) + \cdots + f_p(x_{ip}) 
\]
for which $f \in \cF_\lambda \equiv \cF_{\lambda_1}\oplus\cdots\oplus\cF_{\lambda_p}$, and $\cF_{\lambda_k}$, $k=1,\dots,p$ are unidimensional canonical RKHSs with kernels $h_{\lambda_k}(x_{ik},x_{jk}) = \lambda_k x_{ik} x_{jk}$.
%, where $\bar x_k = \sum_{i=1}^n x_{ik}/n$.
In effect, we now have $p$ scale parameters, one for each of the RKKSs associated with the $p$ covariates.
%Denote $\tilde x_{ik} = x_{ik}-\bar x_k$, the centred covariates.
The RKKS $\cF_\lambda$ therefore has kernel
\[
  h(\bx_i,\bx_j) = \sum_{k=1}^p \lambda_k  x_{ik} x_{jk},
\]
and hence each regression coefficient can now be written as $\beta_k =  \sum_{j=1}^n  \lambda_k x_{jk}w_j$.
Thus, the corresponding I-prior for $\boldsymbol{\beta}$ is
\[
  \boldsymbol{\beta} \sim \N_p(\bzero, \lambda^2 \bX^\top\bLambda \bPsi \bLambda \bX),
\]
with $\bLambda = \diag(\lambda_1,\dots,\lambda_p)$.
%Note that the overall effect $\beta_0$ can be treated in a number of ways, the simplest of which is as an intercept to be estimated.
Note that $\cF_\lambda$ can be seen as a special case of the ANOVA RKKS, in which only the main effects are considered, in which case the \emph{centred canonical RKHSs} should be considered instead.
This approach is disadvantageous when $p$ is large, in which case there would be numerous scale parameters to estimate.

\begin{remark}
  The I-prior for $\boldsymbol{\beta}$ in \eqref{eq:ipriorcanonical} bears resemblance to the $g$-prior \citep{zellner1986assessing}, and in fact, the $g$-prior can be interpreted as an I-prior if the inner product of $\cX$ is the Mahalonobis inner product.
  See Miscellanea \ref{misc:gprior} for a discussion.
\end{remark}

\subsection{Multilevel linear modelling}
\label{sec:multilevelmodels}

Let $\cX\equiv\bbR^p$, and suppose that alongside the covariates, there is information on group levels $\cM= \{1,\dots,m\}$ for each unit $i$.
That is, every observation for unit $i$ is known to belong to a specific group $j$, and we write $\bx_i^{(j)}$ to indicate this.
Let $n_j$ denote the sample size for cluster $j$, and the overall sample size be $n = \sum_{j=1}^m n_j$.
When modelled linearly with the responses $y_i^{(j)}$, the model is known as a multilevel (linear) model, although it is known by many other names: random-effects models, random coefficient models, hierarchical models, and so on.
As this model is seen as an extension of linear models, applications are plenty, especially in research designs for which the data varies at more than one level.

Consider a functional ANOVA decomposition of the regression function as follows:
\begin{align}\label{eq:anovamultilevel}
  f(\bx_i^{(j)}, j) = \alpha + f_1(\bx_i^{(j)}) + f_2(j) + f_{12}(\bx_i^{(j)}, j).  
\end{align}
To mimic the multilevel model, assume $f_1\in\cF_1$ the Pearson RKHS, $f_2\in\cF_2$ the centred canonical RKHS, and $f_{12} \in \cF_{12} = \cF_1 \otimes \cF_2$, the tensor product space of $\cF_1$ and $\cF_2$.
As we know, $\alpha$ is the overall intercept, and the varying intercepts are given by the function $f_2$.
While $f_1$ is the (main) linear effect of the covariates, $f_{12}$ provides the varying linear effect of the covariates by each group.
The I-prior for $f-\alpha$ is assumed to lie in the function space $\cF-\alpha$, which is an ANOVA RKKS with kernel
\[
  h_\lambda\big((\bx_i^{(j)},j),(\bx_i^{(j')},j')\big) = \lambda_1 h_1(\bx_i^{(j)},\bx_{i'}^{(j')}) + \lambda_2 h_2(j,j') + \lambda_1\lambda_2 h_1(\bx_i^{(j)},\bx_{i'}^{(j')})h_2(j,j'),
\]
with $h_1$ the centred canonical kernel and $h_2$ the Pearson kernel.
The reason for not including an RKHS of constant functions in $\cF$ is because the overall intercept is usually simpler to estimate as an external parameter (see \hltodo{Section X}).

We can show that the regression function \eqref{eq:anovamultilevel} corresponds to the standard way of writing the multilevel model, 
\[
  f(\bx_i^{(j)}, j) = \beta_0 + \bx_i^{(j)\top}\boldsymbol{\beta}_1 + \beta_{0j} + \bx_i^{(j)\top}\boldsymbol{\beta}_{1j}.   
\]
and determine the prior distributions on $(\beta_{0j},\boldsymbol{\beta}_{1j}^\top)^\top \in \bbR^{p+1}$.
For the interested reader, the details are in Miscellanea \ref{misc:multilevelmodels}.
The standard multilevel random effects assumption is that $(\beta_{0j},\boldsymbol{\beta}_{1j}^\top)^\top$ is normally distributed with mean zero and covariance matrix $\bPhi$.
In total, there are $p+1$ regression coefficients and $(p+1)(p+2)/2$ covariance parameters in $\Phi$ to be estimated.
In contrast, the I-prior model is parameterised by only two RKKS scale parameters---one for $\cF_1$ and one for $\cF_2$---and the error precision $\psi$.
While the estimation procedure for $\bPhi$ in the standard multilevel model can result in non-positive covariance matrices, the I-prior model has the advantage that positive definiteness is taken care of automatically\footnote{By virtue of the estimate of the regression function belonging to $\cF_n$, an RKHS with a positive definite kernel equal to the Fisher information for $f$.}.
%This is seen from the calculations for $\Var \beta_{0j}$, $\Var \boldsymbol\beta_{1j}$ and the respective covariances.
%An example of multilevel modelling using I-priors is given in \hltodo{Section 4.3.1}.

As a remark, the following regression functions are nested 
\begin{itemize}
  \item $f(\bx_i^{(j)}, j) = f_0 + f_1(\bx_i^{(j)}) + f_2(j)$ (random intercept model);  %\cF_0 \oplus \lambda_1\cF_1 \oplus \lambda_2\cF_2
  \item $f(\bx_i^{(j)}, j) = f_0 + f_1(\bx_i^{(j)})$ (linear regression model);
  \item $f(\bx_i^{(j)}, j) = f_0 + f_2(j)$ (ANOVA model);
  \item $f(\bx_i^{(j)}, j) = f_0 $ (intercept only model),
\end{itemize}
and thus one may compare likelihoods to ascertain the best fitting model.
In addition, one may add flexibility to the model in two possible ways:
\begin{enumerate}
  \item \textbf{More than two levels}. The model can be easily adjusted to reflect the fact that that the data is structured in a hierarchy containing three or more levels. For the three level case, let the indices $j\in\{1,\dots,m_1\}$ and $k\in\{1,\dots,m_2\}$ denote the two levels, and simply decompose the regression function accordingly:
  \begin{align*}
    \begin{gathered}
      f(\bx_i^{(j,k)}, j, k) = f_0 + f_1(\bx_i^{(j,k)}) + f_2(j) + f_3(k) + f_{12}(\bx_i^{(j,k)}, j) + f_{13}(\bx_i^{(j,k)}, k)\\ 
      + f_{23}(j, k) + f_{123}(\bx_i^{(j,k)}, j, k).
    \end{gathered}
  \end{align*}
  \item \textbf{Covariates not varying with levels}. Suppose now we would like to add covariates with a fixed effect to the model, i.e., covariates $\bz_i^{(j)}$ which are not assumed to affect the responses differently in each group. The regression function would be:
  \begin{align*}
    \begin{gathered}
      f(\bx_i^{(j)}, j, \bz_j) = f_0 + f_1(\bx_i^{(j)}) + f_2(j) + f_3(\bz_i^{(j)}) + f_{12}(\bx_i^{(j)}, j).
    \end{gathered}
  \end{align*}
  This can be seen as a limited functional ANOVA decomposition of $f$.
\end{enumerate}

\begin{remark}
  \setlength{\parindent}{0em}
  Indexing can be tricky, but we find the following helpful.
  Supposing $m=2$, and $n_1 = n_2 = 3$, then a typical panel data set looks like this:
  \begin{table}[H]
  \centering
  \begin{tabular}{lllrrr}
  \hline
  $y$ & $x$  & $z$ & $i$ & $j$ & $k$ \\ 
  \hline
  $y_{11}$      & $x_{11}$    & $z_{1}$ &1&1& 1 \\
  $y_{21}$      & $x_{21}$    & $z_{1}$ &2&1& 2 \\
  $y_{31}$      & $x_{31}$    & $z_{1}$ &3&1& 3 \\
  $y_{12}$      & $x_{12}$    & $z_{2}$ &1&2& 4 \\
  $y_{22}$      & $x_{22}$    & $z_{2}$ &2&2& 5 \\
  $y_{32}$      & $x_{32}$    & $z_{2}$ &3&2& 6 \\  
  \hline
  \end{tabular}
  \end{table}
  \vspace{-1.5em}
  The $y$'s are the responses, $x$'s covariates, and $z$'s group-level covariates.
  If $\iota:(i,j)\mapsto k$ is a function which maps the dual index set $(i,j)$ to the single index set $k\in\{1,\dots,n\}$, then the multilevel regression function can be expressed as the regression function in model \eqref{eq:model1}.
\end{remark}


\subsection{Longitudinal modelling}

Longitudinal or panel data observes covariate measurements $x_i\in\cX$ and responses $y_i(t)\in\bbR$ for individuals $i=1,\dots,n$ across a time period $t \in \{1,\dots,T\} =: \cT$. 
Often, the time indexing set $\cT$ may be unique to each individual $i$, so measurements for unit $i$ happens across a time period $\{t_{i1},\dots,t_{iT_i} \} =: \cT_i$---this is known as an unbalanced panel.
It is also possible that covariate measurements vary across time too, so appropriately they are denoted $x_i(t)$.
For example, $x_i(t)$ could be repeated measurements of the variable $x_i$ at time point $t\in\cT_i$.
The relationship between the response variables $y_i(t)$ at time $t\in\cT_i$ is captured through the equation
\[
  y_i(t) = f\big(x_i, t \big) + \epsilon_{i}(t)
\]
where the distribution of $\bepsilon_i = \big(\epsilon_i(t_{i1}),\dots,\epsilon_i(t_{iT_i}) \big)^\top$ is Gaussian with mean zero and covariance matrix $\bPsi_i$.
Assuming $\bPsi_i=\psi_i\bI_{T_i}$ or even $\bPsi_i=\psi\bI_{T_i}$ are perfectly valid choices, even though this seemingly ignores any time dependence between the observations.
In reality, the I-prior induces time dependence of the observations via the kernels in the prior covariance matrix for $f$.
Additionally, the random vectors $\bepsilon_i$ and $\bepsilon_{i'}$ are assumed to be independent for any two distinct $i,i'\in\{1,\dots,n\}$.
%, in which case, $y_i(t)$ are viewed as multidimensional responses.

Using the functional ANOVA decomposition on the regression function, we obtain
\begin{align}\label{eq:longitudinalanova}
  f(x_i,t) = f_0 + f_1(x_i) + f_2(t) + f_{12}(x_i,t),
\end{align}
where $f_0$ is an overall constant, $f_1\in\cF_1$, $f_2\in\cF_2$, and $f_{12}\in\cF_1\otimes\cF_2$.
Choices for $\cF_1$ and $\cF_2$ are plentiful.
In fact, any of the RKHS/RKKS described in Chapter 3 can be used to either model a linear dependence (canonical RKHS), nominal dependence (Pearson RKHS), polynomial dependence (polynomial RKKS) or smooth dependence (fBm or SE RKHS) on the $x_i$'s and $t$'s on $f$.

\begin{remark}
  Although \eqref{eq:longitudinalanova} is a special case of the multilevel model decomposition \eqref{eq:anovamultilevel} for which $x_i = x_i(t)$ (time-varying covariates), it is different to how longitudinal models are normally treated using a mixed effects model.
  As a multilevel model, longitudinal models treat the individuals as the groups or clusters (level two), and the time points as the various measurements within the clusters (level one).
\end{remark}

\subsection{Smoothing models}

Single- and multi-variable smoothing models can be fitted under the I-prior methodology using the fBm RKHS.
In standard kernel based smoothing methods, the squared exponential kernel is often used, and the corresponding RKHS contains analytic functions.
There are several attractive properties of using the fBm RKHS, and for one-dimensional smoothing, these are discussed below.

Assume that, up to a constant, the regression function lies in the scaled, centred fBm RKHS $\cF$ of functions over $\cX \equiv \bbR$ with Hurst index $1/2$.
Thus, with a centring with respect to the empirical distribution $\Prob_n$ of $\{x_1,\dots,x_n\}$ and using the absolute norm on $\bbR$, $\cF$ has kernel
\[
  h_\lambda(x,x') = \frac{\lambda}{2n^2} \sum_{i=1}^n\sum_{j=1}^n \big( \abs{x-x_i} + \abs{x'-x_j} - \abs{x-x'} - \abs{x_i-x_j} \big).
\]
According to \citet[Section 10]{van2008reproducing}, $\cF$ contains absolutely continuous functions possessing a square integrable weak derivative satisfying $f(0)=0$.
The norm is given by $\norm{f}^2_\cF = \int \dot f^2 \d x$.
The posterior mean of $f$ based on an I-prior is then a (one-dimensional) smoother for the data.
For $f$ of the form $f = \sum_{i=1}^n h(\cdot,x_i)w_i$, i.e., $f\in\cF_n$, the finite subspace of $\cF$ as in \hltodo{Section}, then \citet{bergsma2017} shows that $f$ can be represented as 
\begin{align}\label{eq:ipriorbrownianbridge}
  f(x) = \int_{-\infty}^x \beta(t) \dint t
\end{align}
where 
\begin{align}
  \beta(t) = \sum_{i:x_i \leq t} w_i =  \frac{f(x_{i_t + 1}) - f(x_{i_t})}{x_{i_t + 1} - x_{i_t}}
\end{align}
with $i_t = \max_{x_i \leq t} i$.
Under the I-prior with an iid assumption on the errors, the $w_i$'s are zero mean normal random variables with variance $\psi$, so that $\beta$ as defined above is an ordinary Brownian bridge with respect to the empirical distribution $\Prob_n$.
The I-prior for $f$ is piecewise linear with knots at $x_1,\dots,x_n$, and the same holds true for the posterior mean.
The implication is that the I-prior automatically adapts to irregularly spaced $x_i$: in any region where there are no observations, the resulting smoother is linear.
This is explained by the reduced Fisher information about the derivative of the regression curve in regions with no observation.

In \citep{bergsma2017}, it is stated that the covariance function for $\beta$ is 
\[
  \Cov\big(\beta(x),\beta(x') \big) = n\big( \min\{\Prob_n(X < x), \Prob_n(X_n < x') \} -  \Prob_n(X < x) \Prob_n(X_n < x') \big)
\]
From this, notice that $\Var \beta(x) = \Prob_n(X_n < x)\big(1 - \Prob_n(X_n < x) \big)$, which shows an automatic boundary correction: close to the boundary there is little Fisher information on the derivative of the regression function $\beta(x)$, so the prior variance is small.
This will lead to more shrinkage of the posterior derivative of $f$ towards the derivative of the prior mean $f_0$.

%\[
%  \norm{f}^2_\cF = \sum_{i=1}^{n-1} \frac{\big( f(x_{i+1}) - f(x_i) \big)^2}{x_{i+1} - x_i}
%\]
%because $\cF_n$ is the set of functions which integrate to zero and are piecewise linear with knots at $x_1,\dots,x_n$.
%Assuming $x_1\leq x_2\leq \cdots \leq x_n$, then it can be shown that
%\[
%  w_1 = \frac{f(x_2) - f(x_1)}{x_2 - x_1}
%\]

%With $w_k\iid\N(0,\psi)$, functions in $\cF$ are of the form
%\begin{align*}
%  f(x) 
%  &= \sum_{k=1}^n h(x,x_k)w_k \\
%  &= \sum_{k=1}^n \left( \frac{\lambda}{2n^2} \sum_{i=1}^n\sum_{j=1}^n \big( \abs{x-x_i} + \abs{x_k-x_j} - \abs{x-x_k} - \abs{x_i-x_j} \big) \right) w_k.
%\end{align*}
%The derivative of $f(x)$ with respect to $x$ is
%\begin{align*}
%  \frac{\d }{\d x}f(x) 
%  &= \frac{\lambda}{2n^2} \sum_{k=1}^n  \sum_{i=1}^n\sum_{j=1}^n \left( \frac{\d}{\d x}\abs{x-x_i}  - \frac{\d}{\d x}\abs{x-x_k}  \right) w_k \\
%  &= \frac{\lambda}{2n^2} \sum_{k=1}^n  \sum_{i=1}^n\sum_{j=1}^n \left( \sign(x-x_i)  - \sign(x-x_k)  \right) w_k \\
%  &= \frac{\lambda}{2n} \sum_{k=1}^n  \sum_{i=1}^n           \left( \sign(x-x_i)  - \sign(x-x_k)  \right) w_k \\
%\end{align*}

Another advantage of the I-prior methodology is the ability to fit single or multidimensional smoothing models with just two parameters to be estimated: the RKHS scale parameter $\lambda$ and the error precision $\bPsi$.
The Hurst parameter $\gamma \in (0,1)$ of the fBm RKHS can also be treated as a free parameter for added flexibility, but for most practical applications, we find that the default setting of $\gamma = 1/2$ performs sufficiently well.

\begin{remark}
  From \eqref{eq:ipriorbrownianbridge}, the prior process for $f$ is thus an integrated Brownian bridge. 
  This shows a close relation with cubic spline smoothers, which can be interpreted as the posterior mean when the prior is an integrated Wiener process \citep{wahba1990spline}.
  Unlike I-priors however, cubic spline smoothers do not have automatic boundary corrections, and typically the additional assumption is made that the smoothing curve is linear at the boundary knots.
\end{remark}

\subsection{Regression with functional covariates}
\label{sec:regfunctionalcov}

Suppose that we have functional covariates $x$ in the real domain, and that $\cX$ is a set of differentiable functions.
If so, it is reasonable to assume that $\cX$ is a Hilbert-Sobolev space with inner product
%
\[
  \langle x,x' \rangle_\cX = \int \dot{x}(t) \dot{x}'(t) \dint t,
\]
%
so that we may apply the linear, fBm or any other kernels which make use of inner products by making use of the polarisation identity.
Furthermore, let $z \in \bbR^T$ be the discretised realisation of the function $x \in \cX$ at regular intervals $t = 1,\dots,T$. Then
%
\[
  \langle x,x' \rangle_\cX \approx \sum_{t=1}^{T-1} (z_{t+1} - z_t)(z'_{t+1} - z_t').
\]
%
For discretised observations at non-regular intervals $\{t_1,\dots,t_T\}$ then a more general formula to the above one might be used, for instance,
\[
  \langle x,x' \rangle_\cX \approx \sum_{i=1}^{T-1} \frac{(z_{t_{i+1}} - z_{t_i})(z'_{t_{i+1}} - z_{t_i}')}{t_{i+1} - t_{i}}.
\]

%\subsection{Structural equation modelling}
%
%Let $y_i = (y_{i1},\dots,y_{ip})$, $i=1,\dots,n$ and $j=1,\dots,p$.
%Here, $y_{ij}$ denotes the $j$th item measurement for the $i$th individual.
%In a one factor model, this measurement is assumed to rely on an `item intercept', and individual level `factors', plus an error term which depends on the items.
%Specifically, the one factor model is
%\begin{align}
%  y_{ij} = \mu(j) + f(i,j) + \epsilon_{ij}  
%\end{align}
%with $\epsilon_{ij}\sim\N(0,\psi_j)$.







\section{Estimation}
After choosing an appropriate function space...
What are the kernel parameters? 
State the model $y = \alpha + \sum h w + e$.
Goal is to estimate posterior regression function.

After imposing an I-prior on the regression function, the interest is then to obtain the posterior distribution of the regression function.
This has been described in Chapter 1, and repeated here for convenience.
In particular, the posterior distribution for the regression function of the form $f(x) = \sum_{i=1}^n h(x,x_i)w_i$ merely depends on the posterior distribution of $  \bw := (w_1,\dots,w_n)^\top$, which is $  \bw|\by \sim \N_n(\wtilde, )$, where
\begin{align*}
  \begin{gathered}
    \wtilde = 
  \end{gathered}
\end{align*}

explain package


\subsection{The intercept}

Given the regression model \eqref{eq:model1} subject to an I-prior \eqref{eq:model1ass}, the marginal likelihood of the intercept $\alpha$ (after integrating out the I-prior) can be maximised with respect to $\alpha$, which yields the sample mean for $y$ as the ML estimate for intercept.

\subsection{Direct optimisation}

The kernel parameter $\eta$ and the error precision $\psi$ (which we collectively refer to as the model hyperparameters of the covariance kernel $\theta$) can be estimated in several ways.
One of these is direct optimisation of the marginal log-likelihood---the most common method in the Gaussian process literature.
%
\begin{align*}
  \log L(\theta)
  &= \log \int p(\by|\bff)p(\bff) \d \bff \\
  &= -\half[n]\log 2\pi - \half\log\vert \bSigma_\theta \vert - \half \by^\top \bSigma_\theta^{-1} \by
\end{align*}
%
where $\bSigma_\theta = \psi\bH_\eta^2 + \psi^{-1}\bI_n$.
This is typically done using conjugate gradients with a Cholesky decomposition on the covariance kernel to maintain stability, but the \pkg{iprior} package opts for an eigendecomposition of the kernel matrix (Gram matrix) $\bH_{\eta} = \bV\cdot\text{diag}(u_1,\dots,u_n)\cdot\bV^\top$ instead.
Since $\bH_{\eta}$ is a symmetrix matrix, we have that $\bV\bV^\top = \bI_n$, and thus
%
\[
  \bSigma_\theta = \bV \cdot \text{diag} (\psi u_1^2 + \psi^{-1},\dots,\psi u_n^2 + \psi^{-1}) \cdot \bV^\top
\]
%
for which the inverse and log-determinant is easily obtainable.
This method is relatively robust to numerical instabilities and is better at ensuring positive definiteness of the covariance kernel.
The eigendecomposition is performed using the \pkg{Eigen} \proglang{C++} template library and linked to \pkg{iprior} using \pkg{Rcpp} \citep{eddelbuettel2011rcpp}.
The hyperparameters are transformed by the \pkg{iprior} package so that an unrestricted optimisation using the quasi-Newton L-BFGS algorithm provided by \code{optim()} in \proglang R.
Note that minimisation is done on the deviance scale, i.e., minus twice the log-likelihood.
The direct optimisation method can be prone to local optima, in which case repeating the optimisation at different starting points and choosing the one which yields the highest likelihood is one way around this.

\subsection{Expectation-maximisation algorithm}

Alternatively, the expectation-maximisation (EM) algorithm may be used to estimate the hyperparameters, in which case the I-prior formulation in \eqref{eq:ipriorre} is convenient.
Substituting this into \eqref{eq:linmod} we get something that resembles a random effects model.
By treating the $w_i$ as ``missing'', the $t$th iteration of the E-step entails computing
%
\begin{align}
  Q(\theta) = \E \left[ \log p(\by, \bw | \theta) \big\vert \by,\theta^{(t)} \right].
\end{align}
%
As a consequence of the properties of the normal distribution, the required joint and posterior distributions $p(\by, \bw)$ and $p(\bw | \by)$ are easily obtained.
The M-step then maximises the $Q$ function above, which boils down to solving the first order conditions
%
\begin{align}
  \frac{\partial Q}{\partial\eta}
  &= -\half \tr \left(\frac{\partial \bSigma_\theta}{\partial\eta} \tilde\bW^{(t)} \right) + \psi \cdot \by ^\top \frac{\partial \bH_\eta}{\partial\eta} \tilde\bw^{(t)} \label{eq:emtheta} \\
  \frac{\partial Q}{\partial\psi}
  &= -\half \by^\top\by - \tr \left(\frac{\partial \bSigma_\theta}{\partial\psi} \tilde\bW^{(t)} \right) + \by^\top \bH_\eta \tilde\bw^{(t)} \label{eq:empsi}
\end{align}
%
equated to zero.
Here, $\tilde\bw$ and $\tilde\bW$ are the first and second posterior moments of $\bw$.
The solution to \eqref{eq:empsi} can be found in closed-form, but not necessarily for \eqref{eq:emtheta}.
In cases where closed-form solutions exist, then it is just a matter of iterating the update equations until a suitable convergence criterion is met (e.g. no more sizeable increase in successive log-likelihood values).
In cases where closed-form solutions do not exist for $\theta$, the $Q$ function is again optimised with respect to $\theta$ using the L-BFGS algorithm.

The EM algorithm is more stable than direct maximization, and is especially suitable if there are many scale parameters. However, it is typically slow to converge.
The \pkg{iprior} package provides a method to automatically switch to the direct optimisation method after running several EM iterations.
This then combines the stability of the EM with the speed of direct optimisation.

\subsection{Markov chain Monte Carlo methods}

For completeness, it should be mentioned that a full Bayesian treatment of the model is possible, with additional priors on the hyperparameters set.
Markov chain Monte Carlo (MCMC) methods can then be employed to sample from the posteriors of the hyperparameters, with point estimates obtained using the posterior mean or mode, for instance.
Additionally, the posterior distribution encapsulates the uncertainty about the parameter, for which inference can be made.
Posterior sampling can be done using Gibbs-based methods in \pkg{WinBUGS} \citep{lunn2000winbugs} or \pkg{JAGS} \citep{plummer2003jags}, and both have interfaces to \proglang{R} via \pkg{R2WinBUGS} \citep{sturtz2005r2winbugs} and \pkg{runjags} \citep{denwood2016runjags} respectively.
Hamiltonian Monte Carlo (HMC) sampling is also a possibility, and the \proglang{Stan} project \citep{carpenter2016stan} together with the package \pkg{rstan} \citep{rstan}  makes this possible in \proglang{R}.
All of these MCMC packages require the user to code the model individually, and we are not aware of the existence of MCMC-based packages which are able to estimate GPR models.
This makes it inconvenient for GPR and I-prior models, because in addition to the model itself, the kernel functions need to be coded as well and ensuring computational efficiency would be a difficult task.
Note that this full Bayesian method is not implemented in \pkg{iprior}, but described here for completeness.

\subsection{Comparison of estimation methods}

Running example: smoothing in one dimension.
Data simulated, what are the true parameters?
Run three methods of estimation, compare solutions, bias, MSE of prediction.


\section{Computational considerations}
Computational complexity for estimating I-prior models (and in fact, for GPR in general) is dominated by the inversion (by way of eigendecomposition in our case) of the $n \times n$ matrix $\bSigma_\theta = \bH_\eta\bPsi\bH_\eta + \bPsi^{-1}$, which scales as $O(n^3)$ in time.
%Inversion by way of the eigendecomposition of $\bH_\eta$ is $O(n^3)$.
For the direct optimisation method, this matrix inversion is called when computing the log-likelihood, and thus must be computed at each Newton step.
For the EM algorithm, this matrix inversion appears when calculating $\tilde \bw$ and $\tilde \bW$, the first and second posterior moments of the I-prior random effects.
Furthermore, storage requirements for I-priors models are similar to that of GPR models, which is $O(n^2)$.
In what follows, assumptions \ref{ass:A1}--\ref{ass:A3} hold.

\subsection[The Nystrom approximation]{The Nyström approximation}

The shared computational issues of I-prior and GPR models allow us to delve into machine learning literature, which is rich in ways to resolve these issue, as summarised by \citet{quinonero2005unifying}.
One such method is to exploit low rank structures of kernel matrices.
The idea is as follows.
Let $\bQ$ be a matrix with rank $q < n$, and suppose that $\bQ\bQ^\top$ can be used sufficiently well to represent the kernel matrix $\bH_\eta$.
Then
%
\[
  (\psi\bH_\eta^2 + \psi^{-1}\bI_n)^{-1} \approx
  \psi\left[
  \bI_n -
  \bQ\left( \big(\psi^2\bQ^\top\bQ\big)^{-1} +\bQ^\top\bQ \right)^{-1} \bQ^\top
  \right],
\]
%
obtained via the Woodbury matrix identity, is potentially a much cheaper operation which scales $O(nq^2)$: $O(q^3)$ to do the inversion, and $O(nq)$ to do the multiplication (because typically the inverse is premultiplied to a vector).
When using the linear kernel for a low-dimensional covariate then the above method is exact.
This fact is clearly demonstrated by the equivalence of the $p$-dimensional linear model implied by \eqref{eq:ipriorcanonical} with the $n$-dimensional I-prior model using the canonical RKHS.
If $p \ll n$ then certainly using the linear representation is much more efficient.

However, other interesting kernels such as the fractional Brownian motion (fBm) kernel or the squared exponential kernel results in kernel matrices which are full rank.
An approximation to the kernel matrix using a low-rank matrix is the Nystr\"om method \citep{williams2001using}.
The theory has its roots in approximating eigenfunctions, but this has since been adopted to speed up kernel machines.
The main idea is to obtain an (approximation to the true) eigendecomposition of $\bH_\eta$ based on a small subset $m \ll n$ of the data points.

Let $\bH_\eta = \bV\bU\bV^\top = \sum_{i=1}^n u_i \bv_i \bv_i^\top$ be the (orthogonal) decomposition of the symmetric matrix $\bH_\eta$.
As mentioned, avoiding this expensive $O(n^3)$ eigendecomposition is desired, and this is achieved by selecting a subset $\cM$ of size $m$ of the $n$ data points $\{1,\dots,n \}$, so that $\bH_\eta$ may be approximated using the rank $m$ matrix $\bH_\eta \approx \sum_{i\in\cM} \tilde u_i \tilde\bv_i\tilde\bv_i^\top$.
Without loss of generality, reorder the rows and columns of $\bH_\eta$ so that the data points indexed by $\cM$ are used first:
%
\[
  \bH_\eta =
  \begin{pmatrix}
    \bA_{m\times m}         & \bB_{m \times (n-m)} \\
    \bB_{m \times (n-m)}^\top  & \bC_{(n-m) \times (n-m)} \\
  \end{pmatrix}.
\]
%
In other words, the data points indexed by $\cM$ forms the smaller $m\times m$ kernel matrix $\bA$. 
Let $\bA = \bV_m\bU_m\bV_m^\top = \sum_{i=1}^m u_i^{(m)}\bv_i^{(m)}\bv_i^{(m)\top}$ be the eigendeceomposition of $\bA$.
The Nyström method provides the formulae for $\tilde u_i$ and $\tilde\bv_i$ \citep[§8.1, equations 8.2 and 8.3]{rasmussen2006gaussian} as
\begin{align*}
  \tilde u_i &:= \frac{n}{m} u_i^{(m)} \in \bbR \\
  \tilde \bv_i &:= \sqrt{\frac{m}{n}} \frac{1}{u_i^{(m)}}
  \begin{pmatrix}
    \bA & \bB
  \end{pmatrix}^\top
  \bv_i^{(m)} \in \bbR^n.
\end{align*}
Denoting $\bU_m$ as the diagonal matrix of eigenvalues $u_1^{(m)},\dots,u_m^{(m)}$, and $\bV_m$ the corresponding matrix of eigenvectors $\bv_i^{(m)}$, we have
\[
  \bH_\eta \approx
  {\color{gray}
  \overbrace{\color{black}
  \begin{pmatrix}
    \bV_m \\
    \bB^\top\bV_m\bU_m^{-1}
  \end{pmatrix}
  }^{\bar\bV}
  }
  \bU_m
  {\color{gray}
  \overbrace{\color{black}
  \begin{pmatrix}
    \bV_m^\top & \bU_m^{-1}\bV_m^\top\bB
  \end{pmatrix}
  }^{\bar\bV^\top}
  }.
\]
Unfortunately, it may be the case that $\bar\bV\bar\bV^\top \neq \bI_n$, while orthogonality is crucial in order to easily calculate the inverse of $\bSigma_\theta$.
An additional step is required to obtain an orthogonal version of the Nyström decomposition, as studied by \citet{fowlkes2001efficient}.
 Let $\bK = \bA + \bA^{-\half}\bB^\top\bB\bA^{-\half}$, where $\bA^{-\half} = \bV_m\bU_m^{-\half}\bV_m$, and obtain the eigendecomposition of this $m\times m$ matrix $\bK = \bR\hat\bU\bR^\top$.
 Defining
 \[
   \hat\bV = 
   \begin{pmatrix}
     \bA \\
     \bB^\top
   \end{pmatrix}
   \bA^{-\half}\bR\hat\bU^{-\half} \in \bbR^n \times \bbR^m,
 \]
 then 
 \hltodo[Attempt to prove this.]{we have that $\bH_\eta \approx \hat\bV\hat\bU\hat\bV^\top$ such that $\hat\bV\hat\bV^\top = \bI_n$}.
 Estimating I-prior models with the Nystr\"om method including the orthogonalisation step takes roughly $O(nm^2)$ time and $O(nm)$ storage.
 
The issue of selecting the subset $\cM$ remains.
The simplest method, and that which is implemented in the \pkg{iprior} package, 
would be to uniformly sample a subset of size $m$ from the $n$ points.
Although this works well in practice, the quality of approximation might suffer if the points do not sufficiently represent the training set.
In this light, greedy approximations have been suggested to select the $m$ points, so as to reduce some error criterion relating to the quality of approximation.
For a brief review of more sophisticated methods of selecting $\cM$, see \citet[§8.1, pp. 173--174]{rasmussen2006gaussian}.

%\begin{align*}
%  \bH_\eta 
%  &\approx \sum_{i\in\cM} \tilde u_i \tilde\bv_i\tilde\bv_i^\top \\
%  &= \cancel{\frac{n/m}{\sqrt{n/m \times n / m}}}
%  \begin{pmatrix}
%    (\bV_m\bU\bV_m^\top)\bV_m\bU^{-1} \\
%    \bB^\top\bV_m\bU^{-1} \\
%  \end{pmatrix} 
%  \bU_m 
%  \begin{pmatrix}
%    (\bV_m\bU\bV_m^\top)\bV_m\bU^{-1} 
%    & \bB^\top\bV_m\bU^{-1} 
%  \end{pmatrix} \\
%  &=   
%  \begin{pmatrix}
%    \bV_m \\
%    \bB^\top\bV_m\bU_m^{-1}
%  \end{pmatrix}
%  \bU_m
%  \begin{pmatrix}
%    \bV_m^\top & \bU_m^{-1}\bV_m^\top\bB
%  \end{pmatrix}
%\end{align*}

\subsection{An efficient EM algorithm}
\label{sec:efficientEM1}

The evaluation of the $Q$ function in \eqref{eq:QfnEstep} is $O(n^3)$, because a change in the values of $\theta$ requires evaluating $\bSigma_\theta = \psi\bH_\eta^2 + \psi^{-1}\bI_n$, for which squaring $\bH_\eta$ takes the bulk of the computational time.
In this section, we describe an efficient method of evaluating $Q$ if the I-prior model only involves estimating the RKHS scale parameters and the error precision under assumptions \ref{ass:A1}--\ref{ass:A3}.

%Separate the RKHS scale parameters $\lambda$ from the other kernel parameters $\xi$ such as the Hurst index of the fBm RKHS, lengthscale of the SE RKHS, and offset parameter of the polynomial RKKS, and write $\theta = \{\lambda_1,\dots,\lambda_p,\psi,\xi \}$.
Corresponding to $p$ building block RKHSs $\cF_1,\dots,\cF_p$ of functions over $\cX_1,\dots,\cX_p$, there are $p$ scale parameters $\lambda_1,\dots,\lambda_p$ and reproducing kernels $h_1,\dots,h_p$.
Write $\theta = \{\lambda_1,\dots,\lambda_p,\psi\}$.
The most common modelling scenarios that will be encountered are listed below:
\begin{enumerate}
  \item \textbf{Single scale parameter}. With $p=1$, $f\in\cF\equiv \lambda_1\cF_1$ of functions over a set $\cX$. $\cF$ may be any of the building block RKHSs. Note that $\cX_1$ itself may be more than one-dimensional. The kernel over $\cX_1\times\cX_1$ is therefore
  \[
    h_\lambda = \lambda_1 h_1.
  \]
  \item \textbf{Multiple scale parameters}. Here, $\cF$ is a RKKS of functions $f:\cX_1\times\cdots\times\cX_p\to\bbR$, and thus $\cF\equiv \lambda_1\cF_1 \oplus \cdots \oplus \lambda_p\cF_p$, where each $\cF_k$ is one of the building block RKHSs. The kernel is
  \[
    h_\lambda = \lambda_1 h_1 + \cdots + \lambda_p h_p.
  \]
  \item \textbf{Multiple scale parameters with level-2 interactions}. This occurs commonly with multilevel and longitudinal models. Suppose that $\cX_1$ is the set of `levels' and there are $p-1$ covariate sets $\cX_k$, $k=2,\cdots,p$. The function space $\cF$ is a special case of the ANOVA RKKS containing only main and two-way interaction effects, and its kernel is
  \[
    h_\lambda = \sum_{j=1}^p \lambda_j h_j + \sum_{j < k} \lambda_j\lambda_k h_j h_k,
  \]
  where $\cF_1$ is the Pearson RKHS, and the remaining are any of the building block RKHSs.
  \item \textbf{Polynomial RKKS}. When using the polynomial RKKS of degree $d$ to incite a polynomial relationship of the covariate set $\cX_1$ on the function $f\in\cF$ (excluding an intercept), then the kernel of $\cF$ is
  \[
    h_\lambda = \sum_{k=1}^d b_k \lambda_1^k h_1^k.
  \]
  where $b_k = \frac{d!}{k!(d-k)!}$, $k=1,\dots,d$ are constants.
\end{enumerate}
Of course, many other models are possible, such as the ANOVA RKKS with all $p$ levels of interactions.
What we realise is that any of these scenarios are simply a sum-product of a manipulation of the set of scale parameters $\lambda = \{\lambda_1,\dots,\lambda_p\}$ and the set of kernel functions $h = \{h_1,\dots,h_p\}$.
%Furthermore, scenarios 1--3 are special cases of the ANOVA RKKS excluding the grand mean\footnote{As discussed, for simplicity the RKHS of constant functions is ignored and the model includes an intercept to be estimated instead.}: in 1. and 2., $\cF$ is the ANOVA RKKS of main effects only, and in 3., $\cF$ is the ANOVA RKKS of main effects and level-two interactions.

Let us be more concrete about what we mean by `manipulation' of the sets $\lambda$ and $h$.
Define an `instruction operator' which expands out both sets identically as required by the modelling scenario.
Computationally speaking, this instruction could be as simple as a list containing the indices to multiply out.
For the four scenarios above, the list $\cQ$ is
\begin{enumerate}
  \item $\cQ = \big\{ \{1\} \big\}$.
  \item $\cQ = \big\{ \{1\},\dots,\{p\}\big\}$.
  \item  $\cQ = \big\{ \{1\},\dots,\{p\},\{1,2\},\dots, \{p-1,p\}\big\}$.
  \item $\cQ = \big\{ \{1\},\{1,1\},\dots, \{\myoverbrace{1,\dots,1}{d}\}\big\}$.
\end{enumerate}
For the polynomial RKKS in the fourth example, one must also multiply the constants $b_k$ to the $\lambda$'s as appropriate.
Let $q$ be the cardinality of the set $\cQ$, which is the number of summands required to construct the kernel for $\cF$.
Denote the instructed sets as $\xi = \{\xi_1,\dots,\xi_q \}$ for $\lambda$ and $a = \{a_1,\dots,a_q\}$ for $h$.
We can write the kernel $h_\lambda$ as a linear combination of $\xi$ and $a$,
\[
  h_\lambda = \xi_1a_1 + \cdots + \xi_qa_q.
\]
The reason this is important is because changes in $\lambda$ for $h_\lambda$ only changes the $\xi_k$'s, but not the $a_k$'s.
This allows us to compute and store all of the required $n\times n$ kernel matrices $\bA_1,\dots,\bA_q$ from the application of instruction set on $h$ evaluated at all pairs of data points $(x_i,x_j)\in\cX\times\cX$.
This process of initialisation need only be done once prior to commencing the EM algorithm---a step we refer to as `kernel loading'.
In the \pkg{iprior} package, kernel loading is performed using the \code{kernL()} command.
%The application of the instruction set $\cQ$ to $\lambda$ to obtain $\xi$ is computationally effortless.

Notice that
\begin{align*}
  \tr\big(\bSigma_\theta \Wtilde^{(t)}\big)
  &=  \tr\big( (\psi\bH_\eta^2 + \psi^{-1}\bI_n ) \Wtilde^{(t)} \big) \\
  &= \psi \tr (\bH_\eta^2\Wtilde^{(t)}) + \psi^{-1}\tr \Wtilde^{(t)} \\
  &= \psi \tr \left(\sum_{j,k=1}^q \xi_j \xi_k \big(\bA_j \bA_k + (\bA_j \bA_k)^\top\big)
   \Wtilde^{(t)}\right) + \psi^{-1}\tr \Wtilde^{(t)} \\
   &= 2\psi \sum_{j,k=1}^q \xi_j \xi_k \tr \left(  \bA_j \bA_k 
   \Wtilde^{(t)}\right) + \psi^{-1}\tr \Wtilde^{(t)}.
\end{align*}
Provided that we have the matrices $\bA_{jk} = \bA_j\bA_k$, $j,k=1,\dots,q$ in addition to $\bA_1,\dots,\bA_q$ pre-calculated and stored, then evaluating $\tr\big(\bA_{jk} \Wtilde^{(t)} \big) = \vecc(\bA_{jk})^\top \vecc(\Wtilde^{(t)} )$ is $O(n^2)$, although this  only need to be done once per EM iteration.
Thus, with the kernels loaded, the overall time complexity to evaluate $Q$ is $O(n^2)$ at the beginning of each iteration, but roughly linear in $\xi$ thereafter.

As a remark, we have achieved efficiency at the expense of storage and a potentially long initialisation phase of kernel loading.
The storing of the kernel matrices $a$ can be very expensive, especially if the sample size is very large.
On the bright side, once the kernel matrices are stored in memory, the \pkg{iprior} package allows them to be reused again and again.
A practical situation where this might be useful is when we would like to repeat the EM at various initial values.

\subsection{The exponential family EM algorithm}

In the original EM paper by \citet{dempster1977maximum}, the EM algorithm was demonstrated to be easily administered to complete data likelihoods belonging to the exponential family for which the maximum likelihood estimates are easily computed.
If this is the case, then the M-step simply involves replacing the unknown sufficient statistics in the ML estimates with their \emph{conditional expectations} (see Appendix \ref{apx:expem} for details).
Certain I-prior models emit this property, namely regression functions belonging to the full or limited ANOVA RKKS, and we describe its estimation below.

Assume \ref{ass:A1}--\ref{ass:A3} applies, and that only the error precision $\psi$ and the RKHS scale parameters $\lambda_1,\dots,\lambda_p$ need to be estimated, i.e. all other kernel parameters are fixed---a similar situation was described in the previous subsection.
For the full ANOVA RKKS, the kernel is
\begin{align*}
  h_\lambda 
  &= \sum_{i=1}^p \lambda_i h_i + \sum_{i<j} \lambda_i \lambda_j h_i h_j + \cdots + \prod_{i=1}^p \lambda_i h_i \\
  &= \lambda_k 
  \myoverbrace{
  \Bigg(  
  {\color{colblu} h_k + \sum_{i} \lambda_i h_i h_k + \cdots + h_k \prod_{i\neq k} \lambda_i h_i}
  \Bigg)
  }{\text{terms of $\lambda_k$}} 
  + 
  \myoverbrace{\color{colred}
  \phantom{\Bigg(}
  \sum_{i\neq k} \lambda_i h_i + \sum_{i,j \neq k} \lambda_i \lambda_j h_i h_j + \cdots + 0
  }{\text{no $\lambda_k$ here}} \\
  &= \lambda_k {\color{colblu}r_k} + {\color{colred}s_k}
\end{align*}
where $r_k$ and $s_k$ are both functions over $\cX\times\cX$, defined respectively as the terms of the ANOVA kernel involving $\lambda_k$, and the terms not involving $\lambda_k$.
The reason for splitting $h_\lambda$ like this will become apparently momentarily.

Programmatically this looks complicated to implement in software, but in fact it is not.
Consider again the instruction list $\cQ$ for the ANOVA RKKS (Example 3, Section \ref{sec:efficientEM1}).
We can split this list into two: $\cR_k$ as those elements of $\cQ$ which involve the index $k$, and $\cS_k$ as those elements of $\cQ$ which do not involve the index $k$.
%Additionally, define $\cR_k^\lambda$ as the index set $\cR_k$ which removes occurrences of $k$ from its elements.
Let $\zeta_k$, $e_k$ be the sets of $\lambda$ and $h$ after applying the instructions of $\cR_k$ 
%and $\cR_k$ respectively
, and let $\xi_k$ and $a_k$ be the sets of $\lambda$ and $h$ after applying the instructions of $\cS_k$.
Now, we have 
\[
  r_k = \frac{1}{\lambda_k} \sum_{i=1}^{\abs{\cR_k}} \zeta_{ik} e_{ik} 
  \hspace{0.5cm}\text{and}\hspace{0.5cm}
  s_k = \sum_{i=1}^{\abs{\cS_k}} \xi_{ik} a_{ik}.   
\]
Defining $\bR_k$ and $\bS_k$ as the kernel matrices with $(i,j)$ entries $r_k(x_i,x_j)$ and $s_k(x_i,x_j)$ respectively, we have that
\[
  \bH_\eta^2 = \lambda_k^2\bR_k^2 + \lambda_k \myoverbrace{\big(\bR_k\bS_k + (\bR_k\bS_k)^\top \big)}{\bU_k} + \bS_k^2.
\]

Consider now the full data log-likelihood for $\lambda_k$, $k=1,\dots,p$, conditionally dependent on the rest of the unknown parameters $\psi$ and $\lambda_{-k} = \{\lambda_1,\dots,\lambda_p\} \backslash \{ \lambda_k \}$:
\begin{align}
  L(\lambda_k|\by,\bw,\lambda_{-k},\psi)
  &= \const 
  - \half \tr \Big( (
  \psi\bH_\eta^2 + \psi^{-1}\bI_n
  )\bw\bw^\top \Big)
  + \psi \tilde\by^\top \bH_\eta \bw \label{eq:logliklambdak} \\
  &= \const 
  - \lambda_k^2 \cdot \half[\psi] \tr(\bR_k^2 \bw\bw^\top)
  + \lambda_k \cdot \left( 
  \psi \tilde\by^\top \bR_k \bw - \half[\psi] \tr(\bU_k \bw\bw^\top)
  \right). \nonumber
\end{align}
Notice that the above likelihood is an exponential family distribution with the natural parameterisation $\beta = (-\lambda_k^2, \lambda_k)$ and sufficient statistics $T_1$ and $T_2$ defined by
\[
  T_1 = \half[\psi] \tr(\bR_k^2 \bw\bw^\top)
  \hspace{0.5cm}\text{and}\hspace{0.5cm}
  T_2 =  \psi\tilde\by^\top \bR_k \bw - \half[\psi]\tr(\bU_k^2 \bw\bw^\top).
\]
This likelihood is maximised at $\hat\lambda_k = T_2/2T_1$, but of course, the variables $w_1,\dots,w_n$ are never observed.
As per the exponential family EM routine, replace occurrences of $\bw$ and $\bw\bw^\top$ with their respective conditional expectations, i.e. $\bw\mapsto\E[\bw|\by] = \wtilde$ and $\bw\bw^\top\mapsto\E[\bw\bw^\top|\by] = \tilde\bV_w + \wtilde\wtilde^\top$ as defined in \eqref{eq:posteriorw}.
That the $\lambda_k$'s have closed-form expressions, together with the closed-form expression for $\psi$ in \eqref{eq:closedformpsi}, greatly simplifies the EM algorithm.
At the M-step, one simply updates the parameters in turn, and as such, there is no maximisation per se.
This form of the EM algorithm is known as the \emph{conditional expectation-maximisation} algorithm \citep{meng1993maximum}.

The algorithm is summarised in Algorithm \ref{alg:EM2}.
The exponential family EM for ANOVA-type I-prior models require $O(n^3)$ computational time at each step, which is spent on computing the matrix inverse in the E-step.
The M-step takes at most $O(n^2)$ time to compute.
As a remark, it is not necessary that $h_\lambda$ is the full ANOVA RKKS; any of the examples 1--3 in \autoref{sec:efficientEM1} can be estimated using this method, since they are seen as special cases of the ANOVA decomposition.
%This is also true if we decide to drop any of the terms in the ANOVA kernel.

\begin{algorithm}[hbt]
\caption{Exponential family EM for ANOVA-type I-prior models}\label{alg:EM2}
\begin{algorithmic}[1]
  \Procedure{Initialisation}{}
    \State Initialise $\lambda_1^{(0)},\dots,\lambda_p^{(0)}, \psi^{(0)}$
    \State Compute and store matrices as per $\cR_k$ and $\cS_k$.
    \State $t \gets 0$
  \EndProcedure 
  \Statex
  \While{not converged}{}
    \Procedure{E-step}{}
      \State $\wtilde \gets \psi^{(t)} \bH_{\eta^{(t)}} \big(\psi^{(t)} \bH_{\eta^{(t)}}^2 + \psi^{-(t)}\bI_n \big)^{-1} \tilde\by$
      \State $\Wtilde \gets \big(\psi^{(t)} \bH_{\eta^{(t)}}^2 + \psi^{-(t)}\bI_n \big)^{-1} + \wtilde\wtilde^{\top}$
    \EndProcedure
    \Statex
    \Procedure{M-step}{}
      \For{$k=1,\dots,p$}
        \State $T_{1k} \gets \half \tr(\bR_k^2 \Wtilde)$
        \State $T_{2k} \gets \tilde\by^\top \bR_k \wtilde - \half \tr(\bU_k^2 \Wtilde^\top)$
        \State $\lambda_k^{(t+1)} \gets T_{2k} / 2T_{1k}$
      \EndFor
      \State $T_3 \gets \tilde\by^\top\tilde\by + \tr(\bH_{\eta^{(t)}}^2\Wtilde^{(t)}) - 2\tilde\by^\top\bH_{\eta^{(t)}}\wtilde^{(t)}$
      \State $\psi^{(t+1)} \gets \tr \Wtilde^{(t)} / T_3$
    \EndProcedure
    \State $t \gets t+1$
  \EndWhile
\end{algorithmic}
\end{algorithm}

%\bPsi \bH_\eta \bV_y^{-1} (\by - \alpha\bone_n - \bff_0)
%    \hspace{0.5cm}\text{and}\hspace{0.5cm}
%    \tilde\bV_w = \big(\bH_\eta\bPsi\bH_\eta + \bPsi^{-1}\big)^{-1} = \bV_y^{-1},

%For these three examples, the specific form of the matrices $\bR_k$ and $\bS_k$ are given below
%\begin{enumerate}
%  \item \textbf{Single scale parameter}.
%  \[
%    \bR_1 = \bH_1 
%    \hspace{0.5cm}\text{and}\hspace{0.5cm}
%    \bS_1 = \bzero
%  \]
%  \item \textbf{Multiple scale parameters}.
%  \[
%    \bR_k = \bH_k
%    \hspace{0.5cm}\text{and}\hspace{0.5cm}
%    \bS_k = \sum_{j\neq k} \lambda_j \bH_j
%  \]
%  \item \textbf{Multiple scale parameters with level-2 interactions}.
%  \[
%    \bR_k = \bH_k + \sum_{j\neq k} \lambda_j (\bH_j \circ \bH_k)
%    \hspace{0.5cm}\text{and}\hspace{0.5cm}
%    \bS_k = \sum_{j\neq k} \lambda_j \bH_j + \sum_{j<j', j,j' \neq k} \lambda_j\lambda_{j'} (\bH_j \circ \bH_{j'})
%  \]
%\end{enumerate}

While the exponential family EM algorithm takes similar computational time as the efficient EM algorithm described in \autoref{sec:efficientEM1}, there is one compelling reason to consider Algorithm \ref{alg:EM2}: conjugacy of the exponential family of distributions.
Realise that $\lambda_k|(\by,\bw,\lambda_{-k},\psi)$ is in fact normally distributed, with mean and variance given by $T_2/2T_1$ and $1/2T_1$ respectively.
If we were so compelled to assign a normal prior on each of the $\lambda_k$'s, then the conditionally dependent log-likelihood of $\lambda_k$, $L(\lambda_k|\by,\bw,\lambda_{-k},\psi)$, would have a normal log-likelihood prior involving $\lambda_k$ added on.
Importantly, viewed as a posterior log-density for $\lambda_k$, the posterior density for $\lambda_k$ would also be a normal distribution.
The EM as a whole would then generate maximum a posteriori (MAP) estimates for the parameters.
Although not shown here, similar conjugacy benefits for the $\psi$ parameter can be argues, whereby the gamma distribution is the density in question.
The usual EM algorithm without using any priors can be viewed as using improper priors for the parameters, i.e. $p(\lambda_k) \propto \const$ and $p(\psi) \propto \const$.

In the next chapter on binary and multinomial regression using I-priors, the exponential family EM algorithm described here is especially relevant, as it is connected to the variational Bayesian algorithm \citep{bernardo2003variational} that will be used for estimating the models described therein.

\begin{remark}
  Earlier, we restricted attention to ANOVA RKKS. 
  Hopefully, it is now apparent that ANOVA kernels are a requirement for Algorithm \ref{alg:EM2} to work easily.
  As soon as higher degrees of the $\lambda_k$'s come into play, e.g. using the polynomial kernel, then the ML estimate for $\lambda_k$ involve solving a polynomial of degree $2d-1$ the FOC equations.
  Although this is not in itself hard to do, the elegance of the algorithm, especially viewed as having the normal conjugacy property for the $\lambda_k's$, is lost.
\end{remark}

\subsection{Accelerating the EM algorithm}

A criticism of the EM algorithm is that it may take many iterations to converge.
Several novel ideas have been looked at in a bid to `accelerate the EM algorithm', as it were.
One such approach, which does not require any amendment to the particular EM algorithm at hand, is called the \emph{monotonically over-relaxed EM algorithm} (MOEM) by \citet{yu2012monotonically}.

The idea of MOEM is as follows.
At every iteration of the MOEM, perform as usual the E-step and M-step to obtain an updated parameter value $\theta^{(t+1)}_\text{EM}$.
Instead of using this update value of the parameter, modify it instead, and use
\[
  \theta^{(t+1)} = (1 + \omega) \theta^{(t+1)}_\text{EM} - \omega \theta^{(t)},
\]
where $\omega$ is an \emph{over-relaxation} parameter.
Under mild conditions, among them that $Q(\theta^{(t+1)}) > Q(\theta^{(t)})$, the MOEM estimate does not decrease the log-likelihood at each step.
This condition is a slight inconvenience to check under the usual EM algorithm, but is a great companion to exponential family EM algorithm.
From \eqref{eq:logliklambdak}, we see that $Q(\lambda_k) = \E_\bw \big[ L(\lambda_k|\theta\backslash\{\lambda_k \} ) | \by,\theta^{(t)} \big]$ is quadratic in $\lambda_k$, therefore any $\omega \in [0,1]$ will maintain monotonicity of the EM algorithm.


\section{Post-estimation}
\input{04c-iprior-post-estimation}

\section{Examples}
%\input{04c-examples}

\section{Conclusion}

The steps for I-prior modelling are basically three-fold:
\begin{enumerate}
  \item Select an appropriate function space; equivalently, the kernels for which a specific effect is desired on the covariates. Several modelling examples are described in Section \ref{sec:various-regression}.
%  Choices included a linear effect (canonical RKHS), a polynomial effect (polynomial RKKS), smoothing effect (fBm or SE RKHS), 
  \item Estimate the hyperparameters (these included the RKHS scale parameter(s), error precision, and any other kernel parameters such as the Hurst index of fBm) of the I-prior model and obtain the posterior regression function.
  \item Post-estimation procedures include
  \begin{itemize}
    \item Posterior predictive checks;
    \item Model comparison via log-likelihood ratio tests/empirical Bayes factors; and
    \item Prediction of new data point.
  \end{itemize}
\end{enumerate}

The main sticking point with the estimation procedure is the involvement of the $n\times n$ kernel matrix, for which its inverse is needed.
This requires $O(n^2)$ storage and $O(n^3)$ computational time.
The Nyström method of approximating the kernel matrix reduces complexity to $O(nm)$ storage and approximately $O(nm^2)$, and is highly advantageous if $m \ll n$.
The computational issue faced by I-priors are mirrored in Gaussian process regression, so the methods to overcome these computational challenges in GPR can be explored further.
However, most efficient computational solutions exploit the nature of the SE kernel structure, which is the most common kernel used in GPR.

Several avenues have been discussed to make the estimation procedure more efficient, but improvements can be had.
One promising avenue to achieve efficient estimation for I-prior models is by using variational methods.
A sparse variational approximation (typically by using inducing points) or stochastic variational inference can greatly reduce computational storage and speed requirements.
A recent paper by \citet{cheng2017variational} suggested a variational algorithm with linear complexity for GPR-type models.

On the topic of accelerating the EM algorithm, besides the MOEM procedure, there are two other algorithms that could be explored.
The first is called parameter-expansion EM algorithm (PXEM) by \citep{liu1998parameter}, which has been shown to be promising for random-effects type models.
It involves correcting the M-step by a `covariance adjustment', so that extra information can be capitalised on to improve convergence rates.
The second is a quasi-Newton acceleration of the EM algorithm as proposed by \citet{lange1995quasi}.
A slight change to the EM gradient algorithm in the M-step steers the EM algorithm to the Newton-Raphson algorithm, thus exploiting the benefits of the EM algorithm in the early stages (monotonic increase in likelihood) and avoiding the pitfalls of Newton-Raphson (getting stuck in local optima).
The PXEM and quasi-Newton EM algorithms require an in-depth reassessment of the EM algorithm to specifically tailor them to I-prior models, which we leave as future work.

\section*{Miscellanea}
\input{04-omitted}

\ifstandalone
  \section*{Appendix}
  \section[Deriving the posterior distribution for w]{Deriving the posterior distribution for $\bw$}
\label{apx:posteriorw}

In the following derivation, we implicitly assume the dependence on $\bff_0$ and $\theta$.
The distribution of $\by|\bw$ is $\N_n(\balpha + \bff_0 +\bH_\eta\bw,\bPsi^{-1})$, where $\balpha = \alpha\bone_n$, while the prior distribution for $\bw$ is $\N_n(\bzero,\bPsi)$.
Since $p(\bw|\by) \propto p(\by|\bw)p(\bw)$, we have that
\begin{align*}
  \log p(\bw|\by) 
  &=  \log p(\by|\bw) + \log p(\bw) \\
  &= \const + \cancel{\half\log\abs{\bPsi}} - \half (\by - \balpha - \bff_0 - \bH_\eta\bw)^\top \bPsi (\by - \balpha - \bff_0 - \bH_\eta\bw) \\
  &\phantom{==} -\cancel{\half\log\abs{\bPsi}} - \half \bw^\top\bPsi^{-1}\bw \\
  &= \const - \half \bw^\top(\bH_\eta\bPsi\bH_\eta + \bPsi^{-1})\bw + (\by - \balpha - \bff_0)^\top\bPsi\bH_\eta\bw.
\end{align*}
Setting $\bA = \bH_\eta\bPsi\bH_\eta + \bPsi^{-1}$, $\ba^\top = (\by - \balpha - \bff_0)^\top\bPsi\bH_\eta$, and using the fact that 
\[
  \bw^\top \bA \bw - 2 \ba^\top\bw = (\bw - \bA^{-1}\ba)^\top\bA(\bw - \bA^{-1}\ba),
\]
we have that $\bw|\by$ is normally distributed with the required mean and variance.

Alternatively, one could have shown this using standard results of multivariate normal distributions.
Noting that the covariance between $\by$ and $\bw$ is  %\sim\N_n(\balpha +\bff_0, \bV_y)
\begin{align*}
  \Cov(\by,\bw) 
  &= \Cov(\balpha + \bff_0 + \bH_\eta\bw + \bepsilon, \bw) \\
  &= \bH_\eta\Cov(\bw,\bw) \\
  &= \bH_\eta\bPsi 
%  &= \Cov(\bw,\by),
\end{align*}
and that $\Cov(\bw,\by) = \bPsi\bH_\eta = \bH_\eta\bPsi = \Cov(\by,\bw)$ by symmetry, the joint distribution $(\by,\bw)$ is
\[
  \begin{pmatrix}
    \by \\
    \bw
  \end{pmatrix}
  \sim \N_{n+n}
  \left(
    \begin{pmatrix}
      \balpha + \bff_0 \\
      \bzero
    \end{pmatrix},
    \begin{pmatrix}
      \bV_y         & \bH_\eta\bPsi \\
      \bH_\eta\bPsi & \bPsi \\
    \end{pmatrix}
  \right).
\] 
Thus,
\begin{align*}
  \E [\bw|\by] 
  &= \E \bw + \Cov(\bw,\by) (\Var \by)^{-1} (\by - \E \by) \\
  &= \bH_\eta\bPsi\bV_y^{-1}(\by - \balpha - \bff_0),
\end{align*}
and
\begin{align*}
  \Var [\bw|\by] 
  &= \Var \bw - \Cov(\bw,\by) (\Var \by)^{-1} \Cov(\by,\bw) \\
  &= \bPsi - \bH_\eta\bPsi\bV_y^{-1}\bH_\eta\bPsi \\
  &= \bPsi - \bPsi\bH_\eta \left(\bPsi^{-1} + \bH_\eta\bPsi\bH_\eta \right)^{-1}\bH_\eta\bPsi  \\
  &= \left(\bPsi^{-1} + \bH_\eta\bPsi\bH_\eta \right)^{-1} \\
  &= \bV_y^{-1}
%  &= \bPsi\left(\bPsi^{-1} - \bH_\eta\bV_y^{-1}\bH_\eta \right)\bPsi
%  &= \left[ \bPsi^{-1} + \bPsi^{-1}\bH_\eta\bPsi \left( \right) \right]^{-1}
\end{align*}
as a direct consequence of the Woodbury matrix identity.
  \section{A recap on the exponential family EM algorithm}
\label{apx:expem}

Consider the density function $p(\cdot|\btheta)$ of the complete data $\bz = \{\by,\bw\}$, which depends on parameters $\btheta = (\theta_1,\dots,\theta_s)^\top \in\Theta\subseteq\bbR^s$, belonging to an exponential family of distributions.
This density takes the form $p(\bz|\btheta) = B(\bz) \exp \big( \ip{\bfeta(\btheta), \bT(\bz)} -  A(\btheta) \big)$, where $\bfeta:\bbR^s \mapsto \bbR$ is a link function,  $\bT(\bz) = \big(T_1(\bz),\dots,T_s(\bz)\big)^\top \in \bbR^s$ are the sufficient statistics of the distribution, and $\ip{\cdot,\cdot}$ is the usual Euclidean dot product.
It is often easier to work in the \emph{natural parameterisation} of the exponential family distribution
\begin{align}\label{eq:pdfexpfamnat}
  p(\bz|\bfeta) = B(\bz) \exp \big( \ip{\bfeta, \bT(\bz)} -  A^*(\bfeta) \big)
\end{align}
by defining $\bfeta := \big(\eta_1(\btheta),\dots,\eta_r(\btheta)\big) \in \cE$, and $\exp A^*(\bfeta) = \int B(\bz) \, \exp \, \ip{\bfeta, \bT(\bz)}  \dint \bz$ to ensure the density function normalises to one.
As an aside, the set $\cE := \big\{ \bfeta = (\eta_1,\dots,\eta_s) \,|\, \int  \exp A^*(\bfeta) < \infty \big\}$ is called the \emph{natural parameter space}.
If $\dim \cE = r < s = \dim \Theta$, then the the pdf belongs to the \emph{curved exponential family} of distributions.
If $\dim \cE = r = s = \dim \Theta$, then the family is a \emph{full exponential family}.

Assuming the latent $\bw$ variables are observed and working with the natural parameterisation, then the complete maximum likelihood (ML) estimate for $\bfeta$ is obtained by solving 
\begin{align}\label{eq:expEM1}
  \frac{\partial}{\partial\bfeta}\log p(\bz|\bfeta)
  &= \bT(\bz) - \frac{\partial}{\partial\bfeta} A^*(\bfeta) = 0.
\end{align}
Of course, the variable $\bw$ are never observed, so the ML estimate for $\bfeta$ can only be informed from what is observed.
Let $p(\by|\bfeta) = \int p(\by,\bw|\bfeta) \dint \bw$ represent the marginal density of the observations $\by$.
Now, the ML estimate for $\bfeta$  is obtained by solving
\begin{align}
  \frac{\partial}{\partial\bfeta}\log p(\by|\bfeta)
  &= \frac{1}{p(\by|\bfeta)} \cdot \frac{\partial}{\partial\bfeta}  p(\by|\bfeta) \nonumber \\
  &= \frac{1}{p(\by|\bfeta)} \cdot \frac{\partial}{\partial\bfeta} \left( \int p(\by,\bw|\bfeta) \dint \bw \right) \nonumber \\
  &= \frac{1}{p(\by|\bfeta)} \cdot \int \left( \frac{\partial}{\partial\bfeta} p(\by,\bw|\bfeta) \right) \dint \bw \nonumber \\
  &= \frac{1}{p(\by|\bfeta)} \cdot \int \left( p(\by,\bw|\bfeta) \frac{\partial}{\partial\bfeta} \log p(\by,\bw|\bfeta) \right) \dint \bw \nonumber \\
  &= \int \left( \bT(\by,\bw) - \frac{\partial}{\partial\bfeta} A^*(\bfeta) \right) p(\bw|\by,\bfeta) \dint \bw \nonumber \\
  &= \E_\bw \big[ \bT(\by,\bw) | \by \big] - \frac{\partial}{\partial\bfeta} A^*(\bfeta) \label{eq:expEM2}
\end{align}
equated to zero.
Note that we are allowed to change the order of integration and differentation provided the integrand is continuously differentiable.
So the only difference between the first order condition of \eqref{eq:expEM1} and that of \eqref{eq:expEM2} is that the sufficient statistics involving the unknown $\bw$ are replaced by their conditional or posterior expectations.
%but this is not an issue: by the law of total expectations, $\E\bT(\bz) = \E \bT(\by,\bw) = \E \big[ \E_\bw[\bT(\by,\bw)|\by] \big]$
%so solving $\bT(\bz) = \E \bT(\bz)$ for $\btheta$ is not possible without some manipulation.

A useful identity to know is that $\frac{\partial}{\partial\bfeta} A^*(\bfeta) = \E_\bz \bT(\bz)$ \citep[Theorem 3.4.2 \& Exercise 3.32(a)]{casella2002statistical}, which can be expressed in terms of the original parameters $\btheta$.
As a consequence, solving for the ML estimate for $\btheta$ from the FOC equations \eqref{eq:expEM2} is possible without having to deal with the derivative of $A^*$ with respect to the natural parameters.
Having said this, an analytical solution in $\btheta$ may not exist, because the relationship of $\btheta$ could be implicit in the set of equations $\E_\bw \big[ \bT(\bw,\by) | \by, \btheta \big] = \E_{\by,\bw}\left[ \bT(\by,\bw) | \btheta \right]$.
One way around this is to employ an iterative procedure, as detailed in Algorithm \ref{alg:EM3}.

\begin{algorithm}[hbt]
\caption{Exponential family EM}\label{alg:EM3}
\begin{algorithmic}[1]
  \State \textbf{initialise} $\btheta^{(0)}$ and $t\gets 0$
  \While{not converged}
    \State E-step: $\tilde\bT^{(t+1)}(\by,\bw) \gets \E_\bw \big[ \bT(\bw,\by) | \by, \btheta^{(t)} \big]$
    \State M-step: $\btheta^{(t+1)} \gets$ solution to $\tilde\bT^{(t+1)}(\by,\bw) = \E_{\by,\bw}\left[ \bT(\by,\bw) | \btheta \right]$
    \State $t \gets t + 1$
  \EndWhile
\end{algorithmic}
\end{algorithm}

To see how Algorithm \ref{alg:EM3} motivates the EM algorithm, consider the following argument.
Recall that for the EM algorithm, the function $Q_t(\bfeta) = \E_\bw[\log p(\by,\bw|\bfeta) | \by,\bfeta^{(t)}]$ is maximised at each iteration $t$.
For exponential families of the form \eqref{eq:pdfexpfamnat}, the $Q_t$ function turns out to be
\[
 Q_t(\bfeta) = \E_\bw \big[ \ip{\bfeta, \bT(\bz)} | \by,\bfeta^{(t)} \big] -  A^*(\bfeta) + \log B(\bz),
\]
and this is maximised at the value of $\bfeta$ satisfying
\begin{align*}
  \frac{\partial}{\partial\bfeta} Q_t(\bfeta)
  &= \E_\bw \big[ \bT(\by,\bw) | \by,\bfeta^{(t)} \big] - \frac{\partial}{\partial\bfeta}A^*(\bfeta) = 0,
\end{align*}
a similar condition to \eqref{eq:expEM2} when obtaining ML estimate of $\bfeta$.
Thus, $Q_t$ is maximised by the solution to line 4 in Algorithm \eqref{alg:EM3}.



%In other words, the ML estimate for $\btheta$ satisfies $\{ \btheta | \bT(\bz) = \E \bT(\bz) \} $
%Assume the inverse mapping $\bfeta^{-1}$ exists, then the ML estimates $\hat\btheta$ can be obtained as $\bfeta^{-1}(\hat\btheta)$ due to the invariance property of ML estimates.



  \section{Deriving the posterior predictive distribution}
\label{apx:postpred}

A priori, assume that $y_\new \sim \N(\hat\alpha, v_\new)$, where $v_\new =  \bh_{\hat\eta}(x_\new)^\top \hat\bPsi \bh_{\hat\eta}(x_\new) + \psi^{-1}_\new $.
Consider the joint distribution of $(y_\new,\by^\top)^\top$, which is multivariate normal (since both $y_\new$ and $\by$ are.
Write
\[
  \begin{pmatrix}
    y_\new \\
    \by
  \end{pmatrix}
  \sim \N_{n+1}
  \left(
    \begin{pmatrix}
      \hat\alpha \\
      \hat\alpha\bone_n
    \end{pmatrix},
    \begin{pmatrix}
      v_\new &\Cov(y_\new,\by) \\
      \Cov(y_\new,\by)^\top &\tilde\bV_y \\
    \end{pmatrix}
  \right),
\]
where 
\begin{align*}
  \Cov(y_\new,\by)
  &= \Cov(f_\new + \epsilon_\new, \bff + \bepsilon) \\
  &= \Cov(f_\new, \bff) + \Cov(\epsilon_\new, \bepsilon) \\
  &= \Cov\left(\bh_{\hat\eta}(x_\new)^\top \wtilde,\bH_{\tilde\eta}\wtilde \right) + (\sigma_{\new,1},\dots,\sigma_{\new,n}) \\
  &= \bh_{\hat\eta}(x_\new)^\top\hat\bPsi\bH_{\tilde\eta} + \bsigma_\new.
\end{align*}
The vector of covariances $\bsigma_\new$ between observations $y_1,\dots,y_n$ and the predicted point $y_\new$ would need to be prescribed a priori (treated as extra parameters), or estimated again, which seems excessive.
Assuming $\bsigma_\new = \bzero$ would be acceptable, especially under an iid assumption the error precisions.
In any case, using standard multivariate normal results, we get that $y_\new|\by$ is also normally distributed with mean
\begin{align*}
  E[y_\new|\by] 
  &= \hat\alpha + (\bh_{\hat\eta}(x_\new)^\top\hat\bPsi\bH_{\tilde\eta} + \bsigma_\new)\tilde\bV_y^{-1}\tilde\by  \\
  &= \hat\alpha + \bh_{\hat\eta}(x_\new)^\top
  \myoverbrace{\bh_{\hat\eta}(x_\new)^\top\hat\bPsi\bH_{\tilde\eta}\tilde\bV_y^{-1}\tilde\by}{\hat\bw}
  + \bsigma_\new \tilde\bV_y^{-1}\tilde\by \\
  &= \hat\alpha + \E \big[ f(x_\new) |\by \big] + \text{mean correction term}
\end{align*}
and variance
\begin{align*}
  \Var[y_\new|\by] 
  &= v_\new - (\bh_{\hat\eta}(x_\new)^\top\hat\bPsi\bH_{\tilde\eta} + \bsigma_\new)\tilde\bV_y^{-1}(\bh_{\hat\eta}(x_\new)^\top\hat\bPsi\bH_{\tilde\eta} + \bsigma_\new)^\top \\
  &= \bh_{\hat\eta}(x_\new)^\top \hat\bPsi\hat \bh_{\hat\eta}(x_\new) + \psi^{-1}_\new - \bh_{\hat\eta}(x_\new)^\top\hat\bPsi\bH_{\tilde\eta}\tilde\bV_y^{-1}\bH_{\tilde\eta}\hat\bPsi\bh_{\hat\eta}(x_\new) \\
  &\phantom{==}+ \text{variance correction term} \\
  &= \bh_{\hat\eta}(x_\new)^\top 
  \big(
  \hat\bPsi - \hat\bPsi\bH_{\tilde\eta}\tilde\bV_y^{-1}\bH_{\tilde\eta}\hat\bPsi
  \big)
  \bh_{\hat\eta}(x_\new) + \psi^{-1}_\new \\
  &\phantom{==} + \text{variance correction term} \\
  &= \bh_{\hat\eta}(x_\new)^\top \hat\bV_y^{-1}\bh_{\hat\eta}(x_\new) + \psi^{-1}_\new + \text{variance correction term} \\
  &= \Var\big[f(x_\new)|\by\big] + \psi^{-1}_\new + \text{variance correction term}.
\end{align*}
\fi

\hClosingStuffStandalone
\end{document}