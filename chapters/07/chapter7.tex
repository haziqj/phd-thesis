\documentclass[a4paper,showframe,11pt]{report}
\usepackage{standalone}
\standalonetrue
\ifstandalone
  \usepackage{../../haziq_thesis}  
  \usepackage{../../haziq_maths}
  \usepackage{../../haziq_glossary}
  \usepackage{../../knitr}
  \addbibresource{../../bib/haziq.bib}
  \externaldocument{../01/.texpadtmp/chapter1}
  \externaldocument{../02/.texpadtmp/chapter2}
  \externaldocument{../03/.texpadtmp/chapter3}
  \externaldocument{../04/.texpadtmp/chapter4}
  \externaldocument{../05/.texpadtmp/chapter5}
  \externaldocument{../06/.texpadtmp/chapter6}  
\fi

\begin{document}
\hChapterStandalone[7]{Discussion}

Wrap-up---what did I do? Summarise contributions. Raise questions.

\section{Summary}

The I-prior for the regression model \cref{eq:model1} subject to \cref{eq:model1ass} is seemingly data dependent, which violates Bayesian first principles.
That is, an I-prior for $f$ as per \cref{eq:iprior2} makes use of the same data $\bx:=\{x_1,\dots,x_n\}$ in the covariance matrix for $f$ that appears in the model.
However, the whole model is implicitly conditional on $\bx$.
If the prior depended instead on the responses $\by$, then the state of knowledge a priori and a posteriori is exactly the same, and this violates Bayesian principles.

\section{Questions}

\section{Conclusion}

\hClosingStuffStandalone
\end{document}