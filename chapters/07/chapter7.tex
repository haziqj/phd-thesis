\documentclass[a4paper,showframe,11pt]{report}
\usepackage{standalone}
\standalonetrue
\ifstandalone
  \usepackage{../../haziq_thesis}  
  \usepackage{../../haziq_maths}
  \usepackage{../../haziq_glossary}
  \usepackage{../../knitr}
  \addbibresource{../../bib/haziq.bib}
  \externaldocument{../01/.texpadtmp/chapter1}
  \externaldocument{../02/.texpadtmp/chapter2}
  \externaldocument{../03/.texpadtmp/chapter3}
  \externaldocument{../04/.texpadtmp/chapter4}
  \externaldocument{../05/.texpadtmp/chapter5}
  \externaldocument{../06/.texpadtmp/chapter6}  
\fi

\begin{document}
\hChapterStandalone[7]{Discussion}

Wrap-up---what did I do? Summarise contributions. Raise questions.

\hltodo{Summary of I-priors, what it is, how it's used (regression, classification, variable selection) and how it's estimated.}

The I-prior for the regression model \cref{eq:model1} subject to \cref{eq:model1ass} is seemingly data dependent, which violates Bayesian first principles.
That is, an I-prior for $f$ as per \cref{eq:iprior2} makes use of the same data $\bx:=\{x_1,\dots,x_n\}$ in the covariance matrix for $f$ that appears in the model.
However, the whole model is implicitly conditional on $\bx$.
If the prior depended instead on the responses $\by$, then the state of knowledge a priori and a posteriori is exactly the same, and this violates Bayesian principles.

\section{Summary of contributions}

\begin{itemize}
  \item extension of fisher information to infinite-dimensional parameters. the technology for derivatives applicable to other spaces like Banach spaces, could look at RKBS. infinite dimensional vectors e.g. for exponential family type distributions. 
  \item efficient implementation of estimation methods, in particular EM algorithm. iprior package.
  \item extension of iprior methodology to categorical responses. used variational EM
  \item BVS. simple hands off approach. works well in multicollinearity.
\end{itemize}

\section{Questions}

\begin{itemize}
  \item starting values for EM or direct optimisation. what is the best approach?
  \item variational approximation. asymptotic distribution of the parameters? any way to get SE from EM? do 
\end{itemize}

\section{Conclusion}

\hClosingStuffStandalone
\end{document}