\section{Data sets}

%\subsection{Aerobic data}
%\label{apx:aerobic}

\subsection{Air pollution data}
\label{apx:airpollution}

\begin{table}[H]
\centering
\caption{Description of the air pollution data set for the analysis done in \cref{sec:airpollution}}
\label{tab:airpollution}
\begin{tabular}{ll}
\toprule
Variable                 & Description \\
\midrule
Mortality                & Total age adjusted mortality rate \\
Precipitation            & Mean annual precipitation (in) \\
Relative humidity        & Percent relative humidity, annual average at 1 p.m. \\
January temperature      & Mean January temperature ($^\circ$F) \\
July temperature         & Mean July temperature ($^\circ$F) \\
Population density       & Population per square mile in urbanised area \\
Household size & Population per household  \\
Education                & Median school years completed for those over 25 \\
Sound housing units      & Percentage of sound housing units (no defects) \\
Age >65 years            & Percent of population that is 65 years of age or over \\
Non-white                & Percent of urbanised area population that is non-white \\
White collar             & Percent employment in white-collar urbanised occupations \\
Income <\$3,000          & Percent of families with income under \$3,000 \\      
HC                       & Relative population potential of hydrocarbons \\     
NO\textsubscript{x}      & Relative population potential of oxides of nitrogen \\
SO\textsubscript{2}      & Relative population potential of sulphur dioxide \\       
\end{tabular}
\end{table}

\subsection{Ozone data}
\label{apx:ozone}

\begin{table}[H]
\centering
\caption{Description of the ozone data set for the analysis done in \cref{sec:ozone}}
\label{tab:ozone}
\begin{tabular}{ll}
\toprule
Variable     & Description \\
\midrule
$y$      & Daily maximum one-hour-average ozone reading (ppm) at Upland, CA \\
$X_1$    & Month: $1 = \text{January}, \dots, 12 = \text{December}$\\
$X_2$    & Day of month: $1,2,\dots$ \\
$X_3$    & Day of week: $1 = \text{Monday}, \dots, 7 = \text{Sunday}$ \\
$X_4$    & 500-millibar pressure height (m) measured at Vandenberg AFB \\
$X_5$    & Wind speed (mph) at Los Angeles International Airport (LAX) \\
$X_6$    & Humidity (\%) at LAX \\
$X_7$    & Temperature ($^\circ$F) measured at Sandburg, CA \\
$X_8$    & Inversion base height (feet) at LAX \\
$X_9$    & Pressure gradient (mmHg) from LAX to Daggett, CA \\
$X_{10}$ & Visibility (mi) measured at LAX \\
$X_{11}$ & Temperature ($^\circ$F) measured at El Monte, CA \\      
$X_{12}$ & Inversion base temperature (degrees Fahrenheit) at LAX \\             
\end{tabular}

\end{table}
