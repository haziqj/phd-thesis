In what follows, each of the kernel functions will have its associated scale parameter denoted by $\lambda$.
Further, to make the distinction between centred and non-centred versions of the kernels, we use the notation $h$ to denote the uncentred version, and $\bar h$ to denote the centred version.

\subsection{The RKHS of constant functions}

The vector space of constant functions $\cF$ over a set $\cX$ contains the functions $f:\cX \to \bbR$ such that $f(x) = c_f \in \bbR$, $\forall x \in \cX$.
These functions would be useful to model an overall average, i.e. an ``intercept effect''.
The space $\cF$ can be equipped with a norm to form an RKHS, as shown in the following lemma.

\begin{proposition}[RKHS of constant functions]
%  Let $\cF$ be the RKHS of functions over a set $\cX$ with reproducing kernel $h:\cX\times\cX\to\bbR$ as defined rather simply by
%  \[
%    h(x,x') = 1.
%  \]
%  Then $\cF$ consists of constant functions over $\cX$.
%  Furthermore, if $f(x) = c_f \in \bbR$, $\forall x \in \cX$, then $\norm{f}_\cF = \vert c_f \vert$.
  The space $\cF$ as described above endowed with the norm $\norm{f}_\cF = \vert c_f \vert$ forms an RKHS with the reproducing kernel $h:\cX\times\cX\to\bbR$ as defined, rather simply by,
  \[
    h(x,x') = 1,
  \]
  known as the constant kernel.
\end{proposition}

\begin{proof}
  If $\cF$ is an RKHS with kernel $h$ as described, then $\cF$ is spanned by the  functions $h(\cdot,x) = 1$, so it is clear that $\cF$ consists of constant functions over $\cX$.
  On the other hand, if the space $\cF$ is equipped with the inner product $\ip{f,f'}_\cF = c_f c_{f'}$, then the reproducing property follows, since $\ip{f,h(\cdot,x)}_\cF = c_f = f(x)$.
  Hence, $\norm{f}_\cF = \sqrt{\ip{f,f}_\cF} = \vert c_f \vert$.
\end{proof}

In I-prior modelling, one need not consider any scale parameter on reproducing kernel, as the scale parameter would not be identified otherwise.
See later chapter for details.
\hltodo{I think the scale parameter $\lambda$ would just be absorbed by the norm, which is a single value of interest and that is what is ``observed'', and the decomposition $\lambda\cdot c_f$ is not so interesting.}

\subsection{The canonical (linear) RKHS}

Consider a function space $\cF$ over $\cX$ which consists of functions of the form $f_\beta:\cX\to\bbR$, $f_\beta: x \mapsto \ip{x,\beta}_\cX$ for some $\beta\in\bbR$.
Suppose that $\cX \equiv \bbR^p$, then $\cF$ consists of the linear functions $f_\beta(x) = x^\top\beta$.
More generally, if $\cX$ is a Hilbert space, then its continuous dual consists of elements of the form $f_\beta = \ip{\cdot,\beta}_\cX$.
We can show that the continuous dual space of $\cX$ is a RKHS which consists of these linear functions.

\begin{proposition}[The canonical RKHS]
  The continuous dual space a Hilbert space $\cX$, denoted by $\cX'$, is a RKHS of linear functions over $\cX$ of the form $\ip{\cdot,\beta}_\cX$, $\beta\in\cX$. Its reproducing kernel $h:\cX\times\cX\to\bbR$ is defined by
  \[
    h(x,x') = \ip{x,x'}_{\cX}.
  \]
\end{proposition}

\begin{proof}
  Define $f_\beta := \ip{\cdot,\beta}_\cX$ for some $\beta \in \cX$.
  Clearly this is linear and continuous, so $f_\beta\in\cX'$, and so $\cX'$ is a Hilbert space containing functions $f:\cX\to\bbR$ of the form $f_\beta(x) = \ip{x,\beta}_\cX$.
  By the Riesz representation theorem, every element of $\cX'$ has the form $f_\beta$.
  It also gives us a natural isometric isomorphism such that the following is true:
  \[
    \ip{\beta,\beta'}_\cX = \ip{f_\beta,f_{\beta'}}_{\cX'}.
  \]
  Hence, for any $f_\beta\in\cX'$, 
  \begin{align*}
    f_\beta(x) 
    &= \ip{x,\beta}_\cX \\
    &= \ip{f_x,f_{\beta}}_{\cX'} \\
    &= \big\langle \ip{\cdot,x}_\cX,f_{\beta} \big\rangle_{\cX'}.
  \end{align*}
  Thus, $h:\cX\times\cX\to\bbR$ as defined by $h(x,x') = \ip{x,x'}_\cX$ is the reproducing kernel of $\cX'$.
\end{proof}

In many other literature, the kernel $h(x,x') = \ip{x,x'}_\cX$ is also known as the \emph{linear kernel}.
The use of the term `canonical' is fitting not just due to the relation between a Hilbert space and its continuous dual space.
Let $\phi:\cX\to\cV$ be the feature map from the space of covariates (inputs) to some feature space $\cV$.
Suppose both $\cX$ and $\cV$ is a Hilbert space, then a kernel is defined as 
\[
  h(x,x') = \ip{\phi(x),\phi(x')}_\cV.
\]
Taking the feature map to be $\phi(x) = \ip{\cdot,x}_\cX$, we can prove the reproducing property to obtain $h(x,x') = \ip{x,x'}_\cX$, which implies $\phi(x) = h(\cdot,x)$, and thus $\phi$ is the \emph{canonical feature map} \citep[Lemma 4.19]{steinwart2008support}.

The origin of a Hilbert space may be arbitrary, in which case a centring may be appropriate.
We define the centred canonical RKHS as follows.

\begin{definition}[Centred canonical RKHS]
  Let $\cX$ be a Hilbert space, $\Prob$ be a probability measure over $\cX$, and $\mu\in\cX$ be the mean (i.e. $\E\ip{x,x'}_{\cX}  = \ip{\mu,x'}_{\cX}$ for all $x' \in \cX$) with respect to this probability measure.
  Define $(\cX - \mu)'$, the continuous dual space of $\cX - \mu$, to be the \emph{centred canonical RKHS}.
  $(\cX - \mu)'$ consists of the centred linear functions $f_\beta(x)=\ip{x-\mu,\beta}_\cX$, for $\beta\in\cX$, such that $\E f_\beta(x) = 0$.
  The reproducing kernel of $(\cX - \mu)'$ is
  \[
    h(x,x') = \ip{x-\mu,x'-\mu}_\cX.
  \]
\end{definition}

\begin{proof}
  Proof of the claim $\E f_\beta(x) = 0$:
  \begin{align*}
    \E f_\beta(x) 
    &= \E \ip{x-\mu,\beta}_\cX \\
    &= \E \ip{x,\beta}_\cX - \ip{\mu,\beta}_\cX,
  \end{align*}
  and since $\E \ip{x,\beta}_\cX = \ip{\mu,\beta}_\cX$ for any $\beta\in\cX$, the results follows.
\end{proof}

\begin{remark}
  In practice, the probability measure $\Prob$ over $\cX$ is unknown, so we may use the empirical distribution over $\cX$, so that $\cX$ is centred by the sample mean $\hat\mu = \frac{1}{n}\sum_{i=1}^n x_i$.  
\end{remark}


\subsection{The fractional Brownian motion RKHS}

Brownian motion (also known as the Wiener process) has been an inquisitive subject in the mathematical sciences, and here, we describe a function space influenced by a generalised version of Brownian motion paths.

Suppose $B_\gamma(t)$ is a continuous-time Gaussian process on $[0,T]$, i.e. for any finite set of indices $t_1,\dots,t_k$, where each $t_j \in [0,T]$, $\big(B_\gamma(t_1),\dots,B_\gamma(t_k)\big)$ is a multivariate normal random variable.
$B_\gamma(t)$ is said to be a \emph{fractional Brownian motion} (fBm) if $\E B_\gamma(t) = 0$ for all $t \in [0,T]$ and 
\[
  \Cov\big(B_\gamma(t),B_\gamma(s) \big) = \half\big( |t|^{2\gamma} + |s|^{2\gamma} - |t-s|^{2\gamma} \big) \hspace{1cm} \forall t,s \in [0,T],
\]
where $\gamma \in (0,1)$ is called the Hurst index or Hurst parameter.
Introduced by \citet{mandelbrot1968fractional}, fBms are a generalisation of Brownian motion.
The Hurst parameter plays two roles: 1) It describes the raggedness of the resultant motion, with higher values leading to smoother motion; and 2) it determines the type of process the fBm is, as past increments of $B_\gamma(t)$ are weighted by $(t-s)^{\gamma-1/2}$.
When $\gamma=1/2$ exactly, then the fBm is a standard Brownian motion and its increments are independent; when $\gamma > 1/2$ ($\gamma < 1/2$) its increments are positively (negatively) correlated.

%\citet{schoenberg1937certain} has shown that, for $0 < \gamma\leq 1$, there exists a Hilbert space $\cB$ and a function $\phi_\gamma:\cX\to\cB$ such that $\forall x,x' \in \cX$,
%\[
%  \big\Vert \phi_\gamma(x) - \phi_\gamma(x') \big\Vert_\cB = \norm{x-x'}_\cX^\gamma.
%\]

Let $\cX$ be a Hilbert space. 
Defining a kernel function $h:\cX\times\cX\to\bbR$ identical to the fBm covariance kernel yields the so-called \emph{fractional Brownian motion RKHS}.

\begin{definition}[Fractional Brownian motion RKHS]\label{def:fbmrkhs}
  The fractional Brownian motion (fBm) RKHS $\cF$ is the space of functions on the Hilbert space $\cX$ possessing the reproducing kernel $h:\cX\times\cX\to\bbR$ defined by
  \[
    h(x,x') = \half\big( \norm{x}_\cX^{2\gamma} + \norm{x'}_\cX^{2\gamma} - \norm{x-x'}_\cX^{2\gamma} \big),
  \]
  which depends on the Hurst coefficient $\gamma \in (0,1)$.
  We shall reference this space as the fBm-$\gamma$ RKHS.
\end{definition}

\begin{remark}
  When $\gamma=1$, by the polarisation identity we get $h(x,x') = \ip{x,x'}_\cX$, which is the (reproducing) kernel of the canonical RKHS.
\end{remark}

From its construction, it is clear that the fBm kernel is positive definite, and thus defines an RKHS.
That the fBm RKHS describes a space of functions is proved in \citet{cohen2002}, who studied this space in depth. 
It is also noted in the collection of examples of \citet[pp.71 \& 319]{berlinet2011reproducing}.

The Hurst coefficient $\gamma$ controls the ``smoothness'' of the functions in the RKHS. 
We can talk about smoothness in the context of Hölder continuity of functions.

\begin{definition}[Hölder condition]
  A function $f$ over a set $(\cX, \norm{\cdot}_\cX)$ is said to be \emph{Hölder continuous} of order $0 <\gamma\leq 1$ if there exists a $C>0$ such that $\forall x,x'\in\cX$,
  \[
    \vert f(x) - f(x') \vert \leq C \norm{x-x'}^\gamma.
  \]
\end{definition}

Functions in the Hölder space $\text{C}^{k,\gamma}(\cX)$, where $k\geq 0$ is an integer, consists of those functions over $\cX$ having continuous derivatives up to order $k$ and such that the $k$th partial derivatives are Hölder continuous of order $\gamma$.
Unlike realisations of actual fBm paths with Hurst index $\gamma$, which are well-known to be almost surely Hölder continuous of order less than $\gamma$ \citep[Theorem 4.1.1]{embrechts2002selfsimilar}, functions in its namesake RKHS are strictly smoother.

%\begin{claim}
%  Let $\cF$ denote the fBm RKHS of functions over $\cX$ with Hurst parameter $\gamma \in (0,1)$, and the kernel $h$ as defined in Definition \ref{def:fbmrkhs}.
%  Then,
%  \begin{enumerate}[label=(\roman*)]
%    \item The functions in $\cF$ are Hölder continuous of order $\gamma$.  
%    \item The basis functions $h(\cdot,x)$ are Hölder continuous of order $2\gamma$.
%  \end{enumerate}
%\end{claim}

\begin{claim}
  The fBm-$\gamma$ RKHS $\cF$ of functions over $(\cX, \norm{\cdot}_\cX)$ are Hölder continuous of order $\gamma$.
\end{claim}

\begin{proof}
  For some $f \in \cF$ we have $f(x) = \ip{f,h(\cdot,x)}_\cF$ by the reproducing property of the kernel $h$ of $\cF$.
  It follows from the Cauchy-Schwarz inequality that for any $x,x'\in\cX$,
  \begin{align*}
    \vert f(x) - f(x') \vert 
    &= \vert \ip{f,h(\cdot,x) - h(\cdot,x')}_\cF \vert \\
    &\leq \norm{f}_\cF \cdot \big\Vert h(\cdot,x) - h(\cdot,x') \big\Vert_\cF \\
    &= \norm{f}_\cF \cdot \norm{x-x'}_\cX^{\gamma},
  \end{align*}
  since
  \begin{align*}
    \big\Vert h(\cdot,x) - h(\cdot,x') \big\Vert_\cF ^2
    &= \big\Vert h(\cdot,x) \big\Vert_\cF ^2 + \big\Vert h(\cdot,x') \big\Vert_\cF ^2 - 2 \ip{h(\cdot,x),h(\cdot,x')}_\cF \\
    &= h(x,x) + h(x',x') - 2 h(x,x') \\
    &= \norm{x-x'}_\cX^{2\gamma},
  \end{align*}  
  and thus proving the claim.
\end{proof}

\hltodo[This is the same for any RKHS?]{The fBm-$\gamma$ RKHS is spanned by the functions $h(\cdot,x)$, which means that $f(0)=0$ for all $f \in \cF$, which may be undesirable}.
We define the centred fBm RKHS as follows.

\begin{definition}[Centred fBm RKHS]
  Let $\cX$ be a Hilbert space, $\Prob$ be a probability measure over $\cX$, and $\mu\in\cX$ be the mean (i.e. $\E\ip{x,x'}_{\cX}  = \ip{\mu,x'}_{\cX}$ for all $x' \in \cX$) with respect to this probability measure.
  The kernel $\bar h:\cX\times\cX\to\bbR$ defined by
  \[
    \bar h(x,x') = \half \E \left[ \norm{x-X}_\cX^{2\gamma} + \norm{x'-X'}_\cX^{2\gamma} - \norm{x-x'}_\cX^{2\gamma} - \norm{X-X'}_\cX^{2\gamma} \right]
  \]
  is the reproducing kernel of the \emph{centred} fBm-$\gamma$ RKHS, which consists of functions $f$ in the fBm-$\gamma$ RKHS \hltodo[Proof?]{such that $\E f(X) = 0$}.
  In the above definition, $X,X' \sim \Prob$ are two independent copies of a random vector $X \in \cX$.
\end{definition}

\begin{remark}
  Again, when $\gamma=1$, we get the reduction 
  \begin{align*}
    \bar h(x,x') 
    &= \half \E \left[ \norm{x-X}_\cX^{2} + \norm{x'-X'}_\cX^{2} - \norm{x-x'}_\cX^{2} - \norm{X-X'}_\cX^{2} \right] \\
    &= \half \E \left[ \ip{X,X}_\cX + \ip{X',X'}_\cX + 2\ip{x,x'}_\cX - 2\ip{x,X}_\cX - 2\ip{x',X'}_\cX\right] \\
    &= \ip{\mu,\mu}_\cX + \ip{x,x'}_\cX - \ip{x,\mu}_\cX - \ip{\mu,x'}_\cX \\
    &= \ip{x-\mu,x'-\mu}_\cX,
  \end{align*}
  which is the (reproducing) kernel of the centred canonical RKHS.
\end{remark}

\subsection{The squared exponential RKHS}

The \gls{SE} kernel function is indeed known to be the default kernel used for Gaussian process regression in machine learning.
It is a positive definite function, and hence defines an RKHS.
The definition of the \gls{SE} RKHS is as follows.

\begin{definition}[Squared exponential RKHS]
  The squared exponential (SE) RKHS $\cF$ of functions over some set $\cX \subseteq \bbR^p$ equipped with the 2-norm $\norm{\cdot}_2$ is defined by the positive definite kernel $\hXXR$ 
  \[
    h(x,x') = \exp\left(-\frac{\norm{x-x'}_2^2}{2l^2} \right).
  \]
  The real-valued parameter $l > 0$ is called the \emph{lengthscale} parameter, and is a smoothing parameter for the functions in the RKHS.
\end{definition}

It is known by many other names, including the Gaussian kernel, due to its semblance to the kernel of the Gaussian pdf. 
Especially in the machine learning literature, the term Gaussian radial basis functions (RBF) is used, and commonly the simpler parameterisation $\gamma = 1 / 2l^2$ is utilised.
\citet{duvenaud2014automatic} remarks that ``exponentiated quadratic'' is a more fitting descriptive name for this kernel.

Despite being used extensively for learning algorithms using kernels, an explicit study of the RKHS defined by the SE kernel was not done until recently by \citet{steinwart2006explicit}.
In that work, the authors describe the nature of real-valued functions in the SE RKHS by considering a a real restriction on the SE RKHS of functions over complex values.
Their derivation of an orthonormal basis of such an RKHS proved the SE kernel to be the reproducing kernel for the SE RKHS.

\hltodo{Are SE smoother than fBm? Lipschitz continuous. Compact convergence. May be smoother than functions in an fBm RKHS?}

SE kernels are known to be ``universal''. That is, it satisfied the following definition of universal kernels due to \citet{micchelli2006universal}.

\begin{definition}[Universal kernel]
  Let $\text{C}(\cX)$ is the space of all continuous, complex-valued functions $f:\cX\to\bbC$ equipped with the maximum norm $\norm{\cdot}_\infty$, and denote $\cK(\cX)$ as the space of \emph{kernel sections} $ \overline{\text{span}}\{ h(\cdot,x) | x \in \cX \}$, where here, $h$ is a complex-valued kernel function.
  A kernel $h$ is said to be \emph{universal} if given any compact subset $\cZ \subset \cX$, any positive number $\epsilon$ and any function $f \in \text{C}(\cZ)$, there is a function $g \in \cK(\cZ)$ such that $\norm{f-g}_\cZ \leq \epsilon$.
\end{definition}

The consequence of this universal property vis-à-vis regression modelling is that any (continuous) regression function $f$ may be approximated very well by a function $\hat f$ from the SE RKHS, and these two functions can get arbitrarily close to each other in the max norm sense.
This, together with some very convenient computational advantages that the SE kernel brings (more on this in a later chapter), is a testament to the popularity of SE kernels.

In a similar manner to the two previous subsections, we may also derive the \emph{centred} SE RKHS. 

\begin{definition}[Centred SE RKHS]
  Let $\cX \subseteq \bbR^p$ be equipped with the 2-norm $\norm{\cdot}_2$, and let $\Prob$ denote the distribution over $\cX$.
  The \emph{centred} squared exponential (SE) RKHS (with lengthscale $l$) of functions over $\cX$ is defined by the positive definite kernel $\hXXR$ 
  \[
    h(x,x') = \exp\left(-\frac{\ip{x-\mu,x'-\mu}}{2l^2} \right),
  \]
  where $\mu =: \E X \in\cX $ under $\Prob$, and $\ip{\cdot,\cdot}$ represents the usual dot product in Euclidean space.
  \hltodo[Proof?]{This ensures that $\E f(X) = 0$ for any $f$ in this RKHS}.
\end{definition}

%\begin{proof}
%  \begin{align*}
%    \E f(X) 
%    &= \E\ip{f,h(\cdot,X)}_\cF \\
%    &= \E \left\langle f, \exp\left(-\frac{\ip{\cdot,X-\mu}}{2l^2} \right) \right\rangle_\cF
%  \end{align*}
%  but the term on the right side of the inner product can be shown to have expectation 0 no matter what value is plugged in
%\end{proof}

\subsection{The Pearson RKHS}

In all of the previous RKHS of functions, the domain $\cX$ was taken to be some Euclidean space. 
The Pearson RKHS is a vector space of functions whose domain $\cX$ is a finite set.
Let $\Prob$ be a probability measure over the finite set $\cX$. 
The Pearson RKHS is defined as follows.

\begin{definition}[Pearson RKHS]
  The \emph{Pearson RKHS} is the RKHS of functions over a finite set $\cX$ defined by the reproducing kernel
  \[
    h(x,x') = \frac{\delta_{xx'}}{\Prob(X=x)} - 1,
  \]
  where $X \sim \Prob$ and $\delta$ is the Kronecker delta.
\end{definition}

The Pearson RKHS contains functions which are centred, and has the desirable property that the contribution of $f(x)^2$ to the squared norm of $f$ is proportional to $\Prob(X=x)$.

\begin{claim}
  Let $\cF$ be the Pearson RKHS of functions over a finite set $\cX$.
  Then,
  \[
    \cF = \{f:\cX\to\bbR | \E f(X) = 0 \}
  \]
  with
  \[
    \norm{f}_\cF^2 = \Var f(X) = \sum_{x\in\cX} \Prob(X=x)f(x)^2, \ \forall f \in \cF.
  \]
\end{claim}

\begin{proof}
  Write $p_x = \Prob(X=x)$.
  The set of functions $\{h(\cdot,x) | x \in \cX\}$ form a basis for $\cF$, and thus each $f \in \cF$ can be written as $f(x) = \sum_{x'\in\cX} w_{x'}h(x,x')$ for some scalars $w_i\in\bbR$, $i\in\cX$.
  But $\E h(X,x') = \E [\delta_{Xx'}] / p_{x'} - 1 = p_{x'} / p_{x'} - 1 = 0$, and thus $\E f(X) = 0$.
  Conversely, suppose $f:\cX\to\bbR$ is such that $\E f(X) = 0$.
  Taking $w_x = p_xf(x)$, we see that
  \begin{align*}
    \sum_{x'\in\cX} w_{x'}h(x,x') 
    &= \frac{w_x}{p_x} - \sum_{x'\in\cX} w_{x'} \\
    &= \frac{f(x)\cancel{p_x}}{\cancel{p_x}} - \cancelto{\E f(X) = 0}{\sum_{x'\in\cX} p_{x'}f(x')} = f(x)
  \end{align*}
  and thus $h(\cdot,x)$ spans $\cF$ so $f\in\cF$.
  To provide the second part, noting that with the choice $w_x = p_xf(x)$ and due to the reproducing property of $h$ for the RKHS $\cF$, the squared norm is 
  \begin{align*}
    \ip{f,f}_\cF 
    &= \left\langle \sum_{x\in\cX} w_{x}h(\cdot,x), \sum_{x'\in\cX} w_{x'}h(\cdot,x') \right\rangle_\cF \\
    &= \sum_{x\in\cX} \sum_{x'\in\cX} w_{x} w_{x'} \left\langle h(\cdot,x), h(\cdot,x') \right\rangle_\cF \\
%    &= \sum_{x\in\cX} \sum_{x'\in\cX} w_{x} w_{x'} \left( \frac{\delta_{xx'}}{\Prob(X=x)} - 1 \right)
    &= \sum_{x\in\cX} \sum_{x'\in\cX} w_{x} w_{x'} h(x,x')  \\
    &= \sum_{x\in\cX} w_{x} f(x) \\
    &= \sum_{x\in\cX} \Prob(X=x) f(x)^2,
  \end{align*}
  which is also the variance of $f(X)$.
\end{proof}



