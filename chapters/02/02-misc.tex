\subsection{A vector space... of `functions'?}

At first glance, this may seem strange, that the notion of functions (as mappings from input to output space) and vector spaces are somehow equatable.
Upon further thought, one realises that firstly, two functions of a similar, particular form may be added together (in some meaningful way) resulting in a function in that same form. 
Secondly, multiplication of a function by a scalar $c$ can be thought of as $c$ times the output of that function.
Indeed, running through the checklist of what constitutes a vector space, we find that a ``space of functions'' satisfies them all.
In modern linear algebra texts, this checklist is the eight axioms of vector spaces over a field $\bbF$: The vectors forms an abelian group under addition, and this group has an $\bbF$-module structure.

\subsection{Orthogonal functional ANOVA for square integrable functions}

If $f$ is integrable, then the functional ANOVA decomposition of \cref{eq:functionalanova2} yields components which are orthogonal to each other.
Suppose we consider the (Hilbert) space of square integrable functions over $\cX$ with measure $\nu$, $\cF \equiv \text{L}^2(\cX,\nu)$.
There is an isomorphism $\text{L}^2(\cX_1\times\cdots\times\cX_d,\nu_1\times\cdots\times\nu_d) \cong \text{L}^2(\cX_1,\nu_1) \otimes \cdots \otimes \text{L}^2(\cX_d,\nu_d)$. 
See, for example, \citet{reed1972methods,kree1974produits}.
We notice that the decomposition in \cref{eq:functionalanova1} is orthogonal:

\begin{claim}
  For the one-way ANOVA decomposition in \cref{eq:functionalanova1}, $f_0$ and $f_1$ are orthogonal for the usual $\text{L}^2$ inner product.
\end{claim}

\begin{proof}
  Note that $f_0$ is a constant function, and that $f_1 = f- f_0$.
  Thus,
  \begin{align*}
    \ip{f_0,f_1} 
    &= \int f_0f_1 \d\nu \\
    &= f_0 \int \left(f - f_0\right) \d\nu \\
    &= f_0 (f_0 - f_0) = 0. \qedhere
  \end{align*}
\end{proof}

Therefore, the decomposition of $\cF = \cF_0 \displaystyle\mathop{\oplus}^\bot \bar\cF_1$ is in fact orthogonal. 
Indeed, the requirement \cref{eq:funcanovaorth} ensures orthogonality of the summands in the functional ANOVA decomposition formula \cref{eq:functionalanova2} \citep[Definition 1]{sobol2001global}.

%FOR TOPOLOGICAL SPACES ONLY: 
%On the other hand, a set which contains all of its limit points is said to be \emph{closed}.
%Clearly, a complete set must be closed, but a closed set need not necessarily be complete.

%\begin{corollary}[Riesz norm]
%  For any $f\in\cF$ a Hilbert space, define $L(f) = \ip{f,g}_\cF$ for some $g\in\cF$.
%  Then $\norm{L}_{\cF'} = \norm{g}_\cF$. 
%\end{corollary}
%
%\begin{proof}
%%  $L$ as defined by $L = \ip{\cdot,g}_\cF$ for some $g\in\cF$ is a linear operator $L:\cF\to\cF'$,
%%  so from Definition \ref{def:boundedop} of operator norms,
%%  \begin{align*}
%%    \norm{L} &= \sup_{f\in\cF} \frac{\norm{L(f)}_{\cF'}}{\norm{f}_\cF} \\
%%    &= \sup_{f\in\cF} \frac{|\ip{f,g}_{\cF}|}{\norm{f}_\cF} \\
%%    &=  \frac{|\ip{g,g}_{\cF}|}{\norm{g}_\cF} \\
%%    &= \norm{g}_\cF
%%  \end{align*}
%%  by the Cauchy-Schwarz inequality.
%%  Alternative proof:
%  By Cauchy-Schwarz,
%  \[
%    |L(f)| \leq \norm{f}_\cF\norm{g}_\cF
%  \]
%  so $\norm{L}_{\cF'}\leq \norm{g}_\cF$.
%  But  $|L(g)| = \norm{g}_\cF^2$, so in fact $\norm{L}_{\cF'} = \norm{g}$
%\end{proof}