\documentclass[a4paper,showframe,11pt,draft]{report}
\usepackage{standalone}
\standalonetrue
\ifstandalone
  \usepackage{../../haziq_thesis}
  \usepackage{../../haziq_maths}
  \usepackage{../../haziq_glossary}
  \addbibresource{../../bib/haziq.bib}
  \externaldocument{../01/.texpadtmp/introduction}
\fi

\begin{document}
\hChapterStandalone[2]{Vector space of functions}

One of the main assumptions for regression modelling with I-priors is that the regression functions lie in some vector space of functions.
At first glance, this may seem strange, that the notion of functions (as mappings from input to output space) and vector spaces are somehow equatable.
Upon further thought, one realises that firstly, two functions of a similar, particular form may be added together (in some meaningful way) resulting in a function in that same form. 
Secondly, multiplication of a function by a scalar $c$ can be thought of as $c$ times the output of that function.
Indeed, running through the checklist\footnotemark~of what constitutes a vector space, we find that a ``space of functions'' satisfies them all.
\footnotetext{In modern linear algebra texts, this is the eight axioms of vector spaces over a field $\bbF$: The vectors forms an abelian group under addition, and this group has an $\bbF$-module structure.}

The purpose of this chapter is to provide a concise review of functional analysis leading up to the theory of reproducing kernel Hilbert and Krein spaces (RKHS/RKKS).
The interest with these RKHS and RKKS is that these spaces have well-established mathematical structure and offer desirable topologies.
In particular, it allows the possibility of deriving the Fisher information for regression functions---this will be covered in Chapter 3.
As we shall see, RKHS are also extremely convenient in that they may be specified completely via their reproducing kernels.
Several of these function spaces are of interest to us, for example, spaces of linear functions, smoothing functions, and functions whose inputs are nominal values and even functions themselves.
RKHS are widely studied in the literature, but perhaps RKKS are less so.
To provide an early insight, RKKS are simply an extension of RKHS when its kernel is not positive-definite.
The flexibility provided by RKKS will prove both useful and necessary, especially when considering scaled function spaces, as I-prior modelling does.

It is emphasised that a deep knowledge of functional analysis, including RKHS and RKKS theory, is not at all necessary for I-prior modelling, so perhaps the advanced reader may wish to skip Sections 2.1--2.3. 
Section 2.4 describes the fundamental RKHS of interest for I-prior regression, which we refer to as the ``building block'' RKHS/RKKS.
The reason for this is that it is possible to construct new RKKS from existing ones, and this is described in Section 2.5.

A remark on notation: Elements of the vector space $\cF$ of real functions over a set $\cX$ are denoted $f(\cdot)$, or simply $f$.
This distinguishes them from the actual evaluation of the function at an input point $x \in \cX$, denoted $f(x) \in \bbR$.
For a much cleaner read, we dispense with boldface notation for vectors and matrices when talking about them, without ambiguity, in the abstract sense. 

\section{Some functional analysis}
The core study of functional analysis revolves around the treatment of functions as objects in vector spaces over a field\footnote{In this thesis, this will be $\bbR$ exclusively.}.
Vector spaces, or linear spaces as they are sometimes known, may be endowed with some kind of structure so as to allow ideas such as closeness and limits to be conceived.
Of particular interest to us is the structure brought about by \emph{inner products}\index{inner product}, which allow the rigorous mathematical study of various geometrical concepts such as lengths, directions, and orthogonality, among other things.
We begin with the definition of an inner product. 

\begin{definition}[Inner products]\label{def:innerprod}
	Let $\mathcal F$ be a vector space over $\mathbb R$. A function $\langle\cdot,\cdot\rangle_{\mathcal F}:\mathcal F \times \mathcal F \rightarrow \mathbb R$ is said to be an inner product on $\mathcal F$ if all of the following are satisfied:
	\begin{itemize}
	\item \textbf{Symmetry:} $\langle f, g\rangle_{\mathcal F} = \langle g, f	\rangle_{\mathcal F}$, $\forall f,g \in \mathcal F$.
	\item \textbf{Linearity:} $\langle a f_1 + b f_2, g\rangle_{\mathcal F} = a\langle f_1,g \rangle_{\mathcal F} + b\langle f_2,g \rangle_{\mathcal F}$, $\forall f_1, f_2, g \in \mathcal F$ and $\forall a,b \in \mathbb R$.
	\item \textbf{Non-degeneracy:} $\langle f, f\rangle_{\mathcal F} = 0 \Leftrightarrow f=0$.
%	\item \textbf{Positive-definiteness:} $\langle f, f\rangle_{\mathcal F} \geq 0$, $\forall f \in \mathcal F$.
	\end{itemize}
%	Conversely, an inner product is said to be \emph{negative definite} if $\langle f, f\rangle_{\mathcal F} \leq 0$, $\forall f \in \mathcal F$. 
%	An inner product is said to be \emph{indefinite} it is neither positive nor negative definite.
\end{definition}

Additionally, an inner product is said to be \emph{positive definite} if $\langle f, f\rangle_{\mathcal F} \geq 0$, $\forall f \in \mathcal F$.
It is possible that this may not be the case, and we shall revisit this fact later when we cover Krein spaces.
For the purposes of the discussion moving forward, we shall refer to positive definite inner products, unless otherwise stated.

We can always define a \emph{norm} on $\cF$ using the inner product as 
\begin{align}\label{eq:normip}
  \norm{f}_\cF = \sqrt{\ip{f,f}_\cF}.
\end{align}
%In the above definition we had used the term \emph{norm}.
Norms are another form of structure that specifically captures the notion of length. 
This is defined below.

\begin{definition}[Norms]
	Let $\mathcal F$ be a vector space over $\mathbb R$. A non-negative function $||\cdot||_{\mathcal F}:\mathcal F \times \mathcal F \rightarrow \mathbb [0,\infty)$ is said to be a norm  on $\mathcal F$ if all of the following are satisfied:
	\begin{itemize}
	\item \textbf{Absolute homogeneity:} $||\lambda f||_{\mathcal F} = |\lambda| \cdot ||f||_{\mathcal F}$, $\forall \lambda \in \mathbb R$, $\forall f \in \mathcal F$
	\item \textbf{Subadditivity:} $||f+g||_{\mathcal F} \leq ||f||_{\mathcal F} + ||g||_{\mathcal F}$, $\forall f,g \in \mathcal F$
	\item \textbf{Point separating:} $||f||_{\mathcal F} = 0 \Leftrightarrow f=0$
	\end{itemize}
\end{definition}

\index{subadditivity}\index{inequality!triangle}
The subadditivity property is also known as the \emph{triangle inequality}.
Also note that since $\norm{-f}_\cF = \norm{f}_\cF$, and by the triangle inequality and point separating property, we have that $\norm{f}_\cF + \norm{-f}_\cF \geq \norm{f - f}_\cF = \norm{0}_\cF = 0$, thus implying non-negativity of norms.
Several important relationships between norms and inner products hold in linear spaces, namely, the \emph{Cauchy-Schwarz inequality}\index{inequality!Cauchy-Schwarz}
\[
  |\ip{f,g}_\cF| \leq \norm{f}_\cF\cdot\norm{g}_\cF;
\]
the \emph{parallelogram law}
\[
  \norm{f+g}_\cF^2 - \norm{f+g}_\cF^2 = 2\norm{f}_\cF^2 + 2\norm{g}_\cF^2;
\]
and the \emph{polarisation identity}
\[
  \norm{f+g}_\cF^2 + \norm{f+g}_\cF^2 = 4\ip{f,g}_\cF,
\]
for some $f,g\in\cF$.

A vector space endowed with an inner product (c.f. norm) is called an inner product space (c.f. normed vector space).
%A normed vector space is a vector space whose vectors have lengths, as induced by its norm.
As a remark, inner product spaces can always be equipped with a norm using \eqref{eq:normip}, but not always the other way around.
A norm needs to satisfy the parallelogram law for an inner product to be properly defined.

The norm $||\cdot||_{\mathcal F}$, in turn, induces a metric (a notion of distance) on $\mathcal F$: $d(f,g) = ||f-g||_{\mathcal F}$, for $f,g\in\cF$.
With these notions of distances, one may talk about sequences of functions in $\cF$ which are \emph{convergent}, and sequences whose elements become arbitrarily close to one another as the sequence progresses (\emph{Cauchy}).

\begin{definition}[Convergent sequence]
	A sequence $\{f_n\}_{n=1}^\infty$ of elements of a normed vector space $(\mathcal F, ||\cdot ||_{\mathcal F})$ is said to be \emph{converge} to some $f\in\cF$, if for every $\epsilon > 0$, $\exists N=N(\epsilon) \in \mathbb N$, such that $\forall n > N$, $||f_n - f||_{\mathcal F} < \epsilon$.
\end{definition}

\begin{definition}[Cauchy sequence]
	A sequence $\{f_n\}_{n=1}^\infty$ of elements of a normed vector space $(\mathcal F, ||\cdot ||_{\mathcal F})$ is said to be a Cauchy sequence if for every $\epsilon > 0$, $\exists N=N(\epsilon) \in \mathbb N$, such that $\forall n,m > N$, $||f_n - f_m||_{\mathcal F} < \epsilon$.
\end{definition}

Every convergent sequence is Cauchy (from the triangle inequality), but the converse is not true.
If the limit of the Cauchy sequence exists within the vector space, then the sequence converges to it.
If the vector space contains the limits of all Cauchy sequences (or in other words, if every Cauchy sequence converges), then it is said to be \emph{complete}.
On the other hand, a set which contains all of its limit points is said to be \emph{closed}.
Clearly, a complete set must closed, but a closed set need not necessarily be complete.

There are special names given to complete vector spaces.
A complete inner product space is known as a \emph{Hilbert space}, while a complete normed space is called a \emph{Banach space}.
Out of interest, an inner product space that is not complete is sometimes known as a \emph{pre-Hilbert space}, since its completion with respect to the norm induced by the inner product is a Hilbert space.

Being vectors in a vector space, we can discuss mapping the vectors onto a different space, or in essence, having a function acted upon them.
To establish terminology, we define linear functionals, bilinear form, and linear operators.

\begin{definition}[Linear functional]
  Let $\cF$ be a Hilbert space.
  A \emph{functional} $L$ is a map from $\cF$ to $\bbR$, and we denote its action on a function $f$ as $L(f)$. 
  A functional is called \emph{linear} if it satisfies $L(f+g)=L(f)+L(g)$ and $L(\lambda f)=\lambda L(f)$, for all $f,g\in\cF$ and $\lambda\in\bbR$.
\end{definition}

\begin{definition}[Bilinear form]
  Let $\cF$ be a Hilbert space.
  A \emph{bilinear form} $B$ takes inputs $f,g\in\cF$ and returns a real value.
  It is linear in each argument separately, i.e.
  \begin{itemize}
    \item $B(\lambda_1 f +\lambda_2 g, h) = \lambda_1 B(f,h) + \lambda_2 B(g,h)$; and
    \item $B(f, \lambda_1 g +\lambda_2 h) = \alpha B(f,g) + \lambda_2 B(f,h)$,
  \end{itemize} 
  for all $f,g,h \in \cF$ and $\lambda_1,\lambda_2\in\bbR$.
\end{definition}

\begin{definition}[Linear operator]
  Let $\cF$ and $\cG$ be two Hilbert spaces over $\bbR$.
  An operator $A$ is a map from $\cF$ to $\cG$, and we denote its action on a function $f \in\cF$ as $Af \in \cG$.
  A \emph{linear operator} satisfies $A(f+g) = A(f) + A(g)$ and $A(\lambda f) = \lambda A(f)$, for all $f,g \in\cF$ and $\lambda\in\bbR$.
\end{definition}

The term `functional' is classically used in calculus of variations to denote `a function of a function', i.e. a function having another function as its input, and outputs a real number.
Really, from a function space perspective, it is simply a mapping of functions onto another vector space (the reals in this case).
More generally, if the output space is another Hilbert space, then it is an operator.
An interesting property of these operators to look at, besides linearity, is whether or not they are \emph{continuous}.

\index{continuous}
\index{continuous!uniform}
\begin{definition}[Continuity]\label{def:continuity}
  Let $\cF$ and $\cG$ be two Hilbert spaces.
  A function $A:\cF\to\cG$ is said to be \emph{continuous at $g\in\cF$}, if for every $\epsilon>0$, $\exists \delta=\delta(\epsilon,g)>0$ such that
  \[
    \norm{f-g}_\cF < \delta \ \ \Rightarrow \ \ \norm{Af - Ag}_\cG < \epsilon.
  \]
  A is \emph{continuous} on $\cF$, if it is continuous at every point $g \in\cF$.
  If, in addition, $\delta$ depends on $\epsilon$ only, $A$ is said to be \emph{uniformly continuous}.
\end{definition}

Continuity in the sense of linear operators here means that a convergent sequence in $\cF$ can be mapped to a convergent sequence in $\cG$.
For a special case of linear operator, the evaluation functional, this means that a function in $\cF$ is continuous if the evaluation functional is continuous---more on this later in Section \ref{sec:rkhstheory}.
There is an even stronger notion of continuity called the \emph{Lipschitz continuity}.

\begin{definition}[Lipschitz continuity]
  Let $\cF$ and $\cG$ be two Hilbert spaces.  
  A function $A:\cF\to\cG$ is \emph{Lipschitz continuous} if $\exists M >0$ such that $\forall f,f'\in\cF$,
  \[
    \norm{Af - Af'}_\cG \leq M \norm{f - f'}_\cF.
  \]
\end{definition}

Clearly, Lipschitz continuity implies uniform continuity: Choose $\delta = \delta(\epsilon) := \epsilon/M$ and replace this in Definition \ref{def:continuity}.
So important is the concept of linearity and continuity, that there are specially named spaces which contain linear and continuous functionals.

\begin{definition}[Dual spaces]
  Let $\cF$ be a Hilbert space. 
  The space $\cF^*$ of \emph{linear functionals} is called the \emph{algebraic dual space} of $\cF$.
  The space $\cF'$ of \emph{continuous linear functionals} is called the \emph{continuous dual space} or alternatively, the \emph{topological dual space}, of $\cF$.   
\end{definition}

As it turns out, the algebraic dual space and continuous dual space coincide in finite-dimensional Hilbert spaces:
Take any $L\in\cF'$; since $L$ is finite-dimensional, it is bounded, and therefore continuous (see Lemma \ref{thm:boundcont}) so $L\in\cF'$ and $\cF^* \subseteq \cF'$; but $\cF' \subseteq \cF^*$ trivially, so $\cF^* \equiv \cF'$.
For infinite-dimensional Hilbert spaces, this is not so, but in any case, we will only be considering the continuous dual space in this thesis.

\begin{definition}[Bounded operator]\label{def:boundedop}
  The linear operator $A:\mathcal F \rightarrow \mathcal G$ between two Hilbert spaces $\cF$ and $\cG$ is said to be \emph{bounded} if there exists some $M>0$ such that
  \[
    \norm{Af}_\cG \leq M \norm{f}_\cF.
  \] 
  The smallest such $M$ is defined to be the \emph{operator norm}, denoted $\norm{A} := \sup_{f\in\cF} \frac{\norm{Af}_\cG}{\norm{f}_\cF}$.
\end{definition}

\begin{lemma}[Equivalence of boundedness and continuity]\label{thm:boundcont}
  Let $\cF$ and $\cG$ be two Hilbert spaces, and $A:\cF\to\cG$ a linear operator.
  $A$ is a bounded if and only if it is continuous.
\end{lemma}

\begin{proof}
  Suppose that $L$ is bounded.
  Then, $\forall f,f' \in \cF$, there exists some $M>0$ such that $\norm{A(f-g)}_\cG \leq M \norm{f-g}_\cG.$
  Conversely, let $A$ be a continuous linear operator, especially at the zero vector.
  In other words, $\exists \delta > 0$ such that $\norm{A(f)}_\cG = \norm{A(f+0-0)}_\cG = \norm{A(f) - A(0)} \leq 1$, $\forall f\in\cF$ whenever $\norm{f}_\cF \leq \delta$.
  Thus, for all non-zero $f \in\cF$,
  \begin{align*}
    \norm{A(f)}_\cG &= \left\Vert \frac{\norm{f}_\cF}{\delta} A\left(\frac{\delta}{\norm{f}_\cF}f\right)\right\Vert_\cG \\
    &= \left\vert \frac{\norm{f}_\cF}{\delta}\right\vert \cdot \left\Vert A\left(\frac{\delta}{\norm{f}_\cF}f\right)\right\Vert_\cG \\    
    &\leq \frac{\norm{f}_\cF}{\delta} \cdot 1,
  \end{align*}
  and thus $A$ is bounded.
\end{proof}

The following result is an important one, which states that (continuous) linear functionals of an inner product space are nothing more than just inner products.

\begin{theorem}[Riesz representation]
  Let $\cF$ be a Hilbert space.
  Every element $L$ of the continuous dual space $\cF'$, i.e. all continuous linear functionals $L:\cF\to\bbR$, can be uniquely written in the form $L=\ip{\cdot,g}_\cF$, for some $g\in\cF$.
\end{theorem}

\begin{proof}
  Omitted---see \citet[Theorem 4.12]{rudin1987real} for a proof.
\end{proof}

\begin{corollary}[Riesz norm]
  For any $f\in\cF$ a Hilbert space, define $L(f) = \ip{f,g}_\cF$ for some $g\in\cF$.
  Then $\norm{L}_{\cF'} = \norm{g}_\cF$. 
\end{corollary}

\begin{proof}
%  $L$ as defined by $L = \ip{\cdot,g}_\cF$ for some $g\in\cF$ is a linear operator $L:\cF\to\cF'$,
%  so from Definition \ref{def:boundedop} of operator norms,
%  \begin{align*}
%    \norm{L} &= \sup_{f\in\cF} \frac{\norm{L(f)}_{\cF'}}{\norm{f}_\cF} \\
%    &= \sup_{f\in\cF} \frac{|\ip{f,g}_{\cF}|}{\norm{f}_\cF} \\
%    &=  \frac{|\ip{g,g}_{\cF}|}{\norm{g}_\cF} \\
%    &= \norm{g}_\cF
%  \end{align*}
%  by the Cauchy-Schwarz inequality.
%  Alternative proof:
  By Cauchy-Schwarz,
  \[
    |L(f)| \leq \norm{f}_\cF\norm{g}_\cF
  \]
  so $\norm{L}_{\cF'}\leq \norm{g}_\cF$.
  But  $|L(g)| = \norm{g}_\cF^2$, so in fact $\norm{L}_{\cF'} = \norm{g}$\hltodo[Not so convinced.]{}
\end{proof}

The notion of isometry (transformation that preserves distance) is usually associated with metric spaces---two metric spaces being isometric means that they identical in as far as their metric properties are concerned.
For Hilbert spaces (or normed spaces in general), there is an analogous concept as well in \emph{isometric isomorphism} (a bijective isometry), such that two Hilbert spaces being isometrically isomorphic imply that they have exactly the same geometric structure, but may very well contain fundamentally different objects.

\begin{definition}[Isometric isomorphism]
  Two Hilbert spaces $\cF$ and $\cG$ are said to be \emph{isometrically isomorphic} if there is a linear bijective map $A:\cF\to\cG$ which preserves the inner product, i.e. 
  \[
    \ip{f,f'}_\cF = \ip{Af,Af'}_\cG.
  \]
\end{definition}

In Hilbert spaces, this isometry is also known as \emph{linear isometry}.
A consequence of the Riesz representation theorem is that it gives us a canonical isometric isomorphism $A:f\mapsto \ip{\cdot,f}_\cF$ between $\cF$ and its continuous dual $\cF'$, whereby $\norm{Af}_{\cF'} = \norm{f}_\cF$.
Implicitly, this means that $\cF'$ is a Hilbert space as well.

Another important type of mapping is the mapping $P$ of an element in $\cF$ onto a closed subspace $\cG\subset\cF$, such that $Pf \in \cG$ is closest to $f$.
This mapping is called the \emph{orthogonal projection}, due to the fact that such projections yield perpendicularity in the sense that $\ip{f-Pf,g}_\cG = 0$ for any $g\in\cG$.
The remainder $f - Pf$ belongs to the \emph{orthogonal complement} of $\cG$.

\begin{definition}[Orthogonal complement]
  Let $\cF$ be a Hilbert space and $\cG \subset \cF$ be a closed subspace.
  The linear subspace $\cG^\bot = \{ f \,|\, \ip{f,g}_\cG = 0, \forall g \in \cG \}$ is called the orthogonal complement of $\cG$.
\end{definition}

\begin{theorem}[Orthogonal decomposition]
  Let $\cF$ be a Hilbert space and $\cG \subset \cF$ be a closed subspace.
  For every $f \in \cF$, we can write $f = g + g^c$, where $g \in \cG$ and $g^c \in \cG^\bot$, and this decomposition is unique.
\end{theorem}

\begin{proof}
  Omitted---see \citet[Theorem 4.11]{rudin1987real} for a proof.
\end{proof}

We can write $\cF = \cG \oplus \cG^\bot$, where the $\oplus$ symbol denotes the \emph{direct sum}, and such a decomposition is called a \emph{tensor sum decomposition}.
In infinite-dimensional Hilbert spaces, some subspaces are not closed, but all orthogonal complements are closed. 
In such spaces, the orthogonal complement of the orthogonal complement of $\cG$ is the closure of $\cG$, i.e. $(\cG^\bot)^\bot =: \overline \cG$, and we say that $\cG$ is dense in $\overline \cG$.
Another interesting fact regarding the orthogonal complement is that $\cG \cap \cG^\bot = \{ 0 \}$, since any $g\in \cG \cap \cG^\bot$ must be orthogonal to itself, i.e. $\ip{g,g}_\cG = 0$ implying that $g=0$.

\begin{corollary}
  Let $\cG$ be a subspace of a Hilbert space $\cF$. 
  Then, $\cG^\bot = \{0\}$ if and only if $\cG$ is dense in $\cF$.
\end{corollary}

\begin{proof}
  If $\cG^\bot=\{0\}$ then $(\cG^\bot)^\bot = \overline \cG = \cF$.
  Conversely, since $\cG$ is dense in $\cF$, we have $\cG^\bot = \overline\cG^\bot = \cF^\bot = \{0\}$.
%  Conversely, suppose that there exists a non-zero element $h \in \cG^\bot$.
%  Because $\cG$ is dense, we can construct a sequence $\{h_n\}_{n=1}^\infty\in\cG$ converging to $h$.
%  We have
%  \begin{align*}
%    \norm{h}_\cG^2 
%    &= \ip{h,h}_\cG \\
%    &= \ip{h,h}_\cG - \ip{h_n,h}_\cG \hspace{1em} \rlap{\color{gray} since $h$ is in $\cG^\bot$} \\
%    &= \ip{h-h_n,h}_\cG \\
%    &\leq \norm{h-h_n}_\cG \cdot \norm{h}_\cG,
%  \end{align*}
%  but the final term tends to zero since $h_n$ converges to $h$.
%  So $h=0$, a contradiction.
\end{proof}

%https://en.wikibooks.org/wiki/Functional_Analysis/Hilbert_spaces

For the last part of this introductory section on functional analysis, we discuss measures on Hilbert spaces, and in particular, a probability measure.
Let $\cX$ be a real topological space (e.g. real Hilbert spaces), and let $\cB(\cX)$ the Borel $\sigma$-algebra of $\cX$.
A measure $\nu$ on $\big(\cX,\cB(\cX)\big)$ is called a \emph{Borel measure} on $\cX$.
We shall only concern ourselves with finite Borel measures. 
If $\nu(\cX) = 1$ then $\nu$ is a \emph{(Borel) probability measure} and the measure space $\big(\cX,\cB(\cX),\nu\big)$ is a \emph{(Borel) probability space}.

\begin{definition}[Mean vector and covariance operator]
  Let $\nu$ be a Borel probability measure on a real topological space $\cX$.
  Supposing that the function $z \mapsto \ip{z,x}_\cX$ is integrable with respect to $\nu$, the element $\mu\in\cX$ satisfying 
  \[
    \ip{\mu,x} = \int_\cX \ip{z,x}_\cX \d\nu(z), \ \forall x \in \cX
  \]
  is called the \emph{mean vector}.
  If, furthermore, there is a positive, symmetric linear operator $C$ on $\cX$ such that
  \[
    \ip{Cx,x'} = \int_\cX \ip{x,z-\mu}_\cX\ip{x',z-\mu}_\cX \d\nu(z), \ \forall x,x' \in \cX,
  \]
  then $C$ is called the \emph{covariance operator}.
  The conditions requiring existence of the mean vector and covariance operator are $\int_\cX |x| \d\nu(x) < \infty$ and $\int_\cX |x|^2 \d\nu(x) < \infty$ respectively.
\end{definition}

\begin{definition}[Mean and covariance of functions]
  Let $\big(\cX,\cB(\cX),\nu\big)$ be a Borel probability space, and let $\phi:\cX\to\cF$ be a feature map of some Hilbert space of functions $\cF$.
  The \emph{mean element} of $\cF$ is defined as $\mu_f \in \cF$ satisfying
  \[
    \E\ip{f,f'}_\cF = \ip{\mu_f,f'}_\cF
  \]
  for all $f'\in\cF$.
  The quantity $\ip{\mu_f,f}_\cF := \E\ip{\phi(x),f}_\cF$ is denoted $\E f(X)$.
\end{definition}

\hltodo{Slightly confused: Do we need random functions $f\in\cF$ or are the covariates $x\in\cX$ assumed to be random? Later on in Section 2.4 and 2.5, we talk about $\E f(X)$ so there is some measure on $\cX$. However, when we prove the I-prior, we talk about $f$ itself being random.}


\section{Reproducing kernel Hilbert space theory}\label{sec:rkhstheory}
The introductory section sets us up nicely to discuss the coveted reproducing kernel Hilbert space.
This is a subset of Hilbert spaces for which its evaluation functionals are continuous (by definition, in fact).
The majority of this section, apart from defining RKHS, is to convince ourselves that each and every RKHS of functions can be specified solely through its reproducing kernel.
To begin, we consider a fundamental linear functional on a Hilbert space of functions $\cF$, that assigns a value to $f \in \mathcal F$ for each $x \in \mathcal X$.

\begin{definition}[Evaluation functional]
	Let $\mathcal F$ be a vector space of functions $f:\mathcal X \rightarrow \mathbb R$, defined on a non-empty set $\mathcal X$. 
	For a fixed $x \in \mathcal X$, the functional $\delta_x:\mathcal F \rightarrow \mathbb R$ as defined by $\delta_x(f) = f(x)$ is called the (Dirac) evaluation functional at $x$.
\end{definition}

It is easy to see that evaluation functionals are always linear: $\delta_x(\lambda f + g) = (\lambda f + g)(x) = \lambda f(x) + g(x) = \lambda\delta_x(f) + \delta_x(g)$.
This is in fact the linearity that was implied earlier on at the beginning of Chapter 2 when introducing the notion of functions behaving like vectors.
As a remark, the calculation of the (penalised) likelihood functional involves evaluations. 
It is therefore important for the evaluation functional to be continuous.
It turns out, this is exactly what RKHS provide.

\begin{definition}[Reproducing kernel Hilbert space]\label{def:rkhs}
	A Hilbert space $\cF$ of real-valued functions $f:\mathcal X \rightarrow \mathbb R$ on a non-empty set $\mathcal X$ is called a \emph{reproducing kernel Hilbert space} if the evaluation functional $\delta_x: f \mapsto f(x)$ is continuous (equivalently, bounded) on $\cF$, $\forall x \in \cX$. 
\end{definition}

%The continuity condition also represents the weakest condition required for both the existence of an inner product and the evaluation of every function in $\cF$ at every point in the domain $\cX$.
While the continuity condition by definition is what makes an RKHS, it is neither easy to check this condition in practice, nor is it intuitive as to the meaning of its name.
In fact, there isn't even any mention of what a reproducing kernel actually is.
In order to benefit from the desirable continuity property of RKHS, we should look at this from another, more intuitive, perspective. 
By invoking the Riesz representation theorem, we see that for all $x\in\cX$, there exists a unique element $h_x\in\cF$ such that
\[
  f(x) = \delta_x(f) = \ip{f,h_x}_\cF, \forall f \in \cF
\]
holds. 
Since $h_x$ itself is a function in $\cF$, it holds that for every $x' \in \cX$ there exists a $h_{x'}\in\cF$ such that
\[
  h_{x}(x') = \delta_{x'}(h_{x}) = \ip{h_x,h_{x'}}_\cF.
\]
This leads us to the definition of a \emph{reproducing kernel} of an RKHS---the very notion that inspired its name.

\begin{definition}[Reproducing kernels]\label{def:repkern}
  Let $\mathcal F$ be a Hilbert space of functions over a non-empty set $\mathcal X$. A function $h:\mathcal X\times\mathcal X\rightarrow\mathbb R$ is called a reproducing kernel of $\mathcal F$ if $h$ satisfies
  \begin{itemize}
    \item $\forall x \in \mathcal X,\, h(\cdot, x) \in \mathcal F$; and
    \item $\forall x \in \mathcal X, \, f \in \mathcal F, \, \langle f, h(\cdot, x) \rangle_{\mathcal F} = f(x)$ (the reproducing property).
  \end{itemize}
  In particular, for any $x, x' \in \mathcal X$,
  \[
  	h(x,x') = \langle h(\cdot, x), h(\cdot, x') \rangle_{\mathcal F}.
  \]
\end{definition}

Continuity of evaluation functionals in an RKHS means that functions that are close in RKHS norm imply that they are also close pointwise.
\hltodo[Not so sure why this is useful?]{It gives some reassurance when trying to estimate $f$ from $\cF$---just look at the size of the norm.}
More formally,

\begin{corollary}[Norm convergence implies pointwise convergence in RKHS]\label{thm:normpointconv}
  Let $\cF$ be an RKHS of real functions over $\cX$, and let $f_n$ be a sequence of points in $\cF$.
  Then, for some $f\in\cF$,
  \[
    \lim_{n\to\infty} \norm{f_n - f}_\cF = 0 \ \ \Rightarrow \ \ \lim_{n\to\infty} |f_n(x) - f(x)| = 0.
  \]
\end{corollary}

\begin{proof}
  Suppose $\cF$ is an RKHS with reproducing kernel $h$.
  Then,
  \begin{align*}
    |\delta_x(f) - \delta_x(g)| 
    &= |\delta_x(f-g)| \\
    &= |(f-g)(x)|  \\
    &= |\ip{f-g,h(\cdot,x)}_\cF| \hspace{1em} \rlap{\color{gray} (reproducing property)} \\
    &\leq \norm{h(\cdot,x)}_\cF \cdot \norm{f-g}_\cF \hspace{1em} \rlap{\color{gray} (by Cauchy-Schwarz)} \\
    &= \sqrt{h(x,x)}\cdot \norm{f-g}_\cF.
  \end{align*}
\end{proof}

Having defined an RKHS, there are several questions we might like the answer to: What is the relationship between a reproducing kernel and an RKHS? Can we speak to its existence and uniqueness? What other properties does it have?
The rest of this subsection will be dedicated to discuss the following assertion.

\begin{theorem}[RKHS]
  For every reproducing kernel Hilbert space $\cF$ of functions over a set $\cX$, there corresponds a unique, positive-definite reproducing kernel $\hXXR$.
  Conversely, for every positive-definite function $\hXXR$, there corresponds a unique reproducing kernel Hilbert space $\cF$ that has $h$ as its reproducing kernel.
\end{theorem}

In essence, there is a bijection between the set of positive-definite kernels and the set of reproducing kernel Hilbert spaces.
We will take take apart this theorem and inspect its constituent claims.
Firstly, on the definition of kernels and its positive-definiteness.

\begin{definition}[Kernels]\label{def:kernel}
  Let $\mathcal F$ be a Hilbert space (not necessarily a RKHS), $\mathcal X$ a non-empty set, and $\phi:\mathcal X \rightarrow \mathcal F$.   
  A \emph{kernel} is defined to be function $\hXXR$ that satisfies
  \[
    h(x,x') = \langle \phi(x), \phi(x') \rangle_{\mathcal F}
  \]
  $\forall x,x'\in\cX$.
  The map $\phi$ is referred to as the \emph{feature map}, and $\cF$ the \emph{feature space}.
\end{definition}

\begin{lemma}[Positive-definiteness of kernels]\label{thm:posdef}
  The kernel as defined in Definition \ref{def:kernel} is a symmetric and positive definite function, where a symmetric function $h:\mathcal X\times\mathcal X\rightarrow\mathbb R$ is said to be  positive definite if
  \[
    \sum_{i=1}^n\sum_{k=1}^n a_ia_jh(x_i, x_k) \geq 0.
  \]
  for all integers $n>1$, $\forall a_1, \dots, a_n \in \mathbb R$, and $\forall x_1, \dots, x_n \in \mathcal X$.
\end{lemma}

\begin{proof}
  \begin{align*}
    \sum_{i=1}^n\sum_{k=1}^n a_ia_jh(x_i, x_k)	
    &= \sum_{i=1}^n\sum_{k=1}^n \langle a_i\phi(x_i), a_k\phi(x_k) \rangle_{\mathcal F} \\
    &= \Bigg\langle \sum_{i=1}^n a_i\phi(x_i), \sum_{k=1}^n a_k\phi(x_k) \Bigg\rangle_{\mathcal F} \\
    &= \Bigg|\Bigg| \sum_{i=1}^n a_i\phi(x_i) \Bigg|\Bigg|_{\mathcal F}^2 \\
    & \geq 0
  \end{align*}
\end{proof}

\begin{corollary}[Positive-definiteness of reproducing kernels]
  Reproducing kernels of a RKHS are positive definite. 
%  For an RKKS, the reproducing kernel can be shown to be the difference between two positive definite kernels, but need not be itself positive definite.
\end{corollary}

\begin{proof}
  Take $\phi: x \mapsto h(\cdot,x)$. 
  By Definition \ref{def:kernel}, one has $h(x,x') = \langle h(\cdot, x), h(\cdot, x') \rangle_{\mathcal F}$, which is the reproducing property of the kernel in a RKHS, and this is positive-definite by Lemma \ref{thm:posdef}. 
%  The second statement follows by a similar argument and by definition of a RKKS (see Definition \ref{def:krein}).
  Incidentally, the $\phi$ as defined is known as the \emph{canonical feature map}.
\end{proof}

We have established what a kernel is, and that reproducing kernels of an RKHS are positive-definite. 
But do reproducing kernels always exist, and if so, are they unique to an RKHS?
Lemmas \ref{thm:rkhsexist} and \ref{eq:rkhsunique} answer these questions in the positive.

\begin{lemma}[Existence of reproducing kernels]\label{thm:rkhsexist}
  Let $\cF$ be a Hilbert space of functions over $\cX$.
  $\cF$ is a RKHS if and only if $\cF$ has a reproducing kernel.  
\end{lemma}

\begin{proof}
  Suppose $\cF$ is a RKHS with kernel $h$.
  Choose $\delta=\epsilon / \norm{h(\cdot,x)}_\cF$.
  Then, for any $f \in \cF$ such that $\norm{f-g}_\cF < \delta$, we have
  \begin{align*}
    \vert \delta_x (f) - \delta_x (g) \vert 
    &= \vert (f-g)(x) \vert \\
    &= |\ip{f-g,h(\cdot,x)}_\cF| \hspace{1em} \rlap{\color{gray} (reproducing property)} \\
    &\leq \norm{h(\cdot,x)}_\cF \cdot \norm{f-g}_\cF \hspace{1em} \rlap{\color{gray} (by Cauchy-Schwarz)} \\
    &= \epsilon.
  \end{align*}
  Thus, the evaluation functional is (uniformly) continuous on $\cF$.
  To prove the reverse, follow the argument preceding Definition \ref{def:repkern}.
\end{proof}

\begin{lemma}[Uniqueness of reproducing kernels]\label{eq:rkhsunique}
  The reproducing kernel $\hXXR$ of a RKHS $\cF$ of functions over $\cX$ is unique.
\end{lemma}

\begin{proof}
  Assume that $\cF$ has two reproducing kernels $h_1$ and $h_2$. 
  Then, $\forall f\in\cF$ and $\forall x\in\cX$,
  \begin{align*}
    \ip{f,h_1(\cdot,x) - h_2(\cdot,x)}_\cF = f(x) - f(x) = 0.
  \end{align*}
  In particular, if we take $f = h_1(\cdot,x) - h_2(\cdot,x)$, we obtain $\norm{h_1(\cdot,x) - h_2(\cdot,x)}^2_\cF = 0$
\end{proof}

Naturally, having seen that every RKHS corresponds to a unique reproducing kernel, we ask whether the converse is true.
That is, given a reproducing kernel, does it define a unique RKHS?
Astoundingly, the answer is again positive, and this is stated by the much celebrated Moore-Aronszajn theorem below.

\begin{theorem}[Moore-Aronszajn]
  If $\hXXR$ is a positive-definite function then there exists a unique RKHS whose reproducing kernel is $h$.
\end{theorem}

\begin{proof}[Sketch proof]
  Most of the details here have been omitted, except for the parts which we feel are revealing as to the properties of an RKHS.
  For a complete proof, see \citet{berlinet2011reproducing}. 
  Start with the linear space
  \[
    \cF_0 = \left\{ f_n:\cX\to\bbR \, \Big| \, f_n = \sum_{i=1}^n w_i h(\cdot,x_i), x_i\in\cX, w_i\in\bbR, n\in\bbN \right\}
  \]
  and endow this linear space with the following inner product:
  \[\label{eq:rkhsinnerprod}
    \left\langle \sum_{i=1}^n w_i h(\cdot,x_i), \sum_{j=1}^m w_j' h(\cdot,x_j') \right\rangle_{\cF_0} = \sum_{i=1}^n\sum_{j=1}^m w_i w_j' h(x_i,x_j').
  \]
  It may be shown that this indeed a valid inner-product satisfying the conditions laid in Definition \ref{def:innerprod}.
  At this point, the reproducing property is already had:
  \begin{align*}
    \big\langle f_n, h(\cdot,x) \big\rangle_{\cF_0} 
    &= \left\langle \sum_{i=1}^n w_i h(\cdot,x_i), h(\cdot,x) \right\rangle_{\cF_0} \\
    &= \sum_{i=1}^n w_i h(x_i,x) \\
    &= f_n(x),
  \end{align*}
  for any $f_n\in\cF_0$.
  
  Let $\cF$ be the completion of $\cF_0$ with respect to this inner product.
  In other words, define $\cF$ to be the set of functions $f:\cX\to\bbR$ for which there exists a Cauchy sequence $\{f_n\}_{n=1}^\infty$ in $\cF_0$ converging pointwise to $f \in \cF$.
  The inner product for $\cF$ is defined to be
  \[
    \ip{f,f'}_\cF = \lim_{n\to\infty} \ip{f_n,f_n'}_{\cF_0}.
  \]
  The sequence $\{ \ip{f_n,f_n'}_{\cF_0} \}_{n=1}^\infty$ is convergent and does not depend on the sequence chosen, but only on the limits $f$ and $f'$ \citep[Lemma 5]{berlinet2011reproducing}.
  We may check that this indeeds defines a valid inner product.
  The reproducing property carries over to the completion:
  \begin{align*}
    \ip{f,h(\cdot,x)}_\cF 
    &= \lim_{n\to\infty} \ip{f_n,h(\cdot,x)}_{\cF_0} \\
    &= \lim_{n\to\infty} f_n(x) \\
    &= f(x).
  \end{align*}
  
  To prove uniqueness, let $\cG$ be another RKHS with reproducing kernel $h$.
  $\cF$ has to be a closed subspace of $\cG$, since $h(\cdot,x) \in \cG$ for all $x\in\cX$, and because $\cG$ is complete and contains $\cF_0$ and hence its completion.
  Using the orthogonal decomposition theorem, we have $\cG = \cF \oplus \cF^\bot$, i.e. any $g\in\cG$ can be decomposed as $g = f + f^c$, $f\in\cF$ and $f^c\in\cF^\bot$.
  For each element $g\in\cG$ we have that, for all $x\in\cX$,
  \begin{align*}
    g(x) &= \ip{g,h(\cdot,x)}_\cG \\
    &= \big\langle f+f^c, h(\cdot,x) \big\rangle_\cG \\
    &= \big\langle f, h(\cdot,x) \big\rangle_\cG + \cancelto{0}{\big\langle f^c, h(\cdot,x) \big\rangle_\cG} \\
    &= f(x)
  \end{align*}
  so therefore $g\in\cF$ too.
  It must be that $\cF\equiv\cG$.
  %$\cF\cong\cG$.
%  
%  Minor detail: $f^c \in \cF^\bot \Rightarrow f^c \in \{f\in\cF | \ip{f,f'} = 0 \ \forall f'\in\cF \}$. Thus
%  \begin{align*}
%    \big\langle f^c, h(\cdot,x) \big\rangle_\cG 
%    &= f^c(x) \\
%    &= \big\langle f^c, h(\cdot,x) \big\rangle_\cF \\
%    &= 0
%  \end{align*}
%  It is evident that the inner product as defined is symmetric, linear, and positive-definite due to the positive-definiteness of reproducing kernels.
%  The only non-trivial condition to check is that whether $\ip{f_n,f_n}_{\cF_0}$ implies $f_n=0$.
%  To see this, realise that
%  \begin{align*}
%    0 &\leq 
%    1
%  \end{align*}
%  
%  By the Cauchy-Schwarz inequality, we see that
%  \begin{align*}
%    |f_n(x)| &= \big| \big\langle f_n, h(\cdot,x) \big\rangle_{\cF_0} \big| \\
%    &\leq \Vert h(\cdot,x) \Vert_{\cF_0} \cdot \Vert f_n \Vert_{\cF_0}    \\
%    &= \sqrt{h(x,x)} \cdot \Vert f_n \Vert_{\cF_0}
%  \end{align*}
%  so convergence in norm implies pointwise convergence.
%  By a similar argument in the proof of Corollary \ref{thm:normpointconv}, we also get that pointwise convergence is implied by convergence in norm for this space.
%  In particular, for every Cauchy sequence $\{f_n\}_{n=1}^\infty$ in $\cF_0$, $\big\{f_n(x)\big\}_{n=1}^\infty$ is a Cauchy sequence on the real line.
%  Complete the space $\cF_0$ by adjoining all of these limits to it, and call this completed space $\cF$.
%  
%  Furthermore, note that $\big\vert \norm{f_n}_{\cF_0} - \norm{f_m}_{\cF_0} \big\vert \leq \norm{f_n - f_m}_{\cF_0}$ (triangle inequality), so for a Cauchy sequence $\{f_n\}_{n=1}^\infty$ in a complete space, $\norm{f_n}_{\cF_0}$ has a limit.
%  Define the norm of $\cF$ to be $\norm{f}_\cF = \lim_{n\to\infty} \norm{f_n}_{\cF_0}$ for any $f\in\cF$.
%  Next, extend the inner product from $\cF_0$ to $\cF$ by defining $\ip{f,g}_\cF = \lim_{n\to\infty} \langle f_n,g_n \rangle_{\cF_0}$, where
%  \[
%    \lim_{n\to\infty} \langle f_n,g_n \rangle_{\cF_0} = \lim_{n\to\infty} \half\left( \norm{f+g}_{\cF_0}^2 - \norm{f}_{\cF_0}^2 - \norm{g}_{\cF_0}^2\right).
%  \]
%  One can indeed verify this is a well-defined inner product.
\end{proof}

A consequence of the above proof is that we can show that any function $f$ in a RKHS $\mathcal F$ with kernel $h$ can be written in the form $f(x) = \sum_{i=1}^n h(x, x_i)w_i$, with some $(w_1,\dots,w_n)\in\bbR^n$, $n \in \mathbb N$. 
More precisely, $\mathcal F$ is the completion of the space $\mathcal G = \text{span}\{h(\cdot,x) \, | \, x \in \mathcal X \}$ endowed with the inner product as stated in \eqref{eq:rkhsinnerprod}.





\section{Reproducing kernel Krein space theory}
Kreĭn spaces can be seen as a generalisation of Hilbert spaces, which caters for inner products not being positive definite.
To motivate the need for Kreĭn spaces, we first look at several operations on reproducing kernels and the resulting vector space.

\begin{lemma}[Scaling of kernels]\label{thm:scalingkernels}
  If $h$ is a kernel on $\cX$, and $\lambda \geq 0$ a scalar, then $\lambda h$ is a kernel.
  This yields a scaled RKHS $\cF_\lambda = \{\lambda f \,|\, f \in \cF \}$ with reproducing kernel $\lambda h$, where $\cF$ is the RKHS defined by $h$.
\end{lemma}

\begin{proof}
  Multiplying a positive definite function by a positive constant results in a positive definite function still, and thus defines a unique RKHS.
  The scaling of functions is seen through the fact that $\cF$ is the completion of the space spanned by the kernels, and hence $\cF_\lambda$ by the scaled kernels.
\end{proof}

\begin{lemma}[Sum of kernels]
  If $h_1$ and $h_2$ are kernels on $\cX_1$ and $\cX_2$ respectively, then $h = h_1 + h_2$ is a kernel on $\cX_1 \times \cX_2$.
  Moreover, denote $\cF_1$ and $\cF_2$ the RKHS defined by $h_1$ and $h_2$ respectively.
  Then $\cF = \cF_1 \oplus \cF_2$ is an RKHS defined by $h = h_1 + h_2$, where
  \[
    \cF_1 \oplus \cF_2 = \{ f:\cX_1\times\cX_2 \to\bbR \,|\, f = f_1 + f_2, f_1\in\cF_1 \text{ and } f_2\in\cF_2 \}.
  \]
  For all $f\in\cF$,
  \[
    \norm{f}_\cF^2 = \min_{f_1+f_2=f} \left\{ \norm{f_1}_{\cF_1}^2 + \norm{f_2}_{\cF_2}^2 \right\}.
  \]
\end{lemma}

\begin{proof}
  That $h_1+h_2$ is a kernel should be obvious, as the sum of two positive definite functions is also positive definite.
  For a proof of the remaining statements, see \citet[Theorem 5]{berlinet2011reproducing}.
\end{proof}

\begin{lemma}[Products of kernels]\label{thm:prodkernels}
  Let $\cF_1$ and $\cF_2$ be two RKHS of functions over $\cX_1$ and $\cX_2$, with respective reproducing kernels $h_1$ and $h_2$.
  Then, $h = h_1 h_2$ is a kernel on $\cX_1 \times \cX_2$.
  Moreover, the tensor product space $\cF_1 \otimes \cF_2$ is an RKHS with reproducing kernel $h$.
\end{lemma}

To prove this lemma we introduce the following definition of the \emph{kernel matrix}.

\begin{definition}[Kernel matrix]
  Let $\{x_1,\dots,x_n\}$ be a sample of observed data (covariates), where each $x_i\in\cX$, and $h$ a kernel over this set.
  Define the \emph{kernel matrix} $\bH$ for $h$ as the $n \times n$ matrix with $(i,j)$ entries equal to $h(x_i,x_j)$.
\end{definition}

The kernel matrix is also known as the \emph{Gram matrix}.
By definition, the kernel matrix is a positive-definite matrix: $\ba^\top\bH\ba = \sum_{i=1}^n\sum_{j=1}^n a_ia_jh(x_i,x_j) \geq 0$ for any choice of $a\in\bbR^n$.

\begin{proof}[Proof of Lemma \ref{thm:prodkernels}]
  Fix $n\in\bbN$, and let $\bH_1$ and $\bH_2$ be the kernel matrices for $h_1$ and $h_2$ respectively.
  Since these kernel matrices are symmetric and positive-definite, we can write $\bH_1 = \bB^\top\bB$ and $\bH_1 = \bC^\top\bC$ for some matrices $\bB$ and $\bC$---perform an (orthogonal) eigendecomposition of each of the kernel matrices, and take square roots of the eigenvalues.
  Let $\bH$ be the kernel matrix for $h = h_1h_2$.
  With $x_i = (x_{i1}, x_{i2})$, its $(i,j)$ entries are
  \begin{align*}
    h(x_i,x_j)
    &=h_1(x_{i1},x_{i2}) h_2(x_{j1},x_{j2}) \\
    &= (\bB^\top\bB)_{ij}\cdot (\bC^\top\bC)_{ij} \\
    &= \sum_{k=1}^n b_{ik}b_{jk} \sum_{l=1}^n c_{il}c_{jl},
  \end{align*}
  where we have denoted $b_{ij}$ and $c_{ij}$ to be the $(i,j)$th entries of $\bB$ and $\bC$ respectively 
  Then,
  \begin{align*}
    \sum_{i=1}^n\sum_{j=1}^n h(x_i,x_j)
    &= \sum_{k=1}^n \sum_{l=1}^n \sum_{i=1}^n \sum_{j=1}^n  a_ia_jb_{ik}b_{jk}c_{il}c_{jl} \\
    &= \sum_{k=1}^n \sum_{l=1}^n \left(\sum_{i=1}^n a_i b_{ik} c_{il} \right) \left( \sum_{j=1}^n  a_jb_{jk}c_{jl} \right) \\
    &= \sum_{k=1}^n \sum_{l=1}^n \left(\sum_{i=1}^n a_i b_{ik} c_{il} \right)^2 \\
    &\geq 0
  \end{align*}
  Again, for the remainder of the statement in the lemma, we refer to \citet[Theorem 13]{berlinet2011reproducing}.
\end{proof}

A familiar fact from linear algebra is realised here from Lemmas \ref{thm:scalingkernels}--\ref{thm:prodkernels}: 
1) Multiplying a positive definite matrix by a positive constant results in a positive definite matrix; 
2) the addition of positive definite matrices is a positive definite matrix; and 
3) the \emph{Hadamard product}\footnotemark~of two positive definite matrices is a positive definite matrix.
\footnotetext{The Hadamard product is an element-wise multiplication of two matrices $\bA$ and $\bB$ of identical dimensions, denoted $\bA \circ \bB$. That is, $(\bA \circ \bB)_{ij} = \bA_{ij}\bB_{ij}$.}


What if we want to consider scaling or sum or products of kernels that result in a non-positive definite function?
Will we lose all the nice properties of RKHS?
It turns out no---we just have to switch to Kreĭn spaces.
Krein spaces are inner product spaces endowed with Hilbertian topology, yet their inner products are no longer positive.

\begin{definition}[Negative and indefinite inner products]
  If instead of the positive-definiteness condition for inner products, as per the last item in Definition \ref{def:innerprod}, we have that $\forall f\in\cF$
  \[
    \ip{f,f}_\cF \leq 0,
  \]
  then the inner product is said to be \emph{negative-definite}.
  It is \emph{indefinite} if is is neither positive- nor negative-definite.
\end{definition}


\begin{definition}[Krein space]
  An inner product space $(\cF,\ip{\cdot,\cdot}_\cF)$ is a \emph{Krein space} if there exists two Hilbert spaces $\cF_+$ and $\cF_-$ spanning $\cF$ such that
  \begin{itemize}
    \item All $f\in\cF$ can be decomposed into $f = f_+ + f_-$, where $f_+\in\cF_+$ and $f_- \in \cF_-$.
    \item $\forall f,f'\in\cF$, $\ip{f,f'}_\cF = \ip{f_+,f_+'}_{\cF_+}- \ip{f_-,f_-'}_{\cF_-}$
  \end{itemize}
\end{definition}

This suggests that there is an ``associated'' Hilbert space.

\begin{definition}[Associated Hilbert space]
  Let $\cF$ be a Krein space with decomposition into Hilbert spaces $\cF_+$ and $\cF_-$.
  Denote by $\overline \cF$ the associated Hilbert space defined by
  \[
    \overline \cF = \cF_+ \oplus \cF_-,
  \]
  and hence $\ip{f,f'}_{\overline\cF} = \ip{f_+,f_+'}_{\cF_+} + \ip{f_-,f_-'}_{\cF_-}$.
  Likewise,
  \[
    \overline \cF = \cF_+ \ominus \cF_-,
  \]
  and hence $\ip{f,f'}_{\cF} = \ip{f_+,f_+'}_{\cF_+} - \ip{f_-,f_-'}_{\cF_-}$.
\end{definition}

\begin{definition}[Reproducing kernel Krein space]
  A Krein space $\cF$ of real-valued functions $f:\mathcal X \rightarrow \mathbb R$ on a non-empty set $\mathcal X$ is called a \emph{reproducing kernel Krein space} if the evaluation functional $\delta_x: f \mapsto f(x)$ is continuous on $\cF$, $\forall x \in \cX$, endowed with its strong topology. 
\end{definition}

$\overline\cF$ is the smallest Hilbert space majorizing the Krein space $\cF$.
The strong topology on $\cF$ is the Hilbertian topology of $\overline\cF$.
The topology does not depend on the decomposition chosen.
Clearly $|\ip{f,f}_\cF \leq \norm{f}^2_{\overline\cF}|$ for all $f\in\cF$.

For every reproducing kernel Krein space $\cF$ of functions over a set $\cX$, there corresponds a unique, positive-definite reproducing kernel $\hXXR$.

\begin{lemma}
  Let $\cF$ be an RKKS with $\cF = \cF_+ \ominus \cF_-$.
  Then,
  \begin{itemize}
    \item $\cF_+$ and $\cF_-$ are both RKKS with kernel $h_+$ and $h_-$.
    \item There is a unique symmetric $h(x,x')$ with $h(\cdot,x)\in\cF$ such that for all $f\in\cF$, 
    \[
      \ip{f,h(\cdot,x)}_\cF = f(x)
    \]
    \item $h = h_+ + h_-$.
  \end{itemize}
\end{lemma}

On the other hand, for every positive-definite function $\hXXR$, there corresponds at least one reproducing kernel Hilbert space $\cF$ that has $h$ as its reproducing kernel.

\begin{lemma}
  The following are equivalent.
  \begin{itemize}
    \item There exists (at least) one RKKS with kernel $h$.
    \item $h$ admits a positive decomposition, i.e., there exists two positive kernels $h_+$ and $h_-$ such that $h=h_++h_-$.
    \item $h$ is dominated by some kernel $k$, i.e. $h-k$ is a positive kernel.
  \end{itemize}
\end{lemma}

There is no bijection but a surjection between the set of RKKS and the set of generalized kernels defined in the vector space generated out of the cone of positive kernels.






















\section{RKHS building blocks}
\input{rkhs-building-blocks}

\section{Constructing RKKS from existing RKHS}
\input{rkks-construction}

\section{Summary}

Brief notes on functional analysis allows us to describe RKHS and RKKS.
These are of interest because of the convenient topologies.
All linear functionals are continuous in these spaces.
Moreover, RKHS and RKKS can be specified through kernel functions.
Although it is unique for RKHS, not so for RKKS, but need not matter greatly, because the important properties that we care about carry over anyway.
Of importance is the ANOVA functional decomposition, for which we can mimic through similar manipulation of scaled positive definite kernels.
Restricting to positive scalars is unsatisfactory, so RKKS is required.


\hClosingStuffStandalone
\end{document}