Kreĭn spaces can be seen as a generalisation of Hilbert spaces, which caters for inner products not being positive definite.
To motivate the need for Kreĭn spaces, we first look at several operations on reproducing kernels and the resulting vector space.

\begin{lemma}[Scaling of kernels]\label{thm:scalingkernels}
  If $h$ is a kernel on $\cX$, and $\lambda \geq 0$ a scalar, then $\lambda h$ is a kernel.
  This yields a scaled RKHS $\cF_\lambda = \{\lambda f \,|\, f \in \cF \}$ with reproducing kernel $\lambda h$, where $\cF$ is the RKHS defined by $h$.
\end{lemma}

\begin{proof}
  Multiplying a positive definite function by a positive constant results in a positive definite function still, and thus defines a unique RKHS.
  The scaling of functions is seen through the fact that $\cF$ is the completion of the space spanned by the kernels, and hence $\cF_\lambda$ by the scaled kernels.
\end{proof}

\begin{lemma}[Sum of kernels]
  If $h_1$ and $h_2$ are kernels on $\cX_1$ and $\cX_2$ respectively, then $h = h_1 + h_2$ is a kernel on $\cX_1 \times \cX_2$.
  Moreover, denote $\cF_1$ and $\cF_2$ the RKHS defined by $h_1$ and $h_2$ respectively.
  Then $\cF = \cF_1 \oplus \cF_2$ is an RKHS defined by $h = h_1 + h_2$, where
  \[
    \cF_1 \oplus \cF_2 = \{ f:\cX_1\times\cX_2 \to\bbR \,|\, f = f_1 + f_2, f_1\in\cF_1 \text{ and } f_2\in\cF_2 \}.
  \]
  For all $f\in\cF$,
  \[
    \norm{f}_\cF^2 = \min_{f_1+f_2=f} \left\{ \norm{f_1}_{\cF_1}^2 + \norm{f_2}_{\cF_2}^2 \right\}.
  \]
\end{lemma}

\begin{proof}
  That $h_1+h_2$ is a kernel should be obvious, as the sum of two positive definite functions is also positive definite.
  For a proof of the remaining statements, see \citet[Theorem 5]{berlinet2011reproducing}.
\end{proof}

\begin{lemma}[Products of kernels]\label{thm:prodkernels}
  Let $\cF_1$ and $\cF_2$ be two RKHS of functions over $\cX_1$ and $\cX_2$, with respective reproducing kernels $h_1$ and $h_2$.
  Then, $h = h_1 h_2$ is a kernel on $\cX_1 \times \cX_2$.
  Moreover, the tensor product space $\cF_1 \otimes \cF_2$ is an RKHS with reproducing kernel $h$.
\end{lemma}

To prove this lemma we introduce the following definition of the \emph{kernel matrix}.

\begin{definition}[Kernel matrix]
  Let $\{x_1,\dots,x_n\}$ be a sample of observed data (covariates), where each $x_i\in\cX$, and $h$ a kernel over this set.
  Define the \emph{kernel matrix} $\bH$ for $h$ as the $n \times n$ matrix with $(i,j)$ entries equal to $h(x_i,x_j)$.
\end{definition}

The kernel matrix is also known as the \emph{Gram matrix}.
By definition, the kernel matrix is a positive-definite matrix: $\ba^\top\bH\ba = \sum_{i=1}^n\sum_{j=1}^n a_ia_jh(x_i,x_j) \geq 0$ for any choice of $a\in\bbR^n$.

\begin{proof}[Proof of Lemma \ref{thm:prodkernels}]
  Fix $n\in\bbN$, and let $\bH_1$ and $\bH_2$ be the kernel matrices for $h_1$ and $h_2$ respectively.
  Since these kernel matrices are symmetric and positive-definite, we can write $\bH_1 = \bB^\top\bB$ and $\bH_1 = \bC^\top\bC$ for some matrices $\bB$ and $\bC$---perform an (orthogonal) eigendecomposition of each of the kernel matrices, and take square roots of the eigenvalues.
  Let $\bH$ be the kernel matrix for $h = h_1h_2$.
  With $x_i = (x_{i1}, x_{i2})$, its $(i,j)$ entries are
  \begin{align*}
    h(x_i,x_j)
    &=h_1(x_{i1},x_{i2}) h_2(x_{j1},x_{j2}) \\
    &= (\bB^\top\bB)_{ij}\cdot (\bC^\top\bC)_{ij} \\
    &= \sum_{k=1}^n b_{ik}b_{jk} \sum_{l=1}^n c_{il}c_{jl},
  \end{align*}
  where we have denoted $b_{ij}$ and $c_{ij}$ to be the $(i,j)$th entries of $\bB$ and $\bC$ respectively 
  Then,
  \begin{align*}
    \sum_{i=1}^n\sum_{j=1}^n h(x_i,x_j)
    &= \sum_{k=1}^n \sum_{l=1}^n \sum_{i=1}^n \sum_{j=1}^n  a_ia_jb_{ik}b_{jk}c_{il}c_{jl} \\
    &= \sum_{k=1}^n \sum_{l=1}^n \left(\sum_{i=1}^n a_i b_{ik} c_{il} \right) \left( \sum_{j=1}^n  a_jb_{jk}c_{jl} \right) \\
    &= \sum_{k=1}^n \sum_{l=1}^n \left(\sum_{i=1}^n a_i b_{ik} c_{il} \right)^2 \\
    &\geq 0
  \end{align*}
  Again, for the remainder of the statement in the lemma, we refer to \citet[Theorem 13]{berlinet2011reproducing}.
\end{proof}

A familiar fact from linear algebra is realised here from Lemmas \ref{thm:scalingkernels}--\ref{thm:prodkernels}: 
1) Multiplying a positive definite matrix by a positive constant results in a positive definite matrix; 
2) the addition of positive definite matrices is a positive definite matrix; and 
3) the \emph{Hadamard product}\footnotemark~of two positive definite matrices is a positive definite matrix.
\footnotetext{The Hadamard product is an element-wise multiplication of two matrices $\bA$ and $\bB$ of identical dimensions, denoted $\bA \circ \bB$. That is, $(\bA \circ \bB)_{ij} = \bA_{ij}\bB_{ij}$.}

The difference of kernels is not guaranteed to be a positive definite function.
To state the obvious, multiplication of a kernel by a negative scalar results in a negative definite function.
Such actions arise when building new function spaces from existing ones, such as via a polynomial-type or ANOVA-type construction, as we will see later in Section \ref{sec:constructrkks}.
Do we forfeit all the nice properties of RKHS?
As it turns out no, for the most relevant parts anyway.
For the remainder of this section, we shall review basic Kreĭn and reproducing kernel Kreĭn space theory, and comment on the similarity and differences between it and RKHS.
Kreĭn spaces are, first and foremost, characterised by an inner product which is non-positive.

\begin{definition}[Negative and indefinite inner products]
  Let $\ip{\cdot}_\cF$ be an inner product of a vector space $\cF$.
  It is said to be \emph{negative-definite} if for all $f\in\cF$, $\ip{f,f}_\cF \leq 0$.
  An inner product is \emph{indefinite} if is is neither positive- nor negative-definite.
\end{definition}

To quote \citet{ong2004learning}, Kreĭn spaces are indefinite inner product spaces endowed with a Hilbertian topology, yet their inner product is no longer positive. 

\begin{definition}[Kreĭn space]
  An inner product space $\big(\cF,\ip{\cdot,\cdot}_\cF\big)$ is a \emph{Krein space} if there exists two Hilbert spaces $\big(\cF_+,\ip{\cdot,\cdot}_{\cF_+}\big)$ and $\big(\cF_-,\ip{\cdot,\cdot}_{\cF_-}\big)$ spanning $\cF$ such that
  \begin{itemize}
    \item All $f\in\cF$ can be decomposed into $f = f_+ + f_-$, where $f_+\in\cF_+$ and $f_- \in \cF_-$.
    \item This decomposition is orthogonal, i.e. $\cF_+ \cup \cF_- = \{0\}$, and $\ip{f_+,f_-}_\cF=0$ for all $f_+\in\cF_+$ and $f_-\in\cF_-$, with the inner product on $\cF$ defined below.
    \item $\forall f,f'\in\cF$, $\ip{f,f'}_\cF = \ip{f_+,f_+'}_{\cF_+}- \ip{f_-,f_-'}_{\cF_-}$.
  \end{itemize}
\end{definition}

Let $P$ be the projection of the Kreĭn space $\cF$ onto $\cF_+$, and $Q=I-P$ the projection onto $\cF_-$.
These are caleld the \emph{fundamental projections} of $\cF$.
We shall refer to $\cF_+$ as the \emph{positive subspace}, and $\cF_-$ as the \emph{negative subspace}.
These monikers stem from the fact that for all $f,f' \in \cF$, $\ip{Pf,Pf'}_{\cF_+} \geq 0$ while $\ip{Qf,Qf'}_{\cF_-} \leq 0$.
We introduce the notation $\ominus$ to refer to the Kreĭn space decomposition: $\cF = \cF_+ \ominus \cF_-$.
There is then a notion of an \emph{associated Hilbert space}.

\begin{definition}[Associated Hilbert space]
  Let $\cF$ be a Kreĭn space with decomposition into Hilbert spaces $\cF_+$ and $\cF_-$.
  Denote by $\cF_\cH$ the associated Hilbert space defined by $\cF_\cH = \cF_+ \oplus \cF_-$, with inner product 
  \[
    \ip{f,f'}_{\cF_\cH} = \ip{f_+,f_+'}_{\cF_+} + \ip{f_-,f_-'}_{\cF_-},
  \]
  for all $f,f'\in\cF$.
%  Likewise,
%  \[
%    \cF = \cF_+ \ominus \cF_-,
%  \]
%  and hence $\ip{f,f'}_{\cF} = \ip{f_+,f_+'}_{\cF_+} - \ip{f_-,f_-'}_{\cF_-}$.
\end{definition}

The associated Hilbert space can be found via the linear operator $J = P - Q$ called the \emph{fundamental symmetry}.
%, which satisfies $J = J^{-1} = J^\top$.
That is, a Kreĭn space $\cF$ can be turned into its associated Hilbert space by using the positive-definite inner product of the associated Hilbert space as $\ip{f,f'}_{\cF_\cH} = \ip{f,Jf'}_\cF$, for all $f,f'\in\cF$.
The converse is true too: Starting from a Hilbert space $\cF_\cH$ and an operator $J$, the vector space endowed with the inner product $\ip{f,f'}_\cF = \ip{f,Jf'}_{\cF_\cH}$, for all $f,f'\in\cF$, is a Kreĭn space.

We realise that for a Kreĭn space $\cF$, $|\ip{f,f}_\cF| \leq \norm{f}^2_{\cF_\cH}|$ for all $f\in\cF$, and we say that $\cF_\cH$ majorises the $\cF$, and in fact it is the smallest Hilbert space to do so.
The strong topology on $\cF$ is defined to be the topology arising from the norm of $\cF_\cH$, and this does not depend on the decomposition chosen.

\begin{definition}[Reproducing kernel Krein space]
  A Krein space $\cF$ of real-valued functions $f:\mathcal X \rightarrow \mathbb R$ on a non-empty set $\mathcal X$ is called a \emph{reproducing kernel Krein space} if the evaluation functional $\delta_x: f \mapsto f(x)$ is continuous on $\cF$, $\forall x \in \cX$, endowed with its strong topology (i.e. the topology of its associated Hilbert space $\cF_\cH$).
\end{definition}

We wonder if the uniqueness theorem (Theorem \ref{thm:rkhsunique}) holds for RKKS.
On one hand, every RKKS $\cF$ of functions over a set $\cX$, there corresponds a unique reproducing kernel $\hXXR$.

\begin{lemma}[Uniqueness of kernel for RKKS]
  Let $\cF$ be an RKKS of functions over a set $\cX$, with $\cF = \cF_+ \ominus \cF_-$.
  Then, $\cF_+$ and $\cF_-$ are both RKHS with kernel $h_+$ and $h_-$, and the kernel $h = h_+ - h_-$ is a unique, symmetric, reproducing kernel for $\cF$.  
\end{lemma}

\begin{proof}
  Since $\cF$ is a RKKS, evaluation functionals are continuous on $\cF$ with respect to topology of the associated Hilbert space $\cF_\cH = \cF_+ \oplus \cF_-$.
  Therefore, $\cF_\cH$ is a RKHS, and so too are $\cF_+$ and $\cF_-$ with respective kernels $h_+$ and $h_-$.
  
  Furthermore, $h(\cdot,x) \in\cF$ since $h_+(\cdot,x) \in\cF_+$ and $h_-(\cdot,x) \in\cF_-$ for some $x\in\cX$.
  Then, for any $f\in\cF$,
  \begin{align*}
    \ip{f, h(\cdot,x)}_\cF 
    &= \ip{f, h_+(\cdot,x)}_{\cF} - \ip{f, h_-(\cdot,x)}_{\cF} \\ 
    &= \ip{f_+, h_+(\cdot,x)}_{\cF_+} - \cancelto{0}{\ip{f_-, h_+(\cdot,x)}_{\cF_-}} \\
    &\phantom{==} - \cancelto{0}{\ip{f_+, h_-(\cdot,x)}_{\cF_+}} + \ip{f_-, h_-(\cdot,x)}_{\cF_-} \\
    &= f_+(x) + f_-(x) \\
    &= f(x)
  \end{align*}
  The last two lines are achieved by linearity of evaluation functionals ($\delta_x(f_+) + \delta_x(f_-) = \delta_x(f_+ + f_-)$), and the fact that $f = f_+ + f_-$ (by the Kreĭn space decomposition).
  We have that $h=h_+ - h_-$ is a reproducing kernel for $\cF$.
  Uniqueness follows as a consequence of the non-degeneracy condition of the respective inner products for $\cF_+$ and $\cF_-$.
\end{proof}

\begin{remark}
  As we said earlier, difference in kernels may not be positive-definite and therefore not kernels in the truest sense of the word.
  Rather, they should be referred to as \emph{generalised kernels}, as they are defined in the vector space generated out of the cone of positive kernels. 
  Regardless, we shall keep referring to them as kernels for brevity.
\end{remark}

On the other hand, for every kernel $\hXXR$, there corresponds \emph{at least} one RKKS $\cF$ that has $h$ as its reproducing kernel.

\begin{lemma}[Non-uniqueness of RKKS for kernel]
  Let $h$ be a kernel over $\cX$.
  There is (at least) one associated RKKS with kernel $h$ if and only if $h$ can be decomposed as the difference between two positive kernels $h_+$ and $h_-$ over $\cX$, i.e., $h=h_+-h_-$.
\end{lemma}

\begin{proof}
  The existence and (non-)uniqueness of a RKKS associated with a kernel $h$ relies on the ability to complete the span of $h(\cdot,x)$, much like in the proof of the Moore-Aronszajn theorem.
  As it turns out, this is rather involved, so is omitted in the interest of maintaining coherence to the discussion at hand.
  This subject has been studied by various authors, one may refer to a proof of this lemma in works by \citet[Theorem 2 \& Example in Section 4]{alpay1991some}, and \citet[Theorem 2.28]{mary2003hilbertian}.
\end{proof}

The take-away message as we close this section is that there is no bijection, but a surjection, between the set of RKKS and the set of (generalised) kernel functions.
In any case, Hilbertian topology applies to Kreĭn spaces via the associated Hilbert space, and in particular, RKKS provide a functional space for which evaluation functionals are continuous.




















