\begin{definition}[Kernel matrix]
  Let $\{x_1,\dots,x_n\}$ be a sample of observed data (covariates), where each $x_i\in\cX$, and $h$ a kernel over this set.
  Define the \emph{kernel matrix} $\bH$ as the $n \times n$ matrix with $(i,j)$ entries equal to $h(x_i,x_j)$.
\end{definition}

The kernel matrix is also known as the \emph{Gram matrix}.
By definition, the kernel matrix is a positive-definite matrix: $\ba^\top\bH\ba = \sum_{i=1}^n\sum_{j=1}^n a_ia_jh(x_i,x_j) \geq 0$ for any choice of $a\in\bbR^n$.

\begin{lemma}[Scaling of kernels]
  If $h$ is a kernel on $\cX$, and $\lambda \geq 0$ a scalar, then $\lambda h$ is a kernel.
  This yields a scaled RKHS $\cF_\lambda$ with reproducing kernel $\lambda h$.
\end{lemma}

\begin{lemma}[Sum of kernels]
  If $h_1$ and $h_2$ are kernels on $\cX_1$ and $\cX_2$ respectively, then $h = h_1 + h_2$ is a kernel on $\cX_1 \times \cX_2$.
  Moreover, denote $\cF_1$ and $\cF_2$ the RKHS defined by $h_1$ and $h_2$ respectively.
  Then $\cF = \cF_1 \oplus \cF_2$ is an RKHS defined by $h = h_1 + h_2$, where
  \[
    \cF_1 \oplus \cF_2 = \{ f:\cX_1\times\cX_2 \to\bbR | f = f_1 + f_2, f_1\in\cF_1 \text{ and } f_2\in\cF_2 \}.
  \]
  For all $f\in\cF$,
  \[
    \norm{f}_\cF^2 = \min_{f_1+f_2=f} \left\{ \norm{f_1}_{\cF_1}^2 + \norm{f_2}_{\cF_2}^2 \right\}.
  \]
\end{lemma}

\begin{lemma}[Products of kernels]
  If $h_1$ and $h_2$ are kernels on $\cX_1$ and $\cX_2$ respectively, then $h = h_1 h_2$ is a kernel on $\cX_1 \times \cX_2$.
  Moreover, denote $\cF_1$ and $\cF_2$ the RKHS defined by $h_1$ and $h_2$ respectively, and $\cF$ the RKHS defined by $h = h_1 h_2$.
  There is an isometric isomorphism between $\cF$ and the tensor product space $\cF_1 \otimes \cF_2$.
\end{lemma}

We have a familiar fact from linear algebra: 1) Multiplying a positive definite matrix by a positive constant results in a positive definite matrix; 2) Addition of two positive definite matrices results in a positive definite matrix; and 3) the Hadamard product of two positive definite matrices is a positive definite matrix.

What if we want to consider scaling or sum or products of kernels that result in a non-positive definite function?
Will we lose all the nice properties of RKHS?
It turns out no---we just have to switch to Krein spaces.
Krein spaces are inner product spaces endowed with Hilbertian topology, yet their inner products are no longer positive.

\begin{definition}[Negative and indefinite inner products]
  If instead of the positive-definitess condition for inner products, as per the last item in Definition \ref{def:innerprod}, we have that $\forall f\in\cF$
  \[
    \ip{f,f}_\cF \leq 0,
  \]
  then the inner product is said to be \emph{negative-definite}.
  It is \emph{indefinite} if is is neither positive- nor negative-definite.
\end{definition}


\begin{definition}[Krein space]
  An inner product space $(\cF,\ip{\cdot,\cdot}_\cF)$ is a \emph{Krein space} if there exists two Hilbert spaces $\cF_+$ and $\cF_-$ spanning $\cF$ such that
  \begin{itemize}
    \item All $f\in\cF$ can be decomposed into $f = f_+ + f_-$, where $f_+\in\cF_+$ and $f_- \in \cF_-$.
    \item $\forall f,f'\in\cF$, $\ip{f,f'}_\cF = \ip{f_+,f_+'}_{\cF_+}- \ip{f_-,f_-'}_{\cF_-}$
  \end{itemize}
\end{definition}

This suggests that there is an ``associated'' Hilbert space.

\begin{definition}[Associated Hilbert space]
  Let $\cF$ be a Krein space with decomposition into Hilbert spaces $\cF_+$ and $\cF_-$.
  Denote by $\overline \cF$ the associated Hilbert space defined by
  \[
    \overline \cF = \cF_+ \oplus \cF_-,
  \]
  and hence $\ip{f,f'}_{\overline\cF} = \ip{f_+,f_+'}_{\cF_+} + \ip{f_-,f_-'}_{\cF_-}$.
  Likewise,
  \[
    \overline \cF = \cF_+ \ominus \cF_-,
  \]
  and hence $\ip{f,f'}_{\cF} = \ip{f_+,f_+'}_{\cF_+} - \ip{f_-,f_-'}_{\cF_-}$.
\end{definition}

\begin{definition}[Reproducing kernel Krein space]
  A Krein space $\cF$ of real-valued functions $f:\mathcal X \rightarrow \mathbb R$ on a non-empty set $\mathcal X$ is called a \emph{reproducing kernel Krein space} if the evaluation functional $\delta_x: f \mapsto f(x)$ is continuous on $\cF$, $\forall x \in \cX$, endowed with its strong topology. 
\end{definition}

$\overline\cF$ is the smallest Hilbert space majorizing the Krein space $\cF$.
The strong topology on $\cF$ is the Hilbertian topology of $\overline\cF$.
The topology does not depend on the decomposition chosen.
Clearly $|\ip{f,f}_\cF \leq \norm{f}^2_{\overline\cF}|$ for all $f\in\cF$.

For every reproducing kernel Krein space $\cF$ of functions over a set $\cX$, there corresponds a unique, positive-definite reproducing kernel $\hXXR$.

\begin{lemma}
  Let $\cF$ be an RKKS with $\cF = \cF_+ \ominus \cF_-$.
  Then,
  \begin{itemize}
    \item $\cF_+$ and $\cF_-$ are both RKKS with kernel $h_+$ and $h_-$.
    \item There is a unique symmetric $h(x,x')$ with $h(\cdot,x)\in\cF$ such that for all $f\in\cF$, 
    \[
      \ip{f,h(\cdot,x)}_\cF = f(x)
    \]
    \item $h = h_+ + h_-$.
  \end{itemize}
\end{lemma}

On the other hand, for every positive-definite function $\hXXR$, there corresponds at least one reproducing kernel Hilbert space $\cF$ that has $h$ as its reproducing kernel.

\begin{lemma}
  The following are equivalent.
  \begin{itemize}
    \item There exists (at least) one RKKS with kernel $h$.
    \item $h$ admits a positive decomposition, i.e., there exists two positive kernels $h_+$ and $h_-$ such that $h=h_++h_-$.
    \item $h$ is dominated by some kernel $k$, i.e. $h-k$ is a positive kernel.
  \end{itemize}
\end{lemma}

There is no bijection but a surjection between the set of RKKS and the set of generalized kernels defined in the vector space generated out of the cone of positive kernels.




















