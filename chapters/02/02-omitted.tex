At first glance, this may seem strange, that the notion of functions (as mappings from input to output space) and vector spaces are somehow equatable.
Upon further thought, one realises that firstly, two functions of a similar, particular form may be added together (in some meaningful way) resulting in a function in that same form. 
Secondly, multiplication of a function by a scalar $c$ can be thought of as $c$ times the output of that function.
Indeed, running through the checklist\footnotemark~of what constitutes a vector space, we find that a ``space of functions'' satisfies them all.
\footnotetext{In modern linear algebra texts, this is the eight axioms of vector spaces over a field $\bbF$: The vectors forms an abelian group under addition, and this group has an $\bbF$-module structure.}

FOR TOPOLOGICAL SPACES ONLY: 
On the other hand, a set which contains all of its limit points is said to be \emph{closed}.
Clearly, a complete set must be closed, but a closed set need not necessarily be complete.

\begin{corollary}[Riesz norm]
  For any $f\in\cF$ a Hilbert space, define $L(f) = \ip{f,g}_\cF$ for some $g\in\cF$.
  Then $\norm{L}_{\cF'} = \norm{g}_\cF$. 
\end{corollary}

\begin{proof}
%  $L$ as defined by $L = \ip{\cdot,g}_\cF$ for some $g\in\cF$ is a linear operator $L:\cF\to\cF'$,
%  so from Definition \ref{def:boundedop} of operator norms,
%  \begin{align*}
%    \norm{L} &= \sup_{f\in\cF} \frac{\norm{L(f)}_{\cF'}}{\norm{f}_\cF} \\
%    &= \sup_{f\in\cF} \frac{|\ip{f,g}_{\cF}|}{\norm{f}_\cF} \\
%    &=  \frac{|\ip{g,g}_{\cF}|}{\norm{g}_\cF} \\
%    &= \norm{g}_\cF
%  \end{align*}
%  by the Cauchy-Schwarz inequality.
%  Alternative proof:
  By Cauchy-Schwarz,
  \[
    |L(f)| \leq \norm{f}_\cF\norm{g}_\cF
  \]
  so $\norm{L}_{\cF'}\leq \norm{g}_\cF$.
  But  $|L(g)| = \norm{g}_\cF^2$, so in fact $\norm{L}_{\cF'} = \norm{g}$
\end{proof}